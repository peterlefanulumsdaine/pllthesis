
\section{Background}

\begin{para}
A fundamental construction in all flavours of type theory is the \emph{classifying category} $\cl(\T)$ of a theory $\T$.  Objects of $\cl(\T)$ are types (or contexts) of $\T$; morphisms of $\T$ are open terms of one type  with free variables from another, up to syntactic equivalence of some sort.

The uses of this construction are manifold.  First of all, it provides a tool for structural analysis of theories, allowing type-theoretic constructors and rules to be understood as presentations of categorical structure (products, sums, exponential objects, \ldots) familiar from elsewhere in mathematics.

Next, this allows us to model the type theory in any suitably structured category, via a right adjoint $\Lang$ to the functor $\cl$, giving the \emph{internal language} of a structured category.

Putting these together, it allows us to represent type theories as \emph{equivalent} to appropriately structured categories, giving us an alternate presentation with different strengths to the traditional syntax.
\end{para}

\begin{para}
For some logical systems---simple type theory, for instance, and some extensional versions of Martin-Löf type theory---this correspondence is very well understood and leaves little to be desired.  Unfortunately, the situation with the intensional variants of Martin-Löf type theory---ITT, henceforth---is not yet so satisfactory.

Syntactic categories for theories in ITT still work essentially as described above: they are a powerful technical tool for providing models, and for representing theories.  The subtlety is that the structures corresponding to logical constructors in ITT are no longer so categorically familiar as before: they are not quite the products, exponentials, adjoints and so on that one might expect.  The problem\footnote{although, of course, ``it's not a bug, it's a feature''!} is the interaction of its two different kinds of equality.

Firstly, there is \emph{definitional} (aka \emph{intensional}) equality:
\[\Gamma\ \types\ a = a' : A\]
a decidable, quite fine syntactic relation---this covers essentially $\alpha$-equivalence, $\beta$-equivalence, and sometimes $\eta$-, but no more.  Its strictness reflects the extremely well-controlled computational behaviour of the system.  This is the equality used in the classifying category.  Crucially, however, it is a separate judgement of the system, not represented by a type, so cannot be reasoned about by e.g.\ induction within the system.  One cannot prove in ITT, for instance, that 
\[x : \N\ \types\ x+0 = x\ : \N. \]

Secondly, since one does need some kind of equality which can be reasoned about internally, there is \emph{propositional} (sometimes aka \emph{extensional}) equality, which is represented by a dependent type, the \emph{identity type} over each type:
\[ x, y : A\ \types\ \Id_A(x,y)\ \type\]
An inhabitant of this type is seen as a \emph{proof of equality} between $x$ and $y$.  This is a coarser equality than the definitional; it is axiomatised simply by $\intro$/$\elim$ rules of the usual form for an inductive type, but they turn out to yield remarkably subtle consequences.
\end{para}

\begin{para}
In the extensional theory, the two equalities coincide by fiat (the \emph{reflection} rule).  This makes reasoning about equality within the theory much simpler, but at the cost of desirable computational properties: for instance, type-checking becomes undecidable, since it depends (because of the dependency of types on terms) on definitional equality, which is now reduced to propositional, i.e.\ to inhabitation of types, which is undecidable in any sufficiently rich system.  Relatedly, the computational content of terms is lost: terms of different computational behaviour may become identified in this stronger definitional equality.  In these and other proof-theoretic respects, the intensional theory is significantly preferable.

As mentioned above, however, categorical analysis of constructors in the intensional theory is much less clear.  For instance, a unit type\footnote{as usually axiomatised in ITT, i.e.\ as an inductive type with one nullary constructor.} does \emph{not} give a terminal object of the syntactic category.  It may have multiple morphisms from other objects, at least up to definitional equality.  Internally, these terms will be equal, but only up to \emph{propositional} equality---so we need to incorporate this somehow into the structure of the syntactic category.  

(Of course, propositional equality is already present in the classifying category, but as terms of some other type; but we want a description which makes more apparent its rôle in the categorical analysis of the constructors.)

A naïve approach might be to simply quotient by propositional equality; but this destroys too much of the structure of the theory. 
\end{para}

\begin{para}Instead, starting with the Hofmann-Streicher groupoid model \cite{hofmann-streicher}, higher categories and related homotopy-theoretic structures have emerged as a natural solution to this problem: we add the propositional equality, but as extra \emph{structure}, not just as a relation to quotient by.

In the globular approach to higher categories, a higher category has objects (``0-cells'') and arrows (``1-cells'') between objects, as in a category, but also 2-cells between 1-cells, and so on, with various composition operations and laws depending on the kind of category in question (strict or weak, $n$- or $\omega$-, $\ldots$).

Now we see propositional equalities between terms as \emph{2-cells} between morphisms; and also, since propositional equalities themselves are terms of a possibly non-trivial type, we treat further propositional equalities between them as higher cells, and so on \emph{ad infinitum}.  Binary composition of these higher cells corresponds now to the transitivity of equality; higher identity cells, to reflexivity; symmetry of equality, to having some sort of inverses for cells.

There is a compelling analogy here with homotopy theory.  The higher fundamental groupoid of a space consists of points of the space, paths between points, homotopies between paths, \ldots\ Homotopies are composable, and have units, and inverses.

A type, therefore, can be seen as something like a space: a category with cells of arbitrarily high dimension, in which all cells are invertible, i.e.\ an \emph{$\omega$-groupoid}.  An entire theory now looks something like a category of spaces: an $\omega$ category in which all cells of dimension $> 1$ are invertible, i.e.\ an \emph{($\omega,1)$-category} (also known as $(\infty,1)$).

However, in the fundamental groupoid example, it is a familiar fact that associativity of composition, and other laws, hold only up to homotopy, i.e.\ only up to a cell of higher dimension.  We find the same situation in type theory: composition is generally associative, unital etc.\ only up to propositional equality.  (Of course, in the presence of extra axioms, or in certain models, it can sometimes be strict.)  For the fully intensional theory we thus have to work not with strict but with \emph{weak} higher categories.

Various different definitions of weak $\omega$-categories exist; we use the \emph{globular operadic} definition of Batanin, as modified by Leinster. 
\end{para}

\begin{para}
Since \cite{hofmann-streicher}, ideas along these lines have been explored by various authors in various directions.  For ``two-dimensional ITT'' and certain weak 2-categories, the full correspondence is worked out in detail in \cite{garner:2-d-models}.  In the semantic direction, models of ITT in various higher categorical/homotopical settings are investigated in \cite{awodey-warren}, \cite{warren:thesis}, \cite{garner-van-den-berg:top-and-simp-models}, and \cite{voevodsky:univalent-notes}.  Conversely, the structures formed from the syntax of theories have been previously investigated in \cite{gambino-garner}, \cite{lumsdaine:weak-w-cats-from-itt-lmcs}, \cite{garner-van-den-berg}, \cite{awodey-hofstra-warren}.  This dissertation continues the latter line of investigation: the main goal is to construct the \emph{classifying weak $\omega$-category} of a theory.
\end{para}


\section{Overview of results}

\subsection*{The fundamental weak \pdfomega-category of a type (Ch.~\ref{ch:fundamental})}

\begin{para} \label{para:fundamental-sketch} As a warm-up for the main result, in this chapter we construct the ``fundamental weak $\omega$-category''\footnote{named by analogy with the higher fundamental groupoids of a space; in our case the fundamental weak $\omega$-category is again groupoidal, as shown in \cite{garner-van-den-berg}, but we do not discuss this in the present work.} of a single type within a theory.  

Specifically, we show that for any type $A$ and context $\Delta$ in a type theory $\T$ with at least the $\Id$-type rules, there is a weak $\omega$-category in which 0-cells are terms of type $A$ in context $\Delta$, 1-cells are terms of type $\Id_A$ between these, 2-cells are terms of type $\Id_{\Id_A}$ between 1-cells, and so on.

(Note that for this construction the dimensions of cells are always one lower than in the classifying category sketched above.  This comes from the general rule that if $X$, $A$ are objects of an $n$-category $\C$, then $\C(X,A)$ forms an $(n-1)$-category, whose 0-cells are 1-cells of $\C$, and so on.)
\end{para}

\begin{para}Before sketching the construction, we roughly recall the globular operadic definition of weak higher categories.  (The full background required on this material is set out in Ch.~\ref{ch:cat-background}.)

An $\omega$-category $\C$ has a set\footnote{for the present work, we consider \emph{small} higher categories only.} $C_n$ of ``$n$-cells'' for each $n > 0$.  The $0$- and $1$-cells correspond to the objects and arrows of an ordinary category: each arrow $f$ has source and target objects $a = s(f)$, $b = t(f)$.  Similarly, the source and target of a 2-cell $\alpha$ are a parallel pair of 1-cells $f,g: a \two b$, and generally the source and target of an $(n+1)$-cell are a parallel pair of $n$-cells.

Cells of each dimension can be composed along a common boundary in any lower dimension, and in a \emph{strict} $\omega$-category, the composition satisfies various associativity, unit, and interchange laws, captured by the \emph{generalised associativity} law: each labelled pasting diagram has a unique composite. (See illustrations in Fig.\ \ref{figure:assoc-laws}).
\end{para}

\begin{figure}
\[
\begin{array}{c}
\begin{array}{cccc}
\ \xy
(0,0)*{\bullet};
(0,80)*{a};
\endxy \quad
&
\ \xy
(0,0)*{\bullet}="a";
(0,80)*{\scriptstyle a};
(400,0)*{\bullet}="b";
(400,80)*{\scriptstyle b};
{\ar "a";"b"};
(200,80)*{f};
\endxy \ 
&
\ \xy
(0,0)*+{\bullet}="a";
(0,80)*{\scriptstyle a};
(450,0)*+{\bullet}="b";
(450,80)*{\scriptstyle b};
{\ar@/^1pc/^{f} "a";"b"};
{\ar@/_1pc/_{g} "a";"b"};
{\ar@{=>} (210,85)*{};(210,-85)*{}};
(280,0)*{\alpha};
\endxy \ 
&
\ \xy
(0,0)*+{\bullet}="a";
(0,80)*{\scriptstyle a};
(600,0)*+{\bullet}="b";
(600,80)*{\scriptstyle b};
{\ar@/^1.75pc/^{f} "a";"b"};
{\ar@/_1.75pc/_{g} "a";"b"};
{\ar@2{->}@/_0.5pc/ (220,140);(220,-140)} ;
{\ar@2{-}@/_0.5pc/|{\alpha} (203,120);(203,-120)} ;
{\ar@2{->}@/^0.5pc/ (380,140);(380,-140)} ;
{\ar@2{-}@/^0.5pc/|{\beta} (397,120);(397,-120)} ;
{\ar@3{->} (225,-20);(375,-20)};
(300,60)*{\Theta};
\endxy \ 
\end{array} \\
\begin{array}{ccc}
\ \xy(0,0)*{\bullet}="a";
(0,80)*{\scriptstyle a};
(300,0)*{\bullet}="b";
(300,80)*{\scriptstyle b};
(600,0)*{\bullet}="c";
(600,80)*{\scriptstyle c};
{\ar^f "a";"b"};
{\ar^g "b";"c"};
\endxy \ 
&
\ \xy
(0,0)*+{\bullet}="a";
(0,80)*{\scriptstyle a};
(500,0)*+{\bullet}="b";
(500,80)*{\scriptstyle b};
{\ar@/^1.75pc/|f "a";"b"};
{\ar|{f'} "a";"b"};
{\ar@/_1.75pc/|{f''} "a";"b"};
{\ar@{=>}^{\alpha} (250,160)*{};(250,50)*{}} ;
{\ar@{=>}^{\gamma} (250,-50)*{};(250,-160)*{}} ;
(0,-250)*{\ };
\endxy \ 
&
\ \xy
(0,0)*+{\bullet}="a";
(0,80)*{\scriptstyle a};
(400,0)*+{\bullet}="b";
(400,80)*{\scriptstyle b};
(800,0)*+{\bullet}="c";
(800,80)*{\scriptstyle c};
{\ar@/^1.1pc/|f "a";"b"};
{\ar@/_1.1pc/|{f'} "a";"b"};
{\ar@/^1.1pc/|g "b";"c"};
{\ar@/_1.1pc/|{g'} "b";"c"};
{\ar@{=>}^{\alpha} (200,80)*{};(200,-80)*{}} ;
{\ar@{=>}^{\beta} (600,80)*{};(600,-80)*{}} ;
\endxy \ \\
g \cdot_0 f &
\gamma \cdot_1 \alpha &
\beta \cdot_0 \alpha
\end{array}
\\
\begin{array}{cc}
\ \xy(0,0)*{\bullet}="a";
%(0,80)*{\scriptstyle a};
(300,0)*{\bullet}="b";
%(300,80)*{\scriptstyle b};
(600,0)*{\bullet}="c";
%(600,80)*{\scriptstyle c};
(900,0)*{\bullet}="d";
%(900,80)*{\scriptstyle d};
{\ar^f "a";"b"};
{\ar^g "b";"c"};
{\ar^h "c";"d"};
\endxy \ &
\ \xy
(0,0)*+{\bullet}="a";
(400,0)*+{\bullet}="b";
{\ar@/^1.5pc/ "a";"b"};
{\ar "a";"b"};
{\ar@/_1.5pc/ "a";"b"};
{\ar@{=>}^{\alpha} (200,150)*{};(200,25)*{}} ;
{\ar@{=>}^{\gamma} (200,-25)*{};(200,-150)*{}} ;
(800,0)*+{\bullet}="c";
{\ar@/^1.5pc/ "b";"c"};
{\ar "b";"c"};
{\ar@/_1.5pc/ "b";"c"};
{\ar@{=>}^{\beta} (600,150)*{};(600,25)*{}} ;
{\ar@{=>}^{\delta} (600,-25)*{};(600,-150)*{}};
(0,250)*{\ };
(0,-220)*{\ };
\endxy \ \\
\begin{array}{c} h \cdot_0 (g \cdot_0 f) =  \\ (h \cdot_0 g) \cdot_0 f \end{array} &
\begin{array}{c}(\delta \cdot_0 \gamma) \cdot_1 (\beta \cdot_0 \alpha) = \\
(\gamma \cdot_1 \alpha) \cdot_0 (\delta \cdot_1 \beta)\end{array}
\end{array}
\end{array}
\]
\caption{Some cells, composites, and associativities in a strict $\omega$-category \label{figure:assoc-laws}} 
\end{figure}

\begin{para} \label{para:intro-examples} In a \emph{weak} $\omega$-category, we do not expect strict associativity, so may have multiple composites for a given pasting diagram, but we do demand that these composites agree up to cells of the next dimension (``up to homotopy''), and that these associativity cells satisfy certain coherence laws of their own, again up to cells of higher dimension, and so on.
This is exactly the situation we find in intensional type theory.  For instance, even in constructing a term witnessing the transitivity of identity---that is, a composition law for the pasting diagram $(\xymatrix{ \bullet \ar[r] & \bullet \ar[r] & \bullet })$, or explicitly a term $c$ such that 
\[x,y,z:X,\ p:\Id(x,y),\ q:\Id(y,z) \types c(q,p): \Id(x,z)\]
---one finds that there is no single canonical candidate: most obvious are the two equally natural terms $c_l$, $c_r$ obtained by applying ($\Id$-\elim) to $p$ or to $q$ respectively.  These are not definitionally equal, but are propositionally equal, i.e.\ equal up to a 2-cell: there is a term $e$ with
\[x,y,z:X,\ p:\Id(x,y),\ q:\Id(y,z) \types e(q,p): \Id(c_l(q,p),c_r(q,p)).\]
Similarly, for either of these operations (or any combination), we can derive a ``propositional associative law'':
\begin{multline*}
x,y,z,w:X,\ p:\Id(x,y),\ q:\Id(y,z),\ r:\Id(z,w)\ \types \\
a(r,q,p) : \Id ( c(r,c(q,p))\,,\,c(c(r,q),p) ).
\end{multline*}
\end{para} 

\begin{para}In Leinster's definition \cite{leinster:book}, a system of composition laws of this sort is wrapped up in the algebraic structure of a \emph{globular operad with contraction}, and a weak $\omega$-category is given by a globular set equipped with an \emph{action} of such an operad.  We generalise this slightly, to define an \emph{internal weak $\omega$-category} in any suitable category $\C$.

Accordingly, we would like to find an operad-with-contraction $P_\Id$ of all such type-theoretically definable composition laws, acting on terms of any type and its identity types.  Formally, we consider $\T_\Id[X]$, the theory axiomatised just by the structural and $\Id$-rules plus a single generic base type $X$.  The operad $P_\Id$ of definable composition laws therein may then be formally constructed as an endomorphism operad in (presheaves on) its syntactic category $\cl(\T_\Id [X])$; and by some analysis of $\T_\Id[X]$, we show that $P_\Id$ is contractible.
\end{para}

Since $X$ is generic, composition laws defined on it can be implemented on all other types, giving the main result of Chapter \ref{ch:fundamental}: 

\begin{mainthmfund}Let $\T$ be any type theory extending $\T_\Id$, and $A$ any type of $\T$.  Then the system of types $(A, \Id_A, \Id_{\Id_A}, \ldots)$ is equipped naturally with a $P_\Id$-action, and hence with the structure of an internal weak $\omega$-category in $\C(\T)$.
\end{mainthmfund}

From this it follows that the terms of these types in any fixed context form an (external) $P_\Id$ algebra and weak $\omega$-category.  Indeed, one can construct, using the $\Id$-rules, \emph{identity contexts} $\Id_\Gamma$ over all contexts, satisfying rules analogous to those for identity types; so for any contexts $\Delta$, $\Gamma$, there is an $\omega$-category of context morphisms from $\Delta$ into $\Gamma$ and its identity types.

The material of this chapter is essentially based on my earlier paper \cite{lumsdaine:tlca-proceedings}/\cite{lumsdaine:weak-w-cats-from-itt-lmcs}.  A very similar construction (overlapping in some parts and using different techniques in others)  was discovered independently by Benno van den Berg and Richard Garner, and appears in \cite{garner-van-den-berg}. \\

\subsection*{The classifying weak \pdfomega-category of a theory (Ch.~\ref{ch:classifying})}

\begin{para}In this chapter, the heart of the dissertation, we construct the classifying $\omega$-category $\cl_\omega(\T)$ of a theory $T$.

As intimated above, $\cl_\omega(\T)$ should in dimensions $\leq 1$ be just the usual classifying category $\cl(\T)$ of contexts and context morphisms; its 2-cells should be morphisms into identity contexts, and higher cells should be maps into higher identity contexts.  From another point of view, recall that as a category has hom-sets, a 2-category has hom-categories, and generally a $(1+n)$-category has hom-$n$-categories between its $0$-cells.  Then the hom-$\omega$-category $\cl_\omega(\T)(\Delta,\Gamma)$ is just the fundamental $\omega$-category of $\Gamma$ in context $\Delta$, as sketched in the previous section.

In fact we first construct a weak $\omega$-category $\cl_\omega^-(\T)$, as above but with 0-cells just types, not more general contexts; then, we bump this up formally to include contexts as well.

The core technique used is analogous again to the familiar construction of the higher fundamental group(oid)s of a space: \emph{representability}.  Points, paths, homotopies, \ldots\ in a space $X$ are constructed as maps into $X$ from the point, the interval, the 1-disc, \ldots\ ; correspondingly, the cells of $\cl_\omega^-(\T)$ can be seen as interpretations in $\T$ of the theories of a free type, a free open term, \ldots\ ---maps into $\T$ from the \emph{type-theoretic globes}, which we denote $\globes$.  (This is why we start with $\cl_\omega^-$ not $\cl_\omega$: there is no ``theory of a free context'', since contexts can have various lengths.)

Then, as usual with representable algebraic structures\footnote{in fact inevitably, by the Yoneda lemma.}, the weak $\omega$-category structure on $\cl_\omega(\T)$ is induced formally from a weak $\omega$-cocategory structure on the representing objects $\globes$.

To give this structure, we once again consider an endomorphism operad---in fact now the \emph{co-endomorphism} operad of the globes, whose operations\todo{to be conscientious we should really be calling them co-operations.} now represent composition laws that can be defined generically over multiple types and morphisms between them, and we show this is contractible by techniques extending those of Ch.~\ref{ch:fundamental}.
\end{para}

\begin{para}
To carry this out, we require two main additional pieces of technical machinery.  Firstly, to handle the representing objects, co-endomorphism operads, and so on, we need a well-behaved category $\DTT$ of type theories in which to work.  In fact, we consider various such categories $\DTT_\stuff$, given by fixing some set $\stuff$ of constructors and rules ($\Id$-types, $\Pi$-types, etc.), and considering all theories extending these purely algebraically.

This type-theoretic background is set out in Ch.~\ref{ch:dtt-background}.  In order to facilitate the categorical analysis of $\DTT$, we recall the representation of theories as certain \emph{categories with attributes}, and the equivalence of this with more traditional syntactic presentations.

On the other hand, for showing contractibility of the operads in these categories, we need analogues in $\DTT_\stuff$ of $\Id$-elimination in syntactic categories.  To this end, we introduce a weak factorisation system in $\DTT_\stuff$, of \emph{term-extensions} and \emph{term-contractible maps}, and a type-theoretic principle $\Jbar$, giving (in terms of these maps) the required analogue in $\DTT_\stuff$ of the identity-type eliminator $\Jterm$.
\end{para}

\begin{para}
This leads us to the main restrictions on our eventual result.  As an external form of $\Jterm$, and particularly as one concerning propositional identity of open terms, it comes as no surprise that $\Jbar$ is intimately related to the various \emph{functional extensionality} rules that can be imposed in ITT.

(Despite the names, these are not to be confused with the extra rules of \emph{extensional type theory} discussed earlier, which trivialise identity types.  These rules just force the $\Id$-types over function types to be \emph{extensional} in the sense that two functions are propositionally equal if they provably give propositionally equal outputs.)

In particular, we show that $\Jbar$ (and hence our construction of a contractible operad) holds in the category of theories with suitably extensional $\Pi$-types, but does \emph{not} hold in the category of theories with $\Pi$-types but without extensionality rules.  

We conjecture that in fact $\Jbar$ holds in $\DTT_\Id$, and hence that $\Id$-types alone are sufficient to construct a contractible operad acting on $\cl_\omega$; but we are unable to prove this.
\end{para}

Our main eventual result is thus: \todo{``star'' these theorems, make them unnumbered!  and fix the alignment, here and in main body!}
\begin{mainthmclass} $\ $
\begin{enumerate}
\item There is a functor $\cl_\omega \colon \DTT_{\PiIdelim} \to \wkwCat$, as outlined above, giving the ``classifying weak $\omega$-category'' of any theory with at least $\Id$-types, $\Pi$-types, and the $\Piext$, $\Piextapp$ rules.
\item If $\Jbar$ holds for $\Id$, then we moreover have $\cl_\omega \colon \DTT_{\Id} \to \wkwCat$, giving the classifying weak $\omega$-category for any theory with at least $\Id$-types.
\end{enumerate}
\end{mainthmclass}

Additionally, we give a variant construction $\clpi_\omega$, with a slightly different underlying set (using closed terms of $\Pi$-types, rather than open terms), which works for theories with just $\Id$-types, $\Pi$-types, and the $\eta$ rule for $\Pi$-types, giving an alternative candidate for the classifying $\omega$-categories of these theories in lieu of a proof of $\Jbar$ for $\DTT_\Id$.

this is an xypic test and should be removed: 
$$\xy 0;/r.22pc/:
(0,15)*{};
(0,-15)*{};
(0,8)*{}="A";
(0,-8)*{}="B";
{\ar@{=>}@/_.75pc/ "A"+(-4,1) ; "B"+(-3,0)};
(-10,0)*{\alpha};
{\ar@{=}@/_.75pc/ "A"+(-4,1) ; "B"+(-4,1)};
{\ar@{=>}@/^.75pc/ "A"+(4,1) ; "B"+(3,0)};
(10,0)*{\beta};
{\ar@{=}@/^.75pc/ "A"+(4,1) ; "B"+(4,1)};
{\ar@3{->} (-6,0)*{} ; (6,0)*+{}};
(0,4)*{\vartheta};
(-15,0)*+{\bullet}="1";
(-15,4)*{\scriptstyle a};
(15,0)*+{\bullet}="2";
(15,4)*{\scriptstyle b};
{\ar@/^2.75pc/^{f} "1";"2"};
{\ar@/_2.75pc/_{g} "1";"2"};
\endxy$$