% Peter LeFanu Lumsdaine, June 2010
% macros for my thesis

% Contents:
%
% - Binary relations
% - Category names
% - Single letters


%%%%
% Theorem-type environments
%%%%

%% following Cisinski's style, which I found excellent, the theorem-like environments are set up to number _all_ paragraphs [in the conceptual rather than typographic sense] consecutively.  the major advantage of this is making any paragraph referenceable, and hence making the (always rather arbitrary) decision of what to pick out as theorems, definitions, etc. much less consequential and more flexible.

\theoremstyle{plain} 
\newtheorem{thm}{Theorem}[section]
\newtheorem{proposition}[thm]{Proposition}
\newtheorem{lemma}[thm]{Lemma}
\newtheorem{cor}[thm]{Corollary}
\newtheorem{sch}[thm]{Scholium}

\theoremstyle{definition}
\makeatletter
\newtheoremstyle{mydefinition}{}{}{}{}{\bfseries}{.}{5\p@ plus\p@ minus\p@}{}
\makeatother
\theoremstyle{mydefinition}
\newtheorem{definition}[thm]{Definition}
\newtheorem{para}[thm]{}
\newtheorem{exercise}[thm]{Exercise}

%\theoremstyle{remark}
\newtheorem{remark}[thm]{Remark}
\newtheorem{notation}[thm]{Notations}
\newtheorem{example}[thm]{Example}
\newtheorem{examples}[thm]{Examples}

\newtheorem{mydefinition}[thm]{Definition}

% \setcounter{tocdepth}{3}
\setcounter{secnumdepth}{2}


%%%%
% Binary relations, operators
%%%%

\newcommand{\cotensor}{\pitchfork}
\renewcommand{\equiv}{\simeq}
\newcommand{\Iff}{\Leftrightarrow}
\newcommand{\Imp}{\Rightarrow}
\newcommand{\into}{\hookrightarrow}
\newcommand{\iso}{\cong}
\newcommand{\tensor}{\otimes}
\newcommand{\To}{\Rightarrow}
\newcommand{\types}{\vdash}

%%%% 
% Single styled characters (or almost single) and character-like symbols
%%%%

\newcommand{\Two}{\mathbf{2}}
\newcommand{\A}{A_\bullet}
% \newcommand{\uA}[1][]{\underline{A}_{#1}}
% \newcommand{\B}{B_\bullet}
% \newcommand{\ML}{\mathit{ML_I}}
% \newcommand{\MLfrag}{\mathit{ML}^\Id}
\newcommand{\C}{\mathcal{C}}
\newcommand{\CC}{\mathbb{C}}
% \newcommand{\bigC}{\mathcal{C}}
% \newcommand{\bC}{\mathbf{C}}
% \newcommand{\Chat}{\widehat{\mathbb{C}}}
% \newcommand{\D}{\mathbb{D}}
% \newcommand{\bigD}{\mathcal{D}}
% \newcommand{\bD}{\mathbf{D}}
\newcommand{\diag}{\delta}
% \renewcommand{\d}{\partial}
\newcommand{\E}{\mathcal{E}}
\newcommand{\f}{\vec f}
\newcommand{\F}{\mathcal{F}}
\newcommand{\FF}{\mathbb{F}}
\newcommand{\g}{\vec g}
\newcommand{\G}{\mathbb{G}}
\newcommand{\I}{\mathcal{I}}   % generating cofibrations.  mathscr is prettier,
\newcommand{\J}{\mathcal{J}}   % but I find its I, J confusing.
\newcommand{\NN}{\mathbb{N}}   % Natural numbers
\newcommand{\N}{\mathcal{N}}   % Nerve
% \renewcommand{\P}{P_{\MLfrag}}
\newcommand{\PML}{P_{\MLId}}
% \newcommand{\Pfull}{P_{\ML}}
\newcommand{\PARA}{\textparagraph}
\newcommand{\pow}{\mathcal{P}}
% \newcommand{\p}{\vec p}
\newcommand{\SEC}{\textsection}
% \renewcommand{\S}{\mathcal{S}}    % Another generic type theory
\newcommand{\T}{\mathsf{T}}      % A generic type theory
\newcommand{\TT}{\mathbb{T}}    % A generic type theory, seen as a categorical structure
\renewcommand{\u}{\vec u}
\newcommand{\V}{\mathcal{V}}
\renewcommand{\v}{\vec v}
\newcommand{\W}{\mathcal{W}}
\newcommand{\WW}{\mathbb{W}}
\newcommand{\w}{\vec w}
\newcommand{\X}{X_\bullet}
\newcommand{\x}{\vec x}
% \newcommand{\uX}[1][]{\underline{X}_{#1}}
\newcommand{\Y}{Y_\bullet}
\newcommand{\y}{\vec y}
\newcommand{\yon}{\mathbf{y}}
\newcommand{\z}{\vec z}

%%%%
% Styled words: general
%%%%

\newcommand{\Alg}[1]{#1\mbox{-}\mathbf{Alg}}
\newcommand{\IntAlg}[2]{\mathbf{Alg}_{#2}(#1)}
\newcommand{\AMS}{AMS}
\newcommand{\AWFS}{AWFS}
\newcommand{\Cat}{\mathbf{Cat}}
% \newcommand{\cat}[1][-]{\mathbf{Cat}(#1)}
\newcommand{\enrCat}[1][\V]{#1\mbox{-}\mathbf{Cat}}
\newcommand{\cl}{\mathbf{cl}}
\newcommand{\Coll}{\mathbf{Coll}}
\newcommand{\CwA}{\mathbf{CwA}}
\newcommand{\CwAId}{\mathbf{CwA}^{\Id}}
\newcommand{\cod}{\mathrm{cod}}
\newcommand{\dom}{\mathrm{dom}}
\newcommand{\End}{\mathrm{End}}
% \newcommand{\ev}{\mathbf{ev}}
\newcommand{\FibSpans}{\mathbf{FibSpans}}
\newcommand{\FSCC}{\mathbf{FSCC}}
\newcommand{\fscc}{\textsc{fscc}}
\newcommand{\fsccs}{\textsc{fscc}'s}
\newcommand{\FSCS}{\mathbf{FSCS}}
\newcommand{\fscs}{\textsc{fscs}}
\newcommand{\fscss}{\textsc{fscs}'s}
\newcommand{\globe}[1]{\mathfrak{g}_{#1}}
\newcommand{\globes}{\mathfrak{g}_\bullet}
% \newcommand{\longGSets}{[\mathbb{G}^\op,\mathbf{Sets}]}
\newcommand{\GSets}{\widehat{\mathbb{G}}}
% \renewcommand{\lim}{\varprojlim}
\newcommand{\Lan}{\mathrm{Lan}}
\newcommand{\lax}{\mathrm{lax}}
\newcommand{\MonGlobCat}{\mathbf{MonGlobCat}}
\newcommand{\MLId}{\mathsf{ML}^{\Id}}
\newcommand{\ob}{\operatorname{ob}}
\newcommand{\op}{\mathrm{op}}
% \newcommand{\Operads}{\mathbf{Operads}}
% \newcommand{\pd}{\mathbf{pd}}
\newcommand{\QCat}{\mathbf{QCat}}
\newcommand{\qcat}{\mathit{qcat}}
\newcommand{\Sets}{\mathbf{Sets}}
\newcommand{\Spans}[1][]{\mathbf{Spans}_{#1}}
\newcommand{\strat}{\textrm{strat}}
\renewcommand{\th}{\mathbf{th}}
\newcommand{\Th}{\mathbf{Th}}
\newcommand{\ThId}{\mathbf{Th}^{\Id}}
\newcommand{\ThIdPi}{\mathbf{Th}^{\Id,\Pi}}
\newcommand{\Top}{\mathbf{Top}}
\newcommand{\strMonGlobCat}{\mathbf{MonGlobCat}}
\newcommand{\strwCat}{\mathbf{str}\mbox{-}\omega\mbox{-}\mathbf{Cat}}
\newcommand{\strnCat}[1][n]{\mathbf{str}\mbox{-}#1\mbox{-}\mathbf{Cat}}
\newcommand{\SynPres}{\mathbf{SynPres}}
\newcommand{\SynThy}{\mathbf{SynThy}}
\newcommand{\wkwCat}{\mathbf{wk}\mbox{-}\omega\mbox{-}\mathbf{Cat}}
\newcommand{\wknCat}[1][n]{\mathbf{wk}\mbox{-}#1\mbox{-}\mathbf{Cat}}

% \newcommand{\wkwCat}{\mathbf{wk}\mbox{-}\omega\mbox{-}\mathbf{Cat}}

%%%%
% Styled words: type theory syntax
%%%%

\newcommand{\Bool}{\mathsf{Bool}}
\newcommand{\comp}{\textsc{comp}}
\newcommand{\CONG}{\textsc{cong}}
% \newcommand{\Contr}{\mathsf{Contr}}
\newcommand{\cons}{\mathsf{cons}}
\newcommand{\cxt}{\mathsf{cxt}}
\newcommand{\elim}{\textsc{elim}}
% \newcommand{\Exch}{\mathsf{Exch}}
\newcommand{\form}{\textsc{form}}
\newcommand{\Id}{\mathrm{Id}}
% \newcommand{\varidelim}[5]{#4\mathsf{ for }#3\mathsf{ in }#1.#2\mathsf{ via }#5}
% \newcommand{\idelim}[5]{J_{#1.#2}(#3,#4,#5)}
\newcommand{\intro}{\textsc{intro}}
\newcommand{\refl}{\mathsf{refl}}
\newcommand{\subst}{\mathsf{subst}}
% \newcommand{\src}{\mathsf{src}}
% \newcommand{\scterm}{\textsc{term}}
\newcommand{\sym}{\mathsf{sym}}
% \newcommand{\tgt}{\mathsf{tgt}}
\newcommand{\term}{\mathsf{term}}
\newcommand{\trans}{\mathsf{trans}}
\newcommand{\type}{\mathsf{type}}
% \newcommand{\sctype}{\textsc{type}}
% \newcommand{\Weak}{\mathsf{wkg}}
\newcommand{\var}{\mathsf{var}}

%%%%
% Other operators
%%%%

\newcommand{\Clw}{\mathbb{Cl}_\omega}
\newcommand{\ClwQCat}{\mathbb{Cl}^\qcat_\omega}

%%%%
% Other symbols
%%%%

% \newcommand{\irule}[3]{\inferrule*[#1]{#2}{\quad #3 \quad}}  I can't seem to get this to work, not sure why, so just putting in extra spacing by hand...

% \newcommand{\lscott}{[\![}
% \newcommand{\rscott}{]\!]}


%%%
%%% Diagram annotations, work with diagxy
%%%


\newdir{|>}{!/4.7pt/\dir{|}
        *:(1,-.2)\dir^{>}
        *:(1,+.2)\dir_{>}}

\newbox\pbbox
\setbox\pbbox=\hbox{\xy \POS(75,0)\ar@{-} (0,0) \ar@{-} (75,75)\endxy}
\def\pb{\copy\pbbox}
\newbox\urpbbox
\setbox\urpbbox=\hbox{\xy \POS(0,0)\ar@{-} (75,0) \ar@{-} (0,75)\endxy}
\def\urpb{\copy\urpbbox}
\newbox\pobox
\setbox\pobox=\hbox{\xy \POS(0,75)\ar@{-} (0,0) \ar@{-} (75,75) \endxy}
\def\po{\copy\pobox}

% \newbox\tiltvdashbox
% \setbox\tiltvdashbox{\xy \POS( 

%% typical usage:
%
% $$\bfig \square[A`B`C`D;```]
% \place(100,400)[\pb]
% \place(400,100)[\po]
% \efig$$

