\documentclass{amsbook}

\usepackage{ifpdf}
\usepackage{mathpartir}

% \usepackage{makeindex}

%% following Cisinski's style, which I found excellent, the theorem-like environments are set up to number _all_ paragraphs [in the conceptual rather than typographic sense] consecutively.  the major advantage of this is making any paragraph referenceable, and hence making the (always rather arbitrary) decision of what to pick out as theorems, definitions, etc. much less consequential and more flexible.

\theoremstyle{plain} 
\newtheorem{thm}{Theorem}[section]
\newtheorem{prop}[thm]{Proposition}
\newtheorem{lemma}[thm]{Lemma}
\newtheorem{cor}[thm]{Corollary}

\theoremstyle{definition}
\newtheorem{definition}[thm]{Definition}
\newtheorem{para}[thm]{}

%\theoremstyle{remark}
\newtheorem{rem}[thm]{Remark}
\newtheorem{notation}[thm]{Notations}
\newtheorem{example}[thm]{Example}
\newtheorem{examples}[thm]{Examples}
\newtheorem{sch}[thm]{Scholium}

% Peter LeFanu Lumsdaine, June 2010
% macros for my thesis

% Contents:
%
% - Binary relations
% - Category names
% - Single letters


%%%%
% Theorem-type environments
%%%%

%% following Cisinski's style, which I found excellent, the theorem-like environments are set up to number _all_ paragraphs [in the conceptual rather than typographic sense] consecutively.  the major advantage of this is making any paragraph referenceable, and hence making the (always rather arbitrary) decision of what to pick out as theorems, definitions, etc. much less consequential and more flexible.

\theoremstyle{plain} 
\newtheorem{thm}{Theorem}[section]
\newtheorem{proposition}[thm]{Proposition}
\newtheorem{lemma}[thm]{Lemma}
\newtheorem{cor}[thm]{Corollary}
\newtheorem{sch}[thm]{Scholium}

\theoremstyle{definition}
\makeatletter
\newtheoremstyle{mydefinition}{}{}{}{}{\bfseries}{.}{5\p@ plus\p@ minus\p@}{}
\makeatother
\theoremstyle{mydefinition}
\newtheorem{definition}[thm]{Definition}
\newtheorem{para}[thm]{}
\newtheorem{exercise}[thm]{Exercise}

%\theoremstyle{remark}
\newtheorem{remark}[thm]{Remark}
\newtheorem{notation}[thm]{Notations}
\newtheorem{example}[thm]{Example}
\newtheorem{examples}[thm]{Examples}

\newtheorem{mydefinition}[thm]{Definition}

% \setcounter{tocdepth}{3}
\setcounter{secnumdepth}{2}


%%%%
% Binary relations, operators
%%%%

\newcommand{\cotensor}{\pitchfork}
\renewcommand{\equiv}{\simeq}
\newcommand{\Iff}{\Leftrightarrow}
\newcommand{\Imp}{\Rightarrow}
\newcommand{\into}{\hookrightarrow}
\newcommand{\iso}{\cong}
\newcommand{\tensor}{\otimes}
\newcommand{\To}{\Rightarrow}
\newcommand{\types}{\vdash}

%%%% 
% Single styled characters (or almost single) and character-like symbols
%%%%

\newcommand{\Two}{\mathbf{2}}
\newcommand{\A}{A_\bullet}
% \newcommand{\uA}[1][]{\underline{A}_{#1}}
% \newcommand{\B}{B_\bullet}
% \newcommand{\ML}{\mathit{ML_I}}
% \newcommand{\MLfrag}{\mathit{ML}^\Id}
\newcommand{\C}{\mathcal{C}}
\newcommand{\CC}{\mathbb{C}}
% \newcommand{\bigC}{\mathcal{C}}
% \newcommand{\bC}{\mathbf{C}}
% \newcommand{\Chat}{\widehat{\mathbb{C}}}
% \newcommand{\D}{\mathbb{D}}
% \newcommand{\bigD}{\mathcal{D}}
% \newcommand{\bD}{\mathbf{D}}
\newcommand{\diag}{\delta}
% \renewcommand{\d}{\partial}
\newcommand{\E}{\mathcal{E}}
\newcommand{\f}{\vec f}
\newcommand{\F}{\mathcal{F}}
\newcommand{\FF}{\mathbb{F}}
\newcommand{\g}{\vec g}
\newcommand{\G}{\mathbb{G}}
\newcommand{\I}{\mathcal{I}}   % generating cofibrations.  mathscr is prettier,
\newcommand{\J}{\mathcal{J}}   % but I find its I, J confusing.
\newcommand{\NN}{\mathbb{N}}   % Natural numbers
\newcommand{\N}{\mathcal{N}}   % Nerve
% \renewcommand{\P}{P_{\MLfrag}}
\newcommand{\PML}{P_{\MLId}}
% \newcommand{\Pfull}{P_{\ML}}
\newcommand{\PARA}{\textparagraph}
\newcommand{\pow}{\mathcal{P}}
% \newcommand{\p}{\vec p}
\newcommand{\SEC}{\textsection}
% \renewcommand{\S}{\mathcal{S}}    % Another generic type theory
\newcommand{\T}{\mathsf{T}}      % A generic type theory
\newcommand{\TT}{\mathbb{T}}    % A generic type theory, seen as a categorical structure
\renewcommand{\u}{\vec u}
\newcommand{\V}{\mathcal{V}}
\renewcommand{\v}{\vec v}
\newcommand{\W}{\mathcal{W}}
\newcommand{\WW}{\mathbb{W}}
\newcommand{\w}{\vec w}
\newcommand{\X}{X_\bullet}
\newcommand{\x}{\vec x}
% \newcommand{\uX}[1][]{\underline{X}_{#1}}
\newcommand{\Y}{Y_\bullet}
\newcommand{\y}{\vec y}
\newcommand{\yon}{\mathbf{y}}
\newcommand{\z}{\vec z}

%%%%
% Styled words: general
%%%%

\newcommand{\Alg}[1]{#1\mbox{-}\mathbf{Alg}}
\newcommand{\IntAlg}[2]{\mathbf{Alg}_{#2}(#1)}
\newcommand{\AMS}{AMS}
\newcommand{\AWFS}{AWFS}
\newcommand{\Cat}{\mathbf{Cat}}
% \newcommand{\cat}[1][-]{\mathbf{Cat}(#1)}
\newcommand{\enrCat}[1][\V]{#1\mbox{-}\mathbf{Cat}}
\newcommand{\cl}{\mathbf{cl}}
\newcommand{\Coll}{\mathbf{Coll}}
\newcommand{\CwA}{\mathbf{CwA}}
\newcommand{\CwAId}{\mathbf{CwA}^{\Id}}
\newcommand{\cod}{\mathrm{cod}}
\newcommand{\dom}{\mathrm{dom}}
\newcommand{\End}{\mathrm{End}}
% \newcommand{\ev}{\mathbf{ev}}
\newcommand{\FibSpans}{\mathbf{FibSpans}}
\newcommand{\FSCC}{\mathbf{FSCC}}
\newcommand{\fscc}{\textsc{fscc}}
\newcommand{\fsccs}{\textsc{fscc}'s}
\newcommand{\FSCS}{\mathbf{FSCS}}
\newcommand{\fscs}{\textsc{fscs}}
\newcommand{\fscss}{\textsc{fscs}'s}
\newcommand{\globe}[1]{\mathfrak{g}_{#1}}
\newcommand{\globes}{\mathfrak{g}_\bullet}
% \newcommand{\longGSets}{[\mathbb{G}^\op,\mathbf{Sets}]}
\newcommand{\GSets}{\widehat{\mathbb{G}}}
% \renewcommand{\lim}{\varprojlim}
\newcommand{\Lan}{\mathrm{Lan}}
\newcommand{\lax}{\mathrm{lax}}
\newcommand{\MonGlobCat}{\mathbf{MonGlobCat}}
\newcommand{\MLId}{\mathsf{ML}^{\Id}}
\newcommand{\ob}{\operatorname{ob}}
\newcommand{\op}{\mathrm{op}}
% \newcommand{\Operads}{\mathbf{Operads}}
% \newcommand{\pd}{\mathbf{pd}}
\newcommand{\QCat}{\mathbf{QCat}}
\newcommand{\qcat}{\mathit{qcat}}
\newcommand{\Sets}{\mathbf{Sets}}
\newcommand{\Spans}[1][]{\mathbf{Spans}_{#1}}
\newcommand{\strat}{\textrm{strat}}
\renewcommand{\th}{\mathbf{th}}
\newcommand{\Th}{\mathbf{Th}}
\newcommand{\ThId}{\mathbf{Th}^{\Id}}
\newcommand{\ThIdPi}{\mathbf{Th}^{\Id,\Pi}}
\newcommand{\Top}{\mathbf{Top}}
\newcommand{\strMonGlobCat}{\mathbf{MonGlobCat}}
\newcommand{\strwCat}{\mathbf{str}\mbox{-}\omega\mbox{-}\mathbf{Cat}}
\newcommand{\strnCat}[1][n]{\mathbf{str}\mbox{-}#1\mbox{-}\mathbf{Cat}}
\newcommand{\SynPres}{\mathbf{SynPres}}
\newcommand{\SynThy}{\mathbf{SynThy}}
\newcommand{\wkwCat}{\mathbf{wk}\mbox{-}\omega\mbox{-}\mathbf{Cat}}
\newcommand{\wknCat}[1][n]{\mathbf{wk}\mbox{-}#1\mbox{-}\mathbf{Cat}}

% \newcommand{\wkwCat}{\mathbf{wk}\mbox{-}\omega\mbox{-}\mathbf{Cat}}

%%%%
% Styled words: type theory syntax
%%%%

\newcommand{\Bool}{\mathsf{Bool}}
\newcommand{\comp}{\textsc{comp}}
\newcommand{\CONG}{\textsc{cong}}
% \newcommand{\Contr}{\mathsf{Contr}}
\newcommand{\cons}{\mathsf{cons}}
\newcommand{\cxt}{\mathsf{cxt}}
\newcommand{\elim}{\textsc{elim}}
% \newcommand{\Exch}{\mathsf{Exch}}
\newcommand{\form}{\textsc{form}}
\newcommand{\Id}{\mathrm{Id}}
% \newcommand{\varidelim}[5]{#4\mathsf{ for }#3\mathsf{ in }#1.#2\mathsf{ via }#5}
% \newcommand{\idelim}[5]{J_{#1.#2}(#3,#4,#5)}
\newcommand{\intro}{\textsc{intro}}
\newcommand{\refl}{\mathsf{refl}}
\newcommand{\subst}{\mathsf{subst}}
% \newcommand{\src}{\mathsf{src}}
% \newcommand{\scterm}{\textsc{term}}
\newcommand{\sym}{\mathsf{sym}}
% \newcommand{\tgt}{\mathsf{tgt}}
\newcommand{\term}{\mathsf{term}}
\newcommand{\trans}{\mathsf{trans}}
\newcommand{\type}{\mathsf{type}}
% \newcommand{\sctype}{\textsc{type}}
% \newcommand{\Weak}{\mathsf{wkg}}
\newcommand{\var}{\mathsf{var}}

%%%%
% Other operators
%%%%

\newcommand{\Clw}{\mathbb{Cl}_\omega}
\newcommand{\ClwQCat}{\mathbb{Cl}^\qcat_\omega}

%%%%
% Other symbols
%%%%

% \newcommand{\irule}[3]{\inferrule*[#1]{#2}{\quad #3 \quad}}  I can't seem to get this to work, not sure why, so just putting in extra spacing by hand...

% \newcommand{\lscott}{[\![}
% \newcommand{\rscott}{]\!]}


%%%
%%% Diagram annotations, work with diagxy
%%%


\newdir{|>}{!/4.7pt/\dir{|}
        *:(1,-.2)\dir^{>}
        *:(1,+.2)\dir_{>}}

\newbox\pbbox
\setbox\pbbox=\hbox{\xy \POS(75,0)\ar@{-} (0,0) \ar@{-} (75,75)\endxy}
\def\pb{\copy\pbbox}
\newbox\urpbbox
\setbox\urpbbox=\hbox{\xy \POS(0,0)\ar@{-} (75,0) \ar@{-} (0,75)\endxy}
\def\urpb{\copy\urpbbox}
\newbox\pobox
\setbox\pobox=\hbox{\xy \POS(0,75)\ar@{-} (0,0) \ar@{-} (75,75) \endxy}
\def\po{\copy\pobox}

% \newbox\tiltvdashbox
% \setbox\tiltvdashbox{\xy \POS( 

%% typical usage:
%
% $$\bfig \square[A`B`C`D;```]
% \place(100,400)[\pb]
% \place(400,100)[\po]
% \efig$$



% \makeindex

%%
%% PDFJUNK
%% Can add /CreationDate, /Creator, /Subject, /Keywords
%%
\ifpdf
\pdfinfo{
  /Author (Peter LeFanu Lumsdaine) 
  /Title (TODO: put thesis title here when decided!)
}
\fi
%%
%% BEGIN DOCUMENT:
%\onehalfspacing
\begin{document}


\frontmatter

%% TITLE INFORMATION

\title[Higher-categorical Structures and Type Theories]{Higher-categorical Strucures and Type Theories: DRAFT!}

\author[P. LeF. Lumsdaine]{{\large\bf Peter LeFanu Lumsdaine}\\~\\\normalsize
  \bf July 2010}

\maketitle
\newpage
\thispagestyle{empty}
\begin{center}
  {\Large\textbf{Carnegie Mellon University}}\\
  \vspace{1cm}
  {\large\textbf{TODO: Title Here!}}\\
  \vspace{.5cm}
  Peter LeFanu Lumsdaine\\
  \vspace{.5cm}
  July 2010\\
  \vspace{3cm}
  \textbf{Committee}\\~\\
  Steven M. Awodey (advisor)\\
  someone \\
  someone else \\
  some more \\

\end{center}
 
\setcounter{page}{3}


% \clearpage
% \thispagestyle{empty}
% \vspace*{13.5pc}
% \begin{center}
%   Dedication text (use \\[2pt] for line break if necessary)
% \end{center}
% \cleardoublepage 


\tableofcontents


% Unnumbered chapters



\section{Background}

\begin{para}
A fundamental construction in all flavours of type theory is the \emph{classifying category} $\cl(\T)$ of a theory $\T$.  Objects of $\cl(\T)$ are types (or contexts) of $\T$; morphisms of $\T$ are open terms of one type  with free variables from another, up to syntactic equivalence of some sort.

The uses of this construction are manifold.  First of all, it provides a tool for structural analysis of theories, allowing type-theoretic constructors and rules to be understood as presentations of categorical structure (products, sums, exponential objects, \ldots) familiar from elsewhere in mathematics.

Next, this allows us to model the type theory in any suitably structured category, via a right adjoint $\Lang$ to the functor $\cl$, giving the \emph{internal language} of a structured category.

Putting these together, it allows us to represent type theories as \emph{equivalent} to appropriately structured categories, giving us an alternate presentation with different strengths to the traditional syntax.
\end{para}

\begin{para}
For some logical systems---simple type theory, for instance, and some extensional versions of Martin-Löf type theory---this correspondence is very well understood and leaves little to be desired.  Unfortunately, the situation with the intensional variants of Martin-Löf type theory---ITT, henceforth---is not yet so satisfactory.

Syntactic categories for theories in ITT still work essentially as described above: they are a powerful technical tool for providing models, and for representing theories.  The subtlety is that the structures corresponding to logical constructors in ITT are no longer so categorically familiar as before: they are not quite the products, exponentials, adjoints and so on that one might expect.  The problem\footnote{although, of course, ``it's not a bug, it's a feature''!} is the interaction of its two different kinds of equality.

Firstly, there is \emph{definitional} (aka \emph{intensional}) equality:
\[\Gamma\ \types\ a = a' : A\]
a decidable, quite fine syntactic relation---this covers essentially $\alpha$-equivalence, $\beta$-equivalence, and sometimes $\eta$-, but no more.  Its strictness reflects the extremely well-controlled computational behaviour of the system.  This is the equality used in the classifying category.  Crucially, however, it is a separate judgement of the system, not represented by a type, so cannot be reasoned about by e.g.\ induction within the system.  One cannot prove in ITT, for instance, that 
\[x : \N\ \types\ x+0 = x\ : \N. \]

Secondly, since one does need some kind of equality which can be reasoned about internally, there is \emph{propositional} (sometimes aka \emph{extensional}) equality, which is represented by a dependent type, the \emph{identity type} over each type:
\[ x, y : A\ \types\ \Id_A(x,y)\ \type\]
An inhabitant of this type is seen as a \emph{proof of equality} between $x$ and $y$.  This is a coarser equality than the definitional; it is axiomatised simply by $\intro$/$\elim$ rules of the usual form for an inductive type, but they turn out to yield remarkably subtle consequences.
\end{para}

\begin{para}
In the extensional theory, the two equalities coincide by fiat (the \emph{reflection} rule).  This makes reasoning about equality within the theory much simpler, but at the cost of desirable computational properties: for instance, type-checking becomes undecidable, since it depends (because of the dependency of types on terms) on definitional equality, which is now reduced to propositional, i.e.\ to inhabitation of types, which is undecidable in any sufficiently rich system.  Relatedly, the computational content of terms is lost: terms of different computational behaviour may become identified in this stronger definitional equality.  In these and other proof-theoretic respects, the intensional theory is significantly preferable.

As mentioned above, however, categorical analysis of constructors in the intensional theory is much less clear.  For instance, a unit type\footnote{as usually axiomatised in ITT, i.e.\ as an inductive type with one nullary constructor.} does \emph{not} give a terminal object of the syntactic category.  It may have multiple morphisms from other objects, at least up to definitional equality.  Internally, these terms will be equal, but only up to \emph{propositional} equality---so we need to incorporate this somehow into the structure of the syntactic category.  

(Of course, propositional equality is already present in the classifying category, but as terms of some other type; but we want a description which makes more apparent its rôle in the categorical analysis of the constructors.)

A naïve approach might be to simply quotient by propositional equality; but this destroys too much of the structure of the theory. 
\end{para}

\begin{para}Instead, starting with the Hofmann-Streicher groupoid model \cite{hofmann-streicher}, higher categories and related homotopy-theoretic structures have emerged as a natural solution to this problem: we add the propositional equality, but as extra \emph{structure}, not just as a relation to quotient by.

In the globular approach to higher categories, a higher category has objects (``0-cells'') and arrows (``1-cells'') between objects, as in a category, but also 2-cells between 1-cells, and so on, with various composition operations and laws depending on the kind of category in question (strict or weak, $n$- or $\omega$-, $\ldots$).

Now we see propositional equalities between terms as \emph{2-cells} between morphisms; and also, since propositional equalities themselves are terms of a possibly non-trivial type, we treat further propositional equalities between them as higher cells, and so on \emph{ad infinitum}.  Binary composition of these higher cells corresponds now to the transitivity of equality; higher identity cells, to reflexivity; symmetry of equality, to having some sort of inverses for cells.

There is a compelling analogy here with homotopy theory.  The higher fundamental groupoid of a space consists of points of the space, paths between points, homotopies between paths, \ldots\ Homotopies are composable, and have units, and inverses.

A type, therefore, can be seen as something like a space: a category with cells of arbitrarily high dimension, in which all cells are invertible, i.e.\ an \emph{$\omega$-groupoid}.  An entire theory now looks something like a category of spaces: an $\omega$ category in which all cells of dimension $> 1$ are invertible, i.e.\ an \emph{($\omega,1)$-category} (also known as $(\infty,1)$).

However, in the fundamental groupoid example, it is a familiar fact that associativity of composition, and other laws, hold only up to homotopy, i.e.\ only up to a cell of higher dimension.  We find the same situation in type theory: composition is generally associative, unital etc.\ only up to propositional equality.  (Of course, in the presence of extra axioms, or in certain models, it can sometimes be strict.)  For the fully intensional theory we thus have to work not with strict but with \emph{weak} higher categories.

Various different definitions of weak $\omega$-categories exist; we use the \emph{globular operadic} definition of Batanin, as modified by Leinster. 
\end{para}

\begin{para}
Since \cite{hofmann-streicher}, ideas along these lines have been explored by various authors in various directions.  For ``two-dimensional ITT'' and certain weak 2-categories, the full correspondence is worked out in detail in \cite{garner:2-d-models}.  In the semantic direction, models of ITT in various higher categorical/homotopical settings are investigated in \cite{awodey-warren}, \cite{warren:thesis}, \cite{garner-van-den-berg:top-and-simp-models}, and \cite{voevodsky:univalent-notes}.  Conversely, the structures formed from the syntax of theories have been previously investigated in \cite{gambino-garner}, \cite{lumsdaine:weak-w-cats-from-itt-lmcs}, \cite{garner-van-den-berg}, \cite{awodey-hofstra-warren}.  This dissertation continues the latter line of investigation: the main goal is to construct the \emph{classifying weak $\omega$-category} of a theory.
\end{para}


\section{Overview of results}

\subsection*{The fundamental weak \pdfomega-category of a type (Ch.~\ref{ch:fundamental})}

\begin{para} \label{para:fundamental-sketch} As a warm-up for the main result, in this chapter we construct the ``fundamental weak $\omega$-category''\footnote{named by analogy with the higher fundamental groupoids of a space; in our case the fundamental weak $\omega$-category is again groupoidal, as shown in \cite{garner-van-den-berg}, but we do not discuss this in the present work.} of a single type within a theory.  

Specifically, we show that for any type $A$ and context $\Delta$ in a type theory $\T$ with at least the $\Id$-type rules, there is a weak $\omega$-category in which 0-cells are terms of type $A$ in context $\Delta$, 1-cells are terms of type $\Id_A$ between these, 2-cells are terms of type $\Id_{\Id_A}$ between 1-cells, and so on.

(Note that for this construction the dimensions of cells are always one lower than in the classifying category sketched above.  This comes from the general rule that if $X$, $A$ are objects of an $n$-category $\C$, then $\C(X,A)$ forms an $(n-1)$-category, whose 0-cells are 1-cells of $\C$, and so on.)
\end{para}

\begin{para}Before sketching the construction, we roughly recall the globular operadic definition of weak higher categories.  (The full background required on this material is set out in Ch.~\ref{ch:cat-background}.)

An $\omega$-category $\C$ has a set\footnote{for the present work, we consider \emph{small} higher categories only.} $C_n$ of ``$n$-cells'' for each $n > 0$.  The $0$- and $1$-cells correspond to the objects and arrows of an ordinary category: each arrow $f$ has source and target objects $a = s(f)$, $b = t(f)$.  Similarly, the source and target of a 2-cell $\alpha$ are a parallel pair of 1-cells $f,g: a \two b$, and generally the source and target of an $(n+1)$-cell are a parallel pair of $n$-cells.

Cells of each dimension can be composed along a common boundary in any lower dimension, and in a \emph{strict} $\omega$-category, the composition satisfies various associativity, unit, and interchange laws, captured by the \emph{generalised associativity} law: each labelled pasting diagram has a unique composite. (See illustrations in Fig.\ \ref{figure:assoc-laws}).
\end{para}

\begin{figure}
\[
\begin{array}{c}
\begin{array}{cccc}
\ \xy
(0,0)*{\bullet};
(0,80)*{a};
\endxy \quad
&
\ \xy
(0,0)*{\bullet}="a";
(0,80)*{\scriptstyle a};
(400,0)*{\bullet}="b";
(400,80)*{\scriptstyle b};
{\ar "a";"b"};
(200,80)*{f};
\endxy \ 
&
\ \xy
(0,0)*+{\bullet}="a";
(0,80)*{\scriptstyle a};
(450,0)*+{\bullet}="b";
(450,80)*{\scriptstyle b};
{\ar@/^1pc/^{f} "a";"b"};
{\ar@/_1pc/_{g} "a";"b"};
{\ar@{=>} (210,85)*{};(210,-85)*{}};
(280,0)*{\alpha};
\endxy \ 
&
\ \xy
(0,0)*+{\bullet}="a";
(0,80)*{\scriptstyle a};
(600,0)*+{\bullet}="b";
(600,80)*{\scriptstyle b};
{\ar@/^1.75pc/^{f} "a";"b"};
{\ar@/_1.75pc/_{g} "a";"b"};
{\ar@2{->}@/_0.5pc/ (220,140);(220,-140)} ;
{\ar@2{-}@/_0.5pc/|{\alpha} (203,120);(203,-120)} ;
{\ar@2{->}@/^0.5pc/ (380,140);(380,-140)} ;
{\ar@2{-}@/^0.5pc/|{\beta} (397,120);(397,-120)} ;
{\ar@3{->} (225,-20);(375,-20)};
(300,60)*{\Theta};
\endxy \ 
\end{array} \\
\begin{array}{ccc}
\ \xy(0,0)*{\bullet}="a";
(0,80)*{\scriptstyle a};
(300,0)*{\bullet}="b";
(300,80)*{\scriptstyle b};
(600,0)*{\bullet}="c";
(600,80)*{\scriptstyle c};
{\ar^f "a";"b"};
{\ar^g "b";"c"};
\endxy \ 
&
\ \xy
(0,0)*+{\bullet}="a";
(0,80)*{\scriptstyle a};
(500,0)*+{\bullet}="b";
(500,80)*{\scriptstyle b};
{\ar@/^1.75pc/|f "a";"b"};
{\ar|{f'} "a";"b"};
{\ar@/_1.75pc/|{f''} "a";"b"};
{\ar@{=>}^{\alpha} (250,160)*{};(250,50)*{}} ;
{\ar@{=>}^{\gamma} (250,-50)*{};(250,-160)*{}} ;
(0,-250)*{\ };
\endxy \ 
&
\ \xy
(0,0)*+{\bullet}="a";
(0,80)*{\scriptstyle a};
(400,0)*+{\bullet}="b";
(400,80)*{\scriptstyle b};
(800,0)*+{\bullet}="c";
(800,80)*{\scriptstyle c};
{\ar@/^1.1pc/|f "a";"b"};
{\ar@/_1.1pc/|{f'} "a";"b"};
{\ar@/^1.1pc/|g "b";"c"};
{\ar@/_1.1pc/|{g'} "b";"c"};
{\ar@{=>}^{\alpha} (200,80)*{};(200,-80)*{}} ;
{\ar@{=>}^{\beta} (600,80)*{};(600,-80)*{}} ;
\endxy \ \\
g \cdot_0 f &
\gamma \cdot_1 \alpha &
\beta \cdot_0 \alpha
\end{array}
\\
\begin{array}{cc}
\ \xy(0,0)*{\bullet}="a";
%(0,80)*{\scriptstyle a};
(300,0)*{\bullet}="b";
%(300,80)*{\scriptstyle b};
(600,0)*{\bullet}="c";
%(600,80)*{\scriptstyle c};
(900,0)*{\bullet}="d";
%(900,80)*{\scriptstyle d};
{\ar^f "a";"b"};
{\ar^g "b";"c"};
{\ar^h "c";"d"};
\endxy \ &
\ \xy
(0,0)*+{\bullet}="a";
(400,0)*+{\bullet}="b";
{\ar@/^1.5pc/ "a";"b"};
{\ar "a";"b"};
{\ar@/_1.5pc/ "a";"b"};
{\ar@{=>}^{\alpha} (200,150)*{};(200,25)*{}} ;
{\ar@{=>}^{\gamma} (200,-25)*{};(200,-150)*{}} ;
(800,0)*+{\bullet}="c";
{\ar@/^1.5pc/ "b";"c"};
{\ar "b";"c"};
{\ar@/_1.5pc/ "b";"c"};
{\ar@{=>}^{\beta} (600,150)*{};(600,25)*{}} ;
{\ar@{=>}^{\delta} (600,-25)*{};(600,-150)*{}};
(0,250)*{\ };
(0,-220)*{\ };
\endxy \ \\
\begin{array}{c} h \cdot_0 (g \cdot_0 f) =  \\ (h \cdot_0 g) \cdot_0 f \end{array} &
\begin{array}{c}(\delta \cdot_0 \gamma) \cdot_1 (\beta \cdot_0 \alpha) = \\
(\gamma \cdot_1 \alpha) \cdot_0 (\delta \cdot_1 \beta)\end{array}
\end{array}
\end{array}
\]
\caption{Some cells, composites, and associativities in a strict $\omega$-category \label{figure:assoc-laws}} 
\end{figure}

\begin{para} \label{para:intro-examples} In a \emph{weak} $\omega$-category, we do not expect strict associativity, so may have multiple composites for a given pasting diagram, but we do demand that these composites agree up to cells of the next dimension (``up to homotopy''), and that these associativity cells satisfy certain coherence laws of their own, again up to cells of higher dimension, and so on.
This is exactly the situation we find in intensional type theory.  For instance, even in constructing a term witnessing the transitivity of identity---that is, a composition law for the pasting diagram $(\xymatrix{ \bullet \ar[r] & \bullet \ar[r] & \bullet })$, or explicitly a term $c$ such that 
\[x,y,z:X,\ p:\Id(x,y),\ q:\Id(y,z) \types c(q,p): \Id(x,z)\]
---one finds that there is no single canonical candidate: most obvious are the two equally natural terms $c_l$, $c_r$ obtained by applying ($\Id$-\elim) to $p$ or to $q$ respectively.  These are not definitionally equal, but are propositionally equal, i.e.\ equal up to a 2-cell: there is a term $e$ with
\[x,y,z:X,\ p:\Id(x,y),\ q:\Id(y,z) \types e(q,p): \Id(c_l(q,p),c_r(q,p)).\]
Similarly, for either of these operations (or any combination), we can derive a ``propositional associative law'':
\begin{multline*}
x,y,z,w:X,\ p:\Id(x,y),\ q:\Id(y,z),\ r:\Id(z,w)\ \types \\
a(r,q,p) : \Id ( c(r,c(q,p))\,,\,c(c(r,q),p) ).
\end{multline*}
\end{para} 

\begin{para}In Leinster's definition \cite{leinster:book}, a system of composition laws of this sort is wrapped up in the algebraic structure of a \emph{globular operad with contraction}, and a weak $\omega$-category is given by a globular set equipped with an \emph{action} of such an operad.  We generalise this slightly, to define an \emph{internal weak $\omega$-category} in any suitable category $\C$.

Accordingly, we would like to find an operad-with-contraction $P_\Id$ of all such type-theoretically definable composition laws, acting on terms of any type and its identity types.  Formally, we consider $\T_\Id[X]$, the theory axiomatised just by the structural and $\Id$-rules plus a single generic base type $X$.  The operad $P_\Id$ of definable composition laws therein may then be formally constructed as an endomorphism operad in (presheaves on) its syntactic category $\cl(\T_\Id [X])$; and by some analysis of $\T_\Id[X]$, we show that $P_\Id$ is contractible.
\end{para}

Since $X$ is generic, composition laws defined on it can be implemented on all other types, giving the main result of Chapter \ref{ch:fundamental}: 

\begin{mainthmfund}Let $\T$ be any type theory extending $\T_\Id$, and $A$ any type of $\T$.  Then the system of types $(A, \Id_A, \Id_{\Id_A}, \ldots)$ is equipped naturally with a $P_\Id$-action, and hence with the structure of an internal weak $\omega$-category in $\C(\T)$.
\end{mainthmfund}

From this it follows that the terms of these types in any fixed context form an (external) $P_\Id$ algebra and weak $\omega$-category.  Indeed, one can construct, using the $\Id$-rules, \emph{identity contexts} $\Id_\Gamma$ over all contexts, satisfying rules analogous to those for identity types; so for any contexts $\Delta$, $\Gamma$, there is an $\omega$-category of context morphisms from $\Delta$ into $\Gamma$ and its identity types.

The material of this chapter is essentially based on my earlier paper \cite{lumsdaine:tlca-proceedings}/\cite{lumsdaine:weak-w-cats-from-itt-lmcs}.  A very similar construction (overlapping in some parts and using different techniques in others)  was discovered independently by Benno van den Berg and Richard Garner, and appears in \cite{garner-van-den-berg}. \\

\subsection*{The classifying weak \pdfomega-category of a theory (Ch.~\ref{ch:classifying})}

\begin{para}In this chapter, the heart of the dissertation, we construct the classifying $\omega$-category $\cl_\omega(\T)$ of a theory $T$.

As intimated above, $\cl_\omega(\T)$ should in dimensions $\leq 1$ be just the usual classifying category $\cl(\T)$ of contexts and context morphisms; its 2-cells should be morphisms into identity contexts, and higher cells should be maps into higher identity contexts.  From another point of view, recall that as a category has hom-sets, a 2-category has hom-categories, and generally a $(1+n)$-category has hom-$n$-categories between its $0$-cells.  Then the hom-$\omega$-category $\cl_\omega(\T)(\Delta,\Gamma)$ is just the fundamental $\omega$-category of $\Gamma$ in context $\Delta$, as sketched in the previous section.

In fact we first construct a weak $\omega$-category $\cl_\omega^-(\T)$, as above but with 0-cells just types, not more general contexts; then, we bump this up formally to include contexts as well.

The core technique used is analogous again to the familiar construction of the higher fundamental group(oid)s of a space: \emph{representability}.  Points, paths, homotopies, \ldots\ in a space $X$ are constructed as maps into $X$ from the point, the interval, the 1-disc, \ldots\ ; correspondingly, the cells of $\cl_\omega^-(\T)$ can be seen as interpretations in $\T$ of the theories of a free type, a free open term, \ldots\ ---maps into $\T$ from the \emph{type-theoretic globes}, which we denote $\globes$.  (This is why we start with $\cl_\omega^-$ not $\cl_\omega$: there is no ``theory of a free context'', since contexts can have various lengths.)

Then, as usual with representable algebraic structures\footnote{in fact inevitably, by the Yoneda lemma.}, the weak $\omega$-category structure on $\cl_\omega(\T)$ is induced formally from a weak $\omega$-cocategory structure on the representing objects $\globes$.

To give this structure, we once again consider an endomorphism operad---in fact now the \emph{co-endomorphism} operad of the globes, whose operations\todo{to be conscientious we should really be calling them co-operations.} now represent composition laws that can be defined generically over multiple types and morphisms between them, and we show this is contractible by techniques extending those of Ch.~\ref{ch:fundamental}.
\end{para}

\begin{para}
To carry this out, we require two main additional pieces of technical machinery.  Firstly, to handle the representing objects, co-endomorphism operads, and so on, we need a well-behaved category $\DTT$ of type theories in which to work.  In fact, we consider various such categories $\DTT_\stuff$, given by fixing some set $\stuff$ of constructors and rules ($\Id$-types, $\Pi$-types, etc.), and considering all theories extending these purely algebraically.

This type-theoretic background is set out in Ch.~\ref{ch:dtt-background}.  In order to facilitate the categorical analysis of $\DTT$, we recall the representation of theories as certain \emph{categories with attributes}, and the equivalence of this with more traditional syntactic presentations.

On the other hand, for showing contractibility of the operads in these categories, we need analogues in $\DTT_\stuff$ of $\Id$-elimination in syntactic categories.  To this end, we introduce a weak factorisation system in $\DTT_\stuff$, of \emph{term-extensions} and \emph{term-contractible maps}, and a type-theoretic principle $\Jbar$, giving (in terms of these maps) the required analogue in $\DTT_\stuff$ of the identity-type eliminator $\Jterm$.
\end{para}

\begin{para}
This leads us to the main restrictions on our eventual result.  As an external form of $\Jterm$, and particularly as one concerning propositional identity of open terms, it comes as no surprise that $\Jbar$ is intimately related to the various \emph{functional extensionality} rules that can be imposed in ITT.

(Despite the names, these are not to be confused with the extra rules of \emph{extensional type theory} discussed earlier, which trivialise identity types.  These rules just force the $\Id$-types over function types to be \emph{extensional} in the sense that two functions are propositionally equal if they provably give propositionally equal outputs.)

In particular, we show that $\Jbar$ (and hence our construction of a contractible operad) holds in the category of theories with suitably extensional $\Pi$-types, but does \emph{not} hold in the category of theories with $\Pi$-types but without extensionality rules.  

We conjecture that in fact $\Jbar$ holds in $\DTT_\Id$, and hence that $\Id$-types alone are sufficient to construct a contractible operad acting on $\cl_\omega$; but we are unable to prove this.
\end{para}

Our main eventual result is thus: \todo{``star'' these theorems, make them unnumbered!  and fix the alignment, here and in main body!}
\begin{mainthmclass} $\ $
\begin{enumerate}
\item There is a functor $\cl_\omega \colon \DTT_{\PiIdelim} \to \wkwCat$, as outlined above, giving the ``classifying weak $\omega$-category'' of any theory with at least $\Id$-types, $\Pi$-types, and the $\Piext$, $\Piextapp$ rules.
\item If $\Jbar$ holds for $\Id$, then we moreover have $\cl_\omega \colon \DTT_{\Id} \to \wkwCat$, giving the classifying weak $\omega$-category for any theory with at least $\Id$-types.
\end{enumerate}
\end{mainthmclass}

Additionally, we give a variant construction $\clpi_\omega$, with a slightly different underlying set (using closed terms of $\Pi$-types, rather than open terms), which works for theories with just $\Id$-types, $\Pi$-types, and the $\eta$ rule for $\Pi$-types, giving an alternative candidate for the classifying $\omega$-categories of these theories in lieu of a proof of $\Jbar$ for $\DTT_\Id$.

this is an xypic test and should be removed: 
$$\xy 0;/r.22pc/:
(0,15)*{};
(0,-15)*{};
(0,8)*{}="A";
(0,-8)*{}="B";
{\ar@{=>}@/_.75pc/ "A"+(-4,1) ; "B"+(-3,0)};
(-10,0)*{\alpha};
{\ar@{=}@/_.75pc/ "A"+(-4,1) ; "B"+(-4,1)};
{\ar@{=>}@/^.75pc/ "A"+(4,1) ; "B"+(3,0)};
(10,0)*{\beta};
{\ar@{=}@/^.75pc/ "A"+(4,1) ; "B"+(4,1)};
{\ar@3{->} (-6,0)*{} ; (6,0)*+{}};
(0,4)*{\vartheta};
(-15,0)*+{\bullet}="1";
(-15,4)*{\scriptstyle a};
(15,0)*+{\bullet}="2";
(15,4)*{\scriptstyle b};
{\ar@/^2.75pc/^{f} "1";"2"};
{\ar@/_2.75pc/_{g} "1";"2"};
\endxy$$
%\include{ack}
%\include{decl}


% Main Matter

\mainmatter

% this file is called up by thesis.tex
% content in this file will be fed into the main document

\chapter{Universal-algebraic aspects}


% ----------------------- contents from here ------------------------

Possibly fold this chapter into the next??

\section{A category of Type Theories}

\para Give brief, semi-formal outline of the type theory, referring to Appendix

\para In fact, we will not work directly with the category of type theories, but with an equivalent category of algebraic models.
but 
The theory of such models is attractive, but suffers from rather an \emph{embaras de richesse} of frameworks: for instance \cite{jacobs:comprehension-categories}, \cite{pitts:categorical-logic}, \cite{hofmann:syntax-and-semantics} and \cite{dybjer:internal-type-theory} each [TODO: chronologicise these] define slightly different notions of categorical models (several after the unpublished \cite{cartmell:thesis}), all equivalent (in some sense) to each other and to syntactically presented dependent type theories.   (TODO: also look up ``categories with families''.)

Of course, these notions each have advantages and disadvantages: some are more elementary to present; some are more categorically elegant; some are more easily adaptable to extensions of the type theory\ldots\  We thus take this opportunity to survey several of the various options, and the comparisons between them.  (TODO: do this!  Mention all the definitions above, at least briefly; and discuss stratified/reachable versions.)

\begin{definition} A \emph{split full comprehension category}\cite{jacobs:comprehension-categories} (\fscc) is a category $\C$ together with a split fibration $p : \E \to \C$ and a factorisation $p = \cod \cdot \pow : \E \to \C^\rightarrow \to \C$, such that (a) $\pow$ maps cartesion arrows to pullback squares, and (b) $\pow$ is full and faithful.  By abuse of notation, we will often refer to $\C$ itself as the comprehension category; the pair $(p,\pow)$ is called a \emph{comprehension structure} on $\E$.  

A (strict) map of \fsccs{} is just a functor $F: \C \to \C'$ and a map of fibrations $p \to p'$ over $F$, commuting with the factorisations $\pow$, $\pow'$.  A \emph{map of comprehension structures} on $\C$ is just the case $F = 1_C$.
\end{definition}

If $\C$ has all pullbacks, then condition (a) just says that $\pow$ is a map of fibrations.
 
In fact, for fixed $\C$, $\FSCS(\C)$ is (equivalent to) a presheaf category---specifically, to to the slice of $\hat{\C}$ over the presheaf $\cod^{\mathrm{spl}}$ in which an element of $P(A)$ is a map $f : B \to A$ together with chosen pullbacks along all maps $j : A' \to A$.  (Explain how??)  comprehension categories will frequently be presented in this form.


\begin{example}For any type theory $\T$, its category of context $\C(\T)$ is a comprehension category, in which objects of $p(\Gamma)$ are types dependent over $\Gamma$, and $\pow$ sends a type $A \in p(\Gamma)$ to the dependent projection $\Gamma, x : A \to \Gamma$.
\end{example}

In fact, this construction is part of an equivalence between $\Th$ and a certain full co-reflexive subcategory of $\FSCC$: see Appendix \ref{app:??} for details.  In light of this, we will typically refer to the objects of any \fscc\ as contexts, and the objects of the fibration as dependent types.

\para{Categories with attributes} 

From \cite{pitts:categorical-logic}, \cite{hoffman:syntax-and-semantics}, \cite{dybjer:internal-type-theory}, under various names.  Give definitions; (in notation to match that used above); give equivalence (honest 1-equivalence) with comprehension categories.

\para{Stratification}

Define \emph{stratified} cwa, \& map of; note that it's (perhaps unexpectedly) a \emph{full subcategory}(!) of cwa's, closed under connected limits and colimits. (NB: this deinition seems to be new (though comparable to Cartmell); have I missed something?)

Proposition: there is an \emph{honest 1-equivalence} between stratified cwa's and dependent algebraic theories as laid out in appendix.

Note: since morphisms of DAT's are most easily defined by transfer from cwa's, the content of this proposition is really just that there are maps of \emph{objects} from stratified cwa's to dat's and back, with an isomorphism $\C \iso FG(\G)$.

Define \emph{reachable} cwa's after Pitts.

Theorem: there is a \emph{2-equivalence} between reachable cwa's and stratified cwa's.

Discuss significance of 1- versus 2-equivalence: latter gives equivalence of ``type theories'', which is fine on the categorical side, but not on the syntactic side: we care about the difference between \emph{isomorphism} and \emph{equivalence} there as syntactic presentations of theories are \emph{0-categorical objects} (Voevodsky slogan!).

Also point out: type theory with \emph{context and their maps and equalities as primitive judgements} (hence allowing equalities and maps between contexts of different lengths) should correspond to general cwa's.

\para{Constructors}
This all extends to theories with type constructors.

Define: a lax map of comprehension structures (do I mean colax??)

Point out: the 2-category $\mathbf{\FSCS(\C)_\lax}$ has finite products.  (And these are probably better seen as 2-limits in $\FSCS(\C)$.)

\begin{definition}
An \emph{\fscs\ with units} on a category $\C$ is an \fscs\ $(p,\pow_0)$ together with a strict map of \fscss\ $1 : (1_B,\textit{id}) \to (p,\pow_0)$. 

An \emph{\fscs\ with binary products} on $\C$ is an \fscs\ $(p,\pow_0)$ with a strict map $ \times : (p times_B p, \pow_0 \times_B) \to (p,\pow_0)$. 
\end{definition}

Note: this implies a certain adjunction, so does corresponds to Jacobs' ``fscc with units''.  [Show this??]

To introduce the structure corresponding to identity types, we will need a little more terminology.

\begin{definition}[Dependent contexts]
Given any comprehension category $(\C,p,\pow)$, we may construct another comprehension structure $(p^\cxt,\pow^\cxt)$ on $\C$:

An object of $p^\cxt[\Gamma]$ is a list $A_1,\ldots,A_n$, where $A_i \in p(\Gamma . A_1 \ldots . A_{i-1})$, for each $i \leq n$; context extension is defined by $\Gamma . (A_1 \ldots , A_n) = \Gamma . A_1 \ldots . A_n$, and pullback $f^* : p^\cxt(\Gamma) \to p^\cxt(\Delta)$ is similarly defined in terms of pullback in $p$.

This is the object part of an evident functor $(-)*$ on $\FSCS(\C)$.
\end{definition}

The $(-)^\cxt$-construction has a natural type-theoretic interpretation: if $(\C,p,\pow)$ was obtained from a type theory, then for any $\Gamma$, $p^\cxt(\Gamma)$ is (isomorphic to) the category of dependent contexts over $\Gamma$ and dependent context morphisms between them.

Morevoer, $(-)^\cxt$ has a natural monad structure, and indeed is the ``free monoid'' monad for a certain monoidal structure on $\FSCS(\C)$; and all this is natural in $\C$, giving a total monad $(-)^\cxt$ on $\FSCC$ over $\Cat$.  However, these aspects will not concern us further.

(If I included a discussion of ``stratified comp cats'' earlier, mention how this construction naturally takes us outside of them unless we soup it up; and how the souped-up version gives a ``strong $\Sigma$-types'' monad; but why it \emph{doesn't} at the moment.)

\para[The nice slice] Even if $(\C,p,\pow)$ was stratified, $(\C,p^\cxt,\pow^\cxt)$ will generally not be: extending a context by a dependent context may increase its length by more than 1!

However, for any $(\C,p,\pow)$ and $\Gamma \in \C$, there is an evident stratified attributes structure on $p^\cxt(\Gamma)$; the resulting cwa may be called the \emph{nice slice} $(\C,p,\pow)/\Gamma$, and corresponds type-theoretically to working in context $\Gamma$.

\begin{definition}An \emph{elim-structure} on a map $f \colon \Xi \to \Theta$ is a function $E$, assigning to each $C \in p(\Theta)$ and each map $d \colon \Xi \to \Theta.C$ over $\Theta$ a section $E(C,d) \colon \Theta \to \Theta.C$ of the dependent projection $\pi_C$ satisfying $E(C,d) \cdot f = d$.
\end{definition}

Syntactically, this corresponds to the usual style elimination rule
$$\inferrule*{ \y : \Theta \types C(\y)\ \type \\
\x : \Xi \types d(\x) : C(f(\x)) }
{\y : \Theta \types E(C,d;\y) : C(\y)}$$
with computation rule concluding $E(C,d;f(\x)) = d(\x)$.  (Compare $\Id$-elim.)

Categorically, $E$ gives fillers for certain triangles:
$$\xymatrix{ \Xi \ar[d]_f \ar[r]^-d & \Theta.C \ar@/^/@{->>}[dl] \\
\Theta \ar@/^/@{..>}[ur]|-{E(C,d)} } $$  %TODO: prettify the spacing of this a bit!

This in turn implies a more familiar square-filling
$$\xymatrix{ \Xi \ar[d]_f \ar[r] & \Gamma.\Delta \ar@{->>}[d] \\
\Theta \ar@{.>}[ur] \ar[r] & \Gamma }$$
(exhibiting $f$ as weakly orthogonal to all dependent projections; see Section \ref{???} below, and cf.\ \cite{gambino-garner}), together with some stability conditions on the resulting fillers. 

\begin{definition}A \emph{Frobenius elim-structure} on a map $f \colon \Xi \to \Theta$ is a choice of elim-structure $E_\Delta$ on $(f.\Delta) \colon \Xi.(f^*\Delta) \to \Theta.\Delta$, for each $\Delta \in p^\cxt(\Theta)$.
\end{definition}

Syntactically this corresponds to an extra parameter in all the contexts of the rule:
$$\inferrule*{ \y : \Theta, \z : \Delta(\y) \types C(\y,\z)\ \type \\
\x : \Xi, \z: \Delta(f(\x)) \types d(\x,\z) : C(f(\x),\z) }
{\y : \Theta, \z : \Delta(\y) \types E_\Delta(C,d;\y,\z) : C(\y,\z)}$$

\begin{definition}An \emph{$\Id$-structure} on a (plain or stratified) comprehension category $(\C,p,\pow)$ consists of the following data for each context $\Gamma \in \C$ and type $A \in p(\Gamma)$:

\begin{enumerate}
\item a type $\Id_A \in p(\Gamma . A . A)$;

\item a map $r_A \colon \Gamma.A \to \Gamma.A.A.\Id_A$ lifting the diagonal (contraction) map $\diag_A \colon \Gamma.A \to \Gamma.A.A$ over $\Gamma$

$$\xymatrix{ & \Gamma.A.A.\Id_A \ar@{->>}[d] \\
\Gamma.A \ar[ur]^{r_A} \ar[r]^{\delta_A} \ar@{->>}[dr] & \Gamma.A.A \ar@{->>}[d] \\ & \Gamma}$$

\item a Frobenius elim-structure $J_A$ on $r_A$,
$$\xymatrix{ \Gamma.A.\Delta \ar[dr]_{r_A.\Delta} \ar[rr]^d & & \Gamma.A.A.\Id_A.C \ar@/^/@{->>}[dl] \\
& \Gamma.A.A.\Id_A \ar@/^/@{..>}[ur]|-{J_{A,\Delta}(C,d)} } $$
%TODO: prettify the spacing of this a bit!
\end{enumerate}

all stably in $\Gamma$, in that for $A \in p(\Gamma)$ and $f \colon \Theta \to \Gamma$,

\begin{enumerate}
\item $ (f.A.A)^*\Id_A = \Id_{f^*A} \in p(\Theta.{f^*A}.{f^*A})$: 

$$\xymatrix{\Id_{f^*A} & & Id_A \ar@{|->}[ll] \\
\Theta.f^*A.f^*A \ar[rr]^{f.A.A} & & \Gamma.A.A }$$

\item $f^*(r_A) = r_{f^*A}$; equivalently, the following square commutes:
$$\xymatrix{\Theta.f^*A \ar[d]^{r_{f^*A}} \ar[r] & \Gamma.A \ar[d]^{r_A} \\
\Theta.f^*A.f^*A.\Id_{f^*A} \ar[r] & \Gamma.A.A.\Id_A}$$

\item and, for all suitable $\Delta, C, d$, we have $f^*(J_{A,\Delta}(C,d)) = J_{f^*A,f^*\Delta}(f^*C,f^*d)$; in other words, the square
$$\xymatrix{
\{\mbox{triangles over}\ r_A.\Delta\} \ar[d]^{J_{A,\Delta}} \ar[rr]^{f^*} 
    & & \{\mbox{triangles over} r_{f^*A}.f^*\Delta\} \ar[d]^{J_{f^*A,f^*\Delta}} \\
\{\mbox{filled triangles}\} \ar[rr]^{f^*}
    & & \{\mbox{filled triangles}\} }$$
commutes.
\end{enumerate}
\end{definition}

(TO DO: sleep on this for a while, try to find a nice way of wrapping this up, eg fibrationally or similar!)

TO DO: define the various categories $\CwA^\Id$, etc.

\begin{proposition} \label{prop:thid-equiv-cwaid} If $\T$ is any DTT with $\Id$-types, then $\cl(\T)$ admits a canonical $\Id$-structure.  Conversely, if $\T$ is any (plain or stratified) category with attributes, then $\th(\C)$ admits an interpretation of the $\Id$-rules; and the maps $\epsilon_\C \colon \cl(\th(\C)) \to \C$ and $\eta \colon \T \to \T'$ preserve the resulting $\Id$-structure.

In particular, the equivalence $\Th \equiv \CwA_\strat$ lifts to an equivalence $\ThId \equiv \CwA^\Id$.

\begin{proof}Straighforward verification.
\end{proof}
\end{proposition}

\begin{proposition}[Identity contexts] An $\Id$-structure on $(\C,p,\pow)$ lifts to one on $(\C,p^\cxt,\pow^\cxt)$.
\end{proposition}

Note the interesting type-theoretic content: this shows that from identity \emph{types} for dependent types, we can build identity \emph{contexts} for dependent contexts, satisfying all the same rules.

\begin{proof}
We just sketch the proof here; see \cite[2.3.1]{garner:2d-models} for details.  By Proposition \ref{prop:thid-equiv-cwaid}, we may work type-theoretically.

(TODO: is this worth the notation that it requires developing?)
\end{proof}

\para{The coslice construction}



\section{Internal algebras for operads}  (This should probably be a separate chapter.)

\para Give fuller account of what I rush through in my previous paper: show  correspondence btn different notions of algebras for an operad!  (a) models of ess. alg. (Lawvere) theory (poss with extra structure: "P-maps"); (b) Batanin: monoidal globular categories (as used + nicely expounded in [GvdB]); (c) Leinster: (weak) T-structured categories.

Lovely rarely-cited WEBER paper gives source for most of this!  Possibly even \emph{everything} I need is there, in which case possibly move this section to appendix, and fold the first part of this chapter into the next chapter???

\begin{definition}[Endomorphism operads] For $\E$ any category, $\X$ any globular operad in $\E$, we write $\End_\E(\X)$ for the operad (construction\ldots\ either by monoidal globular categories, or by ``representable'' style of my previous paper; latter is slicker here, but seems very difficult for showing the functoriality).  More generally, really want the $\Coll$-enriched category structure on (appropriate subcategory of) $[\G,\E]$.
\end{definition}

\proposition If $\E$ has enough limits, then for any pasting diagram $\pi$, $$\End_\E(\X)(\pi) \iso [\G/n,\E](X \cotensor \hat{\pi},X \cotensor \yon(n))$$
where the right hand side consists of ``pylon diagrams'':
$$\textrm{draw the diagram here.}$$

\begin{proof} Straightforward (in either construction of $\End$).
\end{proof}

\proposition $\End_\E(\X)$ is functorial in $\E$: a functor $F : \E \to \F$ preserving appropriate limits induces a map $\End_\E(\X) \to \End_\F(F\X)$.

\begin{proof} Straightforward in the ``monoidal globular categories'' approach.  Can't currently see how to do it in the ``representable'' approach!?
\end{proof}

\definition An \emph{algebra} for an operad $P$ on an object $\X$ of $\E$ is an operad map $\xi \colon P \to \End_\E (\X)$ (the \emph{action} of $P$ on $\X$); a map of $P$-algebras is a globular map $\f \colon \X \to \Y$ commuting with the action maps, i.e.\ such that the square 
$$\xymatrix{P \ar[r]^{\xi} \ar[d]^{\upsilon} & [\X,\X] \ar[d]^{f \cdot\ } \\ [\Y,\Y] \ar[r]^{\ \cdot f} & [\X,\Y]}$$
commutes.

In enriched terms, the resulting category $\IntAlg{P}{\E}$ is $\enrCat(\Coll)(P,[\G,\E])$.

% but it's alright now
% I learned my lesson, yeah
% can't please everyone, so...
% Caffé Nero, Davygate, 14.vii

We can also consider $P$ as defining a certain finite-limit sketch $\mathrm{Sk}(P)$, and compare the internal algebras defined here with models of this sketch in $\E$.

\proposition $\IntAlg{P}{\E} \equiv \mathbf{CmpSpan}\mathbf{Mod}_\E(\mathrm{Sk}(P))$


% this file is called up by thesis.tex
% content in this file will be fed into the main document

%: ----------------------- name of chapter  -------------------------
\chapter{Globular structures from type theory}

%: ----------------------- contents from here ------------------------

\section{The fundamental weak omega-groupoid of a type}

Update of my prev paper (+ more detailed comparison with Richard + Benno): type gives internal weak omega-groupoid in the classifying category.

\definition algebraic $\Id$-type categories, $\CatId$.

\para functor $\ThId \to \CatId$ over $\Cat$.

\para functor $\FibSpans: \Th \to \strMonGlobCat$: imitate the original Batanin ``higher spans'' construction, but using doubly-dependent types instead of spans.  Note that it's even strict since fibration was split!  (is it?  maybe not because of reordering!)

\para now for $\C \in \ThId$, get for each context $X \in \C$ a globular element of $fibSpans(\C)$.  


\section{The classifying weak omega-category of a type theory}




% ---------------------------------------------------------------------------
%: ----------------------- end of thesis sub-document ------------------------
% ---------------------------------------------------------------------------


% this file is called up by thesis.tex
% content in this file will be fed into the main document

%: ----------------------- name of chapter  -------------------------

\chapter{Homotopical constructions from globular higher categories}

%: ----------------------- contents from here ------------------------

(NON)SECTION AIM: Motivate constructing the simplicial nerve, and hence the model structure.  TODO: cut down, too discursive??  Maybe move this to end of previous section, or even to introduction, and here give an overview of simplicial structures / methods?

\para % Having constructed weak $\omega$-categories from the syntax of type theory, the natural next step is to use them for something!
The great power of classifying categories in 1-categorical logic come from [`depends on'?] an analysis of the logical constructors and rules in categorical terms: substitution as pullbacks, $\Pi$- and $\Sigma$-types as adjoints, and so on.  So, a natural first impulse is to try to analyse the universal properties of the type constructors within $\Clw(\T)$, which we would expect to be weak-higher-categorical analogues of the usual logical structure.

Unfortunately, the theory of logical structure on globular higher-categories is not yet well-understood.  Of course, we can hope that the developing dictionary with type theory will help understand how such structure should behave!  However, there is an alternative model of higher categories for which the relevant theory is already much further advanced: Joyal's quasi-categories.  Quasi-categories are not a fully general theory of higher categories: they only model so-called ($\infty$,1)-categories, in which all cells above dimension 1 are (weakly) invertible.  However, as we have seen, the classifying categories of type theories are of this form; so quasi-categories seem potentially excellently-suited for our desired analysis, if only we can give quasi-category models for $Clw(\T)$!

\para In other words, we would like to construct a functor
$$ \ClwQCat \colon \Th \to \QCat .$$

TODO: Hmm, this doesn't work if the reader doesn't know yet that quas-cateogries are simplicial things!  Work out how to re-organise to fit that in nicely.

There are two obvious options.  Firstly, we could construct $\ClwQCat(\T)$ directly from the theory $\T$.  [TODO: Ask Michael whether/how much to mention simplicial type theory.]  However, $\Id$-types as they stand are inescapably globular; there seems no obvious way to extract simplicial sets from the theory as cleanly and directly as one can extract globular sets.  (The intriguing approach of re-axiomatising $\Id$-types to be ``naturally simplicial in shape'' has, however, been considered by Warren and Gambino \cite{??}.)

It thus seems natural to take a different approach: to construct $\ClwQCat$ in two steps, composing $\Clw$ from the previous section with a functor
$$ \N^\qcat \colon \Alg{P} \to \QCat $$
giving the ``quasi-category nerve'' of a globular weak $n$-category.  This has the added payoff that such a functor would be of independent interest, since the comparison between globular and simplicial higher categories is as yet little-understood in the fully weak case.

\para In Section \ref{sec:simplicial-nerves}, we will thus construct several candidate nerve functors.  Constructing simplicial objects is straightforward; the hard part is proving the requisite horn filling conditions to show that they are quasi-categories.

It is for this that we construct, in Sections \ref{sec:model-strux-general} and \ref{sec:model-strux-specific}, a Quillen model structure on categories of globular higher categories (under certain extra hypotheses).   The computational tools provided by such a structure provide precisely what we need to show that the horns arising in our nerve constructions can be filled, and hence that the nerves are indeed quasi-categories.

In fact, we construct an \emph{algebraic} model structure in the sense of \cite{riehl:alg-mod-strux}.  This is a Quillen model structure in which both weak factorisation systems are NWS's and there is moreover a comparison map connecting the two.  While we will not need any of the extra power of an algebraic model structure, the algebraicity comes almost for free given the form of our proof.

\section{Cellular algebraic model structures} \label{sec:model-strux-general}

SECTION AIM: The general construction of an algebraic model structure from a collection of generating cells.  (TODO: read/ask around in case I've missed where something closer to this has already been done; work out terminology for this general construction.)

\para Recall what an \AWFS and \AMS are.  (Or put this in appendix?)

\para We start by recalling from \cite{garner:understanding}, \cite{garner:homomorphisms} the construction of an AWFS on $\strnCat$ whose right maps are precisely the contractible maps.

...do it by the Garner small-object argument!  Algebraic freeness shows the right maps are what we think.  And Garner shows that this is also the adjunction with computads, fwiw (maybe leave this out if not needed).

Point out how this comes from the map $D(\ob \G) \to \G \to \GSets \to \wknCat$ of cells/boundaries.

\para In fact, the remainder of the construction of the \AMS (though not the proof that it really is one) can be given entirely in terms of this set of generating cofibrations; and, indeed, $L'$ categories will give another example.  Thus for the remainder of this section, we will fix a category $\E$ that admits the small object argument [TODO: define this here or elsewhere!], a set $\I$ (considered as a discrete category), and a functor $\I \to \E^\Two$.  The functor will remain nameless, but we will write maps in its image as  $d_i \into c_i$, for $i \in \I$.  (They are to be thought of as inclusions of boundaries into cells.)

\para Construct $(\CC,\FF_t)$; construct ``canonical squares''.  Describe how to see them as equivs.

\para Construct $\WW$, $\W$, $(\CC_t,\FF)$ from this.  Infer $\xi$, by \cite[Rmk 3.6]{Riehl} (and hence get $\C_t \Imp \C$, $\F_t \Imp \F$).

\para Give $TF \Leftrightarrow F \cap W$, i.e.\ $\Alg{\FF_t} \Leftrightarrow \Alg{\FF} \times_{\E^\Two} \Alg{\WW}$.  Give: $\WW \to \E^\Two$ ``creates retracts'', so $\W$ closed under retracts.

\para Now the hard stuff!

\section{An algebraic model structure on globular higher categories?} \label{sec:model-strux-specific}

\para Discuss instantiating the theorem of the previous section to (a) L'-categories, (b) P-algebras; prove as many of the lemmas as possible!

\section{Simplicial nerves of globular higher categories} \label{sec:simplicial-nerves}

\para Give aim; different NWFS for different nerves; proof with model structure that these give nerves!

\para Prove from model strux that these really do give quasi-categories. 

TODO: read up Dugger references properly.




% ---------------------------------------------------------------------------
% ----------------------- end of thesis sub-document ------------------------
% ---------------------------------------------------------------------------


% Appendices
\appendix

% this file is called up by thesis.tex
% content in this file will be fed into the main document

%: ----------------------- name of chapter  -------------------------
\chapter{Background: Martin-L\"o{}f Type Theory}

Possibly just make this a presentation of the type theory?

\section{Presentation overview}

\para The type theories we consider are all essentially variants on the basic type theory originally presented in \cite{martin-loef:predicative-part}.  See also (TODO: pick out good references!)

The core of the type theory, its basic judgements and structural rules, is given in Section \ref{para:structural-core}; this will remain constant in all the theories we consider.

The type-constructors and various other rules that will be used are given in Section \ref{para:further-rules}; we will consider theories including various different subsets of these rules.

Section \ref{para:tt-lemmas} contains some fundamental lemmas concerning the resulting type theories: basic proof-theoretic properties, admissible rules, and so on.  TODO: this has changed!  Memo to self: don't write overviews of chapters before starting the chapters themselves...

For further background, see \cite{pitts:categorical-logic} (a notably thorough and precise presentation), \cite[\S6]{jacobs:categorical-logic} (for a thoughtful discussion of this system within the wider type-theoretic context), \cite{hofmann:syntax-and-semantics}, \cite[Ch.3]{n-p-s:programming} (discussion of the use of simply-typed calculus as the ``raw language'), and \cite{martin-loef:predicative-part} (the originial presentation of the theory).  The presentation uses essentially the formal presentation of \cite{pitts:categorical-logic}, in the notation of \cite{jacobs:categorical-logic}, slightly modified to include dependent contexts and their morphisms. 

(TODO: perhaps fully expound the presentation rather than just recapping?)

Notes (re-organise and/or remove later):

--- point of using $\lambda$-calculus as raw language: to separate the subtleties of binding and capture-avoiding substitution from those of dependency and well-formedness.

--- point of having the raw language already simply-typed, not completely un(i)typed: to maintain decidability of $\equiv$. 

--- abuse of notation: making variables explicit

--- dependent contexts and maps: to formulate Frobenius rule

--- dependent maps: really only need $\Gamma \types \vec f : \Delta \To \Delta'$ in case either $\Gamma = \diamond$ (for substitution rules) or $Delta = \diamond$ (for Frobenius rules).  However, it's simpler to give this than those two separately, plus natural for syntactic cwa.

--- syntactic sugar: $\x : \Gamma \types a(\x) : A(\x)$ really means $\Gamma \types a : A$, where all of these are \emph{closed} metaexpressions, but $a$, $A$ have arities $\term^n \to \term$, $\term^n \to \type$, where $n$ is the length of $\Gamma$; and so on.


\para{Raw syntax} \label{para:raw-syntax}

Define: ``raw language''.  (NB: sometimes called \emph{metalanguage}, but not meaning the language in which we reason about the type theory; rather, plays a similar r\^o{}le to that of strings (or trees) of symbols in simpler systems.)

Define: signature, pre-terms, pre-types, pre-terms, judgements.

\para{Judgement forms} \label{para:judgement-forms}

\begin{center}\begin{tabular}{|@{\ }c@{\qquad \qquad}c@{\ }|}
\hline
\multicolumn{2}{|c|}{Basic judgement forms} \\
\hline
$\Gamma \types A \ \type $ & $ \Gamma \types a:A $ \\
$\Gamma \types A = A' \ \type $& $ \Gamma \types a = a' : A $ \\
\hline
\end{tabular}

\begin{tabular}{|@{\ }c@{\qquad \qquad}c@{\ }|}
\hline
\multicolumn{2}{|c|}{Derived judgement forms} \\
\hline
$ \Gamma \types \Delta \ \cxt$ & $\Gamma \types {\vec f} : \Delta \To \Delta' $  \\
$ \Gamma \types \Delta = \Delta' \ \cxt$ & $\Gamma \types {\vec f} = {\vec f}' : \Delta \To \Delta'$ \\
\hline
\end{tabular}

Explain in what sense these are derived judgements!

(Discuss in parens why we use dependent cxts and morphisms.)

Perhaps (if feeling pedantic, or can find way to say it non-pedantically) discuss arities of things in judgements, etc.?

\para{Rules: Structural core} \label{para:structural-core}


\begin{tabular}{c}
Rules for contexts:
\\ \ \\
$\inferrule*[right=$\cxt$-$\diamond\diamond$]{\ }{\diamond \types \diamond \ \cxt}$ \quad $\inferrule*[right=$\cxt$-$\diamond$]{\diamond \types \Gamma\ \cxt}{\Gamma \types \diamond \ \cxt}$ \quad $\inferrule*[right={$\cxt$=-$\diamond$}]{\diamond \types \Gamma\ \cxt}{\diamond \types \diamond = \diamond \ \cxt}$
\\ \ \\
$\inferrule*[right=$\cxt$-$\cons$]{\Gamma \types \Delta\ \cxt \\\\ \Gamma , \Delta \types A \ \type}{\quad \Gamma \types \Delta, A\ \cxt\quad}$ \qquad
$\inferrule*[right={$\cxt$=-$\cons$}]{\Gamma \types \Delta = \Delta'\ \cxt \\\\ \Gamma, \Delta \types A = A'\ \type}{\quad \Gamma \types \Delta, A = \Delta', A'\ \cxt \quad}$ 
\\ \ \\
% $\inferrule*[right=$\cxt$-$\subst$]{\x : \Gamma \types \Delta(\x)\ \cxt \\\\ \diamond \types \f : \Gamma' \To \Gamma}{\quad \y : \Gamma' \types \Delta(\f(\y))\ \cxt \quad}$ \qquad 
% $\inferrule*[right={$\cxt$=-$\subst$}]{\x : \Gamma \types \Delta(\x) = \Delta'(\x)\ \cxt \\\\ \diamond \types \f = \f' : \Gamma' \To \Gamma}{\quad \y : \Gamma' \types \Delta(\f(\y)) = \Delta'(\f'(\y))\ \cxt\quad }$ 
% \\ \ \\ 
\end{tabular}

\begin{tabular}{c}
Rules for types: \\ \ \\
$\inferrule*[right={$\type$=-$\refl$}]{\Gamma \types A\ \type}{\quad \Gamma \types A = A\ \type \quad} \qquad
\inferrule*[right={$\type$=-$\sym$}]{\Gamma \types A = B\ \type}{\quad \Gamma \types B = A\ \type \quad}$ \\ \ \\
$\inferrule*[right={$\type$=-$\trans$}]{\Gamma \types A = B\ \type \\\\ \Gamma \types B = C\ \type}{\quad \Gamma \types A = C\ \type \quad} \qquad 
\inferrule*[right=$\type$-$\subst$]{\x : \Gamma \types A(\x)\ \type \\\\ \diamond \types \f : \Gamma' \To \Gamma}{\quad \y : \Gamma' \types A(\f(\y))\ \type \quad}$ \\ \ \\
$\inferrule*[right={$\type$=-$\subst$}]{\x : \Gamma \types A = A'\ \type \\\\ \diamond \types \f = \f' : \Gamma' \To \Gamma}{\quad \y:\Gamma' \types A(\f(\y)) = A'(\f'(\y))\ \type\quad }$ \\ \ \\
\end{tabular}

\begin{tabular}{c}
Rules for terms: \\ \ \\
$\inferrule*[right={\sf var}]{\Gamma \types A\ \type \\ \Gamma, A \types \Delta\ \cxt}{\quad \Gamma, x:A, \Delta \types x : A \quad}$ \\ \ \\
$\inferrule*[right={\sf term-coerce}]{\Gamma \types a : A \\\\ \Gamma \types A = A'\ \type}{\quad \Gamma \types a : A' \quad}$ \qquad \quad
$\inferrule*[right={\sf term=-coerce}]{\Gamma \types a = a' : A \\\\ \Gamma \types A = A'\ \type}{\quad \Gamma \types a = a' : A' \quad}$
\\ \ \\
$\inferrule*[right={$\term$=-$\refl$}]{\Gamma \types a : A}{\quad \Gamma \types a = a : A\quad} \qquad \qquad
\inferrule*[right={$\term$=-$\sym$}]{\Gamma \types a = b : A}{\quad \Gamma \types b = a : A \quad}$
\\ \ \\
$\inferrule*[right={$\term$=-$\trans$}]{\Gamma \types a = b : A \\\\ \Gamma \types b = c : A}{\quad \Gamma \types a = c : A \quad} \qquad 
\inferrule*[right=$\term$-$\subst$]{\x : \Gamma \types a(\x) : A(\x) \\\\ \diamond \types \f : \Gamma' \To \Gamma}{\quad \y : \Gamma' \types a(\f(\y)) : A(\f(\y)) \quad}$
\\ \ \\
$\inferrule*[right={$\term$=-$\subst$}]{\x : \Gamma \types a(\x) = a'(\x) : A(\x) \\\\ \diamond \types \f = \f' : \Gamma' \To \Gamma}{\quad \y:\Gamma' \types a(\f(\y)) = a'(\f'(\y)) : A(\f(\y)) \quad}$
\end{tabular}


\begin{tabular}{c}
Rules for context maps: 
\\ \ \\
$\inferrule*[right=$\cxt$-$\diamond$]{\diamond \types \Gamma\ \cxt}{\diamond \types \diamond : \Gamma \To \diamond}$ \qquad $\inferrule*[right={$\cxt$=-$\diamond$}]{\diamond \types \Gamma\ \cxt}{\diamond \types \diamond = \diamond : \Gamma \To \diamond}$
\\ \ \\
$\inferrule*[right=$\cxt$-$\cons$]{\Gamma \types \f : \Delta' \To \Delta \\\\ \x : \Gamma , \y : \Delta(\x) \types A(\x,\y) \ \type \\\\ \x : \Gamma, \z : \Delta'(\x) \types a(\x,\z) : A(\x,\f(\x,\z))}{\quad \Gamma \types (\f,a) : \Delta' \To (\Delta, A) \quad}$ \qquad
$\inferrule*[right=$\cxt$-$\cons$]{\Gamma \types \f : \Delta' \To \Delta \\\\ \x : \Gamma , \y : \Delta(\x) \types A(\x,\y) \ \type \\\\ \x : \Gamma, \z : \Delta'(\x) \types a(\x,\z) : A(\x,\f(\x,\z))}{\quad \Gamma \types (\f,a) : \Delta' \To (\Delta, A) \quad}$ \qquad
\\ \ \\
% $\inferrule*[right=$\cxt$-$\subst$]{\x : \Gamma \types \Delta(\x)\ \cxt \\\\ \diamond \types \f : \Gamma' \To \Gamma}{\quad \y : \Gamma' \types \Delta(\f(\y))\ \cxt \quad}$ \qquad 
% $\inferrule*[right={$\cxt$=-$\subst$}]{\x : \Gamma \types \Delta(\x) = \Delta'(\x)\ \cxt \\\\ \diamond \types \f = \f' : \Gamma' \To \Gamma}{\quad \y : \Gamma' \types \Delta(\f(\y)) = \Delta'(\f'(\y))\ \cxt\quad }$ 
% \\ \ \\ 
\end{tabular}
\end{center}

\para{Type constructors and miscellaneous rules} \label{para:further-rules}

Rules for type-constructors go here: $\Id$-, $\Pi$-, $\Sigma$-, with just a tip of the hat to $\NN$-, $\Bool$, \ldots

\begin{center}
\begin{tabular}{c}
Rules for $\Id$-types: 
\\ \ \\
$\inferrule*[right=$\Id$-$\form$]{\Gamma \types A\ \type}{\Gamma, x,y : A \types \Id_A(x,y)\ \type}$ \qquad $\inferrule*[right={$\Id$-$\intro$}]{\Gamma \types A\ \type}{\Gamma, x : A \types r(x) : \Id_A(x,x)}$
\\ \ \\
$\inferrule*[right=$\Id$-$\elim$]{\Gamma, x,y : A, p : \Id_A(x,y), \w : \Delta(x,y,p) \types C(x,y,p,\w)\ \type \\
\Gamma, z:A, \v : \Delta(z,z,r(z)) \types d(z,\v) : C(z,z,r(z),\v)}
{\Gamma, x,y:A, p:\Id_A(x,y), \w:\Delta(x,y,p) \types J_{A;\  x,y,p.\Delta;\ x,y,p,\w.C}(z,\v.\ d(z,\v);\ x,y,p,\w) : C(x,y,p,\w)}$
\\ \ \\
$\inferrule*[right=$\Id$-$\comp$]{\Gamma, x,y : A, p : \Id_A(x,y), \w : \Delta(x,y,p) \types C(x,y,p,\w)\ \type \\
\Gamma, z:A, \v : \Delta(z,z,r(z)) \types d(z,\v) : C(z,z,r(z),\v)}
{\Gamma, x:A, \w:\Delta(x,y,p) \types J_{A;\ x,y,p.\Delta}(z,\v.\ d(z,\v);\ x,x,r(x),\w) = d(x,\w): C(x,y,p,\w)}$
\\ \ \\
$\inferrule*[right={reflection}]{\Gamma \types e : \Id_A(a,b)}{\Gamma \types a = b : A}$
\\ \ \\
$\inferrule*[right=K]{\Gamma, x: A, p : \Id_A(x,x), \w : \Delta(x,p) \types C(x,p,\w)\ \type \\
\Gamma, z:A, \v : \Delta(z,r(z)) \types d(z,\v) : C(z,r(z),\v)}
{\Gamma, x:A, p:\Id_A(x,x), \w:\Delta(x,p) \types K_{A;\ x,p.\Delta;\ x,p,\w.C}(z,\v.\ d(z,\v);\ x,p,\w) : C(x,p,\w)}$
\\ \ \\
\end{tabular}
\end{center}

Include optional things: $\eta$- and extensionality rules for function-types; K and the truncation rules for $\Id$-types. 

Discuss implications between these?

\para{Translations} \label{para:translations}

Define translation, category of type theories.

\section{Categories with attributes}

Define: full split comprehension categories \cite[4.10]{jacobs:comprehension-categories} / categories with attributes.  \cite[\S 6.4]{pitts:categorical-logic}

TODO: ask around about reachable vs.\ ``stratified'' cwa's.

Define: $\Id$-structure, $\Pi$-structure, etc. on cwa's.  (Reference: \cite{awodey-warren}.)

\para{Equivalence with type theories}

Recall (from Pitts): \emph{equivalence} (honest 1-equivalence!) between type theories with no constructors and stratified fscc's.  TO DO: read Jacobs in full (\& chase further\ldots) re theories with constructors!

Note: $\FSCC_\strat$ is a \emph{full} subcat of $\FSCC$, and coreflective (with comonad induced by adjunction from $\Th$, so closed under colims!

Extend equivalence above to constructors: $\ThId$, $\ThIdPi$, etc.  

Fact: since all essentially algebraic (or otherwise), these categories are all co-complete!




% \section{Basic properties} \label{sec:tt-lemmas}
% 
% --- lemma: $\Gamma \types \Delta \cxt$ iff $\diamond \types \Gamma , \Delta \cxt$
% --- lemma: if $\Gamma$ appears on the left of any judgement, then $\diamond \types \Gamma \cxt$.  Generally: can derive boundaries of derived judgements.
% --- judgements are stable under equality of the boundary.
% --- lemma: can derive wkg, subst.
% --- lemma: can concatenate context maps like contexts.
% --- lemma: having given only substitution along closed context maps, can get it along all.  
% --- lemma: $\equiv$ decidable.
% --- lemma: if no equality axioms, then judgemental equality decidable.
% --- lemma: if no equality axioms, then typing unique.
% \para 



% ---------------------------------------------------------------------------
%: ----------------------- end of thesis sub-document ------------------------
% ---------------------------------------------------------------------------


% this file is called up by thesis.tex
% content in this file will be fed into the main document

%: ----------------------- name of chapter  -------------------------
\chapter{Background: globular higher category theory}

%: ----------------------- contents from here ------------------------

\section{Strict higher categories}

\para Define $\strnCat$ and $\strnCat[\omega]$ by enrichment. 


\para Analyse $T$: pasting diagrams, Batanin trees, familial representability.

Fact: Cartesian, as can see from familial rep'bility\cite{street:petit-topos},\cite{carboni-johnstone}.

\section{Globular operads: Leinster presentation}

\definition Contractible map of glob sets.

\definition Leinster operad, as cartesian map of cartesian monads.

\para Equivalent: local presentation of Leinster operad: object over $T1$, with appropriate structure.

\definition 
\para Contractibility.



\section{Globular operads: Batanin/Weber presentation}

\section{Internal algebras for operads}  (This should probably be split up and moved: the globular part moved into the ``Globular background'' appendix, and then the construction of a monoidal globular category from a comprehension category folded into the construction of the fundamental things.)

\subsection*{Monoidal globular categories} Recall definition from Batanin. 

Examples: $\Spans$; $\Spans(\C)$, for $\C$ with pullbacks; $\FibSpans(\C)$, for $\C$ with a distinguished class of ``fibrations'', closed under composition and pullbacks.

\para[Pasting diagrams]

Example of a mon glob cat: $T1$.  Give representation as Batanin trees; also as free monoids.

Notation note: I use $\delta(n)$ for the boundary of $\yon(n)$, which is standard, but also $\delta \widehat{\pi}$ for the boundary of the $\widehat{\pi}$ which is less standard: Batanin, Street (TODO: refs?) and others use $\delta \pi := s \pi = t \pi$ (since these are always equal), whereas I always distinguish these (yes, they're equal as pasting diagrams, but in use they're typically `intensionally' distinct) and my $\delta \widehat{\pi}$ is the pushout of $\widehat{s\pi}$ and $\widehat{t\pi}$ along their common boundary. 

Give realisation as globular sets, by (a) iterated pushout, (b) colimit.

Define ``one-leaf prunings''.  

In free monoid terms, defined inductively: removing an (extremal?) $()$ at some depth.  In tree terms: removing a leaf (an endpoint leaf?) at some height!

(``Any leaf'' is more natural---then these really do correspond to places one can apply $\Id$-elim.  On the other hand, ``extremal leaf'' makes pushout/pullback decompositions/lemmas simpler.  Ah!  Idea: maybe just define $\pi^-$, and then mention that could use more general one-leaf prunings.)

Note: every non-point pasting diagram has some; eg in free monoid terms: 
$$((),\pi_1,\ldots,\pi_{r-1})^- = (\pi_1,\ldots,\pi_{r-1})$$
$$(\pi_0,\ldots,\pi_{r-1})^- = (\pi_0^-,\ldots,\pi_{r-1})$$
is a nice straightforward one, corresponding (in tree terms) to following the leftmost branch to its end, and removing the leaf reached.

Show how realisation is pushout:

\newbox\potlbox
\setbox\potlbox=\hbox{\xy 
(-340,170)*{}="tlleft"; % "left": 0-source
(340,170)*{}="trleft";
(-340,-170)*{}="blleft";
(340,-170)*{}="brleft";
"tlleft";"trleft" **\dir{.};
"tlleft";"blleft" **\dir{.};
"trleft";"brleft" **\dir{.};
"blleft";"brleft" **\dir{.};
(-300,0)*+{\cdot}="a";
(0,0)*+{\cdot}="b";
{\ar "a";"b"};
\endxy}
\def\potl{\copy\potlbox}

\newbox\potrbox
\setbox\potrbox=\hbox{\xy 
(-340,170)*{}="tlleft"; % "left": 0-source
(340,170)*{}="trleft";
(-340,-170)*{}="blleft";
(340,-170)*{}="brleft";
"tlleft";"trleft" **\dir{.};
"tlleft";"blleft" **\dir{.};
"trleft";"brleft" **\dir{.};
"blleft";"brleft" **\dir{.};
(-300,0)*+{\cdot}="a";
(0,0)*+{\cdot}="b";
{\ar@/^1.2pc/ "a";"b"};
{\ar "a";"b"};
{\ar@{=>} (-150,120)*{};(-150,25)*{}} ;
\endxy}
\def\potr{\copy\potrbox}

\newbox\poblbox
\setbox\poblbox=\hbox{\xy 
(-340,170)*{}="tlleft"; % "left": 0-source
(340,170)*{}="trleft";
(-340,-170)*{}="blleft";
(340,-170)*{}="brleft";
"tlleft";"trleft" **\dir{.};
"tlleft";"blleft" **\dir{.};
"trleft";"brleft" **\dir{.};
"blleft";"brleft" **\dir{.};
(-300,0)*+{\cdot}="a";
(0,0)*+{\cdot}="b";
(300,0)*+{\cdot}="c";
{\ar "a";"b"};
{\ar@/_1.2pc/ "a";"b"};
{\ar@{=>} (-150,-20)*{};(-150,-115)*{}} ;
{\ar@/^0.7pc/ "b";"c"};
{\ar@/_0.7pc/ "b";"c"};
{\ar@{=>} (150,60)*{};(150,-60)*{}} ;
\endxy}
\def\pobl{\copy\poblbox}

\newbox\pobrbox
\setbox\pobrbox=\hbox{\xy 
(-340,170)*{}="tlleft"; % "left": 0-source
(340,170)*{}="trleft";
(-340,-170)*{}="blleft";
(340,-170)*{}="brleft";
"tlleft";"trleft" **\dir{.};
"tlleft";"blleft" **\dir{.};
"trleft";"brleft" **\dir{.};
"blleft";"brleft" **\dir{.};
(-300,0)*+{\cdot}="a";
(0,0)*+{\cdot}="b";
(300,0)*+{\cdot}="c";
{\ar@/^1.2pc/ "a";"b"};
{\ar "a";"b"};
{\ar@/_1.2pc/ "a";"b"};
{\ar@{=>} (-150,120)*{};(-150,25)*{}} ;
{\ar@{=>} (-150,-20)*{};(-150,-115)*{}} ;
{\ar@/^0.7pc/ "b";"c"};
{\ar@/_0.7pc/ "b";"c"};
{\ar@{=>} (150,60)*{};(150,-60)*{}} ;
\endxy}
\def\pobr{\copy\pobrbox}

$$\bfig \square[y(i)`y(i-1)`\widehat{\pi^-}`\widehat{\pi};```]
\place(400,100)[\po]
\efig
\quad \quad \quad
\bfig \Square[\potl`\potr`\pobl`\pobr;```]
\place(625,175)[\po]
\efig
$$

\subsection*{Operads in monoidal categories}

Define: globular object.  Define: operad!  Endomorphism operad!

\subsection*{Endomorphism operads, explicitly}

In $\FibSpans(\C)$, a globular object $\X$ is\ldots

Then for $\pi \in T1_n$, the peak of the $n$-span $(\X)\pi$ is the object
$$X^\pi := \lim_{c \in \int\! \hat{\pi}} X_{\dim c}$$

This somewhat cryptic formula is perhaps best illuminated by a couple of examples: if $\pi = (\xymatrix{ \bullet \rtwocell & \bullet \rtwocell & \bullet})$, then
\begin{eqnarray*} X_\pi & := & \lim \left( 
\bfig
\node X0l(0,0)[X_0]
\node X0m(700,0)[X_0]
\node X0r(1400,0)[X_0]
\node X1tl(350,300)[X_1]
\node X1tr(1050,300)[X_1]
\node X1bl(350,-300)[X_1]
\node X1br(1050,-300)[X_1]
\node X2l(350,0)[X_2]
\node X2r(1050,0)[X_2]
\arrow[X1tl`X0l;s]
\arrow[X1tl`X0m;t]
\arrow[X1tr`X0m;s]
\arrow[X1tr`X0r;t]
\arrow[X2l`X1tl;s]
\arrow[X2r`X1tr;s]
\arrow[X2l`X1bl;t]
\arrow[X2r`X1br;t]
\arrow[X1bl`X0l;s]
\arrow[X1bl`X0m;t]
\arrow[X1br`X0m;s]
\arrow[X1br`X0r;t]
\efig
\right) \\
& \iso & X_2 \times_{X_0} X_2,
\end{eqnarray*}
giving the object of 0-composable pairs of 2-cells in $X$.  Similarly, if $\pi$ is the basic $n$-cell, then $X_\pi = X_n$.

\para \label{para:homming-out-of-algebras} Describe the gloulbar hom-functor between $n$-objects \cite[3.6]{batanin:natural-environment} and hence into and out of globular objects.  Note: in the case of $\Spans$, this is monoidal. (Check: Batanin doesn't show this anywhere, does he? although he does use it!  Or, better, is it in Weber somewhere, or easily deducible?)

Really, would like a Yoneda lemma: for a globular object $\A$ of a cat (with fibrations) $\C$, $\End_{\C}(\A) \iso \End_{[\C,\Sets]}(\yon \A)$.

\example{Fundamental $\omega$-groupoid of a space}


\subsection*{Connection with Leinster approach}

Recall from Weber paper.


\subsection*{Enriched point of view}

TODO: the below is from a different earlier approach; fold this into the current presentation!

\begin{definition}[Endomorphism operads] For $\E$ any category, $\X$ any globular operad in $\E$, we write $\End_\E(\X)$ for the operad (construction\ldots\ either by monoidal globular categories, or by ``representable'' style of my previous paper; latter is slicker here, but seems very difficult for showing the functoriality).  More generally, really want the $\Coll$-enriched category structure on (appropriate subcategory of) $[\G,\E]$.
\end{definition}



\proposition If $\E$ has enough limits, then for any pasting diagram $\pi$, $$\End_\E(\X)(\pi) \iso [\G/n,\E](X \cotensor \hat{\pi},X \cotensor \yon(n))$$
where the right hand side consists of ``pylon diagrams'':
$$\textrm{draw the diagram here.}$$

\begin{proof} Straightforward (in either construction of $\End$).
\end{proof}

\proposition $\End_\E(\X)$ is functorial in $\E$: a functor $F : \E \to \F$ preserving appropriate limits induces a map $\End_\E(\X) \to \End_\F(F\X)$.

\begin{proof} Straightforward in the ``monoidal globular categories'' approach.  Can't currently see how to do it in the ``representable'' approach!?
\end{proof}

\definition An \emph{algebra} for an operad $P$ on an object $\X$ of $\E$ is an operad map $\xi \colon P \to \End_\E (\X)$ (the \emph{action} of $P$ on $\X$); a map of $P$-algebras is a globular map $\f \colon \X \to \Y$ commuting with the action maps, i.e.\ such that the square 
$$\xymatrix{P \ar[r]^{\xi} \ar[d]^{\upsilon} & [\X,\X] \ar[d]^{f \cdot\ } \\ [\Y,\Y] \ar[r]^{\ \cdot f} & [\X,\Y]}$$
commutes.

In enriched terms, the resulting category $\IntAlg{P}{\E}$ is $\enrCat(\Coll)(P,[\G,\E])$.

% but it's alright now
% I learned my lesson, yeah
% can't please everyone, so...
% Caffé Nero, Davygate, 14.vii

We can also consider $P$ as defining a certain finite-limit sketch $\mathrm{Sk}(P)$, and compare the internal algebras defined here with models of this sketch in $\E$.

\begin{proposition}$\IntAlg{P}{\E} \equiv \mathbf{CmpSpan}\mathbf{Mod}_\E(\mathrm{Sk}(P))$
\end{proposition}

% ---------------------------------------------------------------------------
%: ----------------------- end of thesis sub-document ------------------------
% ---------------------------------------------------------------------------


% this file is called up by thesis.tex
% content in this file will be fed into the main document

%: ----------------------- name of chapter  -------------------------
\chapter{Background: ``Homotopical'' higher categories}

%: ----------------------- contents from here ------------------------


\section{Algebraic Model Structures}

\para A natural weak factorisation system is\;dots

\para 

\section{Quasi-categories}




% ---------------------------------------------------------------------------
%: ----------------------- end of thesis sub-document ------------------------
% ---------------------------------------------------------------------------


%% Backmatter

\backmatter


%% Bibliography Info

% \nocite{CRM:ACSMHC}
% \bibliographystyle{amsplain}
% \bibliography{disbib}

%% Index Info

%% END DOCUMENT

\end{document}