\documentclass{minimal}
\begin{document}

Things to put on the back burner for now and come back to after the thesis:

\begin{itemize}[---]

\item pattern-matching:
\emph{read} Coquand paper \& McBride thesis; try to understand if/why K is really necessary!  I don't believe it is... but that's accepted wisdom in type theory, so it's probably right, but it should be fun attacking it and seeing why I fail!

\item ``living with incoherence'':
work a bit with type theory in ``world without UIP''.  Work out: are extra coherence conditions necessary?  Can we do much category theory without them?  Two secnarios, I guess:

\begin{enumerate}
\item we need coherence to do anything useful at all!  this would SUCK: since then we'd appear to need serious high technology (i.e. a formalisation of something as high as operads!) to even get started doing maths in the world without UIP; which would be rather bad for the dream!

\item we can do (eg) 1-category theory without coherence beyond the 1-level.  This would be interesting, and just a shade iconoclastic for homotopy/higher-category theory...
\end{enumerate}

In either case: write to MS re ``homotopy-monoid that \emph{can't} be made coherent'' by changing its associator/unitor.

\item omega-op categories!  with Nathan??

\item linear DTT via double categories/globular monoidal categories/similar (for appropriate generalisation of pullbacks)?

\item Some exercises in Garner weak maps: given two contractions on the same operad $P$, they induce different functors $P$-$\mathibf{Alg}$ $\rightarrow$ $L$-$\mathibf{Alg}$.  The two realisations of a $P$-algebra should be weakly equivalent; at least (this is the exercise!) they should be connected by maps.

\end{document}