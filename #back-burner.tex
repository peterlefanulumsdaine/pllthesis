Things to put on the back burner for now and come back to after the thesis:

--- pattern-matching:
_read_ Coquand paper & McBride thesis; try to understand if/why K is really necessary!  I don't believe it is... but that's accepted wisdom in type theory, so it's probably right, but it should be fun attacking it and seeing why I fail!

--- ``living with incoherence'':
work a bit with type theory in ``world without UIP''.  Work out: are extra coherence conditions necessary?  Can we do much category theory without them?  Two secnarios:

1. we need coherence to do anything useful at all!  this would SUCK: since then we'd appear to need serious high technology (i.e. a formalisation of something as high as operads!) to even get started doing maths in the world without UIP; which would be rather bad for the dream!

2. we can do (eg) 1-category theory without coherence beyond the 1-level.  This would be interesting, and just a shade iconoclastic for homotopy/higher-category theory...

In either case: write to MS re "homotopy-monoid that _can't_ be made coherent" by changing its associator/unitor.
