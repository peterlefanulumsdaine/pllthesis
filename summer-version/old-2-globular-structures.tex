% this file is called up by thesis.tex
% content in this file will be fed into the main document

%: ----------------------- name of chapter  -------------------------
\chapter{Globular structures from type theory}

%: ----------------------- contents from here ------------------------


\section{The fundamental weak omega-groupoid of a type}


TODO: either here or in introduction, give historical background: the move from seeing it as a transitive relation, to a groupoid-up-to-prop-eq, to (``the parallels with the fundamental group of a space are inescapable'') a weak $\omega$-groupoid.

\subsection*{Hacking} \label{sec:hacking} To understand how the $\omega$-groupoid structure arises, it is instructive to make a start on its construction by hand.

(A few notes on derivations: read bottom-up; and obvious premises are omitted.  TODO: maybe put this earlier?)

The first term often derived for identity types is a witness for the transitivity of equality on $X$---which for us, of course, will give binary composition of 1-cells in $\Pi_\omega(X)$:
$$ x,y,z:X, u:\Id_X(x,y), v:\Id(y,z) \types c_X(v,u) : \Id(x,z)$$

How do we derive such a term?  We can just use $\Id$-elim:

TODO: give the derivation.

Of course, we could also have eliminated on $v$ first instead of $u$.  Alternatively, we didn't need to stop after one use of $\Id$-elim: we could have applied it to the second variable as well:

TODO: give this version too!

\begin{exercise}Derive terms witnessing one or more of the following:
\begin{enumerate}
\item ``symmetry of equality'', i.e.\ (weak) inverses of 1-cells:
$$x,y:X, u:\Id(x,y) \types ? : \Id(y,x)$$

\item \label{ex:assoc}associativity, for any one of the binary compositions terms above (or for a mixture of them):
\begin{multline*}x,y,z,w:X, t:\Id(x,y), u:\Id(y,z), v:\Id(z,w) \\ 
\types\ ? : \Id_{\Id_X} (\ c(c(v,u),t)\ ,\ c(v,c(u,t))\ )
\end{multline*}
(For bonus points, find a one-size-fits-all approach which works with any of the binary compositions operations.)

\item horizontal composition of 2-cells:
$$ x,y,z:X, u,u':\Id(x,y), p:\Id(u,u'), v,v':\Id(y,z), q:\Id(v,v')\ \types\ ? : ? $$
\end{enumerate}
\end{exercise}

\subsection*{Outline of the proof}
Having tackled a few examples, we can now generalise the tactics involved, giving the main points of the full proof:

\begin{enumerate}
\item We constructed our composition operations as maps in the syntactic category, from contexts built out of the identity types of $X$ back into those identity types themselves.  This suggests that these are operations in some kind of \emph{endomorphism operad in the syntactic category}.

\item We used no special information about the type $X$; in other words, we effectively were deriving composition operations just over a \emph{generic type}, which can then be implemented on any other type.  (On this point, \cite{garner-van-den-berg}\ take a different approach; see \PARA\ref{sec:gvdb-comparison} below.)

\item We now want to see that this composition operad is \emph{contractible}!  What does this mean?  The exercises above, particularly the last two, are typical instances of the liftings that contractibility requires: given two parallel composition operations, we have to find another connecting them one dimension up; that is, given two terms $\sigma,\tau$ we have to derive a term of idenity type $\Id(\sigma,\tau)$ between them.

\item How do we do that?  Our context is built out of identity types, so we \emph{repeatedly apply $\Id$-elim} and hope that our target type simplifies until we can inhabit it.  As in the first derivation above, one might sometimes stop early; but as Ex.\ref{ex:assoc} shows, the most generally powerful approach is to eliminate away \emph{all} higher variables in the context, until only a single variable $x:X$ remains; so it is enough to show that $\sigma$ and $\tau$ are propositionally equal once $x$ and its (higher) reflexivity terms $r(x), r(r(x)),\ldots$ have been plugged in for all the variables of $\sigma$ and $\tau$. 

\item In all the examples above, $\sigma$ and $\tau$ then computed down to higher reflexivity terms themselves, and so were definitionally equal.  In fact this turns out to be typical: \emph{the context $x:X$ is an initial object}, so $\sigma$ and $\tau$ must always simplify in this way, and we are done.
\end{enumerate}

\subsection*{Globular contexts, and their endomorphism operads}

For any cwa $\C$, the class of dependent projections is closed under pullbacks, so by \PARA\ref{para:fibspans}, we can form the monoidal globular category $\FibSpans(\C)$; and maps of cwa's preserve dependent projections and pullbacks of them, so we have a functor $\FibSpans : \CwA \to \MonGlobCat$.

A globular object of $\FibSpans(\C)$ is then just a globular object $\Gamma_\bullet :  \G \to \C$ in $\C$, in which all the maps $s,t$ are dependent projections:
$$\xymatrix{ \Gamma_0 & \ar@<0.5ex>@{->>}[l]^s \ar@<-0.5ex>@{->>}[l]_t \Gamma_1 & \ar@<0.5ex>@{->>}[l]^s \ar@<-0.5ex>@{->>}[l]_t \Gamma_2 & \ar@<0.5ex>@{->>}[l]^s \ar@<-0.5ex>@{->>}[l]_t \cdots}$$

Mention abuses of notation: $\End_\C(\A)$, $\End_\T(\A)$.

\subsection*{The theory of a generic type} TODO: Construct $\MLId[X]$; state universal property.

\begin{definition}Let $\MLId[X] \in \ThId$ be the extension of $\MLId$ by a single new type
$$\inferrule*[right=$X$-$\form$]{\ }{\diamond \types X\ \type}$$
and no further terms or axioms.
\end{definition}

$\MLId[X]$ is thus the free theory on a closed type (in $\ThId$): any closed type $A$ of a theory $\T$ uniquely determines a translation $F_A: \MLId[X] \to \T$ with $F_A(X) = A$.

By the adjunctions of \PARA\ref{para:??}, $\cl_\strat(\MLId[X])$ is thus the free object of $\CwAId_\strat$ with a distinguished closed type, and $\cl(\MLId[X])$ is free among CwA's $\C \in \CwAId$ with a distinguished context $\Gamma$ and type $A \in p(\Gamma)$.

In particular, for any such $\C$, $\diamond$, $A$, this gives a monoidal globular functor $\FibSpans(\cl(\MLId[X])) \to \FibSpans(\C)$, sending $\X$ to $\A$, and hence an operad map $\End_{\MLId[X]}(\X) \to \End_{\C}(\A)$.

So $\End_{\MLId[X]}(\X)$ acts on all other types $A \in p(\Gamma)$, naturally in $(\C,p,\pow)$ and $\Gamma$.  Syntactically, this is the fact that a compositin law on $X$ derived in $\MLId[X]$ could just as well have been derived for any ofther type $A$ in any theory with $\Id$-types.  In view of this, we will give the operad a less unwieldy name:

\begin{definition}We will write $\PML$ for the globular operad $\End_{\MLId[X]}(\X)$ of \emph{generically definable composition laws}.
\end{definition}

\para According to \PARA \ref{para:endo-operad}, a composition law in $\PML(\pi)$ consists of a natural transformation $(S_0,T_0,\ldots T_{n-1},R)$ between pylon diagrams as in Figure \ref{fig:endo-pylons}, where $\Gamma_n$ denotes the context
$$ \Gamma_n\ :=\ x_0,y_0:X, x_1,y_1: \Id(x_0,y_0), \ldots\ z:\Id(x_{n-1},y_{n-1})$$
and $\Gamma_{\widehat{\pi}}$ has\ldots [TODO: complete description!]

\begin{figure}[htbp] \label{fig:endo-pylons}
$$\bfig
%%%%%%%%%%%%%%%%%%%%
% right hand pylon %
%%%%%%%%%%%%%%%%%%%%
\node Gpi(250,0)[{\Gamma_{\widehat{\pi}}}]
\node Gspi(0,-250)[{\Gamma_{\widehat{s\pi}}}]
\node Gtpi(500,-400)[{\Gamma_{\widehat{t\pi}}}]
\node Gs2pi(0,-650)[{\Gamma_{\widehat{s^2\pi}}}]
\node Gt2pi(500,-800)[{\Gamma_{\widehat{t^2\pi}}}]
\node Gs1pi(0,-1150)[{\Gamma_{\widehat{s_1\pi}}}]
\node Gt1pi(500,-1300)[{\Gamma_{\widehat{t_1\pi}}}]
\node Gs0pi(0,-1550)[{\Gamma_{\widehat{s_0\pi}}}]
\node Gt0pi(500,-1700)[{\Gamma_{\widehat{t_0\pi}}}]
\arrow[Gpi`Gspi;]
\arrow[Gpi`Gtpi;]
\arrow[Gspi`Gs2pi;]
\arrow[Gspi`Gt2pi;]
\arrow[Gtpi`Gs2pi;]
\arrow[Gtpi`Gt2pi;]
\arrow/@{}|<>(0.42)\vdots/[Gs2pi`Gs1pi;]
\arrow/@{}|<>(0.42)\vdots/[Gt2pi`Gt1pi;]
\arrow[Gs1pi`Gs0pi;]
\arrow[Gs1pi`Gt0pi;]
\arrow[Gt1pi`Gs0pi;]
\arrow[Gt1pi`Gt0pi;]
%%%%%%%%%%%%%%%%%%%
% left hand pylon %
%%%%%%%%%%%%%%%%%%%
\node Gn(1750,0)[\Gamma_{n}]
\node Gn1l(1500,-250)[\Gamma_{n-1}]
\node Gn1r(2000,-400)[\Gamma_{n-1}]
\node Gn2l(1500,-650)[\Gamma_{n-2}]
\node fakeGn2l(450,-650)[]
\node Gn2r(2000,-800)[\Gamma_{n-2}]
\node G1l(1500,-1150)[\Gamma_{1}]
\node G1r(2000,-1300)[\Gamma_{1}]
\node G0l(1500,-1550)[\Gamma_{0}]
\node G0r(2000,-1700)[\Gamma_{0}]
\arrow[Gn`Gn1l;]
\arrow[Gn`Gn1r;]
\arrow/@{>}|<>(0.316)\hole/[Gn1l`Gn2l;]
\arrow/@{>}|!{(500,-400);(2000,-400)}\hole/[Gn1l`Gn2r;]
\arrow[Gn1r`Gn2l;]
\arrow[Gn1r`Gn2r;]
\arrow/@{}|<>(0.42)\vdots/[Gn2l`G1l;]
\arrow/@{}|<>(0.42)\vdots/[Gn2r`G1r;]
\arrow/@{>}|<>(0.316)\hole/[G1l`G0l;]
\arrow/@{>}|!{(500,-1300);(2000,-1300)}\hole/[G1l`G0r;]
\arrow[G1r`G0l;]
\arrow[G1r`G0r;]

%%%%%%%%%%%%%%%%%%%%
% connecting wires %
%%%%%%%%%%%%%%%%%%%%
\arrow/.>/[Gpi`Gn;R]
\arrow/@{>}|!{(250,0);(500,-400)}\hole^(.67){S_{n-1}}/[Gspi`Gn1l;]
\arrow/@{>}^(.34){T_{n-1}}/[Gtpi`Gn1r;]
\arrow/@{>}|<>(.19)\hole|!{(500,-800);(500,-400)}\hole^(.67){S_{n-2}}/[Gs2pi`Gn2l;]
\arrow/@{>}^(.34){T_{n-2}}/[Gt2pi`Gn2r;]
\arrow/@{>}^(.67){S_1}/[Gs1pi`G1l;]
\arrow/@{>}^(.34){T_1}/[Gt1pi`G1r;]
\arrow/@{>}|<>(.19)\hole|!{(500,-1700);(500,-1300)}\hole^(.67){S_0}/[Gs0pi`G0l;]
\arrow/@{>}^(.34){T_0}/[Gt0pi`G0r;]
\efig$$
\end{figure}


\begin{proposition} \label{prop:x-initial-in-g0}
$X$ is an initial object in $\globe{0}$.
\end{proposition}

\begin{proof}
By the universal property of $\globe{0}$, there is a unique translation $F_{1_X}$ from $\globe{0}$ to the coslice $X/\globe{0}$ sending $X$ to $1_X: X \to X$.  Again by the universal property, the composite $\globe{0} \to X/\globe{0} \to \globe{0}$ is equal to $1_{\globe{0}}$, since it fixes $X$.

Thus the values of $F_{1_X}$ on objects give for each $\Gamma \in \globe{0}$ a natural map $\r^\Gamma := F_{1_X}(\Gamma) \colon X \to \Gamma$, with $\r^X = 1_X$.  From this, we see that $X$ is an initial object by considering the naturality square
$$\xymatrix{ X \ar[r]^{\r^X = 1_X} \ar[d]_{1_X} & X \ar[d]^{f} \\
X \ar[r]^{\r^\Gamma} & \Gamma }$$
for any map $f \colon X \to \Gamma$.
\end{proof}

Syntactically, deducing $F_{1_X}$ from the universal property of $\globe{o}$ corresponds to inductively constructing, for each context $\Gamma$, a context morphism
$$ x:X \types \r^\Gamma(x) : \Gamma $$
such that every other context morphism preserves the point:
$$ \inferrule{\y : \Gamma\ \types\ \f(\y) : \Delta}{x:X\ \types\ \f(\r^\Gamma(x)) = \r^\Delta(x) : \Delta}$$
(See \cite{lumsdaine:tlca-journal} for a full presentation of this syntactic approach.)

Concretely, since all contexts of $\globe{0}$ are constructed from $X$ and identity types over it, $\r^\Gamma(x)$ is built by plugging $x$ in for each variable of type $X$ in $\Gamma$, and reflexivity terms for all variables of $\Id$-types.

% goal: get proof finished by the start of the Rondo-Burleske!
% Goal succeeded!  :-)

\begin{corollary}The operad $\PML$ is contractible.
\end{corollary}
\begin{proof}By \PARA \ref{para:contractibility-as-square}, it suffices to show that for any pasting diagram $\pi$ and map $\Gamma_{\widehat{\pi}} \to \Gamma_{\delta(n)}$, we can complete the diagram to a triangle
$$\xymatrix{ & \Gamma_{\yon(n)} \ar@{->>}[d] \\
\Gamma_{\widehat{\pi}} \ar[r] \ar@{.>}[ur] & \Gamma_{\delta(n)} }$$

But by the initiality of $X$, the diagram can always be completed to a square 
$$\xymatrix{ X \ar[d] \ar[r] & \Gamma_{\yon(n)} \ar@{->>}[d] \\
\Gamma_{\widehat{\pi}} \ar[r] \ar@{.>}[ur] & \Gamma_{\delta(n)} }$$
and since $\r^{\Gamma_{\widehat{\pi}}}$ is an $I$-map, we can find a diagonal filler for this square, completing the original triangle as desired.
\end{proof}

Interpreted syntactically, this is precisely the process described in \PARA \ref{para:PML-outline}.  The ``problem'' we are given, lifting a map into $\Gamma_{\delta(n)}$ to one into $\Gamma_{\yon(n)}$, is just the problem of deriving a term $\y : \Gamma_{\widehat{\pi}}\ \types\ ? : \Id(S_{n-1}(\y),T_{n-1}(\y))$.  The fact that $\r^{\Gamma_{\widehat{\pi}}}$ is an $I$-map tells us that (by repeated application of $\Id$-$\elim$ and structural isomorphisms) it suffices to derive the term in the case where $\r^{\Gamma_{\widehat{\pi}}}(x)$ has been substituted for $\y$.  Finally, the initiality of $X$ ensures that after this substitution, (the last component of) the context morphism $\r^{\Gamma_{\yon(n)}}$ fits the bill---specifically, $S_{n-1}$ and $T_{n-1}$ must compute down to be definitionally equal, and so we can use a reflexivity term.

\subsection*{} Compare: \cite{garner-van-den-berg}, with tailor-made operad from $A/\C$.

\subsection*{} Add: groupoidality!






\section{The classifying weak $\omega$-category of a theory}

\subsection*{} Overview: what are we trying to build?

\subsection*{Representability: the type theoretic globes} More abstract view: set  up Kan situation!  Type-theoretic globes.

Make analogy with $\Top$.  How much detail to go into?  Separate subsection*?  Compare: \cite[\SEC 9]{batanin:natural-environment}, 

\begin{definition}
Let $\globe{n}$, the \emph{type-theoretic $n$-globe}, be the theory defined by the following axioms:
$$\begin{array}{l c}
n = 0: & \inferrule{\ }{\diamond \types X\ \type} \\
n > 0 : & \inferrule{\ }{\diamond \types X^s\ \type} \qquad \inferrule{\ }{\diamond \types X^t\ \type} \\
n = 1: & \inferrule{\ }{x: X_s \types p_1(x) : X_t} \\
n > 1:  & \inferrule{\ }{x:X_s \types p_1^s (x) : X_t} \qquad \inferrule{\ }{x:X_s \types p_1^t (x) : X_t} \\
\quad (1 < i < n) & \inferrule{\ }{x:X_s \types p_i^s(x) : \Id(p_{i-1}^s,p_{i-1}^t)} \qquad \inferrule{\ }{x:X_s \types p_i^t(x) : \Id(p_{i-1}^s,p_{i-1}^t)} \\
& \inferrule{\ }{x:X_s \types p_n(x) : \Id(p_{n-1}^s,p_{n-1}^t)} \\
\end{array}$$
\end{definition}

Then $\cl^-_\omega(\TT) := \ThId(\globes,\TT)$ is the underlying globular set of our desired classifying weak $\omega$-category of types of $\TT$.

Indeed, since $\ThId$ is co-complete, Kan extension of $\globes$ along the Yoneda embedding gives a left adjoint $\TT [-] \dashv \cl^-_\omega$.  Analogously to the geometric realisation $|X|$ of a globular (or simplicial) set, we call $\TT[X]$ the \emph{type-theoretic realisation} of $X$, a theory with axioms implementing 0-cells of $X$ as closed types, 1-cells as open terms mapping between these, and higher cells as terms of $\Id$-type.  (This follows from the usual colimit formula for Kan extensions\cite[??]{maclane:cwm} and the description of colimits in \ref{prop:colims-in-thid} above.)  In particular, $\TT[y(n)] = \globe{n}$.

$$\quad \xymatrix{ \GSets \ar@/_/[rrr]_{\TT [-]\, :=\, \Lan_\yon \globes \qquad \ \, } \ar@{}[rrr]|\top & & & \ThId \ar@/_/[lll]_{\cl^-_\omega\ :=\ \ThId(\globes,-)} \\ \G \ar@{^{ (}->}[u]^\yon \ar@/_/[urrr]_{\globes} }
$$

% spent fucking hours trying to get P-Alg into this diagram too,
% while staying good and readable; eventually gave up...

\para So, by \ref{para:homming-out-of-algebras}, to put a natural $P$-algebra structure on $\cl^-_\omega$ is equivalent to putting a co-$P$-algebra structure on $\globes$.  In particular, its coendomorphism operad $\End_{(\ThId)^\op}(\globes)$ acts naturally on $\cl^-\omega$; in other words, we can lift if to a functor into $\End_{(\ThId)^\op}(\globes)$:
$$\xymatrix{ & & \Alg{\left[\End_{(\ThId)^\op}\globes \right]} \ar[d]^U \\ \ThId \ar[urr]^{\cl^-\omega} \ar[rr]^{\ThId(\globes,-)} & & \GSets }$$

With this in mind, we investigate this operad.  As usual, an operad element of dimension $n$ consists of a natural transformation $(S_0,T_0,\ldots,S_{n-1},T_{n-1},R)$ between pylon diagrams, but this time going from the pylon representing the $n$-globe to one representing a pasting diagram as in \ref{fig:coendo-op} below.

Before proceeding to investigate contractibility in general, however, it is helpful to pause for a few observations on the low dimensions of this operad.

TODO: points to make: by arguments like before, or by normalisation, $X_0$ is again initial in the theory of any one-dimensional pasting diagram (i.e.\ path), and so the only elements of the operad in dimensions 0 and 1 are the ones you'd expect: the identity in dim 0, and a unique $k$-ary composition for each $k$ in dim 1.  But, more importantly, we also get that any map out of a globe (or any colimit of globes) into any $\TT[\hat \pi]$ (for 1-dimensional $\pi$) is determined by what it does to in dimensions 0 and 1.


\para[Contractibility]  Figure \ref{fig:coendo-op}\ depicts the lifting condition required for contractibility of $\End_{(\ThId)^\op}(\globes)$: this amounts to showing that whenever $(S_0,T_0,\ldots,S_{n-1},T_{n-1}$ are given making the lower dimensions commute, some $R$ can be found to complete the diagram.

\begin{figure}[htbp] \label{fig:coendo-op}
$$\bfig
%%%%%%%%%%%%%%%%%%%
% left hand pylon %
%%%%%%%%%%%%%%%%%%%
\node gn(250,0)[\globe{n}]
\node gn1l(0,-250)[\globe{n-1}]
\node gn1r(500,-400)[\globe{n-1}]
\node gn2l(0,-650)[\globe{n-2}]
\node fakegn2l(450,-650)[]
\node gn2r(500,-800)[\globe{n-2}]
\node g1l(0,-1150)[\globe{1}]
\node g1r(500,-1300)[\globe{1}]
\node g0l(0,-1550)[\globe{0}]
\node g0r(500,-1700)[\globe{0}]
\arrow[gn1l`gn;]
\arrow[gn1r`gn;]
\arrow[gn2l`gn1l;]
\arrow[gn2r`gn1l;]
\arrow[gn2l`gn1r;]
\arrow[gn2r`gn1r;]
\arrow/@{}|<>(0.58)\vdots/[g1l`gn2l;]
\arrow/@{}|<>(0.58)\vdots/[g1r`gn2r;]
\arrow[g0l`g1l;]
\arrow[g0r`g1l;]
\arrow[g0l`g1r;]
\arrow[g0r`g1r;]
%%%%%%%%%%%%%%%%%%%%
% right hand pylon %
%%%%%%%%%%%%%%%%%%%%
\node Tpi(1750,0)[{\TT[\widehat{\pi}]}]
\node Tspi(1500,-250)[{\TT[\widehat{s\pi}]}]
\node Ttpi(2000,-400)[{\TT[\widehat{t\pi}]}]
\node Ts2pi(1500,-650)[{\TT[\widehat{s^2\pi}]}]
\node Tt2pi(2000,-800)[{\TT[\widehat{t^2\pi}]}]
\node Ts1pi(1500,-1150)[{\TT[\widehat{s_1\pi}]}]
\node Tt1pi(2000,-1300)[{\TT[\widehat{t_1\pi}]}]
\node Ts0pi(1500,-1550)[{\TT[\widehat{s_0\pi}]}]
\node Tt0pi(2000,-1700)[{\TT[\widehat{t_0\pi}]}]
\arrow[Tspi`Tpi;]
\arrow[Ttpi`Tpi;]
\arrow/@{>}|!{(500,-400);(2000,-400)}\hole/[Ts2pi`Tspi;]
\arrow/@{>}|!{(500,-400);(2000,-400)}\hole/[Tt2pi`Tspi;]
\arrow[Ts2pi`Ttpi;]
\arrow[Tt2pi`Ttpi;]
\arrow/@{}|<>(0.58)\vdots/[Ts1pi`Ts2pi;]
\arrow/@{}|<>(0.58)\vdots/[Tt1pi`Tt2pi;]
\arrow/@{>}|!{(500,-1300);(2000,-1300)}\hole/[Ts0pi`Ts1pi;]
\arrow/@{>}|!{(500,-1300);(2000,-1300)}\hole/[Tt0pi`Ts1pi;]
\arrow[Ts0pi`Tt1pi;]
\arrow[Tt0pi`Tt1pi;]
%%%%%%%%%%%%%%%%%%%%
% connecting wires %
%%%%%%%%%%%%%%%%%%%%
\arrow/.>/[gn`Tpi;R]
\arrow/@{>}|!{(250,0);(500,-400)}\hole/[gn1l`Tspi;S_{n-1}]
\arrow[gn1r`Ttpi;T_{n-1}]
\arrow/@{>}|<>(.19)\hole|!{(500,-800);(500,-400)}\hole/[gn2l`Ts2pi;S_{n-2}]
\arrow[gn2r`Tt2pi;T_{n-2}]
\arrow[g1l`Ts1pi;S_1]
\arrow[g1r`Tt1pi;T_1]
\arrow/@{>}|<>(.21)\hole|!{(500,-1700);(500,-1300)}\hole/[g0l`Ts0pi;S_0]
\arrow[g0r`Tt0pi;T_0]
\efig$$
\end{figure}

We can restate this as an extension problem:
$$\xymatrix{ \TT[\delta (n)] \ar@{^{ (}->}[d] \ar[r]^{(\vec S, \vec T)} & \TT[\delta \widehat{\pi}] \ar@{^{ (}->}[r] & \TT[\widehat{\pi}] \\
\globe{n} = \TT[\yon(n)] \ar@{.>}[urr]^R}$$
By the observations above (TODO: work out where to put them!) this ``corner'' completes to a commutative square, so we in fact have a square-filling problem:
$$\xymatrix{ \TT[\delta (n)] \ar@{^{ (}->}[d] \ar[r]^{(\vec S, \vec T)} & \TT[\delta \widehat{\pi}] \ar@{^{ (}->}[r] & \TT[\widehat{\pi}] \ar[d] \\
\globe{n} = \TT[\yon(n)] \ar@{.>}[urr]^R \ar@{->>}[r] & \globe{1} \ar[r] & \TT[\widehat{s_1 \pi} = \widehat{t_1 \pi}]}$$

(TODO: rephrase this to avoid duplicating the diagram, while still keeping the train of thought clear!)

Rephrase \emph{syntactically}: completing the translation is just showing that a certain identity type is inhabited.   This motivates:

\begin{conjecture}In the theories $\TT[\pi]/X_0$, any two closed terms of the same type are propositionally equal.
\end{conjecture}

If this holds, then the operad $\End_{(\ThId)^\op}(\globes)$ is contractible, and hence $\cl_\omega$ carries a natural weak $\omega$-category structure as desired.  In the following sections, we discuss various approaches to proving this conjecture: by direct syntactic attack on the theories $\TT[\pi]$, or by working more globally, using the structures on $(\ThId)^\op$ discussed in \ref{sec:thid-op}.  [TODO: put discussion of the structures on ${\ThId}^\op$ in later section of this chapter, or in previous chapter?  Big-picture-wise, belongs better there; pragmatically, more motivated here.]  First, however, we pause to discuss a variant of $\cl_\omega$ using $\Pi$-types, which is more easibly seen to be contractible.

\subsection*{Approximation by $\Pi$-types}

\para
Work in $\ThIdPi$.  Define alternate globes using closed terms of $\Pi$-type and of their $\Id$-types (so $\Id_\Pi$ not $\Pi (\Id)$!) instead of open terms.

Show this is \emph{is} contractble: since theories are now just extensions by $\Id$-types, induction works, and (by normalisation) in the base case (i.e.\ theories of 1-dimensional




\subsection*{A NWFS on $\ThId$} Introduce contractible maps; give nwfs by Garner small-object argument!

\subsection*{} Conjecture various forms of $\bar{J}$

\subsection*{} Theorem: if any of above conjectures holds, then have $\cl^-_\omega$.

\begin{para} Corollary: have $\cl_\omega$ by using $(-)^\cxt$!
\end{para}


% ---------------------------------------------------------------------------
%: ----------------------- end of thesis sub-document ------------------------
% ---------------------------------------------------------------------------
