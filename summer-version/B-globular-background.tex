% this file is called up by thesis.tex
% content in this file will be fed into the main document

%: ----------------------- name of chapter  -------------------------
\chapter{Background: globular higher category theory}

%: ----------------------- contents from here ------------------------

\section{Strict higher categories}

\para Define $\strnCat$ and $\strnCat[\omega]$ by enrichment. 


\para Analyse $T$: pasting diagrams, Batanin trees, familial representability.

Fact: Cartesian, as can see from familial rep'bility\cite{street:petit-topos},\cite{carboni-johnstone}.

\section{Globular operads: Leinster presentation}

\definition Contractible map of glob sets.

\definition Leinster operad, as cartesian map of cartesian monads.

\para Equivalent: local presentation of Leinster operad: object over $T1$, with appropriate structure.

\definition 
\para Contractibility.



\section{Globular operads: Batanin/Weber presentation}

\section{Internal algebras for operads}  (This should probably be split up and moved: the globular part moved into the ``Globular background'' appendix, and then the construction of a monoidal globular category from a comprehension category folded into the construction of the fundamental things.)

\subsection*{Monoidal globular categories} Recall definition from Batanin. 

Examples: $\Spans$; $\Spans(\C)$, for $\C$ with pullbacks; $\FibSpans(\C)$, for $\C$ with a distinguished class of ``fibrations'', closed under composition and pullbacks.

\para[Pasting diagrams]

Example of a mon glob cat: $T1$.  Give representation as Batanin trees; also as free monoids.

Notation note: I use $\delta(n)$ for the boundary of $\yon(n)$, which is standard, but also $\delta \widehat{\pi}$ for the boundary of the $\widehat{\pi}$ which is less standard: Batanin, Street (TODO: refs?) and others use $\delta \pi := s \pi = t \pi$ (since these are always equal), whereas I always distinguish these (yes, they're equal as pasting diagrams, but in use they're typically `intensionally' distinct) and my $\delta \widehat{\pi}$ is the pushout of $\widehat{s\pi}$ and $\widehat{t\pi}$ along their common boundary. 

Give realisation as globular sets, by (a) iterated pushout, (b) colimit.

Define ``one-leaf prunings''.  

In free monoid terms, defined inductively: removing an (extremal?) $()$ at some depth.  In tree terms: removing a leaf (an endpoint leaf?) at some height!

(``Any leaf'' is more natural---then these really do correspond to places one can apply $\Id$-elim.  On the other hand, ``extremal leaf'' makes pushout/pullback decompositions/lemmas simpler.  Ah!  Idea: maybe just define $\pi^-$, and then mention that could use more general one-leaf prunings.)

Note: every non-point pasting diagram has some; eg in free monoid terms: 
$$((),\pi_1,\ldots,\pi_{r-1})^- = (\pi_1,\ldots,\pi_{r-1})$$
$$(\pi_0,\ldots,\pi_{r-1})^- = (\pi_0^-,\ldots,\pi_{r-1})$$
is a nice straightforward one, corresponding (in tree terms) to following the leftmost branch to its end, and removing the leaf reached.

Show how realisation is pushout:

\newbox\potlbox
\setbox\potlbox=\hbox{\xy 
(-340,170)*{}="tlleft"; % "left": 0-source
(340,170)*{}="trleft";
(-340,-170)*{}="blleft";
(340,-170)*{}="brleft";
"tlleft";"trleft" **\dir{.};
"tlleft";"blleft" **\dir{.};
"trleft";"brleft" **\dir{.};
"blleft";"brleft" **\dir{.};
(-300,0)*+{\cdot}="a";
(0,0)*+{\cdot}="b";
{\ar "a";"b"};
\endxy}
\def\potl{\copy\potlbox}

\newbox\potrbox
\setbox\potrbox=\hbox{\xy 
(-340,170)*{}="tlleft"; % "left": 0-source
(340,170)*{}="trleft";
(-340,-170)*{}="blleft";
(340,-170)*{}="brleft";
"tlleft";"trleft" **\dir{.};
"tlleft";"blleft" **\dir{.};
"trleft";"brleft" **\dir{.};
"blleft";"brleft" **\dir{.};
(-300,0)*+{\cdot}="a";
(0,0)*+{\cdot}="b";
{\ar@/^1.2pc/ "a";"b"};
{\ar "a";"b"};
{\ar@{=>} (-150,120)*{};(-150,25)*{}} ;
\endxy}
\def\potr{\copy\potrbox}

\newbox\poblbox
\setbox\poblbox=\hbox{\xy 
(-340,170)*{}="tlleft"; % "left": 0-source
(340,170)*{}="trleft";
(-340,-170)*{}="blleft";
(340,-170)*{}="brleft";
"tlleft";"trleft" **\dir{.};
"tlleft";"blleft" **\dir{.};
"trleft";"brleft" **\dir{.};
"blleft";"brleft" **\dir{.};
(-300,0)*+{\cdot}="a";
(0,0)*+{\cdot}="b";
(300,0)*+{\cdot}="c";
{\ar "a";"b"};
{\ar@/_1.2pc/ "a";"b"};
{\ar@{=>} (-150,-20)*{};(-150,-115)*{}} ;
{\ar@/^0.7pc/ "b";"c"};
{\ar@/_0.7pc/ "b";"c"};
{\ar@{=>} (150,60)*{};(150,-60)*{}} ;
\endxy}
\def\pobl{\copy\poblbox}

\newbox\pobrbox
\setbox\pobrbox=\hbox{\xy 
(-340,170)*{}="tlleft"; % "left": 0-source
(340,170)*{}="trleft";
(-340,-170)*{}="blleft";
(340,-170)*{}="brleft";
"tlleft";"trleft" **\dir{.};
"tlleft";"blleft" **\dir{.};
"trleft";"brleft" **\dir{.};
"blleft";"brleft" **\dir{.};
(-300,0)*+{\cdot}="a";
(0,0)*+{\cdot}="b";
(300,0)*+{\cdot}="c";
{\ar@/^1.2pc/ "a";"b"};
{\ar "a";"b"};
{\ar@/_1.2pc/ "a";"b"};
{\ar@{=>} (-150,120)*{};(-150,25)*{}} ;
{\ar@{=>} (-150,-20)*{};(-150,-115)*{}} ;
{\ar@/^0.7pc/ "b";"c"};
{\ar@/_0.7pc/ "b";"c"};
{\ar@{=>} (150,60)*{};(150,-60)*{}} ;
\endxy}
\def\pobr{\copy\pobrbox}

$$\bfig \square[y(i)`y(i-1)`\widehat{\pi^-}`\widehat{\pi};```]
\place(400,100)[\po]
\efig
\quad \quad \quad
\bfig \Square[\potl`\potr`\pobl`\pobr;```]
\place(625,175)[\po]
\efig
$$

\subsection*{Operads in monoidal categories}

Define: globular object.  Define: operad!  Endomorphism operad!

\subsection*{Endomorphism operads, explicitly}

In $\FibSpans(\C)$, a globular object $\X$ is\ldots

Then for $\pi \in T1_n$, the peak of the $n$-span $(\X)\pi$ is the object
$$X^\pi := \lim_{c \in \int\! \hat{\pi}} X_{\dim c}$$

This somewhat cryptic formula is perhaps best illuminated by a couple of examples: if $\pi = (\xymatrix{ \bullet \rtwocell & \bullet \rtwocell & \bullet})$, then
\begin{eqnarray*} X_\pi & := & \lim \left( 
\bfig
\node X0l(0,0)[X_0]
\node X0m(700,0)[X_0]
\node X0r(1400,0)[X_0]
\node X1tl(350,300)[X_1]
\node X1tr(1050,300)[X_1]
\node X1bl(350,-300)[X_1]
\node X1br(1050,-300)[X_1]
\node X2l(350,0)[X_2]
\node X2r(1050,0)[X_2]
\arrow[X1tl`X0l;s]
\arrow[X1tl`X0m;t]
\arrow[X1tr`X0m;s]
\arrow[X1tr`X0r;t]
\arrow[X2l`X1tl;s]
\arrow[X2r`X1tr;s]
\arrow[X2l`X1bl;t]
\arrow[X2r`X1br;t]
\arrow[X1bl`X0l;s]
\arrow[X1bl`X0m;t]
\arrow[X1br`X0m;s]
\arrow[X1br`X0r;t]
\efig
\right) \\
& \iso & X_2 \times_{X_0} X_2,
\end{eqnarray*}
giving the object of 0-composable pairs of 2-cells in $X$.  Similarly, if $\pi$ is the basic $n$-cell, then $X_\pi = X_n$.

\para \label{para:homming-out-of-algebras} Describe the gloulbar hom-functor between $n$-objects \cite[3.6]{batanin:natural-environment} and hence into and out of globular objects.  Note: in the case of $\Spans$, this is monoidal. (Check: Batanin doesn't show this anywhere, does he? although he does use it!  Or, better, is it in Weber somewhere, or easily deducible?)

Really, would like a Yoneda lemma: for a globular object $\A$ of a cat (with fibrations) $\C$, $\End_{\C}(\A) \iso \End_{[\C,\Sets]}(\yon \A)$.

\example{Fundamental $\omega$-groupoid of a space}


\subsection*{Connection with Leinster approach}

Recall from Weber paper.


\subsection*{Enriched point of view}

TODO: the below is from a different earlier approach; fold this into the current presentation!

\begin{definition}[Endomorphism operads] For $\E$ any category, $\X$ any globular operad in $\E$, we write $\End_\E(\X)$ for the operad (construction\ldots\ either by monoidal globular categories, or by ``representable'' style of my previous paper; latter is slicker here, but seems very difficult for showing the functoriality).  More generally, really want the $\Coll$-enriched category structure on (appropriate subcategory of) $[\G,\E]$.
\end{definition}



\proposition If $\E$ has enough limits, then for any pasting diagram $\pi$, $$\End_\E(\X)(\pi) \iso [\G/n,\E](X \cotensor \hat{\pi},X \cotensor \yon(n))$$
where the right hand side consists of ``pylon diagrams'':
$$\textrm{draw the diagram here.}$$

\begin{proof} Straightforward (in either construction of $\End$).
\end{proof}

\proposition $\End_\E(\X)$ is functorial in $\E$: a functor $F : \E \to \F$ preserving appropriate limits induces a map $\End_\E(\X) \to \End_\F(F\X)$.

\begin{proof} Straightforward in the ``monoidal globular categories'' approach.  Can't currently see how to do it in the ``representable'' approach!?
\end{proof}

\definition An \emph{algebra} for an operad $P$ on an object $\X$ of $\E$ is an operad map $\xi \colon P \to \End_\E (\X)$ (the \emph{action} of $P$ on $\X$); a map of $P$-algebras is a globular map $\f \colon \X \to \Y$ commuting with the action maps, i.e.\ such that the square 
$$\xymatrix{P \ar[r]^{\xi} \ar[d]^{\upsilon} & [\X,\X] \ar[d]^{f \cdot\ } \\ [\Y,\Y] \ar[r]^{\ \cdot f} & [\X,\Y]}$$
commutes.

In enriched terms, the resulting category $\IntAlg{P}{\E}$ is $\enrCat(\Coll)(P,[\G,\E])$.

% but it's alright now
% I learned my lesson, yeah
% can't please everyone, so...
% Caffé Nero, Davygate, 14.vii

We can also consider $P$ as defining a certain finite-limit sketch $\mathrm{Sk}(P)$, and compare the internal algebras defined here with models of this sketch in $\E$.

\begin{proposition}$\IntAlg{P}{\E} \equiv \mathbf{CmpSpan}\mathbf{Mod}_\E(\mathrm{Sk}(P))$
\end{proposition}

% ---------------------------------------------------------------------------
%: ----------------------- end of thesis sub-document ------------------------
% ---------------------------------------------------------------------------
