% this file is called up by thesis.tex
% content in this file will be fed into the main document

\chapter{Algebraic structures from dependent type theory}


% ----------------------- contents from here ------------------------

Possibly fold this chapter into the next??

\section{A category of Type Theories}

\subsection*{}Give brief, semi-formal outline of the type theory, referring to Appendix

\subsection*{}In fact, we will not work directly with the category of type theories, but with an equivalent category of algebraic models.
but 
The theory of such models is attractive, but suffers from rather an embarassment of riches: for instance \cite{jacobs:comprehension-categories}, \cite{pitts:categorical-logic}, \cite{hofmann:syntax-and-semantics} and \cite{dybjer:internal-type-theory} each [TODO: chronologicise these] define slightly different notions of categorical models (several after the unpublished \cite{cartmell:thesis}), all equivalent (in some sense) to each other and to syntactically presented dependent type theories.   (TODO: also look up ``categories with families''.)

Of course, these notions each have advantages and disadvantages: some are more elementary to present; some are more categorically elegant; some are more easily adaptable to extensions of the type theory\ldots\  We thus take this opportunity to survey several of the various options, and the comparisons between them.

(TODO: reorganise/tidy all the below! 

OK: new outline!   (See notes of 23.vii for details.)
\begin{enumerate}
\item sketch definition of the type theory, referring to appendix for full details
\item categories with attributes
\item (full split) comprehension categories
\item equivalence of these
\item based, stratified versions
\item adjunction between syntactic theories and stratified CwA's.
\item dependent contexts construction; slice construction.
\item connections between plain, based, stratified, accessible CwA's.
\item nice big figure!
\end{enumerate}

\begin{figure}[htbp]
$$\bfig
\node CC(0,0)[\mathbf{CompCat}]
\node CCd(0,-400)[\mathbf{CompCat}_\diamond]
\node FSCC(600,0)[\FSCC]
\node FSCCd(600,-400)[\FSCC_\diamond]
\node FSCCstrat(600,-800)[\FSCC_\strat]
\node CwA(1200,0)[\CwA]
\node CwAd(1200,-400)[\CwA_\diamond]
\node CwAstrat(1200,-800)[\CwA_\strat]
\node SynThy(1800,-800)[\SynThy]
\node SynPres(1800,-1200)[\SynPres]
\arrow|l|/{@/_/}/[FSCCd`FSCC;]
\arrow|r|/{@/_/}/[FSCC`FSCCd;]
\arrow|m|//[FSCC`FSCCd;\dashv]
\arrow|l|/{@{^{ (}->}@/^/}/[FSCCstrat`FSCCd;]
\arrow|r|/{@/^/}/[FSCCd`FSCCstrat;]
\arrow|m|//[FSCCd`FSCCstrat;\dashv]
\arrow|m|//[FSCC`CwA;\equiv]
\arrow|m|//[FSCCd`CwAd;\equiv]
\arrow|m|//[FSCCstrat`CwAstrat;\equiv]
\arrow|l|/{@/_/}/[CwAd`CwA;]
\arrow|r|/{@/_/}/[CwA`CwAd;]
\arrow|m|//[CwA`CwAd;\dashv]
\arrow|l|/{@{^{ (}->}@/^/}/[CwAstrat`CwAd;]
\arrow|r|/{@/^/}/[CwAd`CwAstrat;]
\arrow|m|//[CwAd`CwAstrat;\dashv]
\arrow|m|//[CwAstrat`SynThy;\equiv]
\arrow|l|/{@/^/}/[SynPres`CwAstrat;]
\arrow|r|/{@/^/}/[CwAstrat`SynPres;]
\place(1500,-1000)[\vdash] 
% TODO: find how to use xyoption{rotate} to rotate this!
% or give up and use the `rotating' package.
% \arrow|m|//[SynPres`CwAstrat;\begin{turn}{45}$\dashv$\end{turn}]
\arrow|l|/{@/^/}/[SynPres`SynThy;]
\arrow|r|/{@/^/}/[SynThy`SynPres;]
\arrow|m|//[SynPres`SynThy;\dashv]
\efig$$
\end{figure}

\begin{definition} A \emph{split full comprehension category}\cite{jacobs:comprehension-categories} (\fscc) is a category $\C$ together with a split fibration $p : \E \to \C$ and a factorisation
$$p = \cod \cdot \pow : \E \to \C^\rightarrow \to \C$$such that (a) $\pow$ maps cartesian arrows to pullback squares, and (b) $\pow$ is full and faithful.  By abuse of notation, we will often refer to $\C$ itself as the comprehension category; the pair $(p,\pow)$ is called a \emph{comprehension structure} on $\E$.  

A (strict) map of \fsccs{} is just a functor $F: \C \to \C'$ and a map of fibrations $p \to p'$ over $F$, commuting with the factorisations $\pow$, $\pow'$.  A \emph{map of comprehension structures} on $\C$ is just the case $F = 1_C$.
\end{definition}

If $\C$ has all pullbacks, then condition (a) just says that $\pow$ is a map of fibrations.
 
In fact, for fixed $\C$, $\FSCS(\C)$ is (equivalent to) a presheaf category---specifically, to the slice of $\hat{\C}$ over the presheaf $\cod^{\mathrm{spl}}$ in which an element of $P(A)$ is a map $f : B \to A$ together with chosen pullbacks along all maps $j : A' \to A$.  (Explain how??)  comprehension categories will frequently be presented in this form.


\begin{example}For any type theory $\T$, its category of context $\C(\T)$ is a comprehension category, in which objects of $p(\Gamma)$ are types dependent over $\Gamma$, and $\pow$ sends a type $A \in p(\Gamma)$ to the dependent projection $\Gamma, x : A \to \Gamma$.
\end{example}

In fact, this construction is part of an equivalence between $\Th$ and a certain full co-reflexive subcategory of $\FSCC$: see Appendix \ref{app:??} for details.  In light of this, we will typically refer to the objects of any \fscc\ as contexts, and the objects of the fibration as dependent types.

\subsection*{Categories with attributes} 

From \cite{pitts:categorical-logic}, \cite{hofmann:syntax-and-semantics}, \cite{dybjer:internal-type-theory}, under various names.  Give definitions; (in notation to match that used above); give equivalence (honest 1-equivalence) with comprehension categories.

\subsection*{Stratification}

Define \emph{stratified} cwa, \& map of; note that it's (perhaps unexpectedly) a \emph{full subcategory}(!) of cwa's, closed under connected limits and colimits. (NB: this deinition seems to be new (though comparable to Cartmell); have I missed something?)

Proposition: there is an \emph{honest 1-equivalence} between stratified cwa's and dependent algebraic theories as laid out in appendix.

Note: since morphisms of DAT's are most easily defined by transfer from cwa's, the content of this proposition is really just that there are maps of \emph{objects} from stratified cwa's to dat's and back, with an isomorphism $\C \iso FG(\G)$.

Define \emph{reachable} cwa's after Pitts.

Theorem: there is a \emph{2-equivalence} between reachable cwa's and stratified cwa's.

Discuss significance of 1- versus 2-equivalence: latter gives equivalence of ``type theories'', which is fine on the categorical side, but not on the syntactic side: we care about the difference between \emph{isomorphism} and \emph{equivalence} there as syntactic presentations of theories are \emph{0-categorical objects} (Voevodsky slogan!).

Also point out: type theory with \emph{context and their maps and equalities as primitive judgements} (hence allowing equalities and maps between contexts of different lengths) should correspond to general cwa's.

\subsection*{Constructors}
This all extends to theories with type constructors.

Define: a lax map of comprehension structures (do I mean colax??)

Point out: the 2-category $\mathbf{\FSCS(\C)_\lax}$ has finite products.  (And these are probably better seen as 2-limits in $\FSCS(\C)$.)

\begin{definition}
An \emph{\fscs\ with units} on a category $\C$ is an \fscs\ $(p,\pow_0)$ together with a strict map of \fscss\ $1 : (1_B,\textit{id}) \to (p,\pow_0)$. 

An \emph{\fscs\ with binary products} on $\C$ is an \fscs\ $(p,\pow_0)$ with a strict map $ \times : (p times_B p, \pow_0 \times_B) \to (p,\pow_0)$. 
\end{definition}

Note: this implies a certain adjunction, so does corresponds to Jacobs' ``fscc with units''.  [Show this??]

To introduce the structure corresponding to identity types, we will need a little more terminology.

\begin{mydefinition}[Dependent contexts]
Given any comprehension category $(\C,p,\pow)$, we may construct another comprehension structure $(p^\cxt,\pow^\cxt)$ on $\C$:

An object of $p^\cxt[\Gamma]$ is a list $A_1,\ldots,A_n$, where $A_i \in p(\Gamma . A_1 \ldots . A_{i-1})$, for each $i \leq n$; context extension is defined by $\Gamma . (A_1 \ldots , A_n) = \Gamma . A_1 \ldots . A_n$, and pullback $f^* : p^\cxt(\Gamma) \to p^\cxt(\Delta)$ is similarly defined in terms of pullback in $p$.

This is the object part of an evident functor $(-)^\cxt$ on $\FSCS(\C)$.
\end{mydefinition}

The $(-)^\cxt$-construction has a natural type-theoretic interpretation: if $(\C,p,\pow)$ was obtained from a type theory, then for any $\Gamma$, $p^\cxt(\Gamma)$ is (isomorphic to) the category of dependent contexts over $\Gamma$ and dependent context morphisms between them.

Morevoer, $(-)^\cxt$ has a natural monad structure, and indeed is the ``free monoid'' monad for a certain monoidal structure on $\FSCS(\C)$; and all this is natural in $\C$, giving a total monad $(-)^\cxt$ on $\FSCC$ over $\Cat$.  However, these aspects will not concern us further.

(If I included a discussion of ``stratified comp cats'' earlier, mention how this construction naturally takes us outside of them unless we soup it up; and how the souped-up version gives a ``strong $\Sigma$-types'' monad; but why it \emph{doesn't} at the moment.)

\begin{definition} A \emph{dependent projection} in $(\C,p,\pow)$ is a map isomorphic to one of the form $\Gamma.\Delta \to \Gamma$, for some $\Delta \in p^\cxt(\Gamma)$.  Graphically, dependent projections will be distinguished as
$\Gamma' \to/->>/ \Gamma$.
\end{definition}

Note that any composite of dependent projections is again one.  [TODO: is this clear?  or draw the diagram for it?  see notes, 16.vii]  Moreover, pulbacks of dependent projections along arbitrary maps exist, and are again dependent projections.

[TODO: in some ways, better to take dependent projection to be extra structure, a \emph{chosen}isomorphism and a \emph{chosen} $\Delta \in p^\cxt(\Gamma)$.  The distinction doesn't seem to seriously affect anything I do; make it, perhaps?]

\subsection*{The nice slice} 

\para $(\C,p^\cxt,\pow^\cxt)$ can never be stratified: for any choice of $\diamond$, the empty dependent context $() \in p^\cxt(\diamond)$ satisfies $\\diamond.() = \diamond$, so lengths cannot be consistently assigned.

However, for any $(\C,p,\pow)$ and $\Gamma \in \C$, there is an evident stratified attributes structure on $p^\cxt(\Gamma)$; the resulting cwa may be called the \emph{nice slice} $(\C,p,\pow)/\Gamma$, and is the algebraic analogue of working in context $\Gamma$.

\section{Type constructors}

Thus far, we have considered only \emph{algebraic} theories, with just a fixed, given set of (dependent) types.  While such theories are interesting in their own right, one is more often interested in theories with type constructors: (dependent) sums and products, $\Id$-types (most crucially for the present work), and beyond.

With this in mind, we look at what structure on the categorical side corresponds to the usual rules for such constructors.  For better comparision, we will again give two equivalent presentations: one very literal to the syntax of the theory (``elim-structure''), one more abstract

\begin{definition}An \emph{elim-structure} on a map $f \colon \Xi \to \Theta$ is a function $E$, assigning to each $C \in p(\Theta)$ and each map $d \colon \Xi \to \Theta.C$ over $\Theta$ a section $E(C,d) \colon \Theta \to \Theta.C$ of the dependent projection $\pi_C$ satisfying $E(C,d) \cdot f = d$.
\end{definition}

Syntactically, this corresponds to the usual style elimination rule
$$\inferrule*{ \y : \Theta \types C(\y)\ \type \\
\x : \Xi \types d(\x) : C(f(\x)) }
{\y : \Theta \types E(C,d;\y) : C(\y)}$$
with computation rule concluding $E(C,d;f(\x)) = d(\x)$.  (Compare $\Id$-elim.)

Categorically, $E$ gives fillers for certain triangles:
$$\xymatrix{ \Xi \ar[d]_f \ar[r]^-d & \Theta.C \ar@/^/@{->>}[dl] \\
\Theta \ar@/^/@{..>}[ur]|-{E(C,d)} } $$  %TODO: prettify the spacing of this a bit!

This in turn implies a more familiar square-filling
$$\xymatrix{ \Xi \ar[d]_f \ar[r] & \Delta \ar@{->>}[d] \\
\Theta \ar@{.>}[ur] \ar[r] & \Gamma }$$
(exhibiting $f$ as weakly orthogonal to all dependent projections; see Section \ref{???} below, and cf.\ \cite{gambino-garner}), together with some stability conditions on the resulting fillers. 

\begin{definition}A \emph{Frobenius elim-structure} on a map $f \colon \Xi \to \Theta$ is a choice of elim-structure $E_\Delta$ on $(f.\Delta) \colon \Xi.(f^*\Delta) \to \Theta.\Delta$, for each $\Delta \in p^\cxt(\Theta)$.
\end{definition}

Syntactically this corresponds to an extra parameter in all the contexts of the rule:
$$\inferrule*{ \y : \Theta, \z : \Delta(\y) \types C(\y,\z)\ \type \\
\x : \Xi, \z: \Delta(f(\x)) \types d(\x,\z) : C(f(\x),\z) }
{\y : \Theta, \z : \Delta(\y) \types E_\Delta(C,d;\y,\z) : C(\y,\z)}$$

\begin{definition}An \emph{$\Id$-structure} on a (plain or stratified) comprehension category $(\C,p,\pow)$ consists of the following data for each context $\Gamma \in \C$ and type $A \in p(\Gamma)$:

\begin{enumerate}
\item a type $\Id_A \in p(\Gamma . A . A)$;

\item a map $r_A \colon \Gamma.A \to \Gamma.A.A.\Id_A$ lifting the diagonal (contraction) map $\diag_A \colon \Gamma.A \to \Gamma.A.A$ over $\Gamma$

$$\xymatrix{ & \Gamma.A.A.\Id_A \ar@{->>}[d] \\
\Gamma.A \ar[ur]^{r_A} \ar[r]^{\delta_A} \ar@{->>}[dr] & \Gamma.A.A \ar@{->>}[d] \\ & \Gamma}$$

\item a Frobenius elim-structure $J_A$ on $r_A$,
$$\xymatrix{ \Gamma.A.\Delta \ar[dr]_{r_A.\Delta} \ar[rr]^d & & \Gamma.A.A.\Id_A.C \ar@/^/@{->>}[dl] \\
& \Gamma.A.A.\Id_A \ar@/^/@{..>}[ur]|-{J_{A,\Delta}(C,d)} } $$
%TODO: prettify the spacing of this a bit!
\end{enumerate}

all stably in $\Gamma$, in that for $A \in p(\Gamma)$ and $f \colon \Theta \to \Gamma$,

\begin{enumerate}
\item $ (f.A.A)^*\Id_A = \Id_{f^*A} \in p(\Theta.{f^*A}.{f^*A})$: 

$$\xymatrix{\Id_{f^*A} & & Id_A \ar@{|->}[ll] \\
\Theta.f^*A.f^*A \ar[rr]^{f.A.A} & & \Gamma.A.A }$$

\item $f^*(r_A) = r_{f^*A}$; equivalently, the following square commutes:
$$\xymatrix{\Theta.f^*A \ar[d]^{r_{f^*A}} \ar[r] & \Gamma.A \ar[d]^{r_A} \\
\Theta.f^*A.f^*A.\Id_{f^*A} \ar[r] & \Gamma.A.A.\Id_A}$$

\item and, for all suitable $\Delta, C, d$, we have $f^*(J_{A,\Delta}(C,d)) = J_{f^*A,f^*\Delta}(f^*C,f^*d)$; in other words, the square
$$\xymatrix{
\{\mbox{triangles over}\ r_A.\Delta\} \ar[d]^{J_{A,\Delta}} \ar[rr]^{f^*} 
    & & \{\mbox{triangles over} r_{f^*A}.f^*\Delta\} \ar[d]^{J_{f^*A,f^*\Delta}} \\
\{\mbox{filled triangles}\} \ar[rr]^{f^*}
    & & \{\mbox{filled triangles}\} }$$
commutes.
\end{enumerate}
\end{definition}

(TO DO: sleep on this for a while, try to find a nice way of wrapping this up, eg fibrationally or similar!)

TO DO: define the various categories $\CwA^\Id$, etc.

\begin{proposition} \label{prop:thid-equiv-cwaid} If $\T$ is any DTT with $\Id$-types, then $\cl(\T)$ admits a canonical $\Id$-structure.  Conversely, if $\T$ is any (plain or stratified) category with attributes, then $\th(\C)$ admits an interpretation of the $\Id$-rules; and the maps $\epsilon_\C \colon \cl(\th(\C)) \to \C$ and $\eta \colon \T \to \T'$ preserve the resulting $\Id$-structure.

In particular, the equivalence $\Th \equiv \CwA_\strat$ lifts to an equivalence $\ThId \equiv \CwA^\Id$.

\begin{proof}Straighforward verification.
\end{proof}
\end{proposition}

\begin{proposition}[Identity contexts] An $\Id$-structure on $(\C,p,\pow)$ lifts to one on $(\C,p^\cxt,\pow^\cxt)$.
\end{proposition}

Note the interesting type-theoretic content: this shows that from identity \emph{types} for dependent types, we can build identity \emph{contexts} for dependent contexts, satisfying all the same rules.

\begin{proof}
We just sketch the proof here; see \cite[2.3.1]{garner:2-d-models} for details.  By Proposition \ref{prop:thid-equiv-cwaid}, we may work type-theoretically.

(TODO: is this worth the notation that it requires developing?)
\end{proof}

Thus the $(-)^\cxt \colon \CwA \to \CwA$ lifts to a functor $\CwAId \to CwAId$.  Note, however, that the monad structure does not lift on the nose: its multiplication fails to preserve $\Id$-structure.  %% see notes of 16.vii

\subsection*{The coslice construction}
\begin{proposition}[Co-slice categories with attributes] If $\C$ is a cwa, and $\Theta$ any context of $\C$, then there is a natural cwa structure on the co-slice category $\Theta/\C$.

If $\C$ was stratified, then so is $\Theta/\C$.

If $\C$ had unit types, sum types, or $\Id$-types, then so does $\Theta/\C$.

In all of these cases, the forgetful functor $\cod \colon \Theta/\C \to \C$ preserves all the given structure.
\end{proposition}

\begin{proof}
Recall that an object of $\Theta/\C$ is a pair $(\Gamma,k)$, where $\Gamma \in \C$ and $\g \colon \Theta \to \Gamma$ (``a $\Theta$-pointed object of $\C$''); a map $f \colon (\Gamma',\g') \to (\Gamma,\g)$ is a map in $\C$ preserving the pointings:
$$\xymatrix{ & \Gamma' \ar[r]^f & \Gamma \\ \Theta \ar[ur]^{\g'} \ar[urr]_\g \\ }$$
% not sure: would it be better to have $\Theta$ above or below?  which will transition more smoothly to mental pictures of the examples we use later?

We define the attributes structure on $\Theta/\C$ by taking an object of $p(\Gamma,\g)$ to be a type $A \in p(\Gamma)$ together with a map $\g.a : \Theta \to \Gamma.A$ over $\g$
$$\xymatrix{ & & \Gamma.A \ar@{->>}[d] \\ \Theta \ar[urr]^{\g.a} \ar[rr]_\g & & \Gamma}$$
and defining $(\Gamma,\g).(A,\g.a) = (\Gamma.A,\g.a)$.

All the remaining structure is determined straightforwardly by the requirement that is preserved by $\cod \colon \Theta/\C \to \C$.  We give here the details just for $\Id$-structure, which is the case we will need later.

Given a type $(A,\g.a) \in p(\Gamma,\g)$, the underlying object of $\Id_{(A,\g.a)}$ has to be $\Id_A$; and in order to make the intro map $r_A$ commute with pointings, we must give $\Id_A$ the pointing $r_A \cdot (\g.a)$:
$$\xymatrix{ & \Gamma.A \ar[r]^-{r_A} \ar@{->>}[dr] & \Gamma.A.A.\Id_A \ar@{->>}[d] \\
\Theta \ar[ur]^{\g.a} \ar@{.>}[urr] \ar[rr]^\g & & \Gamma}$$

This gives us $\Id_{(A,\g.a)}$ and $r_{(A,\g.a)}$.  To lift the elim structure, we just need to check that the map $J_{A,\Delta}(C,d)$ will commute with all pointings involved whenever $d$ does, but this follows straightforwardly from the equation $J_{A,\Delta}(C,d) \cdot r_A.\Delta = d$.
$$\xymatrix{ & & & \Gamma.A.\Delta \ar[d]_{r_A.\Delta} \ar[r]^-d & \Gamma.A.A.\Id_A.\Delta.C \ar@/^/@{->>}[dl] \\
\Theta \ar[rrr] \ar[urrr] & & & \Gamma.A.A.\Id_A.\Delta \ar@/^/@{..>}[ur]|-{J_{A,\Delta}(C,d)} } $$
\end{proof}



