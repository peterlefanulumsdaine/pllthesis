% this file is called up by thesis.tex
% content in this file will be fed into the main document

\chapter{Universal-algebraic aspects}


% ----------------------- contents from here ------------------------

Possibly fold this chapter into the next??

\section{A category of Type Theories}

\para Give brief, semi-formal outline of the type theory, referring to Appendix

\para In fact, we will not work directly with the category of type theories, but with an equivalent category of algebraic models.

\begin{definition} A \emph{split full comprehension category}\cite{jacobs:comprehension-categories} (\fscc) is a category $\C$ together with a split fibration $p : \E \to \C$ and a factorisation $p = \cod \cdot \pow : \E \to \C^\rightarrow \to \C$, such that (a) $\pow$ maps cartesion arrows to pullback squares, and (b) $\pow$ is full and faithful.  By abuse of notation, we will often refer to $\C$ itself as the comprehension category; the pair $(p,\pow)$ is called a \emph{comprehension structure} on $\E$.  

A (strict) map of \fsccs{} is just a functor $F: \C \to \C'$ and a map of fibrations $p \to p'$ over $F$, commuting with the factorisations $\pow$, $\pow'$.  A \emph{map of comprehension structures} on $\C$ is just the case $F = 1_C$.
\end{definition}

If $\C$ has all pullbacks, then condition (a) just says that $\pow$ is a map of fibrations.  
 
In fact, for fixed $\C$, $\FSCS(\C)$ is (equivalent to) a presheaf category---specifically, to to the slice of $\hat{\C}$ over the presheaf $P$ in which an element of $P(A)$ is a map $f : B \to A$ together with chosen pullbacks along all maps $j : A' \to A$.  (This is neat, but maybe not needed!  Perhaps take it out?)

\begin{example}For any type theory $\T$, its category of context $\C(\T)$ is a comprehension category, in which objects of $\pow_0(\Gamma)$ are types dependent over $\Gamma$, and $\pow$ sends a type $A \in \pow_0(\Gamma)$ to the dependent projection $\Gamma, x : A \to \Gamma$.
\end{example}

In fact, this construction is part of an equivalence between $\Th$ and a certain full co-reflexive subcategory of $\FSCC$: see Appendix \ref{app:??} for details.  In light of this, we will typically refer to the objects of any \fscc\ as contexts, and the objects of the fibration as dependent types.

\para


This all extends to theories with type constructors.

Define: a lax map of comprehension structures (do I mean colax??)

Point out: the 2-category $\mathbf{\FSCS(\C)_\lax}$ has finite products.  (And these are probably better seen as 2-limits in $\FSCS(\C)$.)

\begin{definition}
An \emph{\fscs\ with units} on a category $\C$ is an \fscs\ $(p,\pow_0)$ together with a strict map of \fscss\ $1 : (1_B,\textit{id}) \to (p,\pow_0)$. 

An \emph{\fscs\ with binary products} on $\C$ is an \fscs\ $(p,\pow_0)$ with a strict map $ \times : (p times_B p, \pow_0 \times_B) \to (p,\pow_0)$. 
\end{definition}

Note: this implies a certain adjunction, so does corresponds to Jacobs' ``fscc with units''.  [Show this??]

\begin{definition}
\para{The ``dependent contexts'' monad}  Explicitly give the monad on $\ThId$ throwing in all dependent contexts as new types, and interpreting equality as described in loc.~cit.  TODO: look up that interpretation of equality!  Refer to Thorsten's OTT also.
\end{definition}

Point out how this corresponds closely (2-equivalence?) but not precisely to adding (very) strong $\Sigma$-types in the type theory.

Ask: is this induced by the adjunction with $\Id$-type categories?  or categories with appropriate nwfs's, or something?  or Hyland-Pitts style ``categories with \ldots?''

Discuss the Kleisli of this monad: ``modelling types as contexts''.

\section{Internal algebras for operads}

\para Give fuller account of what I rush through in my previous paper: show  correspondence btn different notions of algebras for an operad!  (a) models of ess. alg. (Lawvere) theory (poss with extra structure: "P-maps"); (b) Batanin: monoidal globular categories (as used + nicely expounded in [GvdB]); (c) Leinster: (weak) T-structured categories.

Lovely rarely-cited WEBER paper gives source for most of this!  Possibly even \emph{everything} I need is there, in which case possibly move this section to appendix, and fold the first part of this chapter into the next chapter???

\begin{definition}[Endomorphism operads] For $\E$ any category, $\X$ any globular operad in $\E$, we write $\End_\E(\X)$ for the operad (construction\ldots\ either by monoidal globular categories, or by ``representable'' style of my previous paper; latter is slicker here, but seems very difficult for showing the functoriality).  More generally, really want the $\Coll$-enriched category structure on (appropriate subcategory of) $[\G,\E]$.
\end{definition}

\proposition If $\E$ has enough limits, then for any pasting diagram $\pi$, $$\End_\E(\X)(\pi) \iso [\G/n,\E](X \cotensor \hat{\pi},X \cotensor \yon(n))$$
where the right hand side consists of ``pylon diagrams'':
$$\textrm{draw the diagram here.}$$

\begin{proof} Straightforward (in either construction of $\End$).
\end{proof}

\proposition $\End_\E(\X)$ is functorial in $\E$: a functor $F : \E \to \F$ preserving appropriate limits induces a map $\End_\E(\X) \to \End_\F(F\X)$.

\begin{proof} Straightforward in the ``monoidal globular categories'' approach.  Can't currently see how to do it in the ``representable'' approach!?
\end{proof}

\definition An \emph{algebra} for an operad $P$ on an object $\X$ of $\E$ is an operad map $\xi \colon P \to \End_\E (\X)$ (the \emph{action} of $P$ on $\X$); a map of $P$-algebras is a globular map $\f \colon \X \to \Y$ commuting with the action maps, i.e.\ such that the square 
$$\xymatrix{P \ar[r]^{\xi} \ar[d]^{\upsilon} & [\X,\X] \ar[d]^{f \cdot\ } \\ [\Y,\Y] \ar[r]^{\ \cdot f} & [\X,\Y]}$$
commutes.

In enriched terms, the resulting category $\IntAlg{P}{\E}$ is $\enrCat(\Coll)(P,[\G,\E])$.

% but it's alright now
% I learned my lesson, yeah
% can't please everyone, so...
% Caffé Nero, Davygate, 14.vii

We can also consider $P$ as defining a certain finite-limit sketch $\mathrm{Sk}(P)$, and compare the internal algebras defined here with models of this sketch in $\E$.

\proposition $\IntAlg{P}{\E} \equiv \mathbf{CmpSpan}\mathbf{Mod}_\E(\mathrm{Sk}(P))$
