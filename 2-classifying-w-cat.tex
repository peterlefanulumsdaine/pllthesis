\documentclass{amsart}

\usepackage{color}
\usepackage{ifpdf}
\usepackage{mathpartir}
\usepackage{amssymb}
\usepackage{amsthm}
\usepackage{amsmath}
\usepackage[all]{xypic}
\usepackage{wrapfig}

\xyoption{2cell}
\xyoption{rotate}
%\xyoption{curve}
\UseTwocells
 \def\dated#1{\def\thedate{#1}}%
 \dated{2004-12-08}%
 % This is a package of commutative diagram macros built on top of Xy-pic%
 % by Michael Barr (email:  barr@barrs.org).  Its use is unrestricted.  It%
 % may be freely distributed, unchanged, for non-commercial or commercial%
 % use.  If changed, it must be renamed.  Inclusion in a commercial%
 % software package is also permitted, but I would appreciate receiving a%
 % free copy for my personal examination and use.  There are no guarantees%
 % that this package is good for anything.  I have tested it with LaTeX 2e,%
 % LaTeX 2.09 and Plain TeX.  Although I know of no reason it will not work%
 % with AMSTeX, I have not tested it.%
 % Added 2003-05-10: I now know that the only clash is with \square in%
 % amssymb, which should therefore be loaded first.  If the amssymb%
 % \square is required, you can \let\box\square and use \box instead.%

\newcount\atcode \atcode=\catcode`\@%
\catcode`\@=12%
\input xy
\xyoption{arrow}
\xyoption{curve}

\newdir{ >}{{ }*!/-.9em/@{>}}%
\newdir{ (}{{ }*!/-.5em/@{(}}%
\newdir^{ (}{{ }*!/-.5em/@^{(}}%
\newdir{< }{!/.9em/@{<}*{ }}%

\newdimen\high%
\newdimen\ul%
\newcount\deltax%
\newcount\deltay%
\newcount\deltaX%
\newcount\deltaXprime%
\newcount\deltaY%
\newcount\deltaYprime%

\newdimen\wdth%
\newcount\xend%
\newcount\yend%
\newcount\Xend%
\newcount\Yend%
\newcount\xpos%
\newcount\ypos%
\newcount\default \default=500%
\newcount\defaultmargin \defaultmargin=150%
\newcount\topw%
\newcount\botw%
\newcount\Xpos%
\newcount\Ypos%
\def\ratchet#1#2{\ifnum#1<#2\global #1=#2\fi}%

\catcode`\@=11%
\expandafter\ifx\csname @ifnextchar\endcsname\relax%
\def\ifnextchar#1#2#3{\let\@tempe%
#1\def\@tempa{#2}\def\@tempb{#3}\futurelet%
    \@tempc\@ifnch}%
\def\@ifnch{\ifx \@tempc \@sptoken \let\@tempd\@xifnch%
      \else \ifx \@tempc \@tempe\let\@tempd\@tempa\else\let\@tempd\@tempb\fi%
      \fi \@tempd}%
\def\:{\let\@sptoken= } \:  % this makes \@sptoken a space token%
\def\:{\@xifnch} \expandafter\def\: {\futurelet\@tempc\@ifnch}%
\else%
\let\ifnextchar\@ifnextchar%
\fi%
\ifx\check@mathfonts\undefined%
\else \check@mathfonts%
\fi%
\newdimen\axis \axis=\fontdimen22\textfont2%
\ul=.01em%
\X@xbase =.01em%
\Y@ybase =.01em%
\def\scalefactor#1{\ul=#1\ul \X@xbase=#1\X@xbase \Y@ybase=#1\Y@ybase}%
\catcode`\@=12%

\def\fontscale#1{%
\if#1h\relax%
\font\xydashfont=xydash10 scaled \magstephalf%
\font\xyatipfont=xyatip10 scaled \magstephalf%
\font\xybtipfont=xybtip10 scaled \magstephalf%
\font\xybsqlfont=xybsql10 scaled \magstephalf%
\font\xycircfont=xycirc10 scaled \magstephalf%
\else%
\font\xydashfont=xydash10 scaled \magstep#1%
\font\xyatipfont=xyatip10 scaled \magstep#1%
\font\xybtipfont=xybtip10 scaled \magstep#1%
\font\xybsqlfont=xybsql10 scaled \magstep#1%
\font\xycircfont=xycirc10 scaled \magstep#1%
\fi}%

\def\bfig{\vcenter\bgroup\xy}%
\def\efig{\endxy\egroup}%

\def\car#1#2\nil{#1}%

\def\morphism{\ifnextchar({\morphismp}{\morphismp(0,0)}}%
\def\morphismp(#1){\ifnextchar|{\morphismpp(#1)}{\morphismpp(#1)|a|}}%
\def\morphismpp(#1)|#2|{\ifnextchar/{\morphismppp(#1)|#2|}%
    {\morphismppp(#1)|#2|/>/}}%
\def\morphismppp(#1)|#2|/#3/{%
    \ifnextchar<{\morphismpppp(#1)|#2|/#3/}%
    {\morphismpppp(#1)|#2|/#3/<\default,0>}}%

\def\morphismpppp(#1,#2)|#3|/#4/<#5,#6>[#7`#8;#9]{%
\xend#1\advance \xend by #5%
\yend#2\advance \yend by #6%
\domorphism(#1,#2)|#3|/#4/<#5,#6>[{#7}`{#8};{#9}]}%


\def\domorphism(#1,#2)|#3|/#4/<#5,#6>[#7`#8;#9]{%
 % Check if arrow arg has an @; then don't add it.%
\def\next{\car#4.\nil}%
\if@\next\relax%
 \if#3l%
  \ifnum #6>0%
   \POS(#1,#2)*+!!<0ex,\axis>{#7}\ar#4^-{#9} (\xend,\yend)*+!!<0ex,\axis>{#8}%
  \else%
   \POS(#1,#2)*+!!<0ex,\axis>{#7}\ar#4_-{#9} (\xend,\yend)*+!!<0ex,\axis>{#8}%
  \fi%
 \else \if#3m%
    \setbox0\hbox{$#9$}%
   \ifdim \wd0=0pt%
     \POS(#1,#2)*+!!<0ex,\axis>{#7}\ar#4 (\xend,\yend)*+!!<0ex,\axis>{#8}%
   \else%
     \POS(#1,#2)*+!!<0ex,\axis>{#7}\ar#4|-*+<1pt,4pt>{\labelstyle#9}%
       (\xend,\yend)*+!!<0ex,\axis>{#8}%
   \fi%
 \else \if#3r%
  \ifnum #6<0%
   \POS(#1,#2)*+!!<0ex,\axis>{#7}\ar#4^-{#9} (\xend,\yend)*+!!<0ex,\axis>{#8}%
  \else%
   \POS(#1,#2)*+!!<0ex,\axis>{#7}\ar#4_-{#9} (\xend,\yend)*+!!<0ex,\axis>{#8}%
  \fi%
 \else \if#3a%
  \ifnum #5>0%
   \POS(#1,#2)*+!!<0ex,\axis>{#7}\ar#4^-{#9} (\xend,\yend)*+!!<0ex,\axis>{#8}%
  \else%
   \POS(#1,#2)*+!!<0ex,\axis>{#7}\ar#4_-{#9} (\xend,\yend)*+!!<0ex,\axis>{#8}%
  \fi%
 \else \if#3b%
  \ifnum #5<0%
   \POS(#1,#2)*+!!<0ex,\axis>{#7}\ar#4^-{#9} (\xend,\yend)*+!!<0ex,\axis>{#8}%
  \else%
   \POS(#1,#2)*+!!<0ex,\axis>{#7}\ar#4_-{#9} (\xend,\yend)*+!!<0ex,\axis>{#8}%
  \fi%
 \else%
   \POS(#1,#2)*+!!<0ex,\axis>{#7}\ar#4 (\xend,\yend)*+!!<0ex,\axis>{#8}%
 \fi\fi\fi\fi\fi%
 %% Otherwise, have to add @{..}%
\else%
 \if#3l%
  \ifnum #6>0%
   \POS(#1,#2)*+!!<0ex,\axis>{#7}\ar@{#4}^-{#9} (\xend,\yend)*+!!<0ex,\axis>{#8}%
  \else%
   \POS(#1,#2)*+!!<0ex,\axis>{#7}\ar@{#4}_-{#9} (\xend,\yend)*+!!<0ex,\axis>{#8}%
  \fi%
 \else \if#3m%
    \setbox0\hbox{$#9$}%
   \ifdim \wd0=0pt%
     \POS(#1,#2)*+!!<0ex,\axis>{#7}\ar@{#4} (\xend,\yend)*+!!<0ex,\axis>{#8}%
   \else%
     \POS(#1,#2)*+!!<0ex,\axis>{#7}\ar@{#4}|-*+<1pt,4pt>{\labelstyle#9}%
         (\xend,\yend)*+!!<0ex,\axis>{#8}%
   \fi%
 \else \if#3r%
  \ifnum #6<0%
   \POS(#1,#2)*+!!<0ex,\axis>{#7}\ar@{#4}^-{#9} (\xend,\yend)*+!!<0ex,\axis>{#8}%
  \else%
   \POS(#1,#2)*+!!<0ex,\axis>{#7}\ar@{#4}_-{#9} (\xend,\yend)*+!!<0ex,\axis>{#8}%
  \fi%
 \else \if#3a%
  \ifnum #5>0%
   \POS(#1,#2)*+!!<0ex,\axis>{#7}\ar@{#4}^-{#9} (\xend,\yend)*+!!<0ex,\axis>{#8}%
  \else%
   \POS(#1,#2)*+!!<0ex,\axis>{#7}\ar@{#4}_-{#9} (\xend,\yend)*+!!<0ex,\axis>{#8}%
  \fi%
 \else \if#3b%
  \ifnum #5<0%
   \POS(#1,#2)*+!!<0ex,\axis>{#7}\ar@{#4}^-{#9} (\xend,\yend)*+!!<0ex,\axis>{#8}%
  \else%
   \POS(#1,#2)*+!!<0ex,\axis>{#7}\ar@{#4}_-{#9} (\xend,\yend)*+!!<0ex,\axis>{#8}%
  \fi%
 \else%
   \POS(#1,#2)*+!!<0ex,\axis>{#7}\ar@{#4} (\xend,\yend)*+!!<0ex,\axis>{#8}%
 \fi\fi\fi\fi\fi%
\fi\ignorespaces}%

\def\vect(#1,#2)/#3/<#4,#5>{%
 \xend#1 \yend#2 \advance\xend by #4 \advance\yend by #5%
     \POS(#1,#2)\ar#3 (\xend,\yend)}%

\def\squarepppp(#1,#2)|#3|/#4`#5`#6`#7/<#8>[#9]{%
\xpos#1\ypos#2%
\def\next|##1##2##3##4|{%
 \def\xa{##1}\def\xb{##2}\def\xc{##3}\def\xd{##4}\ignorespaces}%
\next|#3|%
\def\next<##1,##2>{\deltax=##1\deltay=##2\ignorespaces}%
\next<#8>%
\def\next[##1`##2`##3`##4;##5`##6`##7`##8]{%
    \def\nodea{##1}\def\nodeb{##2}\def\nodec{##3}\def\noded{##4}%
    \def\labela{##5}\def\labelb{##6}\def\labelc{##7}\def\labeld{##8}\ignorespaces}%
\next[#9]%
\morphism(\xpos,\ypos)|\xd|/{#7}/<\deltax,0>[\nodec`\noded;\labeld]%
\advance \ypos by \deltay%
\morphism(\xpos,\ypos)|\xb|/{#5}/<0,-\deltay>[\nodea`\nodec;\labelb]%
\morphism(\xpos,\ypos)|\xa|/{#4}/<\deltax,0>[\nodea`\nodeb;\labela]%
 \advance \xpos by \deltax%
\morphism(\xpos,\ypos)|\xc|/{#6}/<0,-\deltay>[\nodeb`\noded;\labelc]%
\ignorespaces}%

\def\square{\ifnextchar({\squarep}{\squarep(0,0)}}%
\def\squarep(#1){\ifnextchar|{\squarepp(#1)}{\squarepp(#1)|alrb|}}%
\def\squarepp(#1)|#2|{\ifnextchar/{\squareppp(#1)|#2|}%
    {\squareppp(#1)|#2|/>`>`>`>/}}%
\def\squareppp(#1)|#2|/#3`#4`#5`#6/{%
    \ifnextchar<{\squarepppp(#1)|#2|/#3`#4`#5`#6/}%
    {\squarepppp(#1)|#2|/#3`#4`#5`#6/<\default,\default>}}%

\def\ptrianglepppp(#1,#2)|#3|/#4`#5`#6/<#7>[#8]{%
\xpos#1\ypos#2%
\def\next|##1##2##3|{\def\xa{##1}\def\xb{##2}\def\xc{##3}}%
\next|#3|%
\def\next<##1,##2>{\deltax=##1\deltay=##2\ignorespaces}%
\next<#7>%
\def\next[##1`##2`##3;##4`##5`##6]{%
    \def\nodea{##1}\def\nodeb{##2}\def\nodec{##3}%
    \def\labela{##4}\def\labelb{##5}\def\labelc{##6}}%
\next[#8]%
\advance\ypos by \deltay%
\morphism(\xpos,\ypos)|\xa|/{#4}/<\deltax,0>[\nodea`\nodeb;\labela]%
\morphism(\xpos,\ypos)|\xb|/{#5}/<0,-\deltay>[\nodea`\nodec;\labelb]%
\advance\xpos by \deltax%
\morphism(\xpos,\ypos)|\xc|/{#6}/<-\deltax,-\deltay>[\nodeb`\nodec;\labelc]%
\ignorespaces}%

\def\qtrianglepppp(#1,#2)|#3|/#4`#5`#6/<#7>[#8]{%
\xpos#1\ypos#2%
\def\next|##1##2##3|{\def\xa{##1}\def\xb{##2}\def\xc{##3}}%
\next|#3|%
\def\next<##1,##2>{\deltax=##1\deltay=##2\ignorespaces}%
\next<#7>%
\def\next[##1`##2`##3;##4`##5`##6]{%
    \def\nodea{##1}\def\nodeb{##2}\def\nodec{##3}%
    \def\labela{##4}\def\labelb{##5}\def\labelc{##6}}%
\next[#8]%
\advance\ypos by \deltay%
\morphism(\xpos,\ypos)|\xa|/{#4}/<\deltax,0>[\nodea`\nodeb;\labela]%
\morphism(\xpos,\ypos)|\xb|/{#5}/<\deltax,-\deltay>[\nodea`\nodec;\labelb]%
\advance\xpos by \deltax%
\morphism(\xpos,\ypos)|\xc|/{#6}/<0,-\deltay>[\nodeb`\nodec;\labelc]%
\ignorespaces}%

\def\dtrianglepppp(#1,#2)|#3|/#4`#5`#6/<#7>[#8]{%
\xpos#1\ypos#2%
\def\next|##1##2##3|{\def\xa{##1}\def\xb{##2}\def\xc{##3}}%
\next|#3|%
\def\next<##1,##2>{\deltax=##1\deltay=##2\ignorespaces}%
\next<#7>%
\def\next[##1`##2`##3;##4`##5`##6]{%
    \def\nodea{##1}\def\nodeb{##2}\def\nodec{##3}%
    \def\labela{##4}\def\labelb{##5}\def\labelc{##6}}%
\next[#8]%
\morphism(\xpos,\ypos)|\xc|/{#6}/<\deltax,0>[\nodeb`\nodec;\labelc]%
\advance\ypos by \deltay\advance \xpos by \deltax%
\morphism(\xpos,\ypos)|\xa|/{#4}/<-\deltax,-\deltay>[\nodea`\nodeb;\labela]%
\morphism(\xpos,\ypos)|\xb|/{#5}/<0,-\deltay>[\nodea`\nodec;\labelb]%
\ignorespaces}%

\def\btrianglepppp(#1,#2)|#3|/#4`#5`#6/<#7>[#8]{%
\xpos#1\ypos#2%
\def\next|##1##2##3|{\def\xa{##1}\def\xb{##2}\def\xc{##3}}%
\next|#3|%
\def\next<##1,##2>{\deltax=##1\deltay=##2\ignorespaces}%
\next<#7>%
\def\next[##1`##2`##3;##4`##5`##6]{%
    \def\nodea{##1}\def\nodeb{##2}\def\nodec{##3}%
    \def\labela{##4}\def\labelb{##5}\def\labelc{##6}}%
\next[#8]%
\morphism(\xpos,\ypos)|\xc|/{#6}/<\deltax,0>[\nodeb`\nodec;\labelc]%
\advance\ypos by \deltay%
\morphism(\xpos,\ypos)|\xa|/{#4}/<0,-\deltay>[\nodea`\nodeb;\labela]%
\morphism(\xpos,\ypos)|\xb|/{#5}/<\deltax,-\deltay>[\nodea`\nodec;\labelb]%
\ignorespaces}%

\def\Atrianglepppp(#1,#2)|#3|/#4`#5`#6/<#7>[#8]{%
\xpos#1\ypos#2%
\def\next|##1##2##3|{\def\xa{##1}\def\xb{##2}\def\xc{##3}}%
\next|#3|%
\def\next<##1,##2>{\deltax=##1\deltay=##2\ignorespaces}%
\next<#7>%
\def\next[##1`##2`##3;##4`##5`##6]{%
    \def\nodea{##1}\def\nodeb{##2}\def\nodec{##3}%
    \def\labela{##4}\def\labelb{##5}\def\labelc{##6}}%
\next[#8]%
\multiply\deltax by 2%
\morphism(\xpos,\ypos)|\xc|/{#6}/<\deltax,0>[\nodeb`\nodec;\labelc]%
\divide\deltax by 2%
\advance\ypos by \deltay\advance\xpos by \deltax%
\morphism(\xpos,\ypos)|\xa|/{#4}/<-\deltax,-\deltay>[\nodea`\nodeb;\labela]%
\morphism(\xpos,\ypos)|\xb|/{#5}/<\deltax,-\deltay>[\nodea`\nodec;\labelb]%
\ignorespaces}%

\def\Vtrianglepppp(#1,#2)|#3|/#4`#5`#6/<#7>[#8]{%
\xpos#1\ypos#2%
\def\next|##1##2##3|{\def\xa{##1}\def\xb{##2}\def\xc{##3}}%
\next|#3|%
\def\next<##1,##2>{\deltax=##1\deltay=##2\ignorespaces}%
\next<#7>%
\def\next[##1`##2`##3;##4`##5`##6]{%
    \def\nodea{##1}\def\nodeb{##2}\def\nodec{##3}%
    \def\labela{##4}\def\labelb{##5}\def\labelc{##6}}%
\next[#8]%
\advance\ypos by \deltay%
\morphism(\xpos,\ypos)|\xb|/{#5}/<\deltax,-\deltay>[\nodea`\nodec;\labelb]%
\multiply\deltax by 2%
\morphism(\xpos,\ypos)|\xa|/{#4}/<\deltax,0>[\nodea`\nodeb;\labela]%
\advance\xpos by \deltax \divide \deltax by 2%
\morphism(\xpos,\ypos)|\xc|/{#6}/<-\deltax,-\deltay>[\nodeb`\nodec;\labelc]%
\ignorespaces}%

\def\Ctrianglepppp(#1,#2)|#3|/#4`#5`#6/<#7>[#8]{%
\xpos#1\ypos#2%
\def\next|##1##2##3|{\def\xa{##1}\def\xb{##2}\def\xc{##3}}%
\next|#3|%
\def\next<##1,##2>{\deltax=##1\deltay=##2\ignorespaces}%
\next<#7>%
\def\next[##1`##2`##3;##4`##5`##6]{%
    \def\nodea{##1}\def\nodeb{##2}\def\nodec{##3}%
    \def\labela{##4}\def\labelb{##5}\def\labelc{##6}}%
\next[#8]%
\advance \ypos by \deltay%
\morphism(\xpos,\ypos)|\xc|/{#6}/<\deltax,-\deltay>[\nodeb`\nodec;\labelc]%
\advance\ypos by \deltay \advance \xpos by \deltax%
\morphism(\xpos,\ypos)|\xa|/{#4}/<-\deltax,-\deltay>[\nodea`\nodeb;\labela]%
\multiply\deltay by 2%
\morphism(\xpos,\ypos)|\xb|/{#5}/<0,-\deltay>[\nodea`\nodec;\labelb]%
\ignorespaces}%

\def\Dtrianglepppp(#1,#2)|#3|/#4`#5`#6/<#7>[#8]{%
\xpos#1\ypos#2%
\def\next|##1##2##3|{\def\xa{##1}\def\xb{##2}\def\xc{##3}}%
\next|#3|%
\def\next<##1,##2>{\deltax=##1\deltay=##2\ignorespaces}%
\next<#7>%
\def\next[##1`##2`##3;##4`##5`##6]{%
    \def\nodea{##1}\def\nodeb{##2}\def\nodec{##3}%
    \def\labela{##4}\def\labelb{##5}\def\labelc{##6}}%
\next[#8]%
\advance\xpos by \deltax \advance\ypos by \deltay%
\morphism(\xpos,\ypos)|\xc|/{#6}/<-\deltax,-\deltay>[\nodeb`\nodec;\labelc]%
\advance\xpos by -\deltax \advance\ypos by \deltay%
\morphism(\xpos,\ypos)|\xb|/{#5}/<\deltax,-\deltay>[\nodea`\nodeb;\labelb]%
\multiply \deltay by 2%
\morphism(\xpos,\ypos)|\xa|/{#4}/<0,-\deltay>[\nodea`\nodec;\labela]%
\ignorespaces}%

\def\ptriangle{\ifnextchar({\ptrianglep}{\ptrianglep(0,0)}}%
\def\ptrianglep(#1){\ifnextchar|{\ptrianglepp(#1)}{\ptrianglepp(#1)|alr|}}%
\def\ptrianglepp(#1)|#2|{\ifnextchar/{\ptriangleppp(#1)|#2|}%
    {\ptriangleppp(#1)|#2|/>`>`>/}}%
\def\ptriangleppp(#1)|#2|/#3`#4`#5/{%
    \ifnextchar<{\ptrianglepppp(#1)|#2|/#3`#4`#5/}%
    {\ptrianglepppp(#1)|#2|/#3`#4`#5/<\default,\default>}}%

\def\qtriangle{\ifnextchar({\qtrianglep}{\qtrianglep(0,0)}}%
\def\qtrianglep(#1){\ifnextchar|{\qtrianglepp(#1)}{\qtrianglepp(#1)|alr|}}%
\def\qtrianglepp(#1)|#2|{\ifnextchar/{\qtriangleppp(#1)|#2|}%
    {\qtriangleppp(#1)|#2|/>`>`>/}}%
\def\qtriangleppp(#1)|#2|/#3`#4`#5/{%
    \ifnextchar<{\qtrianglepppp(#1)|#2|/#3`#4`#5/}%
    {\qtrianglepppp(#1)|#2|/#3`#4`#5/<\default,\default>}}%

\def\dtriangle{\ifnextchar({\dtrianglep}{\dtrianglep(0,0)}}%
\def\dtrianglep(#1){\ifnextchar|{\dtrianglepp(#1)}{\dtrianglepp(#1)|lrb|}}%
\def\dtrianglepp(#1)|#2|{\ifnextchar/{\dtriangleppp(#1)|#2|}%
    {\dtriangleppp(#1)|#2|/>`>`>/}}%
\def\dtriangleppp(#1)|#2|/#3`#4`#5/{%
    \ifnextchar<{\dtrianglepppp(#1)|#2|/#3`#4`#5/}%
    {\dtrianglepppp(#1)|#2|/#3`#4`#5/<\default,\default>}}%

\def\btriangle{\ifnextchar({\btrianglep}{\btrianglep(0,0)}}%
\def\btrianglep(#1){\ifnextchar|{\btrianglepp(#1)}{\btrianglepp(#1)|lrb|}}%
\def\btrianglepp(#1)|#2|{\ifnextchar/{\btriangleppp(#1)|#2|}%
    {\btriangleppp(#1)|#2|/>`>`>/}}%
\def\btriangleppp(#1)|#2|/#3`#4`#5/{%
    \ifnextchar<{\btrianglepppp(#1)|#2|/#3`#4`#5/}%
    {\btrianglepppp(#1)|#2|/#3`#4`#5/<\default,\default>}}%

\def\Atriangle{\ifnextchar({\Atrianglep}{\Atrianglep(0,0)}}%
\def\Atrianglep(#1){\ifnextchar|{\Atrianglepp(#1)}{\Atrianglepp(#1)|lrb|}}%
\def\Atrianglepp(#1)|#2|{\ifnextchar/{\Atriangleppp(#1)|#2|}%
    {\Atriangleppp(#1)|#2|/>`>`>/}}%
\def\Atriangleppp(#1)|#2|/#3`#4`#5/{%
    \ifnextchar<{\Atrianglepppp(#1)|#2|/#3`#4`#5/}%
    {\Atrianglepppp(#1)|#2|/#3`#4`#5/<\default,\default>}}%

\def\Vtriangle{\ifnextchar({\Vtrianglep}{\Vtrianglep(0,0)}}%
\def\Vtrianglep(#1){\ifnextchar|{\Vtrianglepp(#1)}{\Vtrianglepp(#1)|alb|}}%
\def\Vtrianglepp(#1)|#2|{\ifnextchar/{\Vtriangleppp(#1)|#2|}%
    {\Vtriangleppp(#1)|#2|/>`>`>/}}%
\def\Vtriangleppp(#1)|#2|/#3`#4`#5/{%
    \ifnextchar<{\Vtrianglepppp(#1)|#2|/#3`#4`#5/}%
    {\Vtrianglepppp(#1)|#2|/#3`#4`#5/<\default,\default>}}%

\def\Ctriangle{\ifnextchar({\Ctrianglep}{\Ctrianglep(0,0)}}%
\def\Ctrianglep(#1){\ifnextchar|{\Ctrianglepp(#1)}{\Ctrianglepp(#1)|arb|}}%
\def\Ctrianglepp(#1)|#2|{\ifnextchar/{\Ctriangleppp(#1)|#2|}%
    {\Ctriangleppp(#1)|#2|/>`>`>/}}%
\def\Ctriangleppp(#1)|#2|/#3`#4`#5/{%
    \ifnextchar<{\Ctrianglepppp(#1)|#2|/#3`#4`#5/}%
    {\Ctrianglepppp(#1)|#2|/#3`#4`#5/<\default,\default>}}%

\def\Dtriangle{\ifnextchar({\Dtrianglep}{\Dtrianglep(0,0)}}%
\def\Dtrianglep(#1){\ifnextchar|{\Dtrianglepp(#1)}{\Dtrianglepp(#1)|alb|}}%
\def\Dtrianglepp(#1)|#2|{\ifnextchar/{\Dtriangleppp(#1)|#2|}%
    {\Dtriangleppp(#1)|#2|/>`>`>/}}%
\def\Dtriangleppp(#1)|#2|/#3`#4`#5/{%
    \ifnextchar<{\Dtrianglepppp(#1)|#2|/#3`#4`#5/}%
    {\Dtrianglepppp(#1)|#2|/#3`#4`#5/<\default,\default>}}%


\def\Atrianglepairpppp(#1)|#2|/#3`#4`#5`#6`#7/<#8>[#9]{%
\def\next(##1,##2){\xpos##1\ypos##2}%
\next(#1)%
\def\next|##1##2##3##4##5|{\def\xa{##1}\def\xb{##2}%
\def\xc{##3}\def\xd{##4}\def\xe{##5}}%
\next|#2|%
\def\next<##1,##2>{\deltax=##1\deltay=##2\ignorespaces}%
\next<#8>%
\def\next[##1`##2`##3`##4;##5`##6`##7`##8`##9]{%
 \def\nodea{##1}\def\nodeb{##2}\def\nodec{##3}\def\noded{##4}%
 \def\labela{##5}\def\labelb{##6}\def\labelc{##7}\def\labeld{##8}\def\labele{##9}}%
\next[#9]%
\morphism(\xpos,\ypos)|\xd|/{#6}/<\deltax,0>[\nodeb`\nodec;\labeld]%
\advance\xpos by \deltax%
\morphism(\xpos,\ypos)|\xe|/{#7}/<\deltax,0>[\nodec`\noded;\labele]%
\advance\ypos by \deltay%
\morphism(\xpos,\ypos)|\xa|/{#3}/<-\deltax,-\deltay>[\nodea`\nodeb;\labela]%
\morphism(\xpos,\ypos)|\xb|/{#4}/<0,-\deltay>[\nodea`\nodec;\labelb]%
\morphism(\xpos,\ypos)|\xc|/{#5}/<\deltax,-\deltay>[\nodea`\noded;\labelc]%
\ignorespaces}%

\def\Vtrianglepairpppp(#1)|#2|/#3`#4`#5`#6`#7/<#8>[#9]{%
\def\next(##1,##2){\xpos##1\ypos##2}%
\next(#1)%
\def\next|##1##2##3##4##5|{\def\xa{##1}\def\xb{##2}%
\def\xc{##3}\def\xd{##4}\def\xe{##5}}%
\next|#2|%
\def\next<##1,##2>{\deltax=##1\deltay=##2\ignorespaces}%
\next<#8>%
\def\next[##1`##2`##3`##4;##5`##6`##7`##8`##9]{%
 \def\nodea{##1}\def\nodeb{##2}\def\nodec{##3}\def\noded{##4}%
 \def\labela{##5}\def\labelb{##6}\def\labelc{##7}\def\labeld{##8}\def\labele{##9}}%
\next[#9]%
\advance\ypos by \deltay%
\morphism(\xpos,\ypos)|\xa|/{#3}/<\deltax,0>[\nodea`\nodeb;\labela]%
\morphism(\xpos,\ypos)|\xc|/{#5}/<\deltax,-\deltay>[\nodea`\noded;\labelc]%
\advance\xpos by \deltax%
\morphism(\xpos,\ypos)|\xb|/{#4}/<\deltax,0>[\nodeb`\nodec;\labelb]%
\morphism(\xpos,\ypos)|\xd|/{#6}/<0,-\deltay>[\nodeb`\noded;\labeld]%
\advance\xpos by \deltax%
\morphism(\xpos,\ypos)|\xe|/{#7}/<-\deltax,-\deltay>[\nodec`\noded;\labele]%
\ignorespaces}%

\def\Ctrianglepairpppp(#1)|#2|/#3`#4`#5`#6`#7/<#8>[#9]{%
\def\next(##1,##2){\xpos##1\ypos##2}%
\next(#1)%
\def\next|##1##2##3##4##5|{\def\xa{##1}\def\xb{##2}%
\def\xc{##3}\def\xd{##4}\def\xe{##5}}%
\next|#2|%
\def\next<##1,##2>{\deltax=##1\deltay=##2\ignorespaces}%
\next<#8>%
\def\next[##1`##2`##3`##4;##5`##6`##7`##8`##9]{%
 \def\nodea{##1}\def\nodeb{##2}\def\nodec{##3}\def\noded{##4}%
 \def\labela{##5}\def\labelb{##6}\def\labelc{##7}\def\labeld{##8}\def\labele{##9}}%
\next[#9]%
\advance\ypos by \deltay%
\morphism(\xpos,\ypos)|\xe|/{#7}/<0,-\deltay>[\nodec`\noded;\labele]%
\advance\xpos by -\deltax%
\morphism(\xpos,\ypos)|\xc|/{#5}/<\deltax,0>[\nodeb`\nodec;\labelc]%
\morphism(\xpos,\ypos)|\xd|/{#6}/<\deltax,-\deltay>[\nodeb`\noded;\labeld]%
\advance\ypos by \deltay%
\advance\xpos by \deltax%
\morphism(\xpos,\ypos)|\xa|/{#3}/<-\deltax,-\deltay>[\nodea`\nodeb;\labela]%
\morphism(\xpos,\ypos)|\xb|/{#4}/<0,-\deltay>[\nodea`\nodec;\labelb]%
\ignorespaces}%

\def\Dtrianglepairpppp(#1)|#2|/#3`#4`#5`#6`#7/<#8>[#9]{%
\def\next(##1,##2){\xpos##1\ypos##2}%
\next(#1)%
\def\next|##1##2##3##4##5|{\def\xa{##1}\def\xb{##2}%
\def\xc{##3}\def\xd{##4}\def\xe{##5}}%
\next|#2|%
\def\next<##1,##2>{\deltax=##1\deltay=##2\ignorespaces}%
\next<#8>%
\def\next[##1`##2`##3`##4;##5`##6`##7`##8`##9]{%
 \def\nodea{##1}\def\nodeb{##2}\def\nodec{##3}\def\noded{##4}%
 \def\labela{##5}\def\labelb{##6}\def\labelc{##7}\def\labeld{##8}\def\labele{##9}}%
\next[#9]%
\advance\ypos by \deltay%
\morphism(\xpos,\ypos)|\xc|/{#5}/<\deltax,0>[\nodeb`\nodec;\labelc]%
\morphism(\xpos,\ypos)|\xd|/{#6}/<0,-\deltay>[\nodeb`\noded;\labeld]%
\advance\ypos by \deltay%
\morphism(\xpos,\ypos)|\xa|/{#3}/<0,-\deltay>[\nodea`\nodeb;\labela]%
\morphism(\xpos,\ypos)|\xb|/{#4}/<\deltax,-\deltay>[\nodea`\nodec;\labelb]%
\advance\ypos by -\deltay%
\advance\xpos by \deltax%
\morphism(\xpos,\ypos)|\xe|/{#7}/<-\deltax,-\deltay>[\nodec`\noded;\labele]%
\ignorespaces}%
\def\Atrianglepair{\ifnextchar({\Atrianglepairp}{\Atrianglepairp(0,0)}}%
\def\Atrianglepairp(#1){\ifnextchar|{\Atrianglepairpp(#1)}%
{\Atrianglepairpp(#1)|lmrbb|}}%
\def\Atrianglepairpp(#1)|#2|{\ifnextchar/{\Atrianglepairppp(#1)|#2|}%
    {\Atrianglepairppp(#1)|#2|/>`>`>`>`>/}}%
\def\Atrianglepairppp(#1)|#2|/#3`#4`#5`#6`#7/{%
    \ifnextchar<{\Atrianglepairpppp(#1)|#2|/#3`#4`#5`#6`#7/}%
    {\Atrianglepairpppp(#1)|#2|/#3`#4`#5`#6`#7/<\default,\default>}}%

\def\Vtrianglepair{\ifnextchar({\Vtrianglepairp}{\Vtrianglepairp(0,0)}}%
\def\Vtrianglepairp(#1){\ifnextchar|{\Vtrianglepairpp(#1)}%
{\Vtrianglepairpp(#1)|aalmr|}}%
\def\Vtrianglepairpp(#1)|#2|{\ifnextchar/{\Vtrianglepairppp(#1)|#2|}%
    {\Vtrianglepairppp(#1)|#2|/>`>`>`>`>/}}%
\def\Vtrianglepairppp(#1)|#2|/#3`#4`#5`#6`#7/{%
    \ifnextchar<{\Vtrianglepairpppp(#1)|#2|/#3`#4`#5`#6`#7/}%
    {\Vtrianglepairpppp(#1)|#2|/#3`#4`#5`#6`#7/<\default,\default>}}%

\def\Ctrianglepair{\ifnextchar({\Ctrianglepairp}{\Ctrianglepairp(0,0)}}%
\def\Ctrianglepairp(#1){\ifnextchar|{\Ctrianglepairpp(#1)}%
{\Ctrianglepairpp(#1)|lrmlr|}}%
\def\Ctrianglepairpp(#1)|#2|{\ifnextchar/{\Ctrianglepairppp(#1)|#2|}%
    {\Ctrianglepairppp(#1)|#2|/>`>`>`>`>/}}%
\def\Ctrianglepairppp(#1)|#2|/#3`#4`#5`#6`#7/{%
    \ifnextchar<{\Ctrianglepairpppp(#1)|#2|/#3`#4`#5`#6`#7/}%
    {\Ctrianglepairpppp(#1)|#2|/#3`#4`#5`#6`#7/<\default,\default>}}%

\def\Dtrianglepair{\ifnextchar({\Dtrianglepairp}{\Dtrianglepairp(0,0)}}%
\def\Dtrianglepairp(#1){\ifnextchar|{\Dtrianglepairpp(#1)}%
{\Dtrianglepairpp(#1)|lrmlr|}}%
\def\Dtrianglepairpp(#1)|#2|{\ifnextchar/{\Dtrianglepairppp(#1)|#2|}%
    {\Dtrianglepairppp(#1)|#2|/>`>`>`>`>/}}%
\def\Dtrianglepairppp(#1)|#2|/#3`#4`#5`#6`#7/{%
    \ifnextchar<{\Dtrianglepairpppp(#1)|#2|/#3`#4`#5`#6`#7/}%
    {\Dtrianglepairpppp(#1)|#2|/#3`#4`#5`#6`#7/<\default,\default>}}%

\def\pplace[#1](#2,#3)[#4]{\POS(#2,#3)*+!!<0ex,\axis>!#1{#4}\ignorespaces}%
\def\cplace(#1,#2)[#3]{\POS(#1,#2)*+!!<0ex,\axis>{#3}\ignorespaces}%
\def\place{\ifnextchar[{\pplace}{\cplace}}%

\def\pullback#1]#2]{\square#1]\trident#2]\ignorespaces}%

\def\tridentppp|#1#2#3|/#4`#5`#6/<#7,#8>[#9]{%
\def\next[##1;##2`##3`##4]{\def\nodee{##1}\def\labele{##2}%
   \def\labelf{##3}\def\labelg{##4}}%
\next[#9]%
\advance \xpos by -\deltax%
\advance \xpos by -#7\advance \ypos by #8%
\advance\deltax by #7%
\morphism(\xpos,\ypos)|#1|/{#4}/<\deltax,-#8>[\nodee`\nodeb;\labele]%
\advance\deltax by -#7%
\morphism(\xpos,\ypos)|#2|/{#5}/<#7,-#8>[\nodee`\nodea;\labelf]%
\advance\deltay by #8%
\morphism(\xpos,\ypos)|#3|/{#6}/<#7,-\deltay>[\nodee`\nodec;\labelg]%
\ignorespaces}%

\def\trident{\ifnextchar|{\tridentp}{\tridentp|amb|}}%
\def\tridentp|#1|{\ifnextchar/{\tridentpp|#1|}{\tridentpp|#1|/{>}`{>}`{>}/}}%
\def\tridentpp|#1|/#2/{\ifnextchar<{\tridentppp|#1|/#2/}%
  {\tridentppp|#1|/#2/<500,500>}}%

\def\setmorphismwidth#1#2#3#4{%
 \setbox0=\hbox{$#1{\labelstyle#3#3}#2$}#4=\wd0%
 \divide #4 by 2 \divide #4 by \ul%
 \advance #4 by 350 \ratchet{#4}{500}}%

\def\setSquarewidth[#1`#2`#3`#4;#5`#6`#7`#8]{%
 \setmorphismwidth{#1}{#2}{#5}{\topw}%
 \setmorphismwidth{#3}{#4}{#8}{\botw}%
\ratchet{\topw}{\botw}}%

\def\Squarepppp(#1)|#2|/#3/<#4>[#5]{%
 \setSquarewidth[#5]%
 \squarepppp(#1)|#2|/#3/<\topw,#4>[#5]%
\ignorespaces}%

\def\Square{\ifnextchar({\Squarep}{\Squarep(0,0)}}%
\def\Squarep(#1){\ifnextchar|{\Squarepp(#1)}{\Squarepp(#1)|alrb|}}%
\def\Squarepp(#1)|#2|{\ifnextchar/{\Squareppp(#1)|#2|}%
    {\Squareppp(#1)|#2|/>`>`>`>/}}%
\def\Squareppp(#1)|#2|/#3`#4`#5`#6/{%
    \ifnextchar<{\Squarepppp(#1)|#2|/#3`#4`#5`#6/}%
    {\Squarepppp(#1)|#2|/#3`#4`#5`#6/<\default>}}%

\def\hsquarespppp(#1,#2)|#3|/#4/<#5>[#6;#7]{%
\Xpos=#1\Ypos=#2%
\def\next|##1##2##3##4##5##6##7|{%
 \def\Xa{##1}\def\Xb{##2}\def\Xc{##3}\def\Xd{##4}%
 \def\Xe{##5}\def\Xf{##6}\def\Xg{##7}}%
\next|#3|%
\def\next<##1,##2,##3>{\deltaX=##1\deltaXprime=##2\deltaY=##3}%
\next<#5>%
\def\next[##1`##2`##3`##4`##5`##6]{%
 \def\Nodea{##1}\def\Nodeb{##2}\def\Nodec{##3}%
 \def\Noded{##4}\def\Nodee{##5}\def\Nodef{##6}}%
\next[#6]%
\def\next[##1`##2`##3`##4`##5`##6`##7]{%
 \def\Labela{##1}\def\Labelb{##2}\def\Labelc{##3}\def\Labeld{##4}%
 \def\Labele{##5}\def\Labelf{##6}\def\Labelg{##7}}%
\next[#7]%
\dohsquares/#4/}%

\def\dohsquares/#1`#2`#3`#4`#5`#6`#7/{%
\squarepppp(\Xpos,\Ypos)|\Xa\Xc\Xd\Xf|/#1`#3`#4`#6/<\deltaX,\deltaY>%
 [\Nodea`\Nodeb`\Noded`\Nodee;\Labela`\Labelc`\Labeld`\Labelf]%
 \advance \Xpos by \deltaX%
\squarepppp(\Xpos,\Ypos)|\Xb\Xd\Xe\Xg|/#2``#5`#7/<\deltaXprime,\deltaY>%
[\Nodeb`\Nodec`\Nodee`\Nodef;\Labelb``\Labele`\Labelg]%
\ignorespaces}%

\def\hsquares{\ifnextchar({\hsquaresp}{\hsquaresp(0,0)}}%
\def\hsquaresp(#1){\ifnextchar|{\hsquarespp(#1)}{\hsquarespp%
(#1)|aalmrbb|}}%
\def\hsquarespp(#1)|#2|{\ifnextchar/{\hsquaresppp(#1)|#2|}%
    {\hsquaresppp(#1)|#2|/>`>`>`>`>`>`>/}}%
\def\hsquaresppp(#1)|#2|/#3/{%
    \ifnextchar<{\hsquarespppp(#1)|#2|/#3/}%
    {\hsquarespppp(#1)|#2|/#3/<\default,\default,\default>}}%

\def\hSquarespppp(#1,#2)|#3|/#4/<#5>[#6;#7]{%
\Xpos=#1\Ypos=#2%
\def\next|##1##2##3##4##5##6##7|{%
 \def\Xa{##1}\def\Xb{##2}\def\Xc{##3}\def\Xd{##4}%
 \def\Xe{##5}\def\Xf{##6}\def\Xg{##7}}%
\next|#3|%
\deltaY=#5%
\def\next[##1`##2`##3`##4`##5`##6]{%
 \def\Nodea{##1}\def\Nodeb{##2}\def\Nodec{##3}%
 \def\Noded{##4}\def\Nodee{##5}\def\Nodef{##6}}%
\next[#6]%
\def\next[##1`##2`##3`##4`##5`##6`##7]{%
 \def\Labela{##1}\def\Labelb{##2}\def\Labelc{##3}\def\Labeld{##4}%
 \def\Labele{##5}\def\Labelf{##6}\def\Labelg{##7}}%
\next[#7]%
\dohSquares/#4/}%

\def\dohSquares/#1`#2`#3`#4`#5`#6`#7/{%
\Squarepppp(\Xpos,\Ypos)|\Xa\Xc\Xd\Xf|/#1`#3`#4`#6/<\deltaY>%
 [\Nodea`\Nodeb`\Noded`\Nodee;\Labela`\Labelc`\Labeld`\Labelf]%
 \advance \Xpos by \topw%
\Squarepppp(\Xpos,\Ypos)|\Xb\Xd\Xe\Xg|/#2``#5`#7/<\deltaY>%
[\Nodeb`\Nodec`\Nodee`\Nodef;\Labelb``\Labele`\Labelg]%
\ignorespaces}%

\def\hSquares{\ifnextchar({\hSquaresp}{\hSquaresp(0,0)}}%
\def\hSquaresp(#1){\ifnextchar|{\hSquarespp(#1)}{\hSquarespp%
(#1)|aalmrbb|}}%
\def\hSquarespp(#1)|#2|{\ifnextchar/{\hSquaresppp(#1)|#2|}%
    {\hSquaresppp(#1)|#2|/>`>`>`>`>`>`>/}}%
\def\hSquaresppp(#1)|#2|/#3/{%
    \ifnextchar<{\hSquarespppp(#1)|#2|/#3/}%
    {\hSquarespppp(#1)|#2|/#3/<\default>}}%

\def\vsquarespppp(#1,#2)|#3|/#4/<#5>[#6;#7]{%
\Xpos=#1\Ypos=#2%
\def\next|##1##2##3##4##5##6##7|{%
 \def\Xa{##1}\def\Xb{##2}\def\Xc{##3}\def\Xd{##4}%
 \def\Xe{##5}\def\Xf{##6}\def\Xg{##7}}%
\next|#3|%
\def\next<##1,##2,##3>{\deltaX=##1\deltaY=##2\deltaYprime=##3}%
\next<#5>%
\def\next[##1`##2`##3`##4`##5`##6]{%
 \def\Nodea{##1}\def\Nodeb{##2}\def\Nodec{##3}%
 \def\Noded{##4}\def\Nodee{##5}\def\Nodef{##6}}%
\next[#6]%
\def\next[##1`##2`##3`##4`##5`##6`##7]{%
 \def\Labela{##1}\def\Labelb{##2}\def\Labelc{##3}\def\Labeld{##4}%
 \def\Labele{##5}\def\Labelf{##6}\def\Labelg{##7}}%
\next[#7]%
\dovsquares/#4/}%

\def\dovsquares/#1`#2`#3`#4`#5`#6`#7/{%
\squarepppp(\Xpos,\Ypos)|\Xd\Xe\Xf\Xg|/`#5`#6`#7/<\deltaX,\deltaYprime>%
[\Nodec`\Noded`\Nodee`\Nodef;`\Labele`\Labelf`\Labelg]%
 \advance\Ypos by \deltaYprime%
\squarepppp(\Xpos,\Ypos)|\Xa\Xb\Xc\Xd|/#1`#2`#3`#4/<\deltaX,\deltaY>%
 [\Nodea`\Nodeb`\Nodec`\Noded;\Labela`\Labelb`\Labelc`\Labeld]%
\ignorespaces}%

\def\vsquares{\ifnextchar({\vsquaresp}{\vsquaresp(0,0)}}%
\def\vsquaresp(#1){\ifnextchar|{\vsquarespp(#1)}{\vsquarespp%
(#1)|aalmrbb|}}%
\def\vsquarespp(#1)|#2|{\ifnextchar/{\vsquaresppp(#1)|#2|}%
    {\vsquaresppp(#1)|#2|/>`>`>`>`>`>`>/}}%
\def\vsquaresppp(#1)|#2|/#3/{%
    \ifnextchar<{\vsquarespppp(#1)|#2|/#3/}%
    {\vsquarespppp(#1)|#2|/#3/<\default,\default,\default>}}%


\def\vSquarespppp(#1,#2)|#3|/#4/<#5,#6>[#7;#8]{%
\Xpos=#1\Ypos=#2%
\def\next|##1##2##3##4##5##6##7|{%
 \def\Xa{##1}\def\Xb{##2}\def\Xc{##3}\def\Xd{##4}%
 \def\Xe{##5}\def\Xf{##6}\def\Xg{##7}}%
\next|#3|%
\deltaX=#5%
\deltaY=#6%
\def\next[##1`##2`##3`##4`##5`##6]{%
 \def\Nodea{##1}\def\Nodeb{##2}\def\Nodec{##3}%
 \def\Noded{##4}\def\Nodee{##5}\def\Nodef{##6}}%
\next[#7]%
\def\next[##1`##2`##3`##4`##5`##6`##7]{%
 \def\Labela{##1}\def\Labelb{##2}\def\Labelc{##3}\def\Labeld{##4}%
 \def\Labele{##5}\def\Labelf{##6}\def\Labelg{##7}}%
\next[#8]%
\dovSquares/#4/\ignorespaces}%

\def\dovSquares/#1`#2`#3`#4`#5`#6`#7/{%
\setmorphismwidth{\Nodea}{\Nodeb}{\Labela}{\topw}%
\setmorphismwidth{\Nodec}{\Noded}{\Labeld}{\botw}%
\ratchet{\topw}{\botw}%
\setmorphismwidth{\Nodee}{\Nodef}{\Labelg}{\botw}%
\ratchet{\topw}{\botw}%
\square(\Xpos,\Ypos)|\Xd\Xe\Xf\Xg|/`#5`#6`#7/<\topw,\deltaX>%
 [\Nodec`\Noded`\Nodee`\Nodef;`\Labele`\Labelf`\Labelg]%
\advance \Ypos by \deltaX%
\square(\Xpos,\Ypos)|\Xa\Xb\Xc\Xd|/#1`#2`#3`#4/<\topw,\deltaY>%
 [\Nodea`\Nodeb`\Nodec`\Noded;\Labela`\Labelb`\Labelc`\Labeld]%
}%

\def\vSquares{\ifnextchar({\vSquaresp}{\vSquaresp(0,0)}}%
\def\vSquaresp(#1){\ifnextchar|{\vSquarespp(#1)}{\vSquarespp%
(#1)|alrmlrb|}}%
\def\vSquarespp(#1)|#2|{\ifnextchar/{\vSquaresppp(#1)|#2|}%
    {\vSquaresppp(#1)|#2|/>`>`>`>`>`>`>/}}%
\def\vSquaresppp(#1)|#2|/#3/{%
    \ifnextchar<{\vSquarespppp(#1)|#2|/#3/}%
    {\vSquarespppp(#1)|#2|/#3/<\default,\default>}}%

\def\osquarepppp(#1)|#2|/#3`#4`#5`#6/<#7>[#8]{\squarepppp%
 (#1)|#2|/#3`#4`#5`#6/<#7>[#8]%
 \let\Nodea\nodea\let\Nodeb\nodeb%
\let\Nodec\nodec\let\Noded\noded\Xpos=\xpos\Ypos=\ypos%
\deltaX=\deltax \deltaY=\deltay \isquare}%

\def\cube{\ifnextchar({\osquarep}{\osquarep(0,0)}}%
\def\osquarep(#1){\ifnextchar|{\osquarepp(#1)}{\osquarepp(#1)|alrb|}}%
\def\osquarepp(#1)|#2|{\ifnextchar/{\osquareppp(#1)|#2|}%
    {\osquareppp(#1)|#2|/>`>`>`>/}}%
\def\osquareppp(#1)|#2|/#3`#4`#5`#6/{%
    \ifnextchar<{\osquarepppp(#1)|#2|/#3`#4`#5`#6/}%
    {\osquarepppp(#1)|#2|/#3`#4`#5`#6/<1500,1500>}}%

\def\isquarepppp(#1)|#2|/#3`#4`#5`#6/<#7>[#8]{%
 \squarepppp(#1)|#2|/#3`#4`#5`#6/<#7>[#8]%
\ifnextchar|{\cubep}{\cubep|mmmm|}}%
\def\cubep|#1|{\ifnextchar/{\cubepp|#1|}{\cubepp|#1|/>`>`>`>/}}%

\def\isquare{\ifnextchar({\isquarep}{\isquarep(\default,\default)}}%
\def\isquarep(#1){\ifnextchar|{\isquarepp(#1)}{\isquarepp(#1)|alrb|}}%
\def\isquarepp(#1)|#2|{\ifnextchar/{\isquareppp(#1)|#2|}%
    {\isquareppp(#1)|#2|/>`>`>`>/}}%
\def\isquareppp(#1)|#2|/#3`#4`#5`#6/{%
    \ifnextchar<{\isquarepppp(#1)|#2|/#3`#4`#5`#6/}%
    {\isquarepppp(#1)|#2|/#3`#4`#5`#6/<500,500>}}%

\def\cubepp|#1#2#3#4|/#5`#6`#7`#8/[#9]{%
\def\next[##1`##2`##3`##4]{\gdef\Labela{##1}%
\gdef\Labelb{##2}\gdef\Labelc{##3}\gdef\Labeld{##4}}\next[#9]%
\xend\xpos \yend\ypos%
\Xend\xend\advance\Xend by -\Xpos%
\Yend\yend\advance\Yend by -\Ypos%
\domorphism(\Xpos,\Ypos)|#2|/#6/<\Xend,\Yend>[\Nodeb`\nodeb;\Labelb]%
\advance\Xpos by-\deltaX%
\advance\xend by-\deltax%
\Xend\xend\advance\Xend by -\Xpos%
\domorphism(\Xpos,\Ypos)|#1|/#5/<\Xend,\Yend>[\Nodea`\nodea;\Labela]%
\advance\Ypos by-\deltaY%
\advance\yend by-\deltay%
\Yend\yend\advance\Yend by -\Ypos%
\domorphism(\Xpos,\Ypos)|#3|/#7/<\Xend,\Yend>[\Nodec`\nodec;\Labelc]%
\advance\Xpos by\deltaX%
\advance\xend by\deltax%
\Xend\xend\advance\Xend by -\Xpos%
\domorphism(\Xpos,\Ypos)|#4|/#8/<\Xend,\Yend>[\Noded`\noded;\Labeld]%
\ignorespaces}%

\def\setwdth#1#2{\setbox0\hbox{$\labelstyle#1$}\wdth=\wd0%
\setbox0\hbox{$\labelstyle#2$}\ifnum\wdth<\wd0 \wdth=\wd0 \fi}%

\def\topppp/#1/<#2>^#3_#4{\:%
\ifnum#2=0%
   \setwdth{#3}{#4}\deltax=\wdth \divide \deltax by \ul%
   \advance \deltax by \defaultmargin  \ratchet{\deltax}{200}%
\else \deltax #2%
\fi%
\xy\ar@{#1}^{#3}_{#4}(\deltax,0) \endxy%
\:}%

\def\toppp/#1/<#2>^#3{\ifnextchar_{\topppp/#1/<#2>^{#3}}{\topppp/#1/<#2>^{#3}_{}}}%
\def\topp/#1/<#2>{\ifnextchar^{\toppp/#1/<#2>}{\toppp/#1/<#2>^{}}}%
\def\toop/#1/{\ifnextchar<{\topp/#1/}{\topp/#1/<0>}}%
\def\to{\ifnextchar/{\toop}{\toop/>/}}%


\def\mon{\to/ >->/}%
\def\epi{\to/->>/}%
\def\toleft{\to/<-/}%
\def\monleft{\to/<-< /}%
\def\epileft{\to/<<-/}%


\def\twopppp/#1`#2/<#3>^#4_#5{\:%
\ifnum0=#3%
  \setwdth{#4}{#5}\deltax=\wdth \divide \deltax by \ul \advance \deltax%
  by \defaultmargin \ratchet{\deltax}{200}%
\else \deltax#3 \fi%
\xy\ar@{#1}@<2.5pt>^{#4}(\deltax,0)%
\ar@{#2}@<-2.5pt>_{#5}(\deltax,0)\endxy\:}%

\def\twoppp/#1`#2/<#3>^#4{\ifnextchar_{\twopppp/#1`#2/<#3>^{#4}}%
  {\twopppp/#1`#2/<#3>^{#4}_{}}}%
\def\twopp/#1`#2/<#3>{\ifnextchar^{\twoppp/#1`#2/<#3>}{\twoppp/#1`#2/<#3>^{}}}%
\def\twop/#1`#2/{\ifnextchar<{\twopp/#1`#2/}{\twopp/#1`#2/<0>}}%
\def\two{\ifnextchar/{\twop}{\twop/>`>/}}%

\def\twoleft{\two/<-`<-/}%

\def\threeppppp/#1`#2`#3/<#4>^#5|#6_#7{\:%
\ifnum0=#4%
\setbox0\hbox{$\labelstyle#5$}\wdth=\wd0%
\setbox0\hbox{$\labelstyle#6$}\ifnum\wdth<\wd0 \wdth=\wd0 \fi%
\setbox0\hbox{$\labelstyle#7$}\ifnum\wdth<\wd0 \wdth=\wd0 \fi%
\deltax=\wdth \divide \deltax by \ul \advance \deltax by%
\defaultmargin \ratchet{\deltax}{300}%
\else\deltax#4 \fi%
    \xy \ifnum\wd0=0 \ar@{#2}(\deltax,0)%
    \else \ar@{#2}|{#6}(\deltax,0)\fi%
\ar@{#1}@<4.5pt>^{#5}(\deltax,0)%
\ar@{#3}@<-4.5pt>_{#7}(\deltax,0)\endxy\:}%

\def\threepppp/#1`#2`#3/<#4>^#5|#6{\ifnextchar_{\threeppppp%
  /#1`#2`#3/<#4>^{#5}|{#6}}{\threeppppp/#1`#2`#3/<#4>^{#5}|{#6}_{}}}%
\def\threeppp/#1`#2`#3/<#4>^#5{\ifnextchar|{\threepppp%
  /#1`#2`#3/<#4>^{#5}}{\threepppp/#1`#2`#3/<#4>^{#5}|{}}}%
\def\threepp/#1`#2`#3/<#4>{\ifnextchar^{\threeppp/#1`#2`#3/<#4>}%
  {\threeppp/#1`#2`#3/<#4>^{}}}%
\def\threep/#1`#2`#3/{\ifnextchar<{\threepp/#1`#2`#3/}%
  {\threepp/#1`#2`#3/<0>}}%
\def\three{\ifnextchar/{\threep}{\threep/>`>`>/}}%


\def\twoar(#1,#2){{%
 \scalefactor{0.1}%
 \deltax#1\deltay#2%
 \deltaX=\ifnum\deltax<0-\fi\deltax%
 \deltaY=\ifnum\deltay<0-\fi\deltay%
 \Xend\deltax \multiply \Xend by \deltax%
 \Yend\deltay \multiply \Yend by \deltay%
 \advance\Xend by \Yend \multiply \Xend by 3%
 \ifnum \deltaX > \deltaY%
    \multiply \deltaX by 3 \advance \deltaX by \deltaY%
 \else%
    \multiply \deltaY by 3 \advance \deltaX by \deltaY%
 \fi%
 \multiply\deltax by 500%
 \multiply\deltay by 500%
 \xpos\deltax \multiply \xpos by 3 \divide\xpos by \deltaX%
 \Xpos\deltax \multiply \Xpos by \deltaX \divide \Xpos by \Xend%
 \advance \xpos by \Xpos%
 \ypos\deltay \multiply \ypos by 3 \divide\ypos by \deltaX%
 \Ypos\deltay \multiply \Ypos by \deltaX \divide \Ypos by \Xend%
 \advance \ypos by \Ypos%
 \xy \ar@{=>}(\xpos,\ypos) \endxy%
}\ignorespaces}%

\def\iiixiiipppppp(#1,#2)|#3|/#4/<#5>#6<#7>[#8;#9]{%
 \xpos#1\ypos#2\relax%
 \def\next|##1##2##3##4##5##6##7|{\def\xa{##1}\def\xb{##2}%
 \def\xc{##3}\def\xd{##4}\def\xe{##5}\def\xf{##6}\nextt|##7|}%
 \def\nextt|##1##2##3##4##5##6|{\def\xg{##1}\def\xh{##2}%
 \def\xi{##3}\def\xj{##4}\def\xk{##5}\def\xl{##6}}%
 \next|#3|%
 \def\next<##1,##2>{\deltax##1\deltay##2}%
 \next<#5>%
 \def\next<##1,##2>{\deltaX##1\deltaY##2}%
 \next<#7>%
 \def\next##1{\topw##1\relax%
 \ifodd\topw \def\zl{}\else\def\zl{\relax}\fi \divide\topw by 2
 \ifodd\topw \def\zk{}\else\def\zk{\relax}\fi \divide\topw by 2
 \ifodd\topw \def\zj{}\else\def\zj{\relax}\fi \divide\topw by 2
 \ifodd\topw \def\zi{}\else\def\zi{\relax}\fi \divide\topw by 2
 \ifodd\topw \def\zh{}\else\def\zh{\relax}\fi \divide\topw by 2
 \ifodd\topw \def\zg{}\else\def\zg{\relax}\fi \divide\topw by 2
 \ifodd\topw \def\zf{}\else\def\zf{\relax}\fi \divide\topw by 2
 \ifodd\topw \def\ze{}\else\def\ze{\relax}\fi \divide\topw by 2
 \ifodd\topw \def\zd{}\else\def\zd{\relax}\fi \divide\topw by 2
 \ifodd\topw \def\zc{}\else\def\zc{\relax}\fi \divide\topw by 2
 \ifodd\topw \def\zb{}\else\def\zb{\relax}\fi \divide\topw by 2
 \ifodd\topw \def\za{}\else\def\za{\relax}\fi}%
 \next{#6}%
 \def\next[##1`##2`##3`##4`##5`##6`##7`##8`##9]{%
 \def\nodea{##1}\def\nodeb{##2}\def\nodec{##3}%
 \def\noded{##4}\def\nodee{##5}\def\nodef{##6}%
 \def\nodeg{##7}\def\nodeh{##8}\def\nodei{##9}}%
 \next[#8]%
 \def\next[##1`##2`##3`##4`##5`##6`##7]{%
 \def\labela{##1}\def\labelb{##2}\def\labelc{##3}%
 \def\labeld{##4}\def\labele{##5}\def\labelf{##6}\nextt[##7]}%
 \def\nextt[##1`##2`##3`##4`##5`##6]{%
 \def\labelg{##1}\def\labelh{##2}\def\labeli{##3}%
 \def\labelj{##4}\def\labelk{##5}\def\labell{##6}}%
 \next[#9]%
 \def\next/##1`##2`##3`##4`##5`##6`##7/{%
\morphism(\xpos,\ypos)|\xe|/{##5}/<\deltax,0>[\nodeg`\nodeh;\labele]%
 \ifx\zi\empty\relax \morphism(\xpos,\ypos)||/<-/<-\deltaX,0>[\nodeg`0;]\fi%
 \ifx\zd\empty\relax \morphism(\xpos,\ypos)||<0,-\deltaY>[\nodeg`0;]\fi%
 \advance\xpos by \deltax%
 \morphism(\xpos,\ypos)|\xf|/{##6}/<\deltax,0>[\nodeh`\nodei;\labelf]%
 \ifx\ze\empty\relax \morphism(\xpos,\ypos)||<0,-\deltaY>[\nodeh`0;]\fi%
 \advance\xpos by \deltax%
 \ifx\zf\empty\relax \morphism(\xpos,\ypos)||<0,-\deltaY>[\nodei`0;]\fi%
 \ifx\zl\empty\relax \morphism(\xpos,\ypos)||<\deltaX,0>[\nodei`0;]\fi%
 \advance\ypos by \deltay%
 \ifx\zk\empty\relax \morphism(\xpos,\ypos)||<\deltaX,0>[\nodef`0;]\fi%
 \advance\xpos by -\deltax%
 \morphism(\xpos,\ypos)|\xd|/{##4}/<\deltax,0>[\nodee`\nodef;\labeld]%
 \advance\xpos by -\deltax%
 \morphism(\xpos,\ypos)|\xc|/{##3}/<\deltax,0>[\noded`\nodee;\labelc]%
 \ifx\zh\empty\relax \morphism(\xpos,\ypos)||/<-/<-\deltaX,0>[\noded`0;]\fi%
 \advance\ypos by \deltay%
 \morphism(\xpos,\ypos)|\xa|/{##1}/<\deltax,0>[\nodea`\nodeb;\labela]%
 \ifx\zg\empty\relax \morphism(\xpos,\ypos)||/<-/<-\deltaX,0>[\nodea`0;]\fi%
 \ifx\za\empty\relax \morphism(\xpos,\ypos)||/<-/<0,\deltaY>[\nodea`0;]\fi%
 \advance\xpos by \deltax%
 \morphism(\xpos,\ypos)|\xb|/{##2}/<\deltax,0>[\nodeb`\nodec;\labelb]%
 \ifx\zb\empty\relax \morphism(\xpos,\ypos)||/<-/<0,\deltaY>[\nodeb`0;]\fi%
 \advance\xpos by \deltax%
 \ifx\zc\empty\relax \morphism(\xpos,\ypos)||/<-/<0,\deltaY>[\nodec`0;]\fi%
 \ifx\zj\empty\relax \morphism(\xpos,\ypos)||<\deltaX,0>[\nodec`0;]\fi%
 \nextt/##7/}%
 \def\nextt/##1`##2`##3`##4`##5`##6/{%
 \morphism(\xpos,\ypos)|\xi|/{##3}/<0,-\deltay>[\nodec`\nodef;\labeli]%
 \advance\xpos by -\deltax%
 \morphism(\xpos,\ypos)|\xh|/{##2}/<0,-\deltay>[\nodeb`\nodee;\labelh]%
 \advance\xpos by -\deltax%
 \morphism(\xpos,\ypos)|\xg|/{##1}/<0,-\deltay>[\nodea`\noded;\labelg]%
 \advance\ypos by -\deltay%
 \morphism(\xpos,\ypos)|\xj|/{##4}/<0,-\deltay>[\noded`\nodeg;\labelj]%
 \advance\xpos by \deltax%
 \morphism(\xpos,\ypos)|\xk|/{##5}/<0,-\deltay>[\nodee`\nodeh;\labelk]%
 \advance\xpos by \deltax%
 \morphism(\xpos,\ypos)|\xl|/{##6}/<0,-\deltay>[\nodef`\nodei;\labell]}%
 \next/#4/\ignorespaces}%

\def\iiixiii{\ifnextchar({\iiixiiip}{\iiixiiip(0,0)}}%
\def\iiixiiip(#1){\ifnextchar|{\iiixiiipp(#1)}%
  {\iiixiiipp(#1)|aammbblmrlmr|}}%
\def\iiixiiipp(#1)|#2|{\ifnextchar/{\iiixiiippp(#1)|#2|}%
    {\iiixiiippp(#1)|#2|/>`>`>`>`>`>`>`>`>`>`>`>/}}%
\def\iiixiiippp(#1)|#2|/#3/{%
    \ifnextchar<{\iiixiiipppp(#1)|#2|/#3/}%
    {\iiixiiipppp(#1)|#2|/#3/<\default,\default>}}%
\def\iiixiiipppp(#1)|#2|/#3/<#4>{\ifnextchar[{\iiixiiippppp(#1)|#2|/#3/%
   <#4>0<0,0>}{\iiixiiippppp(#1)|#2|/#3/<#4>}}%
\def\iiixiiippppp(#1)|#2|/#3/<#4>#5{\ifnextchar<%
   {\iiixiiipppppp(#1)|#2|/#3/<#4>{#5}}%
   {\iiixiiipppppp(#1)|#2|/#3/<#4>{#5}<400,400>}}%

\def\iiixiipppppp(#1,#2)|#3|/#4/<#5>#6<#7>[#8;#9]{%
 \xpos#1\ypos#2\relax%
 \def\next|##1##2##3##4##5##6##7|{\def\xa{##1}\def\xb{##2}%
 \def\xc{##3}\def\xd{##4}\def\xe{##5}\def\xf{##6}\def\xg{##7}}%
 \next|#3|%
 \def\next<##1,##2>{\deltax##1\deltay##2}%
 \next<#5>%
 \deltaX#7
 \topw#6
 \def\next{%
 \ifodd\topw \def\za{}\else\def\za{\relax}\fi \divide\topw by 2
 \ifodd\topw \def\zb{}\else\def\zb{\relax}\fi \divide\topw by 2
 \ifodd\topw \def\zc{}\else\def\zc{\relax}\fi \divide\topw by 2
 \ifodd\topw \def\zd{}\else\def\zd{\relax}\fi}%
 \next%
 \def\next[##1`##2`##3`##4`##5`##6]{%
 \def\nodea{##1}\def\nodeb{##2}\def\nodec{##3}%
 \def\noded{##4}\def\nodee{##5}\def\nodef{##6}}%
 \next[#8]%
 \def\next[##1`##2`##3`##4`##5`##6`##7]{%
 \def\labela{##1}\def\labelb{##2}\def\labelc{##3}%
 \def\labeld{##4}\def\labele{##5}\def\labelf{##6}\def\labelg{##7}}%
 \next[#9]%
 \def\next/##1`##2`##3`##4`##5`##6`##7/{%
 \ifx\zc\empty\relax\morphism(\xpos,\ypos)<\deltaX,0>[0`\noded;]\fi%
 \advance\xpos by\deltaX%
 \morphism(\xpos,\ypos)|\xc|/##3/<\deltax,0>[\noded`\nodee;\labelc]%
 \advance\xpos by \deltax%
 \morphism(\xpos,\ypos)|\xd|/##4/<\deltax,0>[\nodee`\nodef;\labeld]%
 \advance\xpos by \deltax%
 \ifx\zd\empty\relax  \morphism(\xpos,\ypos)<\deltaX,0>[\nodef`0;]\fi%
 \advance\xpos by -\deltaX  \advance\xpos by -\deltax
 \advance\xpos by -\deltax  \advance\ypos by \deltay
 \ifx\za\empty\relax\morphism(\xpos,\ypos)<\deltaX,0>[0`\nodea;]\fi%
 \advance\xpos by\deltaX%
 \morphism(\xpos,\ypos)|\xa|/##1/<\deltax,0>[\nodea`\nodeb;\labela]%
 \morphism(\xpos,\ypos)|\xe|/##5/<0,-\deltay>[\nodea`\noded;\labele]%
 \advance\xpos by \deltax%
 \morphism(\xpos,\ypos)|\xb|/##2/<\deltax,0>[\nodeb`\nodec;\labelb]%
 \morphism(\xpos,\ypos)|\xf|/##6/<0,-\deltay>[\nodeb`\nodee;\labelf]%
 \advance\xpos by \deltax%
 \morphism(\xpos,\ypos)|\xg|/##7/<0,-\deltay>[\nodec`\nodef;\labelg]%
 \ifx\zb\empty\relax \morphism(\xpos,\ypos)<\deltaX,0>[\nodec`0;]\fi}%
 \next/#4/\ignorespaces}%


\def\iiixii{\ifnextchar({\iiixiip}{\iiixiip(0,0)}}%
\def\iiixiip(#1){\ifnextchar|{\iiixiipp(#1)}%
  {\iiixiipp(#1)|aabblmr|}}%
\def\iiixiipp(#1)|#2|{\ifnextchar/{\iiixiippp(#1)|#2|}%
    {\iiixiippp(#1)|#2|/>`>`>`>`>`>`>/}}%
\def\iiixiippp(#1)|#2|/#3/{%
    \ifnextchar<{\iiixiipppp(#1)|#2|/#3/}%
    {\iiixiipppp(#1)|#2|/#3/<\default,\default>}}%
\def\iiixiipppp(#1)|#2|/#3/<#4>{\ifnextchar[{\iiixiippppp(#1)|#2|/#3/%
   <#4>{0}<0>}{\iiixiippppp(#1)|#2|/#3/<#4>}}%
\def\iiixiippppp(#1)|#2|/#3/<#4>#5{\ifnextchar<%
   {\iiixiipppppp(#1)|#2|/#3/<#4>{#5}}%
   {\iiixiipppppp(#1)|#2|/#3/<#4>{#5}<400>}}%

\def\node#1(#2,#3)[#4]{%
\expandafter\gdef\csname x@#1\endcsname{#2}%
\expandafter\gdef\csname y@#1\endcsname{#3}%
\expandafter\gdef\csname ob@#1\endcsname{#4}%
\ignorespaces}%

\newcount\xfinish%
\newcount\yfinish%
\def\arrow{\ifnextchar|{\arrowp}{\arrowp|a|}}%
\def\arrowp|#1|{\ifnextchar/{\arrowpp|#1|}{\arrowpp|#1|/>/}}%
\def\arrowpp|#1|/#2/[#3`#4;#5]{%
\xfinish=\csname x@#4\endcsname%
\yfinish=\csname y@#4\endcsname%
\advance\xfinish by -\csname x@#3\endcsname%
\advance\yfinish by -\csname y@#3\endcsname%
\morphism(\csname x@#3\endcsname,\csname y@#3\endcsname)|#1|/{#2}/%
<\xfinish,\yfinish>[\csname ob@#3\endcsname`\csname ob@#4\endcsname;#5]%
}%


\def\Loop(#1,#2)#3(#4,#5){\POS(#1,#2)*+!!<0ex,\axis>{#3}\ar@(#4,#5)}%
\def\iloop#1(#2,#3){\xy\Loop(0,0)#1(#2,#3)\endxy}%


\catcode`\@=\atcode%
\endinput%
\entrymodifiers={+!!<0pt,\fontdimen22\textfont2>}%



   the \xybox  does not allow control over *where*,%
inside the <object> that it builds,%
the reference point is to be located.%

Accordingly, I've just devised a variant that builds%
the same kind of compound <object>, but also sets%
its reference-point to be at the <coord> of the%
last <POS> within the box; i.e., the <coord> for%
the <object> that has been built is at the current%
<POS> when the Xy-pic parsing has been completed.%
The LRUD extents are the size of the complete box;%
i.e., *not* the extents of the final <POS>.%

Here is coding that should go in your document's%
preamble -- eventually it should be added to  xy.tex%



---------  start of new Xy-pic definitions  -------%


> \makeatletter   % adjust the \catcode of @%

 this is a better definition for the new  \xyobjbox%

   \xydef@\xyobjbox#1{\xy%
     \let \PATHafterPOS\PATHafterPOS@default%
     \let \arsavedPATHafterPOS@@\relax%
     \let\afterar@@\relax%
     \POS#1\endxyobj\Edge@c={\rectangleEdge}\computeLeftUpness@}%

> \xydef@\endxyobj{\if\inxy@\else\xyerror@{Unexpected \string\endxy}{}\fi%
>  \relax%
>   \dimen@=\Y@max \advance\dimen@-\Y@min%
>   \ifdim\dimen@<\z@ \dimen@=\z@ \Y@min=\z@ \Y@max=\z@ \fi%
>   \dimen@=\X@max \advance\dimen@-\X@min%
>   \ifdim\dimen@<\z@ \dimen@=\z@ \X@min=\z@ \X@max=\z@ \fi%
>   \edef\tmp@{\egroup%
>     \setboxz@h{\kern-\the\X@min \boxz@}%
>     \ht\z@=\the\Y@max \dp\z@=-\the\Y@min \wdz@=\the\dimen@%
>     \noexpand\maybeunraise@ \raise\dimen@\boxz@%
>     \noexpand\recoverXyStyle@ \egroup \noexpand\xy@end%
>     \U@c=\the\Y@max \advance\U@c-\the\Y@c%
>     \D@c=-\the\Y@min \advance\D@c\the\Y@c%
>     \L@c=-\the\X@min  \advance\L@c\the\X@c%
>     \R@c=\the\X@max  \advance\R@c-\the\X@c%
>    }\tmp@}%
>%
> \makeatother   % revert \catcode of @%
>%
> ---------  end of new Xy-pic definitions  -------%
\makeatletter%
\gdef\xymerge@MinMax{}%
\xydef@\twocell{\hbox\bgroup\xysave@MinMax\@twocell}%
\xydef@\uppertwocell{\hbox\bgroup\xysave@MinMax\@uppertwocell}%
\xydef@\lowertwocell{\hbox\bgroup\xysave@MinMax\@lowertwocell}%
\xydef@\compositemap{\hbox\bgroup\xysave@MinMax\@compositemap}%
\xydef@\twocelll#1#{\hbox\bgroup\xysave@MinMax\xy@\save\save@\@twocelll{%
#1}}%

\xydef@\xysave@MinMax{\xdef\xymerge@MinMax{%
   \noexpand\ifdim\X@max<\the\X@max \X@max=\the\X@max\noexpand\fi%
   \noexpand\ifdim\X@min>\the\X@min \X@min=\the\X@min\noexpand\fi%
   \noexpand\ifdim\Y@max<\the\Y@max \Y@max=\the\Y@max\noexpand\fi%
   \noexpand\ifdim\Y@min>\the\Y@min \Y@min=\the\Y@min\noexpand\fi%
  }}%
\xydef@\drop@Twocell{\boxz@ \xymerge@MinMax}%


\xydef@\twocell@DONE{%
  \edef\tmp@{\egroup%
   \X@min=\the\X@min \X@max=\the\X@max%
   \Y@min=\the\Y@min \Y@max=\the\Y@max}\tmp@%
  \L@c=\X@c \advance\L@c-\X@min \R@c=\X@max \advance\R@c-\X@c%
  \D@c=\Y@c \advance\D@c-\Y@min \U@c=\Y@max \advance\U@c-\Y@c%
  \ht\z@=\U@c \dp\z@=\D@c \dimen@=\L@c \advance\dimen@\R@c \wdz@=\dimen@%
  \computeLeftUpness@%
  \setboxz@h{\kern-\X@p \raise-\Y@c\boxz@ }%
  \dimen@=\L@c \advance\dimen@\R@c \wdz@=\dimen@ \ht\z@=\U@c \dp\z@=\D@c%
  \Edge@c={\rectangleEdge}\Invisible@false \Hidden@false%
  \edef\Drop@@{\noexpand\drop@Twocell%
   \noexpand\def\noexpand\Leftness@{\Leftness@}%
   \noexpand\def\noexpand\Upness@{\Upness@}}%
  \edef\Connect@@{\noexpand\connect@Twocell%
   \noexpand\ifdim\X@max<\the\X@max \X@max=\the\X@max\noexpand\fi%
   \noexpand\ifdim\X@min>\the\X@min \X@min=\the\X@min\noexpand\fi%
   \noexpand\ifdim\Y@max<\the\Y@max \Y@max=\the\Y@max\noexpand\fi%
   \noexpand\ifdim\Y@min>\the\Y@min \Y@min=\the\Y@min\noexpand\fi }%
  \xymerge@MinMax%
}%
\makeatother%




% \usepackage{makeindex}

%%%%
% Theorem-type environments
%%%%

%% following Cisinski's style, which I found excellent, the theorem-like environments are set up to number _all_ paragraphs [in the conceptual rather than typographic sense] consecutively.  the major advantage of this is making any paragraph referenceable, and hence making the (always rather arbitrary) decision of what to pick out as theorems, definitions, etc. much less consequential and more flexible.


\makeatletter

\newtheoremstyle{mytheorem}{}{}{\itshape}{}{\bfseries}{.}{5\p@ plus\p@ minus\p@}{}

\newtheoremstyle{mydefinition}{}{}{}{}{\bfseries}{.}{5\p@ plus\p@ minus\p@}{}

%% proof environment taken almost verbatim from amsthm.sty, to remove the small caps and indentation that are used in amsbook.cls
\renewenvironment{proof}[1][Proof]{\par
  \pushQED{\qed}%
  \normalfont \topsep6\p@\@plus6\p@\relax
  \trivlist
  \item[\hskip\labelsep
        \itshape
    #1\@addpunct{.}]\ignorespaces
}{%
  \popQED\endtrivlist\@endpefalse
}

\makeatother



\theoremstyle{mytheorem} 
\newtheorem{thm}{Theorem}[section]
\newtheorem{theorem}[thm]{Theorem}
\newtheorem{proposition}[thm]{Proposition}
\newtheorem{lemma}[thm]{Lemma}
\newtheorem{corollary}[thm]{Corollary}
\newtheorem{scholium}[thm]{Scholium}
\newtheorem{conjecture}[thm]{Conjecture}

\theoremstyle{mydefinition}
\newtheorem{definition}[thm]{Definition}
\newtheorem{para}[thm]{}
\newtheorem{exercise}[thm]{Exercise}

%\theoremstyle{remark}
\newtheorem{remark}[thm]{Remark}
\newtheorem{notation}[thm]{Notations}
\newtheorem{example}[thm]{Example}
\newtheorem{examples}[thm]{Examples}

\newtheorem{mydefinition}[thm]{Definition}


\setcounter{tocdepth}{3}
\setcounter{secnumdepth}{2}

\renewcommand{\baselinestretch}{1.5}


% Peter LeFanu Lumsdaine, June 2010
% macros for my thesis

% Contents:
%
% - Binary relations
% - Category names
% - Single letters


%%%%
% Binary relations, operators
%%%%

% \newcommand{\coslice}{\!\!\; \backslash \!\!\!\; \backslash}
\newcommand{\coslice}{\!\!\; \backslash}
\newcommand{\cotensor}{\pitchfork}
\renewcommand{\equiv}{\simeq}
\newcommand{\Iff}{\Leftrightarrow}
\newcommand{\Imp}{\Rightarrow}
\newcommand{\into}{\hookrightarrow}
\newcommand{\iso}{\cong}
\newcommand{\propeq}{\approx}
\newcommand{\mono}{\to/ >->/}
\newcommand{\slice}{\!\!\; / \!\!\!\; /}
\newcommand{\tensor}{\otimes}
\newcommand{\tightcdot}{\! \cdot \!}
\newcommand{\tightcolon}{\!\!\, : \!\!\,}
\newcommand{\To}{\Rightarrow}
\newcommand{\types}{\vdash}

%%%% 
% Single styled characters (or almost single) and character-like symbols
%%%%

\newcommand{\Two}{\mathbf{2}}
\newcommand{\A}{A_\bullet}
\newcommand{\abar}{\overline{a}}
\newcommand{\uA}{\underline{A}}
\newcommand{\uAbu}{\underline{A}_\bullet}
\newcommand{\B}{B_\bullet}
% \newcommand{\ML}{\mathit{ML_I}}
% \newcommand{\MLfrag}{\mathit{ML}^\Id}
\renewcommand{\c}{\vec c}
\newcommand{\C}{\mathcal{C}}
\newcommand{\CC}{\mathbb{C}}
\newcommand{\Cbf}{\mathcal{C}}
\newcommand{\Cbu}{C_\bullet}
\newcommand{\D}{\mathcal{D}}
% \newcommand{\bC}{\mathbf{C}}
% \newcommand{\Chat}{\widehat{\mathbb{C}}}
% \newcommand{\D}{\mathbb{D}}
% \newcommand{\bigD}{\mathcal{D}}
% \newcommand{\bD}{\mathbf{D}}
\newcommand{\diag}{\delta}
% \renewcommand{\d}{\partial}
\newcommand{\E}{\mathcal{E}}
\newcommand{\Ebu}{\E_\bullet}
\newcommand{\f}{\vec f}
\newcommand{\fbf}{\mathbf{f}}
\newcommand{\F}{\mathcal{F}}
\newcommand{\FF}{\mathbb{F}}
\newcommand{\g}{\vec g}
\newcommand{\gbf}{\mathbf{g}}
\newcommand{\G}{\mathbb{G}}
\newcommand{\h}{\vec h}
\newcommand{\I}{\mathcal{I}}   % generating cofibrations.  mathscr is prettier,
\newcommand{\J}{\mathcal{J}}   % but I find its I, J confusing.
\newcommand{\K}{\mathcal{K}}    % A class of left maps
\renewcommand{\L}{\mathcal{L}}    % A class of left maps
\newcommand{\N}{\mathbb{N}}   % Natural numbers
\newcommand{\NN}{\mathbb{N}}   % ditto
\newcommand{\Ncal}{\mathcal{N}}   % Nerve
% \renewcommand{\P}{P_{\MLfrag}}
\newcommand{\PML}{P_{\MLId}}
% \newcommand{\Pfull}{P_{\ML}}
\newcommand{\PARA}{\textparagraph}
\newcommand{\pow}{\mathcal{P}}
\newcommand{\p}{\vec p}
\newcommand{\SEC}{\textsection}
\newcommand{\R}{\mathcal{R}}    % A class of right maps
\renewcommand{\r}{\vec r}
\renewcommand{\S}{\textsf{\textbf{S}}}    % Another generic type theory
\newcommand{\T}{\textbf{\textsf{T}}}      % A generic type theory
\newcommand{\Tcal}{\mathcal{T}}      % A 2-monad
\newcommand{\TT}{\mathbb{T}}    % A generic type theory, seen as a categorical structure
\newcommand{\Tsf}{\mathsf{T}}
\newcommand{\tsf}{\mathsf{t}}
\renewcommand{\u}{\vec u}
\newcommand{\V}{\mathcal{V}}
% \renewcommand{\v}{\vec v}
\newcommand{\W}{\mathcal{W}}
\newcommand{\WW}{\mathbb{W}}
\newcommand{\w}{\vec w}
\newcommand{\Xcal}{\mathcal{X}}
\newcommand{\X}{X_\bullet}
\newcommand{\Xbullet}{X_\bullet}
\newcommand{\x}{\vec x}
\newcommand{\uX}{\underline{X}}
\newcommand{\uXbu}{\underline{X}_\bullet}
\newcommand{\Ycal}{\mathcal{Y}}
\newcommand{\Y}{Y_\bullet}
\newcommand{\y}{\vec y}
\newcommand{\yon}{\mathbf{y}}
\newcommand{\z}{\vec z}

%%%%
% Styled words: general
%%%%

\newcommand{\Alg}[1]{#1\mbox{-}\mathbf{Alg}}
\newcommand{\IntAlg}[2]{\mathbf{Alg}_{#2}(#1)}
\newcommand{\AMS}{AMS}
\newcommand{\AWFS}{AWFS}
\newcommand{\Arr}{\mathrm{Arr}}
\newcommand{\Cat}{\mathbf{Cat}}
\newcommand{\intCat}[1][-]{\mathbf{Cat}(#1)}
\newcommand{\enrCat}[1][\V]{#1\mbox{-}\mathbf{Cat}}
\newcommand{\nCat}[1][n]{#1\mbox{-}\mathbf{Cat}}
\newcommand{\cl}{\mathcal{C}\! \ell}
\newcommand{\clpi}{{}^\Pi \cl}
\newcommand{\ClovFib}{\mathbf{ClovFib}}
\newcommand{\Coll}{\mathbf{Coll}}
\newcommand{\CwA}{\mathbf{CwA}}
\newcommand{\CwAId}{\mathbf{CwA}^{\Id}}
\newcommand{\CwF}{\mathbf{CwF}}
\newcommand{\CwFId}{\mathbf{CwF}^{\Id}}
\newcommand{\Cxt}{\mathrm{Cxt}}
\newcommand{\cxl}{\mathit{cxl}}
\newcommand{\CofCosps}{\mathbf{CofCosps}}
\newcommand{\cod}{\mathrm{cod}}
\newcommand{\cof}{\mathit{cof}}
\newcommand{\colim}{\varinjlim}
\newcommand{\del}{\partial}
\newcommand{\dom}{\mathrm{dom}}
\newcommand{\DTT}{\mathbf{DTT}}
\newcommand{\End}{\mathrm{End}}
% \newcommand{\ev}{\mathbf{ev}}
\newcommand{\Fib}{\mathbf{Fib}}
\newcommand{\FibSpans}{\mathbf{FibSpans}}
\newcommand{\FSCC}{\mathbf{FSCC}}
\newcommand{\fscc}{\textsc{fscc}}
\newcommand{\fsccs}{\textsc{fscc}'s}
\newcommand{\FSCS}{\mathbf{FSCS}}
\newcommand{\fscs}{\textsc{fscs}}
\newcommand{\fscss}{\textsc{fscs}'s}
\newcommand{\globe}[1][n]{\textsf{\textbf{G}}_{#1}}
\newcommand{\piglobe}[1][n]{{}^\Pi \textsf{\textbf{G}}_{#1}}
  % oh dear, I keep reading this as "pig lobe".  It's clearly too late at night…
\newcommand{\globefig}[1]{\globe[#1]}
  % needed because commands with optional[arguments] seem to interact badly with parsing of square brackets inside \bfig...\efig.
\newcommand{\globes}{\textsf{\textbf{G}}_\bullet}
\newcommand{\piglobes}{{}^\Pi \textsf{\textbf{G}}_\bullet}
\newcommand{\maybepiglobes}{{}^{(\Pi)} \textsf{\textbf{G}}_\bullet}
% \newcommand{\longGSets}{[\mathbb{G}^\op,\mathbf{Sets}]}
\newcommand{\GSets}{\widehat{\mathbb{G}}}
\newcommand{\GnSets}[1][n]{\widehat{\mathbb{G}}_{#1}}
\newcommand{\Hom}{\mathrm{Hom}}
\renewcommand{\lim}{\varprojlim}
\newcommand{\Lan}{\mathrm{Lan}}
\newcommand{\lax}{\mathrm{lax}}
\newcommand{\Mon}{\mathbf{Mon}}
\newcommand{\MonGlobCat}{\mathbf{MonGlobCat}}
\newcommand{\ML}{\textsf{\textbf{ML}}}
\newcommand{\MLId}{\textsf{\textbf{ML}}^{\Id}}
\newcommand{\ob}{\operatorname{ob}}
\newcommand{\op}{\mathrm{op}}
\newcommand{\nOpd}[1][n]{#1\mbox{-}\mathbf{Opd}}
% \newcommand{\Operads}{\mathbf{Operads}}
\newcommand{\pd}{\mathbf{pd}}
\newcommand{\pathpd}{\mathit{path}}
\newcommand{\PsAlg}[2][]{\mathbf{Ps}_{#1}\mbox{-}{#2}\mbox{-}\mathbf{Alg}}
\newcommand{\QCat}{\mathbf{QCat}}
\newcommand{\qcat}{\mathit{qcat}}
\newcommand{\Ran}{\mathrm{Ran}}
% \newcommand{\RefGlob}[1][]{\mathbf{RefGlob}_{#1}}
% \newcommand{\RefOneGlob}[1][]{\mathbf{Ref}_1\mathbf{Glob}_{#1}}
\newcommand{\RefnGlob}[2]{\mathbf{Ref}_{#1}\mathbf{Glob}(#2)}
\newcommand{\Sets}{\mathbf{Sets}}
\newcommand{\Spans}[1][]{\mathbf{Spans}(#1)}
\newcommand{\Spansplain}{\mathbf{Spans}}
\newcommand{\str}{\mathrm{str}}
\newcommand{\strat}{\textrm{strat}}
\newcommand{\pdfDTT}{\texorpdfstring{$\DTT$}{DTT}}
\newcommand{\pdfeta}{\texorpdfstring{$\eta$}{η}}
\newcommand{\pdfId}{\texorpdfstring{$\Id$}{Id}}
\newcommand{\pdfJbar}{\texorpdfstring{$\Jbar$}{J-bar}}
\newcommand{\pdfPi}{\texorpdfstring{$\Pi$}{Π}}
\newcommand{\pdfomega}{\texorpdfstring{$\omega$}{ω}}
\renewcommand{\th}{\mathbf{th}}
\newcommand{\Th}{\mathbf{Th}}
\newcommand{\ThId}{\mathbf{Th}^{\Id}}
\newcommand{\ThIdPi}{\mathbf{Th}^{\Id,\Pi}}
\newcommand{\Tm}{\mathrm{Tm}}
\newcommand{\tm}{\textrm{tm}}
\newcommand{\Top}{\mathbf{Top}}
\newcommand{\Ty}{\mathrm{Ty}}
\newcommand{\ty}{\textrm{ty}}
\newcommand{\strMonGlobCat}{\mathbf{MonGlobCat}}
\newcommand{\strwCat}{\mathbf{str}\mbox{-}\omega\mbox{-}\mathbf{Cat}}
\newcommand{\strnCat}[1][n]{\mathbf{str}\mbox{-}#1\mbox{-}\mathbf{Cat}}
\newcommand{\SynPres}{\mathbf{SynPres}}
\newcommand{\SynThy}{\mathbf{SynThy}}
\newcommand{\wkwCat}{\mathbf{wk}\mbox{-}\omega\mbox{-}\mathbf{Cat}}
\newcommand{\wkwGpd}{\mathbf{wk}\mbox{-}\omega\mbox{-}\mathbf{Gpd}}
\newcommand{\wknCat}[1][n]{\mathbf{wk}\mbox{-}#1\mbox{-}\mathbf{Cat}}


%%%%
% Styled words: type theory syntax
%%%%

\newcommand{\app}{\textsc{app}}
\newcommand{\apprule}{\textsc{app}}
\newcommand{\Bool}{\mathsf{Bool}}
\newcommand{\cellrule}{\mathsf{cell}}
\newcommand{\comp}{\textsc{comp}}
\newcommand{\CONG}{\textsc{cong}}
\newcommand{\contr}{\textsc{contr}}
\newcommand{\cons}{\textsc{cons}}
\newcommand{\Cterm}{\mathsf{C}}
\newcommand{\cterm}{\mathsf{c}}
\newcommand{\cxt}{\mathsf{cxt}}
\newcommand{\cxtmap}{\mathsf{cxtmap}}
\newcommand{\cxtrule}{\cxt}
\newcommand{\defrule}{\textsc{def}}
\newcommand{\exch}{\textsc{exch}}
\newcommand{\elim}{\textsc{elim}}
\newcommand{\empt}{\textsc{empty}}
\newcommand{\extrule}{\textsc{ext}}
\newcommand{\extterm}{\mathsf{ext}}
\newcommand{\form}{\textsc{form}}
\newcommand{\Id}{\mathsf{Id}}
% \newcommand{\varidelim}[5]{#4\mathsf{ for }#3\mathsf{ in }#1.#2\mathsf{ via }#5}
% \newcommand{\idelim}[5]{J_{#1.#2}(#3,#4,#5)}
\newcommand{\Jterm}{\mathsf{J}}
\newcommand{\Jbar}{\overline{J}}
\newcommand{\Kbar}{\overline{K}}
\newcommand{\Jbarrule}{\overline{J}\mbox{-}2\mbox{-}\textsc{o}}
\newcommand{\Jbarterm}{\overline{\mathsf{J}}}
\newcommand{\Kterm}{\mathsf{K}}
\newcommand{\LF}{\mathrm{LF}}
\newcommand{\Lterm}{\mathsf{L}}
\newcommand{\Lang}{\textbf{\textsf{Lang}}}
\newcommand{\intro}{\textsc{intro}}
\newcommand{\PiIdelim}{{\Pi\mbox{-}\Id\mbox{-}\elim}}
\newcommand{\Piext}{{\Pi\mbox{-}\extrule}}
\newcommand{\Piextapp}{{\Pi\mbox{-}\extrule\mbox{-}\apprule}}
\newcommand{\PiIdcomp}{{\Pi\mbox{-}\Id\mbox{-}\comp}}
\newcommand{\Nat}{\mathsf{Nat}}
\newcommand{\One}{\mathsf{1}}
\newcommand{\prop}{\textsc{prop}}
\newcommand{\refl}{\textsc{refl}}
\newcommand{\sourcerule}{\mathsf{src}}
\newcommand{\subst}{\textsc{subst}}
\newcommand{\substterm}{\mathsf{subst}}
\newcommand{\src}{\mathsf{src}}
\newcommand{\Sterm}{\mathsf{S}}
\newcommand{\sterm}{\mathsf{s}}
% \newcommand{\scterm}{\textsc{term}}
\newcommand{\sym}{\textsc{sym}}
\newcommand{\targetrule}{\mathsf{tgt}}
\newcommand{\termrule}{\textsc{term}}
\newcommand{\term}{\mathsf{term}}
\newcommand{\tgt}{\mathsf{tgt}}
\newcommand{\trans}{\textsc{trans}}
\newcommand{\Tterm}{\mathsf{T}}
\newcommand{\tterm}{\mathsf{t}}
\newcommand{\type}{\mathsf{type}}
\newcommand{\typerule}{\textsc{type}}
% \newcommand{\sctype}{\textsc{type}}
\newcommand{\wkg}{\textsc{wkg}}
\newcommand{\var}{\textsc{var}}
\newcommand{\Zero}{\mathsf{0}}

%%%%
% Other operators
%%%%

\newcommand{\Clw}{\mathbf{Cl}_\omega}
\newcommand{\ClwQCat}{\mathbf{Cl}^\qcat_\omega}


%%%%
% Other symbols
%%%%

\newcommand{\lscott}{[\![}
\newcommand{\rscott}{]\!]}


%%%
%%% Diagram annotations, work with diagxy
%%%


%\newdir{|>}{!/4.7pt/@{|}
%        *:(1,-.2)\dir^{>}
%        *:(1,+.2)@_{>}}

\newdir{|>}{!/4.5pt/@{|}*:(1,-.2)@^{>}*:(1,+.2)@_{>}}
       
\newdir{ |>}{{}*!/-1pt/@{|}*!/-6pt/:(1,-.2)@^{>}*!/-6pt/:(1,+.2)@_{>}}

\newbox\pbbox
\setbox\pbbox=\hbox{\xy \POS(75,0)\ar@{-} (0,0) \ar@{-} (75,75)\endxy}
\def\pb{\copy\pbbox}
\newbox\urpbbox
\setbox\urpbbox=\hbox{\xy \POS(0,0)\ar@{-} (75,0) \ar@{-} (0,75)\endxy}
\def\urpb{\copy\urpbbox}
\newbox\upbbox
\setbox\upbbox=\hbox{\xy \POS(0,0)\ar@{-} (-40,40) \ar@{-} (40,40)\endxy}
\def\upb{\copy\upbbox}
\newbox\pobox
\setbox\pobox=\hbox{\xy \POS(0,75)\ar@{-} (0,0) \ar@{-} (75,75) \endxy}
\def\po{\copy\pobox}

% \newbox\tiltvdashbox
% \setbox\tiltvdashbox{\xy \POS( 

%% typical usage:
%
% $$\bfig \square[A`B`C`D;```]
% \place(100,400)[\pb]
% \place(400,100)[\po]
% \efig$$

% \newcommand{\todo}[1]{\marginpar{\textcolor{red}{#1}}}
\newcommand{\todo}[1]{}
\newcommand{\oldtodo}[1]{\todo{#1}}
% \newcommand{\padding}[1]{\textcolor{purple}{#1}}
% \newcommand{\comment}[1]{\textcolor{blue}{#1}}
\newcommand{\comment}[1]{}


\newcommand{\CompCat}{\mathbf{CompCat}}

\newcommand{\arr}{\mathrm{arr}}
\newcommand{\ext}{\mathrm{ext}}
% \newcommand{\Jbar}{\overline{J}}
\newcommand{\tree}{\mathrm{tree}}
\newcommand{\tr}{\mathrm{tr}}
\newcommand{\stuff}{{\Phi}}
% \makeindex

%%
%% PDFJUNK
%% Can add /CreationDate, /Creator, /Subject, /Keywords
%%
\ifpdf
\pdfinfo{
  /Author (Peter LeFanu Lumsdaine) 
  /Title (TODO: put thesis title here when decided!)
}
\fi
%%
%% BEGIN DOCUMENT:
%\onehalfspacing
\begin{document}



%% TITLE INFORMATION

\title{The classifying weak $\omega$-category of a type theory}

\author[P. LeF. Lumsdaine]{Peter LeFanu Lumsdaine}

\maketitle
\tableofcontents

\section{Globular structures from $\DTT$}

\comment{Give: the globes, and variants of globes; the Kan constructions; fact that pasting diagrams get realised by these; resulting functors $\DTT \to \Alg{\End(\globes)}$.]}

\todo{[Add, here or later, the ``topological'' interpretation of $\Kterm$: declaring the inclusion of a basepoint into a loop to be a homotopy equivalence.]}

\subsection{The type-theoretic globes}

\begin{para} Once again, let $\stuff$ be some set of rules/constructors, including at least the $\Id$-rules.  The \emph{type-theoretic globes} over $\stuff$ are then a sequence of theories $\globe[n]^\stuff$ which play a similar r\^o{}le in $\DTT_\stuff$ to that which the discs $D^n$ play in $\Top$: they are an internal weak-$\omega$-cocategory, and as such will---almost---be representing objects for the classifying weak $\omega$-category functor.
\end{para}

\begin{definition} $\globe[n]^\stuff$ is the theory over $\stuff$ generated by axioms $i$-$\sourcerule$, $i$-$\targetrule$ (for $0 \leq i < n$), and $n$-$\cellrule$, as follows:
$$
\inferrule*[right={0-$\sourcerule$}]{\ }{\diamond \types A\ \type} \qquad 
\inferrule*[right={0-$\targetrule$}]{\ }{\diamond \types B\ \type} \qquad 
\inferrule*[right={0-$\cellrule$}]{\ }{\diamond \types C\ \type}
$$
$$ 
\inferrule*[right={1-$\sourcerule$}]{\Gamma \types a : A}{\Gamma \types s_1(a): B} \qquad
\inferrule*[right={1-$\targetrule$}]{\Gamma \types a : A}{\Gamma \types t_1(a): B} \qquad
\inferrule*[right={1-$\cellrule$}]{\Gamma \types a : A}{\Gamma \types c_1(a): B} 
$$
$$
\inferrule*[right={$i$-$\sourcerule$}]{\Gamma \types a : A}{\Gamma \types s_i(a):\Id(s_{i-1}(a),t_{i-1}(a))} \qquad
$$
and $i$-$\targetrule$ , $i$-$\cellrule$\ exactly as $i$-$\sourcerule${} except with term-formers $t_i$, $c_i$ in place of $s_i$.
\end{definition}

Often, when working with some fixed $\stuff$, we will write just $\globe[n]$.

\begin{para} There are evident interpretations between these theories, forming a reflexive coglobular object $\globes$ in $\DTT_\stuff$:

$$ \globe[0]\, \three/->`<-`->/<500>\ \globe[1]\, \three/->`<-`->/<500>\ \globe[2]\, \three/->`<-`->/<500> \ \ldots $$

Leaving aside the reflexivity for now, we can thus see the globes as a functor
$$ \globes \colon \GSets \to \DTT_\stuff .$$
Since $\DTT_\stuff$ is co-complete, this induces by general nonsense(\cite[VII.2]{mac-lane-moerdijk}\todo{(??)}) an adjoint pair of functors between $\GSets$ and $\DTT_\stuff$ (a ``Kan situation'').  Both these functors will be of central interest to us in the sequel:
$$\quad \xymatrix{ \GSets \ar@/_/[rrr]_{\T_\stuff [-]\, :=\, \Lan_\yon \globes \qquad \ \, } \ar@{}[rrr]|\top & & & \DTT_\stuff \ar@/_/[lll]_{\cl^-_\omega\ :=\ \DTT_\stuff(\globes,-)} \\ \G \ar@{ >->}[u]^\yon \ar@/_/[urrr]_{\globes} }
$$

The right adjoint, $\cl^-_\omega \colon \DTT_\stuff\ \to\ \GSets$, is defined by homming out of the globes, i.e.\ by setting $\cl^-_\omega(\T)_n = \DTT_\stuff(\globe[n],\T)$.  Thus, by the definitions of the globes, the 0-cells of $\cl^-_\omega(\T)$ correspond exactly to closed types in $\T$; the 1-cells $A \to B$ to terms of $A$ dependent on a single variable from $B$; the 2-cells to terms of type $\Id_B$ between 1-cells; and so on.

This is very nearly, but not quite, what we wanted for the underlying globular set of $\cl_\omega(\T)$.  The difference is that it has only the \emph{types} of $\T$ as 0-cells, not all the contexts; however, we will proceed for now with $\cl^-_\omega$, and remedy this deficiency later.

Meanwhile, the left adjoint $\T_\stuff [-] \colon \GSets\ \to\ \DTT_\stuff$ is constructed as the left Kan extension of $\globes$ along $\yon$, and may be seen as freely adjoining a globular set to $\T_\stuff$, using the globes as templates.  Explicitly, for a globular set $\X$, the theory $\T[\X]$ has an axiom for each cell of $\X$, realising the 0-cells as closed types, the 1-cells as terms between these types, and the higher-cells as terms of appropriate identity types.\footnote{A related construction is considered in \cite{awodey-hofstra-warren} and \cite{hofstra-warren}\todo{[get Awodey--Hofstra--Warren citation]}, corresponding to a slightly different co-globular theory: they omit our $\globe[0]$, giving instead just a single closed base type, and realise $0$-cells as closed terms of this type, $1$-cells as terms of identity type between these, and so forth.  Their $T_\mathbf{ML}$ is then the monad induced by the Kan adjunction.}

 In particular, $\T_\stuff[\yon(n)] = \globe[n]$.  Also useful will be the boundary of the  $n$-globe, $\del \globe[n] := \T_\stuff[\del \yon(n)]$; up to isomorphism, this is the theory given by $i$-$\sourcerule$ and $i$-$\targetrule$, for $0 \leq i < n$, i.e.\ all the axioms of $\globe[n]$ except for $n$-$\cellrule$ itself.
\end{para}
 
\begin{para} Since $\DTT_\stuff$ is co-complete, we can consider (by Section \ref{sec:endo-operad}) the co-endomorphism operad of the globes, $\End_{\Spans[\DTT_\stuff^\op]}(\globes)$, or briefly just $\End(\globes)$.  We know that its operations of some shape $\pi$ are given by
$$\End(\globes)(\pi) \iso [\G/n,\DTT_\stuff](\globes \cotensor \hat{\pi},\globes \cotensor \yon(n))$$
and hence, unwinding this formula, consist of pylon diagrams as in Fig.\ \ref{fig:endo-pylons}.

\begin{figure}[htbp]
$$\bfig
%%%%%%%%%%%%%%%%%%%
% left hand pylon %
%%%%%%%%%%%%%%%%%%%
\node gn(250,0)[\globefig{n}]
\node gn1l(0,-250)[\globefig{n-1}]
\node gn1r(500,-400)[\globefig{n-1}]
\node gn2l(0,-650)[\globefig{n-2}]
\node fakegn2l(450,-650)[]
\node gn2r(500,-800)[\globefig{n-2}]
\node g1l(0,-1150)[\globefig{1}]
\node g1r(500,-1300)[\globefig{1}]
\node g0l(0,-1550)[\globefig{0}]
\node g0r(500,-1700)[\globefig{0}]
\arrow[gn1l`gn;]
\arrow[gn1r`gn;]
\arrow[gn2l`gn1l;]
\arrow[gn2r`gn1l;]
\arrow[gn2l`gn1r;]
\arrow[gn2r`gn1r;]
\arrow/@{}|<>(0.58)\vdots/[g1l`gn2l;]
\arrow/@{}|<>(0.58)\vdots/[g1r`gn2r;]
\arrow[g0l`g1l;]
\arrow[g0r`g1l;]
\arrow[g0l`g1r;]
\arrow[g0r`g1r;]
%%%%%%%%%%%%%%%%%%%%
% right hand pylon %
%%%%%%%%%%%%%%%%%%%%
\node Tpi(1750,0)[{\T_\stuff[\widehat{\pi}]}]
\node Tspi(1500,-250)[{\T_\stuff[\widehat{s\pi}]}]
\node Ttpi(2000,-400)[{\T_\stuff[\widehat{t\pi}]}]
\node Ts2pi(1500,-650)[{\T_\stuff[\widehat{s^2\pi}]}]
\node Tt2pi(2000,-800)[{\T_\stuff[\widehat{t^2\pi}]}]
\node Ts1pi(1500,-1150)[{\T_\stuff[\widehat{s_1\pi}]}]
\node Tt1pi(2000,-1300)[{\T_\stuff[\widehat{t_1\pi}]}]
\node Ts0pi(1500,-1550)[{\T_\stuff[\widehat{s_0\pi}]}]
\node Tt0pi(2000,-1700)[{\T_\stuff[\widehat{t_0\pi}]}]
\arrow[Tspi`Tpi;]
\arrow[Ttpi`Tpi;]
\arrow/@{>}|!{(500,-400);(2000,-400)}\hole/[Ts2pi`Tspi;]
\arrow/@{>}|!{(500,-400);(2000,-400)}\hole/[Tt2pi`Tspi;]
\arrow[Ts2pi`Ttpi;]
\arrow[Tt2pi`Ttpi;]
\arrow/@{}|<>(0.58)\vdots/[Ts1pi`Ts2pi;]
\arrow/@{}|<>(0.58)\vdots/[Tt1pi`Tt2pi;]
\arrow/@{>}|!{(500,-1300);(2000,-1300)}\hole/[Ts0pi`Ts1pi;]
\arrow/@{>}|!{(500,-1300);(2000,-1300)}\hole/[Tt0pi`Ts1pi;]
\arrow[Ts0pi`Tt1pi;]
\arrow[Tt0pi`Tt1pi;]
%%%%%%%%%%%%%%%%%%%%
% connecting wires %
%%%%%%%%%%%%%%%%%%%%
\arrow[gn`Tpi;H]
\arrow/@{>}|!{(250,0);(500,-400)}\hole/[gn1l`Tspi;F_{n-1}]
\arrow[gn1r`Ttpi;G_{n-1}]
\arrow/@{>}|<>(.19)\hole|!{(500,-800);(500,-400)}\hole/[gn2l`Ts2pi;F_{n-2}]
\arrow[gn2r`Tt2pi;G_{n-2}]
\arrow[g1l`Ts1pi;F_1]
\arrow[g1r`Tt1pi;G_1]
\arrow/@{>}|<>(.21)\hole|!{(500,-1700);(500,-1300)}\hole/[g0l`Ts0pi;F_0]
\arrow[g0r`Tt0pi;G_0]
\efig$$
\caption{Operations in an endomorphism operad\label{fig:endo-pylons}}
\end{figure}
\end{para}

\begin{para} By Lemma \ref{lemma:homming-out-of-P-alg}, $\End(\globes)$ acts naturally on $\cl^-_{\omega}$.  This allows us to lift $\cl^-_\omega$ to a functor into $\Alg{L} = \wkwCat$, which by abuse of notation we still denote $\cl^-_\omega$:

$$\bfig
\node DTT(0,0)[\DTT_\stuff]
\node EndGAlg(1100,500)[\Alg{\End(\globes)}]
\node GSets(1400,0)[\GSets]
%\node wkwCat(900,500)[\wkwCat]
\arrow[DTT`GSets;\cl^-_\omega]
\arrow[DTT`EndGAlg;\cl^-_\omega]
%\arrow[EndGAlg`wkwCat;]
\arrow[EndGAlg`GSets;U]
%\arrow[wkwCat`GSets;]
\efig$$

Moreover, since $U$ reflects and the original $\cl^-_\omega$ preserves all limits, so does the lifted $\cl^-_\omega$, and it is easily seen to be finitary; so by the adjoint functor theorem for locally presentable categories \cite[1.66]{adamek-rosicky}, $\cl^-_\omega$ has a left adjoint, realising any $\End{\globes}$-algebra as a theory.  (Its cells are realised as types and terms as under $\T_\stuff[-]$, and the $\End(\globes)$-action specifies various definitional-equality axioms between them.)
\end{para}

\begin{para} As mentioned before, this is not quite what we want; $\cl^-_\omega(\T)$ only has types as objects, where we would like contexts.  To remedy the situation, we can compose with the ``dependent contexts'' endofunctor $(-)^\cxt \slice \diamond$ on $\DTT_\stuff$, and define $\cl(\T) := \cl^-_\omega(\T^\cxt \slice \diamond)$.

Now the objects of $\cl(\T)$ are closed types of $\T^\cxt \slice \diamond$, i.e.\ closed contexts of $\T$, just as we wanted; and the fullness of the functor $\T^\cxt \slice \diamond \to \T$ ensures that higher cells also are as we intended.
\end{para}

\begin{para} \label{para:map-from-pcat} In Section \ref{sec:contractibility} below, we will investigate the question of when $\End(\globes)$ is contractible, or at least of finding a contractible suboperad.  This will require, however, the development of some more type-theoretic machinery.  For now, we may content ourselves with showing that at least in dimensions $\leq 1$, $\End(\globes)$ is very nice.

Specifically, recall that there is a triple adjunction
$$ \nOpd[1] \three/->`<-`->/^{D}|{\tr^1}_{I} \nOpd[\omega]$$
where $DP$ is the \emph{discrete} $\omega$-operad on a 1-operad $P$, with only unit cells (i.e.\ units of the operad structure) in higher dimensions; $\tr^1Q$ is the 1-truncation of an $\omega$-operad $Q$, with $\tr^1(Q)_i = Q_i$ for $i \leq 1$; and $IP$ is the indiscrete $\omega$-operad on $P$, with a unique operation of shape $\pi$ for each pasting diagram $\pi$ together with a choice of $\leq 1$-dimensional source and target from $P$.  \todo{[Check Batanin's terminology for this: (co-)skeleton, (co-)truncation??]}

Now take $P_\Cat \in \nOpd[1]$ to be the operad for categories, i.e.\ the terminal 1-operad.  Then there should be a map $D(P_{\Cat}) \to \End(\globes)$, or equivalently $\psi \colon P_{\Cat} \to \tr^1 \End(\globes)$, so that truncating algebras and pulling back along this map thus induce a map $\psi^* \Alg{\End(\globes)} \to \Cat$, which when applied to $\cl_\omega(\T)$ recovers the classifying category $\cl(\T)$:
$$\bfig
\node DTT(0,0)[\DTT_\stuff]
\node EndGAlg(1000,500)[\Alg{\End(\globes)}]
\node trEndGAlg(2000,500)[\Alg{\tr^1 \End(\globes)}]
\node Cat(2800,500)[\Cat]
\node GSets(1400,0)[\GSets]
\node G1Sets(2400,0)[{\GnSets[1]}]
\arrow[DTT`EndGAlg;\cl_{\omega}]
\arrow[EndGAlg`trEndGAlg;\tr^1]
\arrow[trEndGAlg`Cat;\psi^*]
\arrow[EndGAlg`GSets;]
\arrow[trEndGAlg`G1Sets;]
\arrow[Cat`G1Sets;]
\arrow[DTT`GSets;\cl_{\omega}]
\arrow[GSets`G1Sets;\tr^1]
\arrow/@{>}@/^0em//[DTT`Cat;\cl]
\efig$$

This points us towards an easy abstract construction of the map $\psi$.  We know that the 1-globular object $\globe[0] \two/<-`<-/ \globe[1]$ represents the original classifying category functor $\cl : \DTT_\stuff \to \Cat$; so by the Yoneda lemma, $\globe[0] \two/<-`<-/ \globe[1]$ must carry some co-category structure; but such a structure corresponds exactly to an operad map of the form we want.
\end{para}

\begin{para} \label{para:map-from-cat} However, constructing $\psi$ concretely gives us an excuse to analyse low dimensions of $\End(\globes)$.  A 0-dimensional operation in $\End(\globes)$ is perforce just a unary map $\globe[0] \to \globe[0]$ (there is only one 0-dimensional pasting diagram); so the single 0-dimensional operation in $\Cat$ (the identity on 0-cells) we send to the identity map $1_{\globe[0]}$.  (There is no freedom here: $\psi$ must preserve the operad structure, and $1_{\globe[0]}$ is the $0$-dimensional operad unit of $\End(\globes)$.)

A 1-dimensional pasting diagram is just a path $\path_l = (\cdot \to<200> \cdot \to<200> \ldots \to<200> \cdot)$ of some length $l \geq 0$.  \todo{[Dangit, need a better notation than $\path_l$!]} An operation of shape $\path_l$ in $\End(\globes)$ with source and target $1_{\globe[0]}$ is a map of cospans
$$\bfig
%%%%%%%%%%%%%%%%%%%
% left hand pylon %
%%%%%%%%%%%%%%%%%%%
\node gn(250,0)[\globefig{1}]
\node gn1l(0,-250)[\globefig{0}]
\node gn1r(500,-400)[\globefig{0}]
\arrow[gn1l`gn;s]
\arrow[gn1r`gn;t]
%%%%%%%%%%%%%%%%%%%%
% right hand pylon %
%%%%%%%%%%%%%%%%%%%%
\node Tpi(1750,0)[{\T_\stuff[\widehat{\path_l}]}]
\node Tspi(1500,-250)[{\globefig{0}}]
\node Ttpi(2000,-400)[{\globe[0]}]
\arrow[Tspi`Tpi;s]
\arrow[Ttpi`Tpi;t]
%%%%%%%%%%%%%%%%%%%%
% connecting wires %
%%%%%%%%%%%%%%%%%%%%
\arrow[gn`Tpi;H]
\arrow/@{=}|!{(250,0);(500,-400)}\hole/[gn1l`Tspi;]
\arrow/@{=}/[gn1r`Ttpi;]
\efig$$

But $\T_\stuff(\path_l)$ admits a very simple axiomatisation
$$
\inferrule{\ }{\diamond\ \types\ A_i\ \type} \quad (0 \leq i \leq l) \qquad 
\inferrule{\Gamma\ \types\ a:A_{j-1}}{\Gamma\ \types\ f_j(a) : A_j } \quad (1 \leq j \leq l) 
$$
simply adjoining basic types and type formers
$$ \xymatrix{ A_0 \ar[r]^{f_1} & A_1 \ar[r]^{f_2} & \ \ar@{}[r]|{\ldots} & \ \ar[r]^{f_l} & A_l} .$$

The source and target maps of the right-hand cospan interpret the type of $\globe[0]$ as $A_0$ and $A_l$ respectively; so suitable maps $H : \globe[1] \to \T_\stuff[\widehat{\path_l}]$ correspond to interpretations of the type-constructor of $\globe[1]$ as some term
$$ x : A_0 \ \types\ t(x) : A_l $$
For the unique $l$-ary operation of $P_\Cat$, we thus use the obvious composite term $f_l(f_{l-1}(\ldots (f_1(x))\ldots))$.  It is routine to check that this indeed gives an operad map.
\end{para}

\begin{para} \label{para:canonicity-in-Tpath} When $\stuff$ gives a particularly well-behaved type system, we can say a little more.

In the case when $\stuff$ consists of just the $\Id$-rules, then firstly $\globe[0]$ has no other closed types besides the basic $C$, so $1_{\globe[0]}$ is its only endomorphism, and the only element of $\End(\globes)$ in dimension $0$; and secondly,  (\todo{\cite{canonicity-reference?}}) $\T_\stuff[\widehat{\path_l}]$ enjoys both normalisation and \emph{canonicity}\footnote{canonicity: the property that every closed normal form is an intro (aka canonical) form; there are no stuck (aka neutral) normal forms}, so the ``obvious term'' we used was in fact the only possible such term: there is only one $l$-ary operation in $\End(\globes)$ with source and target $1_{\globe{0}}$.

So in this case, the map $P_\Cat \to \tr^1\End(\globes)$ (always injective, since $P_\Cat$ is terminal) is moreover surjective, and gives an isomorphism $\tr^1\End(\globes) \iso P_\Cat$.

In richer type systems, $\globe[0]$ will typically have more closed types (e.g.\ $C \rightarrow C$), and hence $\End(\globe[0])$ will have more $0$-dimensional operations.  But in important cases, such as when $\stuff$ consists of just the $\Id$- and $\Pi$-rules, we retain normalisation and canonicity for $\T_\stuff[\widehat{\path_l}]$, so by the argument above our map is at least an isomorphism from $P_\Cat$ to the \emph{normalised core} of $\End(\globes)$. \comment{[Make sure to define the normalised core in the globular background!]}
\end{para}




\subsection{A variant for $\Pi$-types}  % A house for Mr. Biswas?  A penny for a song?

\begin{para} We can also consider a variant set of globes ${}^\Pi \globes^\stuff$, for any set of constructors $\stuff$ including $\Pi$-types.  The axioms for each $\globe^\Pi$ are selected from rules $i$-$\sourcerule$-$\Pi$, $i$-$\targetrule$-$\Pi$, $i$-$\cellrule$-$\Pi$, analogously to the axioms for $\globe$.  These axioms differ from before in dimensions $\geq 1$, by using closed rather than open terms:
$$ 
\inferrule*[right={1-$\sourcerule$-$\Pi$}]{\ }{\Gamma \types s_1: S_0 \rightarrow T_0} \qquad
\inferrule*[right={$i$-$\sourcerule$-$\Pi$}]{\ }{\Gamma \types s_i:\Pi_{x:S_0} \Id(s_{i-1}(x),t_{i-1}(x))} \qquad \mbox{etc.}
$$

As before, we get a Kan adjunction
$$ \GSets \two/->`<-/^{\T_\stuff[-]^\Pi}_{{}^\Pi \cl^-_\omega} \DTT_\stuff $$
and lift $\clpi^-_\omega$ to $\Alg{\End({}^\Pi \globes)}$; also as before, we ``correct'' the functor $\clpi^-_\omega$ by precomposing with $(-)^\cxt \slice \diamond$, to get an alternate candidate for the classifying weak $\omega$-category:
$$ \clpi_\omega \colon \DTT_\stuff \to \Alg{\End({}^\Pi \globes)}$$

The objects of $\clpi_\omega (\T)$ are the same as those of $\cl_\omega(\T)$.  The difference is in the higher cells: rather than open context maps, $1$-cells are now context maps from the empty context $\diamond$ into ``$\Pi$-contexts'', and higher cells are context maps from $\diamond$ into the identity contexts over these.

Thus, while not exactly what we first thought of, this is a reasonable alternative candidate for the ``classifying weak $\omega$-category''. 

\todo{[Discuss a little the natural transformations between the two versions?]}
\end{para}

\begin{para} \label{para:canonicity-for-piglobes}
We would also like to repeat the construction of \ref{para:map-from-pcat} and construct a map
$${}^\Pi \psi \colon P_\Cat \to \tr^1 \End(\piglobes),$$
and indeed we can do so, under the further assumption that $\Phi$ also contains the (definitional) $\eta$-rule for $\Pi$-types.  In this case, we interpret the $l$-ary operation using the map $\piglobe[1] \to \T_\Phi[\widehat{path}_l]$ which interprets the basic term $c_1: S_0 \rightarrow T_0$ as the composite term $\lambda x \tightcolon S_0.\, c_l \tightcdot (c_{l-1} \tightcdot \ldots (c_1 (x))\ldots)$.  This certainly preserves the operad composition; the $\eta$-rule required to ensure that it preserves the operad unit, i.e.\ that in the case $l=1$, the resulting operation (sending $c_1$ to $\lambda x.\, c_1 \tightcdot x$) is just the identity on $piglobe[1]$.

In the case where $\stuff$ is exactly $(\Id, \Pi, \Pi\mbox{-}\eta)$, then both normalisation and canonicity hold \todo{\cite{normalisation-reference}}, but $C_0$ is not the only closed type of $\piglobe[0]$.  So ${}^\Pi \psi \colon P_\Cat \to \tr^1 \End(\piglobes)$ is not an isomorphism in this case, as it is not surjective on $0$-cells; but it is at least full on $1$-cells.  
\end{para}











\section{Homotopical structures on $\DTT$}

\comment{Recall from background: orthogonality; other exples of cofibrantly generated wfs'.  Or include these in main text?  No, put them in background.}

\comment{Include: basic extensions; contractible maps; $\Jbar$ and discussion.}

\subsection{Left and right maps in $\DTT$}

\begin{para} In the next section, we'll construct classifying weak $\omega$-categories for theories with $\Id$-types and $\Pi$-types (with $\eta$-rules and functional extensionality), and discuss how it might be possible to extend this to require only the $\Id$-types.

The construction of the classifying weak $\omega$-category of a theory is closely analogous to that of the fundamental weak $\omega$-groupoid of a space: it is obtained by homming out of a complex of representing objects (``globes''), and so it is enough to show that these representing objects form a co-(weak $\omega$-category), just as the topological globes ($D^0 \two D^1 \two D^2 \two \cdots\quad$) do in $\Top$.

To that end, we set up in this section various classes of left and right maps on $\CwA^{\stuff}_\diamond$, and
then apply the machinery of Section \ref{sec:endo-operads} to show that the endomorphism operad of the globes is contractible.

Among this, only one step (showing that certain maps are absolute right maps) seems to require the $\Pi$-types for its proof.  In particular, we isolate and conjecture a certain type-theoretical principle, $\Jbar$, which would suffice for the proof of this step, and which seems to be of independent interest.  In particular, we discuss equivalent natural statements of $\Jbar$ from several rather different points of view: as a conservativity statement for certain theory extensions; as a second-order form of the $\Id$-elim rule; and as a form of observational equality for $\Pi$-types. 
\end{para}


\begin{para}[Type and term extensions]

For the remainder of this section, fix some collection $\stuff$ of the constructors and rules of Subsection \ref{subsec:construtors}, and work in $\DTT_\stuff$.  (The main cases of interest in the sequel are where $\stuff$ is either $(\Id)$, $(\Id,\Pi,\eta)$, or $(\Id,\Pi,\PiIdelim)$.)

For $n \geq 0$, we define theories $\T_\stuff[\Gamma_{(n)}]$, $\T_\stuff [\Gamma_{(n)} \types A]$, and $\T_\stuff [\Gamma_{(n)} \types a : A]$ to be the free theories on, respectively, a context of length $n$; a dependent type, in context of length $n$; and a term in such a type.  Axiomatically, each may be  specified by some subset of the rules below: $\T_\stuff[\Gamma_{(n)}]$ by the rules $i$-$\cxtrule$, for $0 \leq i < n$; $\T_\stuff [\Gamma_{(n)} \types A]$, by these rules together with $n$-$\typerule$; and $\T_\stuff [\Gamma_{(n)} \types a : A]$ by all of the above, together with $n$-$\termrule$:
$$\inferrule*[right={$i$-$\cxtrule$}]{\Gamma \types a_0:A_0\ \ldots\ \Gamma \types a_{i-1}:A_{i-1}}{\Gamma \types A_i(a_0,\ldots,a_{i-1})\ \type} \qquad \inferrule*[right={$i$-$\typerule$}]{\Gamma \types a_0:A_0\ \ldots\ \Gamma \types a_{i-1}:A_{i-1}}{\Gamma \types A(a_0,\ldots,a_{i-1})\ \type}$$
$$\inferrule*[right={$i$-$\termrule$}]{\Gamma \types a_0:A_0\ \ldots\ \Gamma \types a_{i-1}:A_{i-1}}{\Gamma \types a(a_0,\ldots,a_{i-1}) : A_i(a_0,\ldots,a_{i-1})}$$

(Of course, we have $\T[\Gamma_{(n-1)} \types A] \iso \T[\Gamma_{(n)}]$; we retain the distinction just for notational clarity.)
\end{para}

\begin{para} The importance of these theories lies in their universal mapping properties.  For any theory $\T$, maps $\T_\stuff[\Gamma_{(l)}] \to \T$ correspond precisely to contexts of length $n$ in $\T$; maps $\T_\stuff[\Gamma_{(l)} \types A\ \type] \to \T$, to types over such a context; and maps $\T_\stuff[\Gamma_{(l)} \types a:A] \to \T$ to terms of such a type.

An analogy can profitably be drawn here between type theories and higher categories.  Globular higher categories are made up of cells, which are \emph{represented} by the free $n$-categories on individual cells.  Similarly, type theories are made up of judgements---contexts, types, and terms---which are represented by the theories above.

But now, many important apspects of higher category theory---in particular, their homotopical structure---can be described in terms of the inclusions of boundaries into those basic cells.  Much of this carries over substantially to type theories once we observe that \emph{judgements have boundaries too!}  ---indeed, this idea is already implicit in referring to e.g.\ $\Gamma \types a:A$ as a \emph{term judgement}: we are thinking of $a$ as the essential substance of the judement, and the function of $\Gamma$ and $A$ as just to situate $a$ within its surroundings, as seen in the form of the algebraic rules \ref{para:alg-rules}.
\end{para}

The ``inclusions of boundaries into cells'' are defined as follows:

\begin{definition}
The \emph{universal type (resp.\ term) extensions} are the inclusion maps
$$ i^\ty_n \colon \T_\stuff [\Gamma_{(n)}] \mono \T_\stuff[\Gamma_{(n)} \types A],$$
$$ i^\tm_n \colon \T_\stuff [\Gamma_{(n)} \types A] \mono \T_\stuff[\Gamma_{(n)} \types a : A].$$

A \emph{basic term/type extension} is a pushout of one of the universal extensions.  A \emph{term/type/term-and-type extension} is any composite (possibly transfinite) of basic extensions.

We indicate such extensions in diagrams by tailed arrows: $\T \mono \S$.  \todo{[Maybe use some other tail, since in other categories I'm using this for just monos?]} 
\end{definition}

So in syntactic terms, a basic term extension is just any extension of a theory $\T$ by a new constructor $x_1: A_1,\ \ldots,\ A_{n-1}(\x^{< n})\ \types\ a(\x) : A_n(\x)$, where the $A_i$ are existing types of the theory.  Similarly, a basic type extension is an extension by a single algebraic term-forming axiom.  An arbitrary extension is any extension of theories formed by iteratively adding (possibly sets of) axioms of these forms.

The classes of term, type, and term-and-type extensions are all closed under composition, identities, and pushouts, so form classes of left maps in the sense of Section \ref{sec:endo-operads} above.  \todo{(Point out that they are not, however, closed under retracts? (or at least, not obviously; are they actually?)  This depends on how I set things up in homotopical background, I guess.)}

\begin{definition}A \emph{term-contraction} (resp.\ \emph{type-contraction}, \emph{contraction}) on a map $F \colon \T \to \S$ is an operation assigning a diagonal filler to every square
$$\xymatrix{ \T_\stuff[\Gamma_{(n)} \types A\ \type] \ar@{ >->}[d]_{i^\tm_n} \ar[r] & \T \ar@{->>}[d]^F \\ \T_\stuff[\Gamma_{(n)} \types a: A] \ar[r] \ar@{.>}[ur] & \S }$$
with left-hand-side a universal type (term, term or type) extension.  Assuming the axiom of choice, a map admits such fillers if and only if it is weakly orthogonal to all universal term (type, term and type) extensions, in which case it is called \emph{term-contractible} (\emph{type-contractible}, \emph{contractible}).

We will write $\R_\tm$, $\R_\ty$, $R_{\tm\ty}$ for the classes of contractible maps, and indicate them in diagrams by double-headed arrows: $\T \epi \S$.  \todo{[Maybe use eg $\to/-|>/ \ $ instead, to disambiguate from epis?]} 



\end{definition}

The class of maps to which some given map is right orthogonal is always closed under pushouts and transfinite compositions \cite{hovey:closure}; so a type-contractible (term-contractible, contractible) map is in fact right orthogonal to all type (term, term-and-type) extensions.

\begin{para} Contractibility is familiar in type-theoretic terms as a form of conservativity.  Term-contractibility, for instance, states that whenever we have a type $\Gamma\ \types_\T\ A\ \type$ of $\T$ whose interpretation in $\S$ is inhabited by some term $F(\Gamma)\ \types_\S\ a:F(A)$, it is already inhabited in $\T$ by some term $\Gamma\ \types_\T\ \overline{a}:A$, which moreover is a \emph{lifting} of $a$, in that we can prove $F(\Gamma)\ \types_\S\ F(\overline{a}) = a : F(A)$ in $\S$.  Type-contractibility asserts the same sort of lifting property for types derivable in $\S$ over a context from $\T$.

This syntactic formulation of type-contractibility has been considered previously by Hofmann as a conservativity principle: see the discussion of logical frameworks in \cite[\SEC 4]{hofmann:syntax-and-semantics}, and Example \ref{ex:hofmann-contractibility} below.
\end{para}

\begin{para} Note that while neither form of contractibility directly provides any kind of lifting for definitional equality judgements, in the presence of identity types one can obtain weakforms of such liftings just from term-contractibility.  If for instance $\Gamma\ \types_\T\ a,a': A$ and $F(\Gamma)\ \types_\S\ F(a) = F(a'):F(A)$, then term-contractibility lets us lift $r(a)$ to some term $\Gamma\ \types_\T\ \overline{r(a)} : \Id_A(a,a')$.  

Essentially, definitional equality for terms implies propositional equality, and for types, isomorphism-up-to-propositional-equality (``homotopy-equivalence''); and since these are matters of term-judgements, they can be lifted along a term-contractible map. 

Often, term-contractibility implies type-contractibility.  In particular, in many important theories, the type-forming axioms do not mention any of the specific term-constructors.  From this it follows that any type judgment factors uniquely as a type judgement derivable without any term-formers (the ``shape'' of the type), followed by substitution along some context morphism.  Now if $F \colon \T \to \S$ is a morphism between two theories with this property, and $\S$ has the same type-forming rules as $\T$, then if $F$ is term-contractible, it is also type-contractible.  \todo{[Explain this in more detail??  We never actually need it later, but it's an interesting fact.]}
\end{para}

\begin{para} Since the classes of contractible maps are defined by an orthogonality condition, they are easily seen to be closed under retracts, by the standard argument (see eg \cite[where?]{hovey}).  This can be seen by a brief diagram-chase:  
$$\xymatrix{ \bullet \ar[r] \ar@{ >->}[d]_i & \bullet \ar[r] \ar[d]^f & \bullet  \ar[r] \ar@{->>}[d]^g & \bullet \ar[d]^f \\ \bullet \ar[r] \ar@{.>}[urr] & \bullet \ar[r] & \bullet \ar[r] & \bullet }$$
If $f$ is a retract of $g$, $g$ is contractible, and we are given a square from $i$ to $f$ to fill, take this as the leftmost square above, and build out to the right.  By the lifting property of $g$, we can find a diagonal filler $2 \times 1$, and hence for the overall $3 \times 1$ rectangle; but thanks to the retraction, this rectangle is exactly the original square.  [This proof probably shouldn't be in the main body; omit it entirely, or relegate it to an appendix?  But it is probably helpful for a non-homotopically experienced logician.]

Similarly, any transfinite composition of (term-, type-) contractible maps is again contractible.
\end{para}

\begin{example} \label{ex:elim-gives-contraction}
For any context morphism $f : \Delta \to \Theta$, the induced map of slices $f^*\colon \T \slice \Theta \to \T \slice \Delta$ is orthogonal to $i^\tm_0$ just if $f$ is orthogonal to all dependent projections---that is, if $f$ is a left map in the sense of Gambino--Garner \cite{gambino-garner}---or syntactically, if $f$ admits an ``elimination rule'' (and associated computation rule):
$$\inferrule{\y : \Theta \types C(\y)\ \type \\ \z:\Delta \types d(\z): C(f(\z))}{\x : \Theta\ \types\ \mathsf{elim}_f(\y.C, \z.d; \x) : C(\x)}$$

$$\xymatrix{ 
  \T_\stuff[\diamond \types A\ \type] \ar@{ >->}[d]_{i^\tm_0} \ar[r]^C 
  & \T \slice \Theta \ar@{->>}[d]^{f^*} 
\\ 
  \T_\stuff[\diamond \types a: A] \ar[r]^d \ar@{.>}[ur]
  & \T \slice \Delta
}$$

It is fully term-contractible exactly if every pullback of $f$ along a dependent projection is a Gambino--Garner left map, or equivalently if it supports the ``Frobenius'' form of this elimination rule, with extra dependent premises in the context, i.e.\ 
$$\y : \Theta, w: \Xi(\y) \types C(\y,\w)\ \type \ldots$$

This is implied by $f$ alone being a left map as long as $\T$ has identity types (by \cite[5.2.1]{gambino-garner}), or $\Pi$-types (by standard arguments).

In particular, for every reflexivity map $\r \colon \Delta.B \to \Delta.B.B.\Id_B$, the map $r^*$ between slices is term-contractible.  TODO: and for the variant $\Id$-elim.

Analogously to the above, ``large elimination'' rules give type-contractibility; but this will not concern us here.
\end{example}

\begin{example} \lalbe{ex:hofmann-contractibility}
As remarked above, if $\T$ is any theory and $\T^\mathrm{LF}$ is its presentation in a logical framework, then according to \cite[\SEC 4]{hofmann:syntax-and-semantics}, the interpretation of $\T$ in $\T^\mathrm{LF}$ is type-contractible. \todo{[Give more details, in terms of forgetful functors and the component-wise contractible natural transformation, and emphasising functoriality of the logical framework.]}
\end{example}

\begin{para} Since $\DTT$ is the category of models of an algebraic theory, and so is locally presentable, we can use the machinery of \cite{garner:understanding} (the ``algebraic small-object argument'') to construct an algebraic weak factorisation system\footnote{aka natural weak factorisation system} on $\DTT$, using the universal extensions (term, type, or both) as the generating left maps.  The algebraic right maps in the resulting system are then just maps equipped with (term-, type-) contractions; the algebraic left maps are maps presented as (term, type, term-and-type) extensions.

We can think of the maps $i^\ty_n$, $i^\tm_n$ here as \emph{generating cofibrations} in a putative model structure on $\DTT$, and the contractible maps as the \emph{trivial fibrations}; this idea is discussed a little further in Section \ref{sec:model-strux} below.

Example \ref{ex:elim-gives-contraction} suggests that this factorisation system on $\DTT$ is in some sense dual to the Gambino--Garner systems on the classifying categories of individual theories.  We will make this idea more precise in Section \ref{sec:fam-strux-on-DTT}.
\end{para}






\subsection{Extensions by propositional copies: the conservativity principle $\Jbar$.}

\begin{para}
One of the fundamental lemmas for many logical systems is that \emph{extension by definitions} should be well behaved: that if we extend a theory $\T$ by adding a new term $a'$, and an axiom that $a'$ is equal to some pre-existing term $a$ of $\T$, then the resulting theory is in some sense equivalent to $\T$.

For dependent type theories, this is clear when the new constructor is posited to be \emph{definitionally} equal to an existing one: the resulting theory $\T[a := a']$ is isomorphic to $\T$ itself.

However, one may also wish to understand a weaker situation, where the new term is only posited to be \emph{propositionally} equal to the existing one; precisely, where we extend $\T$ by axioms
$$\inferrule{\ }{\x : \Gamma \types a'(\x) : A(\x)} \qquad \inferrule{\ }{\Gamma \types l(\x) : \Id_A(a'(\x),a(\x)) }$$
Briefly, denote the resulting theory by $\T[a'(\x) :\propeq(\x) a]$, or when more detail is needed, by $\T[\x: \Gamma\ \types\ a'(\x) :\propeq_{l(\x)} a(\x) : A(\x)]$ or similar.

Categorically, extensions of this form are precisely pushouts of the universal ones 
$$\T_\stuff[\Gamma_{(n)} \types a: A] \mono \T_\stuff[\Gamma_{(n)} \types a: A][a'(\x) :\propeq a(\x)].$$
We will call (possibly transfinite) compositions of such pushouts \emph{extensions by propositional copies}, and write them as $\T[a_i'(\x) :\propeq a_i](\x)$

What can we now say about the inclusion $\T \mono \T[a'(\x) :\propeq a(\x)]$?  It is certainly not an isomorphism in general, nor indeed contractible, since $a'$ will rarely be in its image.  On the other hand, it is by definition a term-extension.  It is also certainly a monomorphism, since it has a retraction $\T[a'(\x) :\propeq a(\x)] \epi \T$, given by interpreting $a'$ as $a$ and $l$ as $r(a)$.

Our principle $\Jbar$ describes one sense in which $\T[a'(\x) :\propeq a(\x)]$ may reasonably be equivalent to $\T$:
\end{para}

\begin{definition}Say that \emph{$\Jbar$ holds for $\stuff$} if for every extension by propositional copies, the retraction $\T[a'(\x) :\propeq a(\x)] \epi \T$ is term-contractible.

In other words, the retractions of the universal such extensions
$$\T_\stuff[\Gamma_{(n)} \types a: A] \mono \T_\stuff[\Gamma_{(n)} \types a: A][a'(\x) :\propeq a(\x)].$$
are absolutely term-contractible.  \todo{[Actually, ``absolutely term-contractible'' isn't quite right: that would imply $\overline{K}$!  Subtlety is in what kinds of pushouts we're considering.  Can this phrasing of $\Jbar$ be corrected?  Think on it, throw it out if not.]}
\end{definition}

(Actually, we will not need the full strength of the principle as stated here: for our purposes, it would be enough to show that this holds when $\T$ can be axiomatised over $\stuff$ without any definitional equality axioms, i.e.\ when $\T_\stuff \mono \T$ is a term-extension, or in homotopy-theoretic language, when $\T$ is a \emph{cofibrant} theory.)

Why is this a plausible principle?  If we restrict to the case of adjoining copies of \emph{closed} terms, i.e.\ in the case where $\Gamma = \diamond$, then it is just the $\Id$-$\elim$ rule\footnote{in its ``one-ended'' form, as discussed in \ref{sec:one-vs-two-ended}; analogously, one can also consider a ``two-ended'' form of $\Jbar$, in which $a(\x)$ as well as $a'(\x)$ is freely adjoined}.  Syntactically, this is the fact that working in an extension by closed terms is equivalent to working over an extended context, with the new variables.  Categorically, the extension and its retraction in this case are isomorphic to the maps of slices
$$\bfig
\node Ta'a(0,400)[{\T[a' :\propeq a]}]
\node Tbyxu(1200,400)[\T \slice (x:A, u:\Id(x,a))]
\node T(0,0)[\T]
\node TbyD(1200,0)[\T \slice \diamond]
\arrow|m|/<->/[T`TbyD;\iso]
\arrow/@{ >->}@/^0.4em//[T`Ta'a;]
\arrow/@{->>}@/^0.4em//[Ta'a`T;]
\arrow/@{ >->}@/^0.4em//[TbyD`Tbyxu;]
\arrow/@{->>}@/^0.4em//[Tbyxu`TbyD;]
\arrow|m|/<->/[Ta'a`Tbyxu;\iso]
\efig$$
induced by the retraction of contexts
$$\bfig
\node diamond(0,0)[\diamond\ ]
\node xu(850,0)[(x:A, u:\Id(x,a))]
\arrow|a|/@{ >->}@<0.3em>/[diamond`xu;a,r(a)]
\arrow|b|/@{->>}@<0.3em>/[xu`diamond;!]
\efig .$$

Then the map $(a,r(a)) \colon \diamond \mono x:A, i:\Id(x,a)$ is (up to isomorphism) the introduction map for the one-ended form of $\Id$-$\elim$, so by Example \ref{ex:elim-gives-contraction}, the retraction  of slices $(a,r(a))^*$ is term-contractible.

Thus $\Jbar$---like various other type-theoretic principles---asserts that something which is holds derivably for \emph{closed} terms also holds admissibly for \emph{open} terms: in particular, that open terms of identity types satisfy the same rules as closed ones do. \\

So one way to prove $\Jbar$ is therefore to reduce the general case to the closed case, via $\Pi$-types.  This is possible, as long as we assume enough $\ext$ rules to make sure that their identity types are well-behaved:

\begin{proposition} \label{prop:jbar-holds-1}
$\Jbar$ holds for $(\Id,\Pi,\Pi\mbox{-}\extrule,\Pi\mbox{-}\extrule\mbox{-}\apprule\mbox{-}\defrule)$ and any set of constructors extending this.
\end{proposition}

\begin{proof}
The diagram
$$\bfig
\node Ta(0,400)[{\T[a'(\x) :\propeq a(\x) : A(\x)]}]
\node Tf(1600,400)[{\T[f :\propeq (\lambda \x.\ a(\x)) : \Pi_{\x} A(\x)]}]
\node T1(0,0)[\T]
\node T2(1600,0)[\T]
\arrow/@/^0.4em//[Ta`Tf;]
\arrow/@/^0.4em//[Tf`Ta;]
\arrow/@{->>}/[Ta`T1;]
\arrow/@{->>}/[Tf`T2;]
\arrow/@{=}/[T1`T2;]
%%% \arrow|a|/@{ >->}@/^0.4em//[Ta'`Tf;]
%%% \arrow|b|/@{->>}@/^0.4em//[Tf`Ta';]
\efig$$
% $$\xymatrix{\T[k(\x):K(\x)] \ar[d] \ar@/_/[r] & \ar@/_/[l] \T[\hat{k}:\Pi_{\x} K(\x) \ar[d] \\ 
% \T[k_0(\x),k_1(\x):K(\x),\ l(\x):\Id(k_0(\x),k_1(\x))] \ar@/_/[r] & \ar@/_/[l] \T[\hat{k}_0,\hat{k}_1 : \Pi_{\x} K(\x),\ \hat{l}:\Id(\hat{k}_0,\hat{k}_1)]}$$
exhibits its left-hand side (the map we wish to show contractible) as a retract of its right-hand side.  (The fact that the squares commutes and are a retraction require the computation rules for $\ext$.)  But the right-hand side is just the closed case of $\Jbar$, which we've seen is contractible.

\comment{[The diagram above is incorrect and out-of-date; I'm trying to update it but am having strange LaTeX troubles with the new version, so leaving the old one for now.]}
\end{proof}

\begin{para} \todo{[There are some major subtleties I'm currently glossing over here, re (a) passing in and out of the logical framework presentation (see notes of 2 Nov for how to fix this up), and (b) $\DTT_\PiIdelim$ not really being essentially algebraic.  Clean this up, once background is better set out!  Probably ought to expunge $\DTT_\PiIdelim$ entirely, replacing with something like ``$\stuff$ any set of constructores implying $\PiIdelim$\ldots''.]}

In fact, the hypotheses here can be weakened.  Another way to see $\Jbar$ is as a close cousin of Garner's rule $\PiIdelim$.  This latter asserts that a version of the $\Id$-elim rule holds over products of identity type $\Pi_x\ \Id(f \tightcdot x, g \cdot x)$.  But terms of such types are very close to the open terms of $l(\x)$ of identity type that appear in $\Jbar$---and the connection may be made very clear by working in a second-order formulation, using a logical framework as metalanguage.  In these terms, a second-order version of $\Jbar$ can be seen as being a strong functional extensionality principle, similar to $\PiIdelim$, but stated without $\Pi$-types.   It is then not hard to show that this form of $\Jbar$ is derivable from $\PiIdelim$, and so:
\end{para}

\begin{proposition}\label{prop:jbar-holds-2}
$\Jbar$ holds for $(\Id,\Pi,\PiIdelim)$, and any set of constructors extending this.
\end{proposition}

(This is strictly stronger than the preceding proposition, since as shown in \cite[5.11]{garner:on-the-strength}, $\PiIdelim$ is derivable from $\ext$ with slightly weaker computation rules than ours.)

\begin{proof}
Recall the statement of the rule $\PiIdelim$:

$$ \inferrule*[right={$\PiIdelim$}]{
\Gamma,\ u, v : \Pi_{x:A}B(x), w : \Pi_{x:A} \Id_{B(x)}(u \cdot x,v \cdot x)\ \types\ C(u,v,w)\ \type \\ 
\Gamma,\ f : (x \tightcolon A) B(x)\ \types\ d(f) : C (\lambda f, \lambda f, \lambda (r \circ f)) \\
\Gamma\ \types\ k, k' : \Pi_{x:A} B(x) \qquad \Gamma\ \types\ l : (x \tightcolon A) \Id_{B(x)}(k x, k' x) }
{ \Gamma\ \types\ \Lterm(C,d,k,k',l) : C(k,k',l) } $$
and with associated computation rule $\PiIdcomp$ concluding:
$$ \Lterm(C,d, \lambda h, \lambda h, \lambda ( r \circ h)) = d(h) : C( \lambda h, \lambda h, \lambda (r \circ h)).$$

(Here $(x \tightcolon A)B(x)$ and $[x \tightcolon A] b(x)$ denote abstraction in the metalanguage; $u \cdot x$.)

Correspondingly, a version of $\Jbar$ may be stated in second-order language as:
$$ \inferrule*[right={$\Jbarrule$}]{
  \Gamma,\ \underline{k}, \underline{k}' : (x \tightcolon A) B(x),\ \underline{l} : (x \tightcolon A) \Id_{B(x)}(\underline{k}x,\underline{k}'x)\ \types\ \underline{C}(\underline{k},\underline{k}',\underline{l})\ \type \\ 
  \Gamma,\ \underline{f} : (x \tightcolon A) B(x)\ \types\ \underline{d}(\underline{f}) : \underline{C} ( \underline{f}, \underline{f}, r \circ \underline{f}) \\
\Gamma\ \types\ \underline{k}, \underline{k}' : (x \tightcolon A) B(x) \qquad \Gamma\ \types\ \underline{l} : (x \tightcolon A) \Id_{B(x)}(\underline{k} x,\underline{k}' x) }
{ \Gamma \types \Jbarterm(\underline{C},\underline{d},\underline{k},\underline{k}',\underline{l}) : \underline{C}(\underline{k},\underline{k}',\underline{l}) } $$
and the corresponding computation rule concludes that
$$ \Jbarterm(C,d,k,k,r \circ k) = d(k) : C(k,k',l) . $$

(This is slightly stronger than the original, external formulation of $\Jbar$, since this rule implies stability in the ambient context $\Gamma$.) 

We can now define the eliminator $\Jbarterm$ in terms of $\Lterm$ by:
$$\Jbarterm(\underline{C},\underline{d},\underline{k},\underline{k}',\underline{l})\ :=\  \Lterm(\, [k,k',l]\,\underline{C}([x]k \tightcdot x,\,[x] k' \tightcdot x,\,[x] l \tightcdot x) ,\ \underline{d},\ \lambda \underline{k},\ \lambda \underline{k}',\ \lambda \underline{l} ).$$
It is routine to verify that under the hypotheses of the $\Jbar$ rule, this typechecks, and satisfies the required computational behaviour.
\end{proof}

(It is tempting to wonder if the implication can be reversed; however, it is at least not obviously possible.  The obvious candidate for defining $\Lterm$ in terms of $\Jbarterm$,
$$\Lterm(C,d,k,k',l)\ :=\ \Jbarterm(\, [\underline{k},\underline{k}',\underline{l}]\, C( \lambda\underline{k},\lambda \underline{k}',\lambda\underline{l}),\ d,\ [x]\,k \tightcdot x,\ [x]\,k' \tightcdot x,\ [x]\,l \tightcdot x),$$ 
does indeed typecheck successfully; but the desired computation rule only holds up to propositional equality, not definitional.) \\

Thus in theories with reasonably strong functional extensionality principles, $\Jbar$ holds, and holds robustly: it is derivable, so will continue to hold under strengthenings of the system.  However, it may certainly fail as we weaken the system:

\begin{proposition} \label{prop:jbar-implies-ext}
For any set of constructors $\stuff$ including $\Pi$-types, $\Jbar$ implies a weak form of functional extensionality: if $x:A \types k(x),k'(x) : B$ and $x : A \types l(x):\Id(k(x),k'(x))$, then there is some term $\hat{l}$ for which $\types \hat{l} : \Id ( \lambda x.\,k(x),\, \lambda x.\, k'(x))$.
\end{proposition}

\begin{proof}
$\Jbar$ tells us that the map $\T_\stuff[k(x),k'(x),l(x)] \epi \T_\stuff[k(x)]$ is term-contractible; applying this to the type $\Id ( \lambda x.\,k(x),\, \lambda x.\, k'(x))$ upstairs and the term $r(\lambda x.\, k(x))$ downstairs yields a term as desired.
\end{proof}

\begin{corollary} \label{prop:jbar-fails}
$\Jbar$ fails for $(\Id,\Pi,\eta)$ and $(\Id,\Pi)$. 
\end{corollary}

\begin{proof}
The well-known  \todo{[Citation?]}  failures of $\ext$ in these systems are also failures of the conclusion of Proposition \ref{prop:jbar-implies-ext}.
\end{proof}

However, these failures involve essential use of $\Pi$-types.  

\begin{conjecture}
$\Jbar$ holds for $(\Id)$.
\end{conjecture}

Proposition \ref{prop:jbar-fails} shows that if the conjecture is true, then $\Jbar$ is not stable under extensions of the constructor sets, so can't hold for $(\Id)$ as robustly as it does for $(\Id,\Pi,\ext)$: it may be \emph{admissible} for the type theory with just $\Id$-types, but it cannot be \emph{derivable}. \\

\comment{[Perhaps also mention the analogous ``$\overline{K}$'' principle, and how it fails when UIP does?]}


\begin{para} \todo{[Discuss the relationship of $\Jbar$ to \emph{equivalence vs.\ interderivability of axioms}, which is probably the best argument for why even non-homotopically-inclined type theorists should care about it?]}
\end{para}

























\section{Contractible operads; weak $\omega$-categories from $\DTT$} \label{sec:contractibility}

\comment{Include:  General contractibility of operads.  Give in terms of pylon diagrams.  Prune/contract pasting diagrams.  Recall L09/GvdB ``if whole glob obj is nice, then co-points of pds are ctrble''.  Refine that!  Reduce to more 1-d filling problem.}

\comment{In light of this, give various conditions for classifying weak $\omega$-category to exist: $\Jbar$ plus normalisation plus $(-)^\cxt$, etc.}

 In this section, we investigate various conditions under which we can map some contractible operad into $\End(\globes)$, and hence give a weak $\omega$-category structure.  In summary, we obtain a weak $\omega$-structure:
\begin{enumerate}
\item on $\cl_\omega$, conjecturally (depending on $\Jbar$), for all theories with $\Id$-types;
\item on $\cl_\omega$, unconditionally, for theories with $\Id$- and $\Pi$-types and Garner's rule $\PiIdelim$; and
\item on $\clpi_\omega$, for theories with $\Id$- and $\Pi$-types and rule $\Pi$-$\eta$-$\prop$.
\end{enumerate}

\begin{figure}[htbp]
\caption{Contractibility in an endomorphism operad \label{fig:contractibility-pylons}} 
$$\bfig
%%%%%%%%%%%%%%%%%%%
% left hand pylon %
%%%%%%%%%%%%%%%%%%%
\node gn(250,0)[\globefig{n}]
\node gn1l(0,-250)[\globefig{n-1}]
\node gn1r(500,-400)[\globefig{n-1}]
\node gn2l(0,-650)[\globefig{n-2}]
\node fakegn2l(450,-650)[]
\node gn2r(500,-800)[\globefig{n-2}]
\node g1l(0,-1150)[\globefig{1}]
\node g1r(500,-1300)[\globefig{1}]
\node g0l(0,-1550)[\globefig{0}]
\node g0r(500,-1700)[\globefig{0}]
\arrow[gn1l`gn;]
\arrow[gn1r`gn;]
\arrow[gn2l`gn1l;]
\arrow[gn2r`gn1l;]
\arrow[gn2l`gn1r;]
\arrow[gn2r`gn1r;]
\arrow/@{}|<>(0.58)\vdots/[g1l`gn2l;]
\arrow/@{}|<>(0.58)\vdots/[g1r`gn2r;]
\arrow[g0l`g1l;]
\arrow[g0r`g1l;]
\arrow[g0l`g1r;]
\arrow[g0r`g1r;]
%%%%%%%%%%%%%%%%%%%%
% right hand pylon %
%%%%%%%%%%%%%%%%%%%%
\node Tpi(1750,0)[{\T_\stuff[\widehat{\pi}]}]
\node Tspi(1500,-250)[{\T_\stuff[\widehat{s\pi}]}]
\node Ttpi(2000,-400)[{\T_\stuff[\widehat{t\pi}]}]
\node Ts2pi(1500,-650)[{\T_\stuff[\widehat{s^2\pi}]}]
\node Tt2pi(2000,-800)[{\T_\stuff[\widehat{t^2\pi}]}]
\node Ts1pi(1500,-1150)[{\T_\stuff[\widehat{s_1\pi}]}]
\node Tt1pi(2000,-1300)[{\T_\stuff[\widehat{t_1\pi}]}]
\node Ts0pi(1500,-1550)[{\T_\stuff[\widehat{s_0\pi}]}]
\node Tt0pi(2000,-1700)[{\T_\stuff[\widehat{t_0\pi}]}]
\arrow[Tspi`Tpi;]
\arrow[Ttpi`Tpi;]
\arrow/@{>}|!{(500,-400);(2000,-400)}\hole/[Ts2pi`Tspi;]
\arrow/@{>}|!{(500,-400);(2000,-400)}\hole/[Tt2pi`Tspi;]
\arrow[Ts2pi`Ttpi;]
\arrow[Tt2pi`Ttpi;]
\arrow/@{}|<>(0.58)\vdots/[Ts1pi`Ts2pi;]
\arrow/@{}|<>(0.58)\vdots/[Tt1pi`Tt2pi;]
\arrow/@{>}|!{(500,-1300);(2000,-1300)}\hole/[Ts0pi`Ts1pi;]
\arrow/@{>}|!{(500,-1300);(2000,-1300)}\hole/[Tt0pi`Ts1pi;]
\arrow[Ts0pi`Tt1pi;]
\arrow[Tt0pi`Tt1pi;]
%%%%%%%%%%%%%%%%%%%%
% connecting wires %
%%%%%%%%%%%%%%%%%%%%
\arrow/@{.}/[gn`Tpi;H]
\arrow/@{>}|!{(250,0);(500,-400)}\hole/[gn1l`Tspi;F_{n-1}]
\arrow[gn1r`Ttpi;G_{n-1}]
\arrow/@{>}|<>(.19)\hole|!{(500,-800);(500,-400)}\hole/[gn2l`Ts2pi;F_{n-2}]
\arrow[gn2r`Tt2pi;G_{n-2}]
\arrow[g1l`Ts1pi;F_1]
\arrow[g1r`Tt1pi;G_1]
\arrow/@{>}|<>(.21)\hole|!{(500,-1700);(500,-1300)}\hole/[g0l`Ts0pi;F_0]
\arrow[g0r`Tt0pi;G_0]
\efig$$
\end{figure}

\subsection{Theories with $\Id$-types}

\renewcommand{\stuff}{\Id}
\begin{theorem} \label{thm:ctrble-operad-for-id} If $\Jbar$ holds for $\Id$, then $\End(\globes^\Id)$ is contractible.
\end{theorem}

\begin{proof}
As seen in section \ref{sec:endo-operads}, contractibility for this operad demands that given any pasting diagram $\pi \in T1(n)$, and $(F_0,G_0,\ldots G_{n-1})$ as in Fig.\ \ref{fig:contractibility-pylons}, we must construct $H$ to complete the map of spans; more concisely, we must complete the triangle
$$\xymatrix{ \del \globe[n] \ar[r]^{[F_i,G_i]} \ar@{ >->}[d] & \T_\stuff[\del \hat{\pi}]  \ar@{ >->}[r] & \T_\stuff[\hat{\pi}] \\ \globe[n] \ar@{.>}[urr] & }.$$

The cases $n= 0,1$ are dealt with by the discussion of \ref{para:map-from-cat} above.

On the other hand, when $n > 0$, it is immediate from the axiomatisations given that the map $\del \globe[n] \to/ >->/ \globe[n]$ is a term-extension.  Also, according to the pruning procedure described \ref{para:pruning} above, we can obtain $\T_\stuff[\hat{\pi}]$ as an extension of $\T_\stuff[\widehat{s_1\pi}]$ by propositional copies; so \emph{provided $\Jbar$ holds for $\DTT_\stuff$}, the retraction
$$\T_\stuff[\hat{\pi}] \epi \T_\stuff[\widehat{s_1\pi}]$$
(interpreting all identity cells as reflexivity terms) is term-contractible.

Thus to complete the triangle above, it is sufficient to complete the square
$$\xymatrix{ \del \globe[n] \ar[r]^{[F_i,G_i]} \ar@{ >->}[d] & \T_\stuff[\hat{\pi}] \ar@{->>}[d] \\ \globe[n] \ar@{.>}[r] & \T_\stuff[\widehat{s_1\pi}]},$$
i.e.\ to complete a triangle of the form
$$\xymatrix{ \del \globe[n] \ar[dr] \ar@{ >->}[d] & \\ \globe[n] \ar@{.>}[r] & \T_\stuff[\widehat{s_1\pi}]}.$$

But now $s_1\pi$ is just some $\path_l$, so as in \ref{para:map-from-cat}, \ref{para:canonicity-of-Tpath} we have an explicit axiomatisation of $\T_\Id(\widehat{s_1 \pi})$, and we know that this theory enjoys canonicity.  So in trying to extend $[F_i,G_i]$ along $ \del \globe[n] \mono \globe[n]$, we have interpreted $i$-$\sourcerule$ and $i$-$\targetrule$ in $\T_\stuff[\widehat{s_1\pi}]$, for $i < n$, and wish to interpret $n$-$\cellrule$; i.e.\ we wish to prove a propositional equality between the interpretations of $s_{n-1}(x)$ and $t_{n-1}(x)$.   But by canonicity, and the simplicity of our set of constructors, any two terms of the same type in $\T_\stuff[\widehat{s_1\pi}]$ are \emph{definitionally} equal; so interpreting $c_n$ as a reflexivity term, we are done.  (Specifically, $s_1$, $t_1$ must both be interpreted as the obvious composite of basic constructors described in \ref{para:map-from-cat}, and for $i > 1$, $s_i$ and $t_i$ must be interpreted as the reflexivity term over $s_{i-1}$, $t_{i-1}$.)
\end{proof}

\subsection{$\End(\globes)$, in theories with $\PiIdelim$}

\renewcommand{\stuff}{\PiIdelim}
\begin{para} By turning our attention to theories with not only $\Id$-types but also $\Pi$-types and the $\PiIdelim$ rule, we ensure that $\Jbar$ holds unconditionally.  However, this comes at the possible cost of normalisation and canonicity.  \todo{[Is ``possible'' necessary here?]}  Thus, in trying to repeat the argument above to show that $\End(\globes)$ is contractible, we fall at the last hurdle: we do not know that $[F_i,G_i]$ gives the ``correct'' map $\del \globe[n] \to \T_\stuff[\widehat{\path_l}]$.

To remedy this, we simply restrict to the sub-operad of operations for which this holds.  This is essentially just a slightly more complicated analogue of the tactic used in \cite{garner-van-den-berg}, of restricting to the operad of point-preserving operatons, as discussed in Section \ref{sec:lumsdaine-versus-gvdb}. 
\end{para}


\begin{definition} \label{def:ref-1-glob} For $\E$ any category with pullbacks, define a monoidal category $\RefnGlob{1}{\E}$ as follows:
\end{definition}

\begin{wrapfigure}[15]{r}{0.15\textwidth}
\vskip -1.5em
$\bfig
\node An(0,0)[A_n]
\node An1(0,-400)[A_{n-1}]
\node A2(0,-900)[A_2]
\node A1(0,-1300)[A_1]
\node A0(0,-1700)[A_0]
\arrow|m|/@<0ex>/[An`An1;s]
\arrow|m|/@<1ex>/[An`An1;t]
\arrow/@{}|<>(0.58)\vdots/[A2`An1;]
\arrow|m|/@<0ex>/[A2`A1;s]
\arrow|m|/@<1ex>/[A2`A1;t]
\arrow|m|/@<-0.5ex>/[A1`A0;s]
\arrow|m|/@<0.5ex>/[A1`A0;t]
\arrow|m|/@/^0.5em//[A1`A2;r]
\arrow|m|/@/^1.35em//[A1`An1;r]
\arrow|m|/@/^2.5em//[A1`An;r]
\efig$\caption{\label{fig:modspan} \textcolor{white}{longword}}
\end{wrapfigure}

Objects in dimension $n$ are globular objects $\A$ of $\E$, together with \emph{reflexivity data from dimension 1}: for $1 \leq i \leq n$, a map $r_i \colon A_1 \to A_i$, such that $s_i r_{i+1} = t_i r_{i+1} = r_i$, and $r_1 = 0$.\footnote{Is there pre-existing terminology for this?}

A map between two of these is a map $(f_i,g_i,h)$ between their ``underlying'' spans, with $f_0 = g_0$, $f_1 = g_1$, and commuting with the reflexivity data in that $f_i r_i = g_i r_i = r_i f_1$, and $h r_n = r_n f_1$.

The monoidal globular structure of $\RefnGlob{1}{\E}$ is lifted from that of $\Spans[\E]$.  Any tensor product $\A \tensor_k \B$ in $\Spans[\E]$ of globular objects is again globular, and reflexivity data on the multiplicands lifts naturally to reflexivity data on the product; and similarly, the units over globular objects are globular and carry natural reflexivity data. \\

We thus have a monoidal globular category and faithful forgetful functor
$$\RefnGlob{1}{\E} \to \Spans[\E].$$

(Note that, as the definition of the maps hints, the globularity condition on objects in dimensions $>1$ is not actually required here, and it would arguably be more natural to omit it.  However, all spans occurring in the construction of endomorphism operads remain fully globular, so it makes no difference for present purposes, and it simplifies the specification of the reflexivity data.) \\

Instantiating this construction with $\E = \DTT^\op$, the globes $\globes$ lift (using their reflexivity maps) to a globular object in $\RefnGlob{1}{\DTT^\op}$.  We this obtain a new endomorphism operad for them---a sub-operad of the old, since the functor inducing the map is faithful:
$$\End_{\RefnGlob{1}{\DTT^\op}}(\globes) \mono \End(\globes)$$

\begin{para}From here we need to restrict still a little further before we have a contractible operad: we need to look at just those operations which do the correct thing in dimensions $\leq 1$.  Specifically, the map $\psi \colon P_\Cat \to \tr^1\End(\globes)$ is easily seen to factor through $\End_{\RefnGlob{1}{\DTT^\op}}(\globes)$; so let $Q_\PiIdelim$ be the pullback
% $$\xymatrix{ P \ar[r] \ar[d] & \End_{ModSpans[\DTT^\op]}(\globes) \ar[d]^\eta \\ P_{\strwCat} = I P_\Cat \ar[r]^{I \psi} & I \tr^1 \End_{ModSpans[\DTT^\op]}(\globes)}$$
$$\bfig 
\node Q(0,400)[Q_\PiIdelim]
\node End(1400,400)[\End_{\RefnGlob{1}{\DTT^\op}}(\globes)]
\node Pstr(0,0)[P_{\strwCat} = I P_\Cat]
\node ItrEnd(1400,0)[I \tr^1 \End_{\RefnGlob{1}{\DTT^\op}}(\globes)]
\arrow/@{ >->}/[Q`End;]
\arrow[Q`Pstr;]
\arrow[End`ItrEnd;\eta]
\arrow/@{ >->}/[Pstr`ItrEnd;I \psi]
\place(100,300)[\pb]
\efig$$
where $I \colon \nOpd[1] \to \nOpd[\omega]$ is the ``indiscrete'' functor, right adjoint to $\tr^1$; then $Q_\PiIdelim$ consists precisely of those operations of $\End_{\RefnGlob{1}{\DTT^\op}}(\globes)$ whose $\leq 1$-dimensional parts lie in the image of $\psi$.  We are now set up for:
\end{para}

\begin{proposition}The suboperad $Q_\PiIdelim \mono \End(\globes)$ is contractible.
\end{proposition}

\begin{proof}Contractibility in dimensions $\leq 1$ holds by fiat: in these dimensions, $Q_\PiIdelim$ is isomorphic to the terminal operad.

For higher dimensions, note that operations in this new operad are just as in the old, except that additionally the maps involved must commute with the reflexivity data.  So contractibility now demands that for $\pi \in \pd_n$, and suitable $(F_i,G_i)_{i < n}$, we must produce a map $H : \globes[n] \to \T_\stuff[\widehat{\pi}]$ making both squares in the following commute:
$$\xymatrix{ 
  \del \globe[n] \ar[r]^{[F_i,G_i]} \ar@{ >->}[d] 
  & \T_\stuff[\del \hat{\pi}]  \ar@{ >->}[d] 
\\
  \globe[n] \ar@{.>}[r]  \ar@{->>}[d]
  & \T_\stuff[\hat{\pi}] \ar@{->>}[d]
\\ 
  \globe[1] \ar[r]_{F_1 = G_1}
  & \T_\stuff[\widehat{s_1 \pi}]
}.$$

But the overall rectangle commutes, so the desired filler follows by $\Jbar$.
\end{proof}

\subsection{$\End(\piglobes)$}

Turning our attention to $\piglobes$, and taking $\stuff = (\Id,\Pi,\Pi\mbox{-}\eta)$, and $\Id$, we are now well set up to construct a contractible sub-operad $Q_\Pi$ of $\End(\piglobes)$.  Specifically, take $Q_\Pi$ to be the normalised core of $\End(\piglobes)$---that is, all those operations whose $0$-dimensional source and target are the operad unit $1_{\piglobe[0]}$.  Since we have canonicity, we do not need to restrict further as we did in the construction of $Q_\PiIdelim$.

\begin{theorem} \label{thm:ctrble-operad-for-pi}
The operad $Q_\Pi$ is contractible.
\end{theorem}

\begin{proof}
The proof of Theorem \ref{thm:ctrble-operad-for-id} goes through almost verbatim.  The only difference is that the type-contractibility of the maps into which we factor $\T_\stuff[\widehat{pi}]^\Pi \epi \T_\stuff[\widehat{s_1 \pi}]^\Pi$ does not depend on $\Jbar$, instead following just from Example \label{ex:elim-gives-contraction}. 
\end{proof}

\subsection{Classifying weak $\omega$-categories}
% TODO: should be \\\texorpdfstring here

\begin{para} Theorems \ref{thm:ctrble-operad-for-id}, \ref{thm:ctrble-operad-for-piidelim}, and \ref{thm:ctrble-operad-for-pi} give  three situations in which there is a map from some contractible operad $Q$ into $\End(\globes)$ (or $\End(\piglobes)$).  In each case, this induces a map $L \to Q \to \End(\maybepiglobes)$, and hence a functor (``restriction of scalars'') $\Alg{\End(\maybepiglobes)} \to \Alg{L} = \wkwCat$.  We thus have:
\end{para}

\begin{corollary}
\begin{enumerate}
\item There is a functor $\cl_\omega \colon \DTT_{\PiIdelim} \to \wkwCat$, as outlined in the introduction, giving the ``classifying weak $\omega$-category'' of any theory with at least $\Id$-types, $\Pi$-types, and the $\PiIdelim$ rule.
\item If $\Jbar$ holds for $\Id$, then we moreover have $\cl_\omega \colon \DTT_{\Id} \to \wkwCat$, giving the classifying weak $\omega$-category for any theory with at least $\Id$-types.
\item There is a functor $\cl^\Pi_\omega \colon \DTT_{\Id,\Pi} \to \wkwCat$ giving a variant of the classifying weak $\omega$-category, for any theory with at least $\Pi$-types and $\Id$-types.
\end{enumerate}
\end{corollary}

\begin{para}This statement, while pleasing, has a few loose ends which deserve to be tied up.

Firstly, there is an obvious abuse of notation: if $\Jbar$ holds for $\Id$, then we are overloading $\cl_\omega$ not only with different codomains, a mild and common sin, but also with different domains, a potentially worse one.  Given a theory $\T$ with at least the $\PiIdelim$ rule, we could compute $\cl_\omega(\T)$ as such a theory, or we could treat it as a theory over $\Id$-, and compute $\cl_\omega(\T)$ from there.  Will these agree?  In other words, does $\cl_\omega$ commute with the forgetful functor $U \colon \DTT_\PiIdelim \to \DTT_\Id$, as in the following diagram?
$$\bfig
\node DTTPi(-200,400)[\DTT_\PiIdelim]
\node DTTId(-200,0)[\DTT_\Id]
\node GPiAlg(800,400)[\Alg{\End(\globes^\PiIdelim)}]
\node QAlg(1900,400)[\Alg{Q_\PiIdelim}]
\node GIdAlg(800,0)[\Alg{\End(\globes^\Id)}] % (1200,0) if curved map below included
\node wkwCat(2400,0)[\wkwCat]
\node GSets(1100,-400)[\GSets]
\arrow|l|/@/^0.5em//[DTTId`DTTPi;F]
\arrow|r|/@/^0.5em//[DTTPi`DTTId;U]
\arrow[DTTPi`GPiAlg;]
\arrow[DTTId`GIdAlg;]
\arrow[GPiAlg`GIdAlg;]
\arrow[GPiAlg`QAlg;]
\arrow[QAlg`wkwCat;]
\arrow[GIdAlg`wkwCat;]
\arrow[DTTId`GSets;]
% \arrow/@/_0.5em//[GPiAlg`GSets;]
\arrow[GIdAlg`GSets;]
\arrow[wkwCat`GSets;]
\place(1400,200)[(?)]
\place(-200,190)[\dashv]
\efig$$

Most parts of this diagram are easily seen to commute up to natural isomorphism.  The essential point is that the globes $\globes^\PiIdelim$ are just the image of $\globes^\Id$ under the left adjoint $F \colon \DTT_\Id \to \DTT_\PiIdelim$, since their axiomatisations involve only $\Id$-types; and hence as functors into globular sets, or even into $\End(\globes^\Id)$-algebras, we have $\DTT_\PiIdelim(\globes^\PiIdelim, \T) \iso \DTT_\Id(\globes^\Id, U(\T))$.

This leaves, as the dubious part, the square marked by (?).  This comes down to the question of whether the corresponding square of operad maps commutes
$$\bfig
\node L(0,500)[L]
\node EndGId(1200,350)[\End(\globes^\Id)]
\node Q(400,0)[Q_\PiIdelim]
\node EndGPi(1200,0)[\End(\globes^\PiIdelim)]
\arrow[L`EndGId;]
\arrow[EndGId`EndGPi;]
\arrow[L`Q;]
\arrow[Q`EndGPi;]
\place(700,225)[(?)]
\efig$$
and now the subtlety emerges: the maps out of $L$ are defined in terms of the specific contractions used on $Q_\PiIdelim$, and these in turn depend on how precisely we have implemented $\Jbar$.

What we can at least see is that constructing $Q_\Id$ analogously to $Q_\PiIdelim$, there is a square (indeed, a pullback):
$$\bfig
\node QId(400,350)[Q_\Id]
\node EndGId(1200,350)[\End(\globes^\Id)]
\node QPi(400,0)[Q_\PiIdelim]
\node EndGPi(1200,0)[\End(\globes^\PiIdelim)]
\arrow[QId`EndGId;]
\arrow[EndGId`EndGPi;]
\arrow[QId`QPi;]
\arrow[QPi`EndGPi;]
\efig$$

Moreover, the map $Q_\Id \mono \End(\globes^\Id)$ will certainly be a map of operads-with-contraction; and it is fairly routine but rather notationally fiddly to show that if the forgetful functor $U \colon \DTT_\PiIdelim \to \DTT_\Id$ ``preserves the implementation of $\Jbar$'' in an appropriate sense, then so is the map $Q_\Id \to Q_\PiIdelim$.  If this is the case, then the maps from $L$ will factor through $Q_\Id$, and hence all the squares (?) above will commute.

However, it seems unlikely to the present author \comment{[in fact, I think it's impossible; check this out if time allows!]} that the natural implementation of $\Jbar$ given above could be preserved but $U$.  This defect can (as often with problems of contrations) be finessed by some ad hoc modification of the contractions on the operads involved; but essentially, this is a problem that should be resolved not by strict-higher-categorical fiddling, but by a good theory of weak higher-categorical equivalence, under which the squares (?) would commute \emph{up to equivalence}.  In lieu of such a theory, then, we leave a full resolution of this problem aside for now.
\end{para}

\begin{para}Secondly, there is an abuse of terminology: what, if anything, does the classifying weak $\omega$-category classify?

As given at present, in the form of the functors $\cl^{(\Pi)}_\omega \colon \DTT_\stuff \to \wkwCat$ above, it cannot classify anything: that is, it cannot have a right adjoint, since it does not preserve the initial object, nor a left, since it does not preserve the terminal object.  (The initial theory in $\DTT_\stuff$ is $\T_\stuff$, whose $\cl_\omega$ is certainly non-empty, containing at least one $0$-cell, the empty context $\diamond$.  The terminal theory in each $\DTT_\stuff$ has one context of every length $l \in \N$, and so its $\cl_\omega$ will have $\N$-many 0-cells.)

However, this is not surprising: in the passage from type theories to weak $\omega$-categories, we have forgotten much structure (and also co-structure).  One of the next important steps in the current program should be the axiomatisation of higher-categorical structure corresponding to the structure on the type theories, in such a way that the classifying $\omega$-categories of theories carry such structure, and (hopefully) do indeed appropriately classify some kind of models in these structured higher categories.

For the 2-truncated case, this is well-worked-out in \cite{garner:2-d-models}.  However, in higher dimensions, the understanding of appropriate ``weak logical structure'' is at a very early stage of development; so this is a story for another time.
\end{para}

\subsection{CwA structures on $\DTT$ etc.} \ref{sec:fam-strux-on-DTT}.

\comment{This section is an inessential extra: if time allows and inspiration strikes (or if quantity demands) then I'll update + expand it, otherwise I'll delete it.}

An alternate perspective on $\Jbar$, shows that it can be seen not just as analogous to the $\Id$-elim rule, but actually as instance of it for a certain attributes-structure:

There are various important CwA-structures on categories of CwA's. In particular: there is a canonical CwA structure on $\CwA_\diamond^\op$, given by $\Ty^\mathrm{canon}_{\CwA_\diamond^\op}(\C) := \Ty_\C(\diamond)$, and $\C.A := \C/\!/A$.  The universal properties of slices (Proposition \ref{prop:slicing}), with general facts about free constructions, ensure that the requisite squares [diagram] are pullbacks.  (This is in some sense a universal CwA: certainly every small CwA may be obtained by pullback from it, a more precise statement can probably be formulated.)

This extends to a canonical CwA-structure with $\Id$-types on $(\CwA^\Id_\diamond)^\op$, a CwA-structure with $\Id$- and $\Pi$-types with $\eta$-rule on $(\CwA^{\Id,\Pi,\eta}_\diamond)^\op$, and so on.

However, we can bump up these structures a little further, to include certain ``formal $\Pi$-types'' (independently of what $\Pi$-types may already be present in the theories).  That is, we define $\Ty^\mathrm{canon + Pi}(\C) := \sum_{\Gamma \in \C} \Ty_\C(\Gamma)$; so a type over $\C$, in this attributes structure, is a type $A$ in some context $\Gamma$ of $\C$, to be thought of as the formal dependent product $\prod_\Gamma A$.

Context extension is by adjoining \emph{open} terms.

$\Jbar$ asserts that \emph{open} $\Id$-types in contexts are indeed $\Id$-types in this attributes structure.  (But danger, Will Robinson, danger: $\Jbar$ doesn't assert, and afaics doesn't imply, the stability/coherence conditions required for ``this attributes structure has $\Id$-types''.)

\subsection{A model structure on $\DTT$?} \ref{sec:model-strux}

Another optional bonus section.

\clearpage










































%% Bibliography Info

\bibliographystyle{amsalpha}
\bibliography{pll-thesis-bib}



\end{document}