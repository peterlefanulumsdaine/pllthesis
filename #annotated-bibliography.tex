\documentclass{article}

\begin{document}

\title{Higher categories and type theory: an annotated bibliography}

\section{Homotopy Type Theory proper}

\section{Presentations of MArtin-L\"o{}f Type Theory and its categorical models}

\cite{cartmell:generalised-algebraic-theories} Cartmell, \emph{Generalised algebraic theories and contextual categories}  TO WRITE!

\cite{dybjer:internal-type-theory} Peter Dybjer, \emph{Internal Type Theory}: TO WRITE!

\cite{hofmann:on-the-interpretation} Martin Hofmann, \emph{On the interpretation of DTT in LCCC's} (in Tiuryn and Pacholski, \emph{Computer Science Logic 1994}): Defines categories with attributes.  Gives classifying cwa of type theory, & interpretation of type theories in cwas.  Gives (in nice straightforward way) algebraic interpretation of some connectives; good discussion of coherence of these.  Doesn't give full presentation of

\cite{hofmann:syntax-and-semantics} Martin Hofmann, \emph{Syntax and semantics of dependent types} (in Dybjer, Pitts, \emph{Semantics and logics of computation}): Defines categories with families.  READ FULLY AND WRITE MORE! 

\cite{jacobs:comprehension-categories} Bart Jacobs, \emph{Comprehension categories and the semantics of type dependency}: TO WRITE!

\cite{jacobs:categorical-logic} Bart Jacobs, \emph{Categorical Logic and Type Theory}: Nice wide-ranging exposition of (mainly the fibration approach to) categorical logic, for various different type systems.  Some good discussions of subtle points often either overlooked or taken for granted.  Slight caveat on presentation of DTT: doesn't fully formalise the syntax, refers to \cite{pitts:categorical-logic}, \cite{hofmann:syntax-and-semantics} and others for details.

\cite{martin-loef:predicative-part} Per Martin-L\"o{}f, \emph{An Intuitionistic Theory of Types: Predicative Part}: ``In the beginning was the word\ldots'' and this is of course seminal; but for present purposes, it is essentially superseded by subequent presentations.

\cite{n-p-s:programming} Bengt Nordstr\"o{}m, Ken Petersson, Jan M. Smith, \emph{Programming in Martin-L\"o{}f's type theory}: Lovely, leisurely exposition of MLTT.  Useful discussion of building binding into the raw syntax (Ch. 3).

\cite{pitts:categorical-logic}, Andy Pitts, \emph{Categorical Logic} (in Abramsky, Gabbay, Maibaum, \emph{Handbook of Logic in CS}): The best full, formal presentation of DTT syntax I've been able to find.  Presentation is completely full \& precise, and includes extensions by signatures/axioms.  (No constructors beyond $\Pi$-types are done, but it's set up well for extensibility.)  Sketches the (weak) equivalence between dependent algebraic theories and accessible categories-with-attributes (\emph{type-categories}) (Prop. 6.11, Rmk 6.22), and states it for the case with $\Pi$-types (Prop. 6.25).

Names for categories with attributes: \emph{contextual cats}, Cartmell 1978 (not quite equivalent yet); \emph{categories with attributes}, \cite{hofmann:lcccs}, \cite{hofmann:syntax-and-semantics}; \emph{type-categories} \cite{pitts:categorical-logic}; \emph{categories with families} \cite{dybjer:internal-type-theory}; \emph{full split comprehension categories}, \cite[4.10]{jacobs:comprehension-categories}.

A couple more cited in RG's 2-d models paper: [7]: Ehrhard Une sémantique catégorique, [17]: Hyland, Pitts; [26]: Paul Taylor, Practical Foundations.

\section{Globular higher cats}

\cite{garner:understanding} Richard Garner, \emph{Understanding the small-object argument}:  Fundamental tool for the construction of ``cofibrantly generated'' algebraic weak factorisation systems.


\section{Homotopy Theory}

\bibliographystyle{amsalpha}
\bibliography{pll-thesis-bib}

\end{document}
