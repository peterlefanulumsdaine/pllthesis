\documentclass{amsart}

\usepackage{wasysym}
%\usepackage{ifpdf}
%\usepackage{mathpartir}
\usepackage{amssymb}
%\usepackage[all]{xypic}
%\xyoption{2cell}
%\xyoption{curve}
%\UseTwocells
\input{diagxy}

% \usepackage{makeindex}

% Peter LeFanu Lumsdaine, June 2010
% macros for my thesis

% Contents:
%
% - Binary relations
% - Category names
% - Single letters



%%%% 
% Single styled characters (or almost single)
%%%%

\newcommand{\Two}{\mathbf{2}}
% \newcommand{\A}{A_\bullet}
% \newcommand{\uA}[1][]{\underline{A}_{#1}}
% \newcommand{\B}{B_\bullet}
% \newcommand{\ML}{\mathit{ML_I}}
% \newcommand{\MLfrag}{\mathit{ML}^\Id}
\newcommand{\C}{\mathcal{C}}
\newcommand{\CC}{\mathbb{C}}
% \newcommand{\bigC}{\mathcal{C}}
% \newcommand{\bC}{\mathbf{C}}
% \newcommand{\Chat}{\widehat{\mathbb{C}}}
% \newcommand{\D}{\mathbb{D}}
% \newcommand{\bigD}{\mathcal{D}}
% \newcommand{\bD}{\mathbf{D}}
% \renewcommand{\d}{\partial}
\newcommand{\E}{\mathcal{E}}
\newcommand{\F}{\mathcal{F}}
\newcommand{\FF}{\mathbb{F}}
\newcommand{\G}{\mathbb{G}}
\newcommand{\I}{\mathcal{I}}   % generating cofibrations.  mathscr is prettier,
\newcommand{\J}{\mathcal{J}}   % but I find its I, J confusing.
\newcommand{\NN}{\mathbb{N}}   % Natural numbers
\newcommand{\N}{\mathcal{N}}   % Nerve
% \renewcommand{\P}{P_{\MLfrag}}
% \newcommand{\operadP}{P_{\MLfrag}}
% \newcommand{\Pfull}{P_{\ML}}
% \newcommand{\p}{\vec p}
% \renewcommand{\S}{\mathcal{S}}    % Another generic type theory
\newcommand{\T}{\mathcal{T}}      % A generic type theory
\newcommand{\W}{\mathcal{W}}
\newcommand{\WW}{\mathbb{W}}
% \newcommand{\X}{X_\bullet}
% \newcommand{\x}{\vec x}
% \newcommand{\uX}[1][]{\underline{X}_{#1}}
% \newcommand{\Y}{Y_\bullet}
% \newcommand{\y}{\vec y}
% \newcommand{\yon}{\mathbf{y}}

%%%%
% Styled words: general
%%%%

\newcommand{\Alg}[1]{#1\mbox{-}\mathbf{Alg}}
\newcommand{\AMS}{AMS}
\newcommand{\AWFS}{AWFS}
% \newcommand{\Cat}{\mathbf{Cat}}
% \newcommand{\cat}[1][-]{\mathbf{Cat}(#1)}
% \newcommand{\cod}{\mathrm{cod}}
% \newcommand{\dom}{\mathrm{dom}}
% \newcommand{\End}{\mathrm{End}}
% \newcommand{\ev}{\mathbf{ev}}
% \newcommand{\colim}{\varinjlim}
% \newcommand{\longGSets}{[\mathbb{G}^\op,\mathbf{Sets}]}
\newcommand{\GSets}{\widehat{\mathbb{G}}}
% \renewcommand{\lim}{\varprojlim}
\newcommand{\ob}{\operatorname{ob}}
% \newcommand{\op}{\mathrm{op}}
% \newcommand{\Operads}{\mathbf{Operads}}
% \newcommand{\pd}{\mathbf{pd}}
\newcommand{\QCat}{\mathbf{QCat}}
\newcommand{\qcat}{\mathit{qcat}}
% \newcommand{\Sets}{\mathbf{Sets}}
\newcommand{\Th}{\mathbf{Th}}
\newcommand{\ThId}{\mathbf{Th}_{\Id}}
\newcommand{\strwCat}{\mathbf{str}\mbox{-}\omega\mbox{-}\mathbf{Cat}}
\newcommand{\strnCat}[1][n]{\mathbf{str}\mbox{-}#1\mbox{-}\mathbf{Cat}}
\newcommand{\wkwCat}{\mathbf{wk}\mbox{-}\omega\mbox{-}\mathbf{Cat}}
\newcommand{\wknCat}[1][n]{\mathbf{wk}\mbox{-}#1\mbox{-}\mathbf{Cat}}

% \newcommand{\wkwCat}{\mathbf{wk}\mbox{-}\omega\mbox{-}\mathbf{Cat}}

%%%%
% Styled words: type theory syntax
%%%%

% \newcommand{\comp}{\textsc{comp}}
% \newcommand{\Contr}{\mathsf{Contr}}
% \newcommand{\elim}{\textsc{elim}}
% \newcommand{\Exch}{\mathsf{Exch}}
% \newcommand{\form}{\textsc{form}}
\newcommand{\Id}{\mathrm{Id}}
% \newcommand{\varidelim}[5]{#4\mathsf{ for }#3\mathsf{ in }#1.#2\mathsf{ via }#5}
% \newcommand{\idelim}[5]{J_{#1.#2}(#3,#4,#5)}
% \newcommand{\intro}{\textsc{intro}}
% \newcommand{\Subst}{\mathsf{Subst}}
% \newcommand{\src}{\mathsf{src}}
% \newcommand{\scterm}{\textsc{term}}
% \newcommand{\tgt}{\mathsf{tgt}}
% \newcommand{\type}{\mathsf{type}}
% \newcommand{\sctype}{\textsc{type}}
% \newcommand{\Weak}{\mathsf{Wkg}}
% \newcommand{\Vble}{\mathsf{Vble}}

%%%%
% Other operators
%%%%

\newcommand{\Clw}{\mathbb{Cl}_\omega}
\newcommand{\ClwQCat}{\mathbb{Cl}^\qcat_\omega}

%%%%
% Binary relations
%%%%

% \renewcommand{\equiv}{\simeq}
% \newcommand{\into}{\rightarrowtail}
% \newcommand{\iso}{\cong}
% \newcommand{\types}{\vdash}

%%%%
% Other symbols
%%%%

% \newcommand{\lscott}{[\![}
% \newcommand{\rscott}{]\!]}





\begin{document}

\section{Bibliography}

Citation styles\ldots\ I don't like \textbf{[17]}---it makes them a complete black box, useless unless one flips constantly back and forth.  I quite like \textbf{[Batanin 2009]}, but it takes up a lot of space, so obliges one to cite less liberally; also it's perhaps a little eccentric.  Overall, \textbf{[Bat09]}\ probably a decent compromise---it's readable for someone who knows the literature a little, and to someone who doesn't, nothing short of the full title is much more useful anyway.

Author naming: full name, or surname only?  I prefer full names; again, since bulkier, this means using them a bit sparinglier than with surnames alone, but since I'm using mostly-readable citation style, I don't have to use names too often, so this seems OK.

\section{Sectioning}  numbering: follow Cisinski[?] and number just by paragraph within section?  Even within chapter??  Ask Michael \& Steve about this first...  but I think something like this is nice; it 

OK!  Here's what I think I want (5.vii):

chapters, section: as usual.

Subsections --- use just for their headings, \emph{not} numbered.  (display in toc?  undecided!)

Theorems, definitions, pars: \emph{all} numbered consecutively within section.
Deliniated paras: numbered/named in bold, with no indent; ordinary paras: small indent.  As in: \\

\noindent \textbf{3.2\ Id-Types are awesome.}\ \ I think that $\Id$-types are awesome, and you should think so too!

The most fundamentally compelling reason for this is that\ldots \\
  
\noindent \textbf{3.3}\ \ However, if we look at some of the alternative eliminators for $\Id$-types, \ldots \\

Hmmm\ldots but what should (sub)section headings look like?  Bold and/or small caps?  Centred and/or left-aligned?  Probably: both centred, section in bold small caps, subsections (if used) in just bold?  (So: all amsbook default, except making sections bold (and maybe also giving them more space?), for better prominence.)

Indentation: the amsbook paragraph indent is rather big!  Make it like in amsart instead?

Also note: text beginning directly after a (sub)section heading thus needs a \texttt{\\newline} and a \texttt{\\noindent}.  (Maybe just put a \texttt{\\newline} in the definiton of subsection somehow?)  CHECKLIST at end: make sure of this!

Running page headers and references: if numbering paras (etc.) within sections,  reference that as [2.3.11] (and show it in running header for easy navigation?)

\section{Notations}
\para The classifying $\CwA_\strat$ of a theory: $\cl_\strat(\TT)$?  Problem: $\cl$ looks like ``closure'' not classifying!  $\C(\TT)$ or $\CC(\TT)$ is more usual, but not brilliant by any means\ldots\ and gets worse as we decorate it $\C_\strat$, $\C_\omega$, $\C^-_\omega$, etc.\ldots

\para The boundary of a pasting diagram??  Batanin and others use $\delta \pi$ to mean the source/target of $\pi$ (since they're equal).  I DON'T LIKE THIS: several reasons!  Firstly, it's misleading: it's not the boundary in the usual well-established sense of $\delta$; that's the pushout of the source and target along their common boundary.  Secondly, it thus leaves us \emph{without} a natural notation for that boundary, even though it's something we use all the time!  Thirdly (this is minor and debatable) it's often helpful to distinguish $s\pi$ and $t\pi$ when working with them, since they each have natural embeddings into $\pi$, etc.

I certainly want to avoid supporting this usage: I'll just write $s\pi$ and $t\pi$ for these.  Question: should I actually conflict with Batanin's usage by using $\delta \pi$ for what I argue it \emph{should} mean, or is this too potentially misleading/confusing?  (Probably, I fear.)  If so, what other notation can I use for the boundary?  $B(\pi)$??  Ugh\ldots

In any case, make sure to explain all this usage in the background introducing pasting diagrams.

\section{Terminologies}

Set up and explain in the first section: there are many equivalent presentations of a theory.  Main ones for us: syntactic; comprehension cat; cat with attributes.  For categorical models, have plain, based, and stratified.  Show: stratified.  Stratified: full, co-reflective subcategory of based, but not equivalent.  (Mention: Pitts' accessible = 2-equivalent to stratified, but still not equivalent.)  Having got equivalence, explain: \emph{agnostic} about representation; use $\ThId$ for any of the $\strat$ categories (strategories?? $\smiley$)

\end{document}