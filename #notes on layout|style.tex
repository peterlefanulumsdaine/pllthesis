\documentclass{amsart}

\usepackage{wasysym}
%\usepackage{ifpdf}
%\usepackage{mathpartir}
\usepackage{amssymb}
%\usepackage[all]{xypic}
%\xyoption{2cell}
%\xyoption{curve}
%\UseTwocells
\input{diagxy}

% \usepackage{makeindex}

% Peter LeFanu Lumsdaine, June 2010
% macros for my thesis

% Contents:
%
% - Binary relations
% - Category names
% - Single letters


%%%%
% Binary relations, operators
%%%%

\newcommand{\cotensor}{\pitchfork}
\renewcommand{\equiv}{\simeq}
\newcommand{\Iff}{\Leftrightarrow}
\newcommand{\Imp}{\Rightarrow}
\newcommand{\into}{\hookrightarrow}
\newcommand{\iso}{\cong}
\newcommand{\propeq}{\simeq}
\newcommand{\mono}{\hookrightarrow}
\newcommand{\tensor}{\otimes}
\newcommand{\To}{\Rightarrow}
\newcommand{\types}{\vdash}

%%%% 
% Single styled characters (or almost single) and character-like symbols
%%%%

\newcommand{\Two}{\mathbf{2}}
\newcommand{\A}{A_\bullet}
\newcommand{\abar}{\overline{a}}
% \newcommand{\uA}[1][]{\underline{A}_{#1}}
% \newcommand{\B}{B_\bullet}
% \newcommand{\ML}{\mathit{ML_I}}
% \newcommand{\MLfrag}{\mathit{ML}^\Id}
\newcommand{\C}{\mathcal{C}}
\newcommand{\CC}{\mathbb{C}}
\newcommand{\D}{\mathcal{D}}
% \newcommand{\bigC}{\mathcal{C}}
% \newcommand{\bC}{\mathbf{C}}
% \newcommand{\Chat}{\widehat{\mathbb{C}}}
% \newcommand{\D}{\mathbb{D}}
% \newcommand{\bigD}{\mathcal{D}}
% \newcommand{\bD}{\mathbf{D}}
\newcommand{\diag}{\delta}
% \renewcommand{\d}{\partial}
\newcommand{\E}{\mathcal{E}}
\newcommand{\f}{\vec f}
\newcommand{\fbf}{\mathbf{f}}
\newcommand{\F}{\mathcal{F}}
\newcommand{\FF}{\mathbb{F}}
\newcommand{\g}{\vec g}
\newcommand{\gbf}{\mathbf{g}}
\newcommand{\G}{\mathbb{G}}
\newcommand{\I}{\mathcal{I}}   % generating cofibrations.  mathscr is prettier,
\newcommand{\J}{\mathcal{J}}   % but I find its I, J confusing.
\newcommand{\K}{\mathcal{K}}    % A class of left maps
\renewcommand{\L}{\mathcal{L}}    % A class of left maps
\newcommand{\NN}{\mathbb{N}}   % Natural numbers
\newcommand{\N}{\mathcal{N}}   % Nerve
% \renewcommand{\P}{P_{\MLfrag}}
\newcommand{\PML}{P_{\MLId}}
% \newcommand{\Pfull}{P_{\ML}}
\newcommand{\PARA}{\textparagraph}
\newcommand{\pow}{\mathcal{P}}
\newcommand{\p}{\vec p}
\newcommand{\SEC}{\textsection}
\newcommand{\R}{\mathcal{R}}    % A class of right maps
\renewcommand{\r}{\vec r}
\renewcommand{\S}{\textsf{\textbf{S}}}    % Another generic type theory
\newcommand{\T}{\textbf{\textsf{T}}}      % A generic type theory
\newcommand{\Tcal}{\textsf{\textbf{T}}}      % A 2-monad
\newcommand{\TT}{\mathbb{T}}    % A generic type theory, seen as a categorical structure
\renewcommand{\u}{\vec u}
\newcommand{\V}{\mathcal{V}}
\renewcommand{\v}{\vec v}
\newcommand{\W}{\mathcal{W}}
\newcommand{\WW}{\mathbb{W}}
\newcommand{\w}{\vec w}
\newcommand{\Xcal}{\mathcal{X}}
\newcommand{\X}{X_\bullet}
\newcommand{\Xbullet}{X_\bullet}
\newcommand{\x}{\vec x}
% \newcommand{\uX}[1][]{\underline{X}_{#1}}
\newcommand{\Ycal}{\mathcal{Y}}
\newcommand{\Y}{Y_\bullet}
\newcommand{\y}{\vec y}
\newcommand{\yon}{\mathbf{y}}
\newcommand{\z}{\vec z}

%%%%
% Styled words: general
%%%%

\newcommand{\Alg}[1]{#1\mbox{-}\mathbf{Alg}}
\newcommand{\IntAlg}[2]{\mathbf{Alg}_{#2}(#1)}
\newcommand{\AMS}{AMS}
\newcommand{\AWFS}{AWFS}
\newcommand{\Cat}{\mathbf{Cat}}
\newcommand{\intCat}[1][-]{\mathbf{Cat}(#1)}
\newcommand{\enrCat}[1][\V]{#1\mbox{-}\mathbf{Cat}}
\newcommand{\nCat}[1][n]{#1\mbox{-}\mathbf{Cat}}
\newcommand{\cl}{\mathbf{cl}}
\newcommand{\ClovFib}{\mathbf{ClovFib}}
\newcommand{\Coll}{\mathbf{Coll}}
\newcommand{\CwA}{\mathbf{CwA}}
\newcommand{\CwAId}{\mathbf{CwA}^{\Id}}
\newcommand{\CwF}{\mathbf{CwF}}
\newcommand{\CwFId}{\mathbf{CwF}^{\Id}}
\newcommand{\Cxt}{\mathrm{Cxt}}
\newcommand{\cxl}{\mathit{cxl}}
\newcommand{\CofCosps}{\mathbf{CofCosps}}
\newcommand{\cod}{\mathrm{cod}}
\newcommand{\del}{\partial}
\newcommand{\dom}{\mathrm{dom}}
\newcommand{\DTT}{\mathbf{DTT}}
\newcommand{\End}{\mathrm{End}}
% \newcommand{\ev}{\mathbf{ev}}
\newcommand{\Fib}{\mathbf{Fib}}
\newcommand{\FibSpans}{\mathbf{FibSpans}}
\newcommand{\FSCC}{\mathbf{FSCC}}
\newcommand{\fscc}{\textsc{fscc}}
\newcommand{\fsccs}{\textsc{fscc}'s}
\newcommand{\FSCS}{\mathbf{FSCS}}
\newcommand{\fscs}{\textsc{fscs}}
\newcommand{\fscss}{\textsc{fscs}'s}
\newcommand{\globe}[1][n]{\textsf{\textbf{G}}_{#1}}
\newcommand{\globes}{\textsf{\textbf{G}}_\bullet}
% \newcommand{\longGSets}{[\mathbb{G}^\op,\mathbf{Sets}]}
\newcommand{\GSets}{\widehat{\mathbb{G}}}
% \renewcommand{\lim}{\varprojlim}
\newcommand{\Lan}{\mathrm{Lan}}
\newcommand{\lax}{\mathrm{lax}}
\newcommand{\MonGlobCat}{\mathbf{MonGlobCat}}
\newcommand{\ML}{\textsf{\textbf{ML}}}
\newcommand{\MLId}{\textsf{\textbf{ML}}^{\Id}}
\newcommand{\ob}{\operatorname{ob}}
\newcommand{\op}{\mathrm{op}}
% \newcommand{\Operads}{\mathbf{Operads}}
% \newcommand{\pd}{\mathbf{pd}}
\newcommand{\PsAlg}[2][]{\mathbf{Ps}_{#1}\mbox{-}{#2}\mbox{-}\mathbf{Alg}}
\newcommand{\QCat}{\mathbf{QCat}}
\newcommand{\qcat}{\mathit{qcat}}
\newcommand{\Ran}{\mathrm{Ran}}
\newcommand{\Sets}{\mathbf{Sets}}
\newcommand{\Spans}[1][]{\mathbf{Spans}_{#1}}
\newcommand{\str}{\mathrm{str}}
\newcommand{\strat}{\textrm{strat}}
\renewcommand{\th}{\mathbf{th}}
\newcommand{\Th}{\mathbf{Th}}
\newcommand{\ThId}{\mathbf{Th}^{\Id}}
\newcommand{\ThIdPi}{\mathbf{Th}^{\Id,\Pi}}
\newcommand{\Tm}{\mathrm{Tm}}
% \newcommand{\tm}{\textsf{tm}}
\newcommand{\Top}{\mathbf{Top}}
\newcommand{\Ty}{\mathrm{Ty}}
% \newcommand{\ty}{\textsf{ty}}
\newcommand{\strMonGlobCat}{\mathbf{MonGlobCat}}
\newcommand{\strwCat}{\mathbf{str}\mbox{-}\omega\mbox{-}\mathbf{Cat}}
\newcommand{\strnCat}[1][n]{\mathbf{str}\mbox{-}#1\mbox{-}\mathbf{Cat}}
\newcommand{\SynPres}{\mathbf{SynPres}}
\newcommand{\SynThy}{\mathbf{SynThy}}
\newcommand{\wkwCat}{\mathbf{wk}\mbox{-}\omega\mbox{-}\mathbf{Cat}}
\newcommand{\wkwGpd}{\mathbf{wk}\mbox{-}\omega\mbox{-}\mathbf{Gpd}}
\newcommand{\wknCat}[1][n]{\mathbf{wk}\mbox{-}#1\mbox{-}\mathbf{Cat}}

% \newcommand{\wkwCat}{\mathbf{wk}\mbox{-}\omega\mbox{-}\mathbf{Cat}}

%%%%
% Styled words: type theory syntax
%%%%

\newcommand{\Bool}{\mathsf{Bool}}
\newcommand{\cellrule}{\mathsf{cell}}
\newcommand{\comp}{\textsc{comp}}
\newcommand{\CONG}{\textsc{cong}}
% \newcommand{\Contr}{\mathsf{Contr}}
\newcommand{\cons}{\mathsf{cons}}
\newcommand{\cxt}{\mathsf{cxt}}
\newcommand{\elim}{\textsc{elim}}
% \newcommand{\Exch}{\mathsf{Exch}}
\newcommand{\form}{\textsc{form}}
\newcommand{\Id}{\mathrm{Id}}
% \newcommand{\varidelim}[5]{#4\mathsf{ for }#3\mathsf{ in }#1.#2\mathsf{ via }#5}
% \newcommand{\idelim}[5]{J_{#1.#2}(#3,#4,#5)}
\newcommand{\intro}{\textsc{intro}}
\newcommand{\refl}{\mathsf{refl}}
\newcommand{\sourcerule}{\mathsf{src}}
\newcommand{\subst}{\mathsf{subst}}
% \newcommand{\src}{\mathsf{src}}
% \newcommand{\scterm}{\textsc{term}}
\newcommand{\sym}{\mathsf{sym}}
\newcommand{\targetrule}{\mathsf{tgt}}
\newcommand{\term}{\mathsf{term}}
\newcommand{\trans}{\mathsf{trans}}
\newcommand{\type}{\mathsf{type}}
% \newcommand{\sctype}{\textsc{type}}
% \newcommand{\Weak}{\mathsf{wkg}}
\newcommand{\var}{\mathsf{var}}

%%%%
% Other operators
%%%%

\newcommand{\Clw}{\mathbf{Cl}_\omega}
\newcommand{\ClwQCat}{\mathbf{Cl}^\qcat_\omega}

%%%%
% Other symbols
%%%%

% \newcommand{\irule}[3]{\inferrule*[#1]{#2}{\quad #3 \quad}}  I can't seem to get this to work, not sure why, so just putting in extra spacing by hand...

% \newcommand{\lscott}{[\![}
% \newcommand{\rscott}{]\!]}


%%%
%%% Diagram annotations, work with diagxy
%%%


\newdir{|>}{!/4.7pt/\dir{|}
        *:(1,-.2)\dir^{>}
        *:(1,+.2)\dir_{>}}

\newbox\pbbox
\setbox\pbbox=\hbox{\xy \POS(75,0)\ar@{-} (0,0) \ar@{-} (75,75)\endxy}
\def\pb{\copy\pbbox}
\newbox\urpbbox
\setbox\urpbbox=\hbox{\xy \POS(0,0)\ar@{-} (75,0) \ar@{-} (0,75)\endxy}
\def\urpb{\copy\urpbbox}
\newbox\pobox
\setbox\pobox=\hbox{\xy \POS(0,75)\ar@{-} (0,0) \ar@{-} (75,75) \endxy}
\def\po{\copy\pobox}

% \newbox\tiltvdashbox
% \setbox\tiltvdashbox{\xy \POS( 

%% typical usage:
%
% $$\bfig \square[A`B`C`D;```]
% \place(100,400)[\pb]
% \place(400,100)[\po]
% \efig$$


%%%%
% Theorem-type environments
%%%%

%% following Cisinski's style, which I found excellent, the theorem-like environments are set up to number _all_ paragraphs [in the conceptual rather than typographic sense] consecutively.  the major advantage of this is making any paragraph referenceable, and hence making the (always rather arbitrary) decision of what to pick out as theorems, definitions, etc. much less consequential and more flexible.


\makeatletter

\newtheoremstyle{mytheorem}{}{}{\itshape}{}{\bfseries}{.}{5\p@ plus\p@ minus\p@}{}

\newtheoremstyle{mydefinition}{}{}{}{}{\bfseries}{.}{5\p@ plus\p@ minus\p@}{}

%% proof environment taken almost verbatim from amsthm.sty, to remove the small caps and indentation that are used in amsbook.cls
\renewenvironment{proof}[1][Proof]{\par
  \pushQED{\qed}%
  \normalfont \topsep6\p@\@plus6\p@\relax
  \trivlist
  \item[\hskip\labelsep
        \itshape
    #1\@addpunct{.}]\ignorespaces
}{%
  \popQED\endtrivlist\@endpefalse
}

\makeatother



\theoremstyle{mytheorem} 
\newtheorem{thm}{Theorem}[section]
\newtheorem{theorem}[thm]{Theorem}
\newtheorem{proposition}[thm]{Proposition}
\newtheorem{lemma}[thm]{Lemma}
\newtheorem{corollary}[thm]{Corollary}
\newtheorem{scholium}[thm]{Scholium}
\newtheorem{conjecture}[thm]{Conjecture}

\theoremstyle{mydefinition}
\newtheorem{definition}[thm]{Definition}
\newtheorem{para}[thm]{}
\newtheorem{exercise}[thm]{Exercise}

%\theoremstyle{remark}
\newtheorem{remark}[thm]{Remark}
\newtheorem{notation}[thm]{Notations}
\newtheorem{example}[thm]{Example}
\newtheorem{examples}[thm]{Examples}

\newtheorem{mydefinition}[thm]{Definition}


\setcounter{tocdepth}{3}
\setcounter{secnumdepth}{2}

\renewcommand{\baselinestretch}{1.5}



\begin{document}

\section{Bibliography}

Citation styles\ldots\ I don't like \textbf{[17]}---it makes them a complete black box, useless unless one flips constantly back and forth.  I quite like \textbf{[Batanin 2009]}, but it takes up a lot of space, so obliges one to cite less liberally; also it's perhaps a little eccentric.  Overall, \textbf{[Bat09]}\ probably a decent compromise---it's readable for someone who knows the literature a little, and to someone who doesn't, nothing short of the full title is much more useful anyway.

Author naming: full name, or surname only?  I prefer full names; again, since bulkier, this means using them a bit sparinglier than with surnames alone, but since I'm using mostly-readable citation style, I don't have to use names too often, so this seems OK.

\section{Sectioning}  numbering: follow Cisinski[?] and number just by paragraph within section?  Even within chapter??  Ask Michael \& Steve about this first...  but I think something like this is nice; it makes the often artificial distinction between numbered definitions/propositions and open discussion much less consequential.

OK!  Here's what I think I want (5.vii):

chapters, section: as usual.

Subsections --- use just for their headings, \emph{not} numbered.  (display in toc?  undecided!)

Theorems, definitions, pars: \emph{all} numbered consecutively within section.
Deliniated paras: numbered/named in bold, with no indent; ordinary paras: small indent.  As in: \\

\noindent \textbf{3.2\ Id-Types are awesome.}\ \ I think that $\Id$-types are awesome, and you should think so too!

The most fundamentally compelling reason for this is that\ldots \\
  
\noindent \textbf{3.3}\ \ However, if we look at some of the alternative eliminators for $\Id$-types, \ldots \\

Hmmm\ldots but what should (sub)section headings look like?  Bold and/or small caps?  Centred and/or left-aligned?  Probably: both centred, section in bold small caps, subsections (if used) in just bold?  (So: all amsbook default, except making sections bold (and maybe also giving them more space?), for better prominence.)

Indentation: the amsbook paragraph indent is rather big!  Make it like in amsart instead?  YES!

Also note: text beginning directly after a (sub)section heading thus needs a \texttt{\\newline} and a \texttt{\\noindent}.  (Maybe just put a \texttt{\\newline} in the definiton of subsection somehow?)  CHECKLIST at end: make sure of this!

Running page headers and references: if numbering paras (etc.) within sections,  reference that as [2.3.11] (and show it in running header for easy navigation?)

\para Figures should also, surely, then be numbered in sequence with everything else.  Is it possible to make \LaTeX{} do this?

\section{Notations}
\para The classifying $\CwA_\strat$ of a theory: $\cl_\strat(\TT)$?  Problem: $\cl$ looks like ``closure'' not classifying!  $\C(\TT)$ or $\CC(\TT)$ is more usual, but not brilliant by any means\ldots\ and gets worse as we decorate it $\C_\strat$, $\C_\omega$, $\C^-_\omega$, etc.\ldots

At least, identifying a theory transparently with its classifying (strat) CwA lets us avoid using $\cl$ except for as $\cl_\omega$.  

\para The boundary of a pasting diagram??  Batanin and others use $\delta \pi$ to mean the source/target of $\pi$ (since they're equal).  I DON'T LIKE THIS: several reasons!  Firstly, it's misleading: it's not the boundary in the usual well-established sense of $\delta$ for discs; that's the pushout of the source and target along their common boundary.  Secondly, it thus leaves us \emph{without} a natural notation for that boundary, even though it's something we use all the time!  Thirdly (this is minor and debatable) it's often helpful to distinguish $s\pi$ and $t\pi$ when working with them, since they have different natural embeddings into $\pi$, etc.

I certainly want to avoid supporting this usage: I'll perhaps just write $s\pi$ and $t\pi$ for these.  Question: should I actually conflict with Batanin's usage by using $\delta \pi$ for what I argue it \emph{should} mean, or is this too potentially misleading/confusing?  (Probably, I fear.)  If so, what other notation can I use for the boundary?  $B(\pi)$??  Ugh\ldots

In any case, make sure to explain all this usage in the background introducing pasting diagrams.

\para In ``fundamental $\omega$-groupoid of a type'', what to call $\MLId[X]$?  $\MLId[X]$ is (I think) the best option in isolation (usefully evocative of polynomial rings), but $\globe{0}$ or $\\T[\yon(0)]$ fit better into bigger picture$\ldots$

\para Give the type-theoretic operads catchy names, or just use the definitions (descriptive but unwieldy), or use slightly abbreviated forms of their definitions, i.e.\ $\End(\globes)$ etc.?  Probably this last.

\para Adjunctions? ---remind myself how to place the adjunction symbol them, in each of the various diagram styles I've used!

\section{Terminologies}

\para Set up and explain in the first section: there are many equivalent presentations of a theory.  Main ones for us: syntactic; comprehension cat; cat with attributes.  For categorical models, have plain, based, and stratified.  Show: stratified.  Stratified: full, co-reflective subcategory of based, but not equivalent.  (Mention: Pitts' accessible = 2-equivalent to stratified, but still not equivalent.)  Having got equivalence, explain: \emph{agnostic} about representation; use $\ThId$ for any of the $\strat$ categories (strategories?? $\smiley$)

\para ``pylon diagrama'': fun but probably unnecessary; we already have ``higher spans'', which does the job fine. (Not to mention both ``telescopes'' and ``ladders'', which I abhor; going by which, others might well dislike ``pylon'' just as much!)

\para What term should we use for what $\globes$ form?  A co-weak $\omega$-category, a co-weak-$\omega$-category, a weak co-$\omega$-category, a weak $\omega$-cocategory?  A co-$P$-algebra, or a $P$-co-algebra?

\para Should I conscientiously talk about ``co-operations'' in ``co-endomorphism operads'', or is it clearer to drop the co-?

\section{Checklist}  Notations etc.\ I should check for consistencty during proofing:
\begin{itemize}
\item Any of the notations/terminologies above that I've debated!
\item ``pylon'' $\longrightarrow$ ``(higher) span''
\item \texttt{\\colon} vs.\ :
\end{itemize}

\section{Other}

\para Type-theoretic rules: give in ``hypothetical'' forms.  What does this mean for axioms with no hypotheses?  Shold they be given in the empty context, or in arbitrary context? 
\end{document}