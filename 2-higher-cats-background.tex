\section{Weak factorisation system background}

The study of \emph{weak factorisation systems} originated in homotopy theory, and has since become important in higher category theory, especially with recent devlopments in \emph{algebraic\footnote{aka \emph{natural}} weak factorisation systems}.  We will not require any of the more sophisticated results of this theory; but there are a couple of basic constructions from it which we will meet repeatedly, so which we now recall here.  See \emph[Ch.~1]{hovey:book} for further background on the classical theory, and \emph{garner:understanding} for the algebraic case.

\begin{definition}For maps $f, g$ in a category $\C$, we say \emph{$f$ is (weakly)\footnote{We will never require the strong counterparts of these properties, so will assume weak by deafult.} (left) orthogonal to $g$}, written \emph{$f \pitchfork g$}, if every commutative square from $f$ to $g$ has a filler:
$$\xymatrix{D \ar[r] \ar[d]_f & Y \ar[d]^g \\ C \ar[r] \ar@{..>}[ur]^\exists & X}$$

There are several synonyms: ``$f$ has the (weak) \emph{left lifting property} against $g$'', and corresponding right-handed versions phrasing it as a property of of $g$. 

If $\L$, $\R$ are classes of maps in $\C$, we say $\L \pitchfork \R$ if $f \pitchfork g$ for every $f \in \L$, $g \in \R$.
\end{definition}

Simple examples of such classes are given in $\Sets$ by $\L =$ all maps, $\R =$ split surjections (so all surjections, if we assume choice); or $\L =$ split injections (equivalently, injections $S \mono X$ where if $S$ is empty then so is $X$), $\R = $ all maps.

\begin{para} \label{para:cofib-generated-wfs} A more powerful example, in $\Top$, is: $\L =$ retracts of relative cell complexes, $\R =$ trivial fibrations in the Quillen model structure, aka weakly contractible maps.  The definitions of these are an example of a very general construction: if $\J$ is any class of maps, we define $\J^\pitchfork$ to be the class of maps weakly right orthogonal to all maps in $\J$; and dually we define ${}^\pitchfork \J$.

It is easy to check that a class $\J^\pitchfork$ must contain all identities, and be closed under composition, retracts, and pullback along arbitrary maps.  Dually, ${}^\pitchfork \J$ is closed under identities, composition, retracts, and pushouts.  A little more thought shows that the classes are closed under (co-)transfinite composition (i.e.\ under taking colimits of colimit-preserving diagrams indexed by ordinals (or their opposites)).

Most typically we start with a \emph{set} of maps $\J$ in a category $\C$, then form $\R = \J^\pitchfork$ (``$\J$-fibrations''), $\L = {}^\pitchfork (\J^\pitchfork)$ (``$\J$-cofibrations'').

For the case of Serre trivial cofibrations, we start with the inclusions of spheres into discs (\emph{cells}), as their boundaries:
$$\J = \{\,S^i \mono D^i\ |\ i \in \N\,\}.$$

Then a map $p \colon Y \to X$ is in $\J^\pitchfork$ just if whenever we are given a cell in $X$ and a lift of its boundary to $Y$, we can lift the whole cell to $Y$, with the given boundary.  So any loop in $Y$ that is null-homotopic in $X$ must be null-homotopic in $Y$, and so on: $p$ is forced in particular to be a weak homotopy equivalence.

But now by the closure properties of ${}^\pitchfork (\J^\pitchfork)$, the generating boundary inclusions are not the only maps we can lift against trivial fibrations.  We can lift any \emph{relative cell complex}, i.e.\ any transfinite composite of pushouts of the boundary inclusions, essentially by lifting it one cell at a time; and indeed we can also lift retracts of these.

This exemplifies one of the two main flavours of classes of orthogonal maps: the left maps are \emph{cofibrations} of some sort, the right maps are \emph{trivial fibrations}.  The other flavour is dual: \emph{trivial cofibrations} vs.\ \emph{fibrations}.

Things work similarly from any basic set $\J$ in a co-complete category $\C$; \emph{$\J$-cell complexes} (transfinite compositions of pushouts of maps in $\J$) will always be $\J$-cofibrations.  Under good circumstances (co-completeness and smallness conditions), all $\J$-cofibrations will be retracts of these, and every map of $\C$ will have a factorisation into a $J$-cell complex followed by a $\J$-fibration, giving a \emph{weak factorisation system}; but this is beyond what we will require.
\end{para}

\begin{para}
The above theory can be improved and refined by changing the objects of study from mere \emph{properties} of maps to \emph{structures} on maps.  This both strengthens the resulting theory---constructions become more natural, functorial, etc.---and more naturally describes mathematical practice: demonstrating ``fibrationhood'' of a map typically involves giving some kind of structure on it.  This is also a more naturally constructive approach---an existence property should always be given by exhibiting witnesses---and as such, turns out to develop much better in a constructive setting than the classical theory does.

We recall only the barest rudiments of this theory.  Given a set $\J$ of maps in a category $\C$, a \emph{$\J^\oslash$}-structure on a map $p$ is a function \emph{choosing} a diagonal filler for each commutative square from a $\J$-map to $p$.  Assuming choice, the maps admitting a $\J^\oslash$-structure are exactly the $\J$-fibrations. $\J^\oslash$-structured maps, together with commutative squares between them preserving the structure (in a suitable sense), form a category, with a forgetful functor to $\Arr(\C)$.

Structured versions of the closure properties above can be formulated: any identity carries a natural $\J^\oslash$-structure, as does any composite of maps with $\J^\oslash$-structures; any pullback of a map with $\J^\oslash$-structure again carries a natural such structure, as does any retract of such a map.  When we speak of some category of structured maps being \emph{closed under composition}, and similar, we will mean statements of this form.

Again analogously to the classical theory, any map exhibited as (a retract of) a \emph{formal $\J$-cell complex} lifts against any map with $\J^\oslash$-structure; these liftings are now moreover \emph{natural} in the appropriate structure preserving maps.  Again, under good circumstances, we have factorisations of every map into (the realisation of) a formal $\J$-cell complex, followed by a $\J^\oslash$-structured map, and this is now moreover functorial \cite{garner:understanding}.

The construction of $\J^\oslash$ and $\J$-cell complexes is the one which recurs several times in the present work: see \todo{[give refs!]}.
\end{para}

\begin{para}
The above is all that will be needed in the sequel; but a curious observation is perhaps worth making before moving on.  As we algebraised the theory, we passed from considering (as our right maps) classes of maps closed under identities, composition, and pullback, to considering isofibrations over $\Arr(\C)$\footnote{Indeed the exapmles typically studied in algebraic wfs's are monadic over $\Arr(\C)$.}, ``closed'' in the structured sense under composition and pullback.  But continuing just a hop, skip and jump further in this direction brings one to the notion of a CwA with units and $\Sigma$-types---and one could motivate this step never having thought of type theory, considering just examples from classical mathematics, e.g.\ replacing Grothendieck fibrations (which fit into the awfs framework) with pseudo-functors into $\Cat$ (which do not).
\end{para}

\section{Higher-categorical background}

\subsection{Strict higher categories}

There are various approaches to higher category theory, and in particular many definitions of weak higher categories \cite{leinster:ten-definitions}, \cite{cheng-lauda:guidebook}.  We will use the \emph{globular operadic} approach, as set out originally by Michael Batanin in \cite{batanin:natural-environment}, and later re-presented and slightly modified by Tom Leinster, \cite{leinster:book}.  These two sources are the authoritative references for most of this material; \cite{leinster:survey} also provides an excellent, and more streamlined, introduction.

\begin{figure}
$$
\begin{array}{c}
\begin{array}{cccc}
\ \xy
(0,0)*{\bullet};
(0,80)*{a};
\endxy \quad
&
\ \xy
(0,0)*{\bullet}="a";
(0,80)*{\scriptstyle a};
(400,0)*{\bullet}="b";
(400,80)*{\scriptstyle b};
{\ar "a";"b"};
(200,80)*{f};
\endxy \ 
&
\ \xy
(0,0)*+{\bullet}="a";
(0,80)*{\scriptstyle a};
(450,0)*+{\bullet}="b";
(450,80)*{\scriptstyle b};
{\ar@/^1pc/^{f} "a";"b"};
{\ar@/_1pc/_{g} "a";"b"};
{\ar@{=>} (210,85)*{};(210,-85)*{}};
(280,0)*{\alpha};
\endxy \ 
&
\ \xy
(0,0)*+{\bullet}="a";
(600,0)*+{\bullet}="b";
{\ar@/^1.75pc/^{f} "a";"b"};
{\ar@/_1.75pc/_{g} "a";"b"};
{\ar@2{->}@/_0.5pc/|{\alpha} (220,140);(220,-140)} ;
{\ar@2{->}@/^0.5pc/|{\beta} (380,140);(380,-140)} ;
{\ar@3{->} (225,-20);(375,-20)};
(300,60)*{\Theta};
\endxy \ 
\end{array} \\
\begin{array}{ccc}
\ \xy(0,0)*{\bullet}="a";
(0,80)*{\scriptstyle a};
(300,0)*{\bullet}="b";
(300,80)*{\scriptstyle b};
(600,0)*{\bullet}="c";
(600,80)*{\scriptstyle c};
{\ar^f "a";"b"};
{\ar^g "b";"c"};
\endxy \ 
&
\ \xy
(0,0)*+{\bullet}="a";
(0,80)*{\scriptstyle a};
(500,0)*+{\bullet}="b";
(500,80)*{\scriptstyle b};
{\ar@/^1.75pc/|f "a";"b"};
{\ar|{f'} "a";"b"};
{\ar@/_1.75pc/|{f''} "a";"b"};
{\ar@{=>}^{\alpha} (250,160)*{};(250,50)*{}} ;
{\ar@{=>}^{\gamma} (250,-50)*{};(250,-160)*{}} ;
(0,-250)*{\ };
\endxy \ 
&
\ \xy
(0,0)*+{\bullet}="a";
(0,80)*{\scriptstyle a};
(400,0)*+{\bullet}="b";
(400,80)*{\scriptstyle b};
(800,0)*+{\bullet}="c";
(800,80)*{\scriptstyle c};
{\ar@/^1.1pc/|f "a";"b"};
{\ar@/_1.1pc/|{f'} "a";"b"};
{\ar@/^1.1pc/|g "b";"c"};
{\ar@/_1.1pc/|{g'} "b";"c"};
{\ar@{=>}^{\alpha} (200,80)*{};(200,-80)*{}} ;
{\ar@{=>}^{\beta} (600,80)*{};(600,-80)*{}} ;
\endxy \ \\
g \circ_0 f &
\gamma \circ_1 \alpha &
\beta \circ_0 \alpha
\end{array}
\\
\begin{array}{cc}
\ \xy(0,0)*{\bullet}="a";
%(0,80)*{\scriptstyle a};
(300,0)*{\bullet}="b";
%(300,80)*{\scriptstyle b};
(600,0)*{\bullet}="c";
%(600,80)*{\scriptstyle c};
(900,0)*{\bullet}="d";
%(900,80)*{\scriptstyle d};
{\ar^f "a";"b"};
{\ar^g "b";"c"};
{\ar^h "c";"d"};
\endxy \ &
\ \xy
(0,0)*+{\bullet}="a";
(400,0)*+{\bullet}="b";
{\ar@/^1.5pc/ "a";"b"};
{\ar "a";"b"};
{\ar@/_1.5pc/ "a";"b"};
{\ar@{=>}^{\alpha} (200,150)*{};(200,25)*{}} ;
{\ar@{=>}^{\gamma} (200,-25)*{};(200,-150)*{}} ;
(800,0)*+{\bullet}="c";
{\ar@/^1.5pc/ "b";"c"};
{\ar "b";"c"};
{\ar@/_1.5pc/ "b";"c"};
{\ar@{=>}^{\beta} (600,150)*{};(600,25)*{}} ;
{\ar@{=>}^{\delta} (600,-25)*{};(600,-150)*{}};
(0,250)*{\ };
(0,-220)*{\ };
\endxy \ \\
\begin{array}{c} h \circ_0 (g \circ_0 f) =  \\ (h \circ_0 g) \circ_0 f \end{array} &
\begin{array}{c}(\delta \circ_0 \gamma) \circ_1 (\beta \circ_0 \alpha) = \\
(\gamma \circ_1 \alpha) \circ_0 (\delta \circ_1 \beta)\end{array}
\end{array}
\end{array}
$$
\caption{Some cells, composites, and associativities in a strict higher category \label{figure:assoc-laws}} 
\end{figure}

In the globular approach, the underlying substance of an $\omega$-category $\Cbf$ consists of a set $C_n$ of ``$n$-cells'' for each $n > 0$.  The $0$- and $1$-cells correspond to the objects and arrows of an ordinary category: each arrow $f$ has source and target objects $a = s(f)$, $b = t(f)$.  Similarly, the source and target of a 2-cell $\alpha$ are a parallel pair of 1-cells $f,g: a \two b$, and generally the source and target of an $(n+1)$-cell are a parallel pair of $n$-cells.  Summing this up we arrive at:

\begin{definition}
A \emph{globular set} $\A$ is a diagram of sets and functions
$$ A_0 \two/<-`<-/^{s_0}_{t_0} A_1 \two/<-`<-/^{s_1}_{t_1} A_2 \two/<-`<-/^{s_2}_{t_2} A_3 \two/<-`<-/ \ldots $$
such that $s_i \circ t_{i+1} = s_i \circ s_{i+1}$ and $t_i \circ t_{i+1} = t_i \circ s_{i+1}$---the \emph{globularity} conditions, asserting that the source and target of any cell are parallel.
\end{definition}

Some notation we will use throughout the sequel: we will often drop the subscripts on arrows as far as possible, writing just $s$ and $t$; $s^i$, $t^i$ will denote compositions of these arrows, as usual; and for $c \in A_n$ and $k \leq n$, we take $s_k(c) := s^{n-k}(c)$, $t_k(c) := t^{n-k}(c)$, the $k$-dimensional source and target of $c$.  (Of course when $n = k+1$, this agrees with the original usage of $s_k,t_k$.)

A map of globular objects $f_\bullet \colon \A \to \B$ is a sequence of functions $f_n \colon A_n \to B_n$, preserving the globular structure, in that $s_i \circ f_{i+1} = f_i \circ s_i$ and $t_i \circ f_{i+1} = f_i \circ t_i$, or more compactly, $sf = fs$, $tf = ft$.

More generally, let $\G$ be the category with objects $\N$ and arrows generated by
$$ 0 \two^{s_0}_{t_0} 1 \two^{s_1}_{t_1} 2 \two^{s_2}_{t_2} 3 \two \ldots $$
subject to the equations $ts = ss$, $st = tt$.  Then a \emph{globular object} in a category $\E$ is a functor $\A \colon \G \to \E$, and a map of these is a natural transformation between the functors.

Thus the category of globular sets is just the category $\GSets$ of presheaves on $\G$.  The Yoneda embedding $y \cdot \G \to \GSets$ yields some globular sets which will be of particular use and importance to us: the basic $n$-cells $y(n)$, and their boundaries $\del (n) \subseteq y(n)$, the maximal proper subobject of $y(n)$, consisting of all \emph{non-identity} maps into $n$.  Note that for $n < 0$, we have $\del(n) \iso y(n-1) +_{\del(n-1)} y(n-1)$, a \emph{parallel pair} of $n-1$-cells.

(For the finite-dimensional versions of all the bove, and all that follows, $\G$ is replaced by the category $\G_n$, defined just as $\G$ except with no objects or arrows above $n$; and $n$-globular set is a presheaf $\A \colon \G_n \to \Sets$, and so on.) \\

To complete the definition of strict $\omega$-categories, we simply add the structure of composition on top of this basic data.  As illustrated in Figure \ref{fig:assoc-laws}, we want to be able to compose cells whenever the target of one is the source of another in some lower dimension.  Specifically, for any $k < n$, the set of $k$-composable $n$-cells is the pullback
$$\bfig
\node AnxAn(0,400)[A_n \times_k A_n]
\node Ant(500,400)[A_n]
\node Ans(0,0)[A_n]
\node Ak(500,0)[A_k]
\arrow[AnxAn`Ans;]
\arrow[AnxAn`Ant;]
\arrow[Ans`Ak;s_k]
\arrow[Ant`Ak;t_k]
\place(100,300)[\pb]
\efig .$$

\begin{definition} \footnote{This is a somewhat old-fashioned presentation of the definition; cf.\ eg.\ \cite{street:algebra-of-oriented-simplices}.  It is, however, completely equivalent to the now more usual presentation via iterated enrichment.} 
A \emph{strict $\omega$-category} $\Cbf$ is a globular set $\Cbu$ together with composition operations for each $k < n$
$$\circ_k \colon C_n \times_k C_n \to C_n$$
and unit maps
$$r_n \colon C_{n-1} \to C_n$$
(for which we use index conventions analogous to those for $s$, $t$),
 such that firstly
\begin{itemize}
\item for each $k < n$, $(C_k,C_n, \circ_k, r_n)$ forms a category, i.e.\ the source and target of composites and units are ``what one would expect'', and the associativity and unit laws
$$ c \circ_k (b \circ_k a) = (c \circ_k b) \circ_k a$$
$$ a \circ_k (r_n s_k a) = a = (r_n t_k a) \circ_k a$$
hold (for appropriately composable $a, b, c \in C_n$); and additionally,
\item the \emph{interchange law} holds: for $j < k < n$, and suitable $a,b,c,d \in C_n$,
$$ (d \circ_j c) \circ_k (b \circ_j a) = (d \circ_k b) \circ_j (c \circ_k a)$$
(also as illustrated in Figure \ref{fig:assoc-laws}). 
\end{itemize}
\end{definition}

\begin{para}This presentation exhibits strict $\omega$-categories explicitly as models of an essentially algebraic theory, monadic over $\G$:
$$\bfig 
\node GSets(0,0)[\GSets]
\node strwCat(800,0)[\strwCat]
\arrow|a|/@/^0.6em//[GSets`strwCat;F]
\arrow|b|/@/^0.6em//[strwCat`GSets;U]
\place(400,0)[\bot]
\efig$$
and it is shown in \cite{street:petit-topos}, \cite{leinster:book} that the resulting monad $T = FU$ in $\GSets$ is familially representable, and hence cartesian---both its functor part, and its monad structure $\eta$, $\mu$.

(Recall that a functor is cartesian if it preserves pullbacks, and a natural transformation is cartesian if all its naturality squares are pullbacks.)

The cells of $TX$ may thus be seen as \emph{formal composites}, or \emph{labelled pasting diagrams}, of cells of $X$.  In particular, taking $1$ to be the terminal globular set with a single cell of each dimension, we define:


% \begin{para}
% \todo{[Actually, take this out and put it in the intro!  For here, keep it snappy.]}
% Thus in a strict higher-category, the associativity, unitality and interchange laws hold literally as equations.  These, along with the existence of the composition operations in the first place, may be summed up by the \emph{generalised associativty} principle: each \emph{labelled pasting diagram} has a \emph{unique} composite---where for now, a pasting diagram means something like the pictures appearing in Figure \ref{fig:assoc-laws}; we make this precise in \ref{para:pasting-diagrams} below.
% 
% In a weak higher category, we do not expect these laws to hold in such a strict form: prototypical examples are monoidal categories, where associativity and unitality only hold up to coherent natural isomorphisms $A \tensor (B \tensor C) \iso (A \tensor B) \tensor C$, and the higher fundamental groupoids of spaces, where composition of paths is associative only up to homotopy, and these homotopies themselves are invertible only up to higher homotopies, and so on.
% \end{para}

\begin{definition} \label{def:pasting-diagrams}
A \emph{pasting diagram} is a cell of $T1$, the free strict $\omega$-category on the terminal globular set.  We will often write $\pd$ for $T1$.
\end{definition}

There are several useful combinatorial representations for pasting diagrams.  \cite[8.1]{leinster:book} provides an inductive description in terms of free monoids:
$$\pd_0 := T_{\Mon}(\emptyset) = \{\, () \,\}$$
$$\pd_{n+1} := T_{\Mon}( \pd_m) = \{\, (\pi_1,\ldots,\pi_l)\ |\ l \geq 0,\ \pi_i \in \pd_n\, \}$$

Here, for instance, the element $(((),(),()),(),(())) \in \pd_2$ represents the 2-cell $(c_2 \circ_1 c_2 \circ_1 c_2) \circ_0 (r_2 c_1) \circ_0 (c_2)$ of $T1$ (where $c_i$ is the unique $i$-cell of $1$):
$$\xy
(0,0)*+{\bullet}="a";
(400,0)*+{\bullet}="b";
{\ar@/^0.8pc/ "a";"b"};
{\ar@/_0.8pc/ "a";"b"};
{\ar@{=>} (200,70)*{};(200,-70)*{}} ;
(800,0)*+{\bullet}="c";
{\ar "b";"c"};
% (0,250)*{\ };
% (0,-220)*{\ };
(1200,0)*+{\bullet}="d";
{\ar@/^1.65pc/ "c";"d"};
{\ar@/^0.55pc/ "c";"d"};
{\ar@/_0.55pc/ "c";"d"};
{\ar@/_1.65pc/ "c";"d"};
{\ar@{=>} (1000,185)*{};(1000,85)*{}} ;
{\ar@{=>} (1000,50)*{};(1000,-50)*{}};
{\ar@{=>} (1000,-85)*{};(1000,-185)*{}};
\endxy$$
This representation may be seen syntactically as a normal form theorem: every formal expression in the operations $r_n$, $\circ_k$ may be put into normal form by eliminating identities wherever possible and moving lowest-dimensional composition outermost.

The $k$-dimensional source or target of a pasting diagram (they always coincide) is given by simply discarding everything nested more than $k$ levels deep.

With this representation, we can construct for each $n$-dimensional pasting diagram $\pi$ an $n$-dimensional globular set $\widehat{\pi}$ (formalising our depictions of pasting diagrams):
$$\widehat{(\pi_1,\ldots,\pi_l)}_i\ =\ \left\{ \begin{array}{ll} \sum_{1 \leq j \leq l} (\widehat{\pi_j})_{i-1} & i > 1  \\ \{0,\ldots,l\} & i = 0 \end{array}\right.$$

These now provide the promised familial representation of $T$, decomposing $TX$ into the fibers of the map $T! \colon TX \to T1$:
$$TX_n = \sum_{\pi \in T1_n} \GSets(\widehat{\pi},X).$$

Here and in the next few constructions, it is highly instructive to compare this to the analogous presentation of the ``free monoid'' monad on $\Sets$: $$T_\Mon X = \sum_{n \in \N} \Sets([n],X).$$
\end{para}

\subsection{Globular operads, à la Leinster}

A weak higher category, as outlined in the introduction, should again consist of a globular set together with some composition operations, similar to those in a strict higher category; but now there may be multiple composition operations for each shape of pasting diagram.  To present appropriate algebraic theories of such structures, we introduce the definition:

\begin{definition}[Globular operads, definition 1]
A \emph{globular operad} $P$ is a cartesian monad $T_P$ on $\GSets$, together with a cartesian monad map $\alpha \colon T_P \to T$.
\end{definition}

Thus a globular operad yields again an algebraic theory over globular sets, via its monad part $T_P$; the natural transformation $\alpha \colon T_P \A \to T \A$ ensures that all the operations of $T_P$ can be viewed as composition operations for pasting diagrams, while the cartesianness of $\alpha$ ensures that the set of such operations for any pasting diagram is uniform in $X$.  Precisely,

\begin{definition}[Globular operads, definition 2]
A \emph{globular operad} may equivalently be specified by a globular set $P$ with maps $a\colon P \to T1$ (``arity''), $e\colon  1 \to P$ (``units''), $m \colon  P \times_{T1} TP \to P$ (``composition''), such that
$$\bfig
\node 1(-250,-150)[1]
\node P(250,-150)[P]
\node T1(0,-650)[T1]
\arrow[1`P;e]
\arrow|l|[1`T1;\eta]
\arrow|r|[P`T1;a]
\node PxTP(1000,0)[\ \,P \times_{T1} TP]
\node P'(1750,0)[P]
\node TP(1250,-250)[TP]
\node P''(750,-250)[P]
\node T1'(1000,-500)[T1]
\node TT1(1400,-525)[T^2 1]
\node T1''(1500,-800)[T1]
\arrow[PxTP`P';m]
\arrow[PxTP`TP;]
\arrow[PxTP`P'';]
\arrow|a|[TP`T1';T!]
\arrow|a|[P''`T1';a]
\arrow|r|/{@{>}@/^2pt/}/[TP`TT1;Ta]
\arrow|l|/{@{>}@/^1pt/}/[TT1`T1'';\mu]
\arrow|r|[P'`T1'';a]
\place(1000,-100)[\upb]
\efig$$
commute (i.e.\ $e$ and $m$ are maps over $T1$), satisfying the axioms \todo{[correct the orientation of $m$!]}
$$m \cdot ( \eta \cdot e \times 1_P) = 1_P = m \cdot (\eta \times e) \colon  P \to P,$$
$$m \cdot (\mu \times m) = m \cdot (Tm \times 1_P) : T^2 P \times_{T^2 1} TP \times_{T1} P \to P.$$
\end{definition}

Given an operad in the original form $T_P,\alpha$, we take $P := T_P 1$, $a = \alpha_1$, $m = \mu_1$, $e = \eta_1$.

Conversely given $P, a, e, m$, we recover the full monad as $T_P X := TX \times_T1 P$, and $\eta$, $\mu$, $\alpha$ as approriate pullbacks of $e$, $m$, $a$.  In particular, $T_P$ must again be familially representable:
$$(T_P X)_n \iso \sum_{\pi \in T1_n} P(\pi) \times \GSets(\widehat{\pi},X).$$
where $P(\pi)$ denotes the fiber of $a$ over $\pi \in T1_n$, the set of ``$\pi$-ary operations'' of $P$.  The map $e$ then gives us a unary $n$-cell ``identity'' operation for each $n$, while the map $m$ allows us to compose these operations appropriately. \\

As groups have actions, rings have modules, theories have models, so operads have algebras: 
\begin{definition}An \emph{algebra} for an operad $P$ is an algebra for the monad $T_P$; or equivalently, a globular set $A$ together with a map $c \colon P \times_{T1} TA \to A$, such that
$$c \circ (e \times \eta) = 1_A \colon A \to A,$$
$$c \circ (m \times \mu) = c \circ (1_P \times Tc) \colon P \times_{T1} TP \times_{T^2 1} T^2 P \to A.$$

This structure is also called an \emph{action} of $P$ on $A$.
\end{definition}

Actions may also be reformulated in terms of endomorphism operads:
\begin{definition}
The \emph{endomorphism operad} $\End_{\GSets}(A)$ of a globular set $A$ has underlying set $\sum_{\pi : T1} [TX_\pi,X]$ (interpreted in the internal language of $\GSets$, i.e.\ computed as the internal hom $[TX,X \times T1]$ in the slice $\GSets/T1$), and structure maps constructed using the universal property of the internal hom, together with the monad structure of $T$.
\end{definition}

Actions of an operad $P$ on $A$ then correspond to operad maps $P \to \End_{\GSets}(A)$.  Via either definition, there is an evident category of $P$-algebras $\Alg{P}$.

\begin{example}The object $T1$ itself carries a natural operad structure, making it the terminal operad; its associated monad is just $T$, and its algebras are strict $\omega$-categories. 
\end{example}

We would like to define weak $\omega$-categories as algebras for some operad.  However, to give a reasonable theory of some sort of $\omega$-categories, an operad needs to have enough operations to implement the usual compositions and identities; and given any two parallel operations of the same arity in $P$, they should be an operation of the next dimension connecting them, witnessing that they are ``equal up to homotopy''.  In Leinster's definition, these two conditions are imposed by a single piece of structure:\footnote{This is the sole point where Batanin's original definition differs in more than just presentation.}

\begin{definition}
A \emph{contraction} on a map of globular sets $f \colon Y \to X$ is a diagonal filler for every square
$$\bfig
\node dn(0,400)[\del(n)]
\node yn(0,0)[y(n)]
\node X(500,400)[X]
\node Y(500,0)[Y]
\arrow[dn`X;]
\arrow/@{ >->}/[dn`yn;]
\arrow[X`Y;f]
\arrow[yn`Y;]
\arrow/@{.>}/[yn`X;]
\efig$$
A map is \emph{contractible} if it admits a contraction.  Contractible maps are distinguished diagrammatically as $f \colon Y \to/{-|>}/ X$.
\end{definition}

It is immediate from \ref{para:general-contractions} that maps-with-contraction are closed under identities and composition, and under pullback along arbitrary maps.

\begin{definition}
A \emph{contraction} on an operad $P$ is a contraction on its arity map $a \colon P \to/{-|>}/ T1$, or equivalently, a natural contraction on each component of the natural transformation $\alpha$.
\end{definition}

Roughly, this asserts that $P$ is homotopically equivalent to $T$ (see \cite{garner:homotopy-theoretic-universal-property} for a precise exploration of this idea).  Any contractible operad may be taken as giving a reasonable theory of $\omega$-categories.  In particular, for maximal weakness, let $L$ be the \emph{initial} operad-with-contraction:

\begin{definition}[\cite{leinster:book}]
A \emph{weak $\omega$-category} is an $L$-algebra.  $\wkwCat := \Alg{L}$.
\end{definition}

The universal property $L$ ensures that if $P$ is any other contractible operad, choosing a contraction $\chi$ on $P$ determines a map $f_\chi \colon L \to P$ and hence a functor $f_\chi^* \colon \Alg{P} \to \wkwCat$.  However, the subtleties of contractible vs.\ -with-contraction meet us here: this functor will depend upon the choice of contraction (though it seems reasonable to expect that the functors induced by different contractions will differ only up to some notion of weak equivalence).

\subsection{Globular operads, à la Batanin}

The preceding elegant presentation of globular operads in terms of cartesian monads is due to Leinster.  However, for most of our purposes below it will be easier to construct our desired operads using the original machinery of \cite{batanin:natural-environment}.  We thus take a brief detour to set up the machinery of \emph{monoidal globular categories}, and to investigate pasting diagrams further, before returning to operads in this setting.

\begin{definition}
(See \cite[2.3]{batanin:natural-environment} for all the details here elided.)  A \emph{monoidal globular category} $\E$ is a globular category 
$$ \E_0 \two/<-`<-/^S_T \E_1 \two/<-`<-/^S_T \E_2 \two/<-`<-/^S_T \E_3 \two/<-`<-/ \ldots $$
equipped with composition functors
$$\tensor_k \colon \E_n \times_k \E_n \to \E_n$$
and unit maps
$$Z \colon \E_n \to \E_{n+1}$$
satisfying the same source and target conditions as units and composition in a category, and also with natural transformations
$$ \alpha \colon C \tensor_k (B \tensor_k A) \iso (C \tensor_k B) \tensor_k A$$
$$ \rho \colon A \tensor_k (Z_n S_k A) \iso A \qquad  \lambda \colon (Z_n T_k A) \tensor_k A \iso A$$
$$ \chi \colon (D \tensor_j C) \tensor_k (B \tensor_j A) \iso (D \tensor_k B) \tensor_j (C \tensor_k A)$$
satisfying various coherence axioms.

A \emph{monoidal globular functor} between monoidal globular categories $\E, \F$  is a globular functor $F_\bullet \cdot \E_\bullet \to \F_\bullet$, commuting with the composition and units up to coherent isomorphism.  Together with \emph{monoidal globular transformations}, these form a 2-category $\MonGlobCat$.   
\end{definition}

So monoidal globular categories bear essentially the same relationship to strict $\omega$-categories as a monoidal categories do to monoids: sets are replaced by categories, and axioms hold just up to coherent isomorphism.

This is made precise by Mark Weber in \cite{weber:monoidal-pseudo-algebras}: $\MonGlobCat$ is (2-equivalent to) the 2-category of normalised pseudo-algebras for the ``internal strict $\omega$-category'' 2-monad $\Tcal$ on $[\G,\Cat]$.

\begin{figure}[hbtp]
$$
\bfig
\node Cn(300,-100)[A_n]
\node An1l(0,-400)[A^s_{n-1}]
\node Bn1r(600,-400)[A^t_{n-1}]
\node An2l(0,-800)[A^s_{n-2}]
\node Bn2r(600,-800)[A^t_{n-2}]
\node A1l(0,-1200)[A^s_1]
\node B1r(600,-1200)[A^t_1]
\node A0l(0,-1600)[A^s_0]
\node B0r(600,-1600)[A^t_0]
\arrow[Cn`An1l;s]
\arrow|r|[Cn`Bn1r;t]
\arrow[An1l`An2l;s]
\arrow||/@{->}^(0.35){t}/[An1l`Bn2r;]
\arrow||/@{->}_(0.35){s}/[Bn1r`An2l;]
\arrow|r|[Bn1r`Bn2r;t]
\arrow/@{}|<>(0.58)\vdots/[An2l`A1l;]
\arrow/@{}|<>(0.58)\vdots/[Bn2r`B1r;]
\arrow[A1l`A0l;s]
\arrow||/@{->}^(0.35){t}/[A1l`B0r;]
\arrow||/@{->}_(0.35){s}/[B1r`A0l;]
\arrow|r|[B1r`B0r;t]
\efig
\qquad \qquad \qquad
\bfig
\node Cn(300,-100)[A_n]
\node An1l(0,-400)[A_n]
\node Bn1r(600,-400)[A_n]
\node An2l(0,-800)[A^s_{n-1}]
\node Bn2r(600,-800)[A^t_{n-1}]
\node A1l(0,-1200)[A^s_1]
\node B1r(600,-1200)[A^t_1]
\node A0l(0,-1600)[A^s_0]
\node B0r(600,-1600)[A^t_0]
\arrow[Cn`An1l;1_{A_n}]
\arrow|r|[Cn`Bn1r;1_{A_n}]
\arrow[An1l`An2l;s]
\arrow||/@{->}^(0.35){t}/[An1l`Bn2r;]
\arrow||/@{->}_(0.35){s}/[Bn1r`An2l;]
\arrow|r|[Bn1r`Bn2r;t]
\arrow/@{}|<>(0.4)\vdots/[An2l`A1l;]
\arrow/@{}|<>(0.4)\vdots/[Bn2r`B1r;]
\arrow[A1l`A0l;s]
\arrow||/@{->}^(0.35){t}/[A1l`B0r;]
\arrow||/@{->}_(0.35){s}/[B1r`A0l;]
\arrow|r|[B1r`B0r;t]
\efig$$
\caption{\label{fig:some-spans} A higher span; an identity span.}
\end{figure}

\begin{example}
For any category $\C$ with all pullbacks, the monoidal globular category $\Spans[\C]$ of \emph{higher spans} in $\C$ has
$$\Spans[\C]_n = [(\G/n)^\op, \C]$$

So an object of $\Spans[\C]_n$, an \emph{$n$-span} in $\C$, is a diagram $(A^s_i,A^t_i\:_{(i < n)};\,A_n)$ as in Fig.~\ref{fig:a-span}, satisfying the equations $ss = st$, $ts = tt$ wherever appropriate; and a map of $n$-spans $(f^s_i,f^t_i;f_n) \colon (A^s_i,A^t_i;A_n) \to (B^s_i,B^t_i;B_n)$ is a sequence of maps between them, commuting with all the source and target maps.

The source of $(A^s_i,A^t_i\:_{(i < n)};\,A_n)$ is then $(A^s_i,A^t_i\:_{(i < {n-1})};\,A^s_{n-1})$, and similarly its target is $(A^s_i,A^t_i\:_{(i < {n-1})};\,A^t_{n-1})$.

Identity spans are defined as in Fig.~\ref{fig:some-spans}; composition $\tensor_k$ of spans is defined by pullback, as in Fig.~\ref{fig:composite-spans}.

A functor $\C \to \C'$ preserving pullbacks induces a monoidal globular functor $\Spans[\C] \to \Spans[C']$. 
\end{example}

\begin{figure}[htbp]
$$\bfig
\node A4(-350,-100)[A_4]
\node B4(350,-100)[B_4]
\node As3(-600,-400)[A^s_3]
\node At3(-100,-400)[A^t_3]
\node Bs3(100,-400)[B^s_3]
\node Bt3(600,-400)[B^t_3]
\node As2(-600,-800)[A^s_2]
\node At2(-100,-800)[\phantom{A^t_2}]
\node AtBs(0,-800)[A^t_2\! =\! B^s_2]
\node Bs2(100,-800)[\phantom{B^s_2}]
\node Bt2(600,-800)[B^t_2]
\node As1(-300,-1200)[A^s_1\!=\!B^s_1]
\node At1(300,-1200)[A^t_1\!=\!B^t_1]
\node As0(-300,-1600)[A^s_0\!=\!B^s_0]
\node At0(300,-1600)[A^t_0\!=\!B^t_0]
%
\arrow[A4`As3;]
\arrow[A4`At3;]
%
\arrow[B4`Bs3;]
\arrow[B4`Bt3;]
%
\arrow[As3`As2;]
\arrow[As3`At2;]
\arrow[At3`As2;]
\arrow[At3`At2;]
%
\arrow[Bs3`Bs2;]
\arrow[Bs3`Bt2;]
\arrow[Bt3`Bs2;]
\arrow[Bt3`Bt2;]
%
\arrow[As2`As1;]
\arrow[As2`At1;]
\arrow[AtBs`As1;]
\arrow[AtBs`At1;]
\arrow[Bt2`As1;]
\arrow[Bt2`At1;]
%
\arrow[As1`As0;]
\arrow[As1`At0;]
\arrow[At1`As0;]
\arrow[At1`At0;]
\efig
\qquad \to/{|->}/ \qquad 
\bfig
\node AxB4(0,-100)[A_4 \times_{A^s_2} B_4]
\node AxBs3(-300,-400)[A^s_3 \times_{A^s_2} B^s_3]
\node AxBt3(300,-400)[A^t_3 \times_{A^s_2} B^s_3]
\node As2(-300,-800)[A^s_2]
\node Bt2(300,-800)[B^t_2]
\node As1(-300,-1200)[A^s_1\!=\!B^s_1]
\node At1(300,-1200)[A^t_1\!=\!B^t_1]
\node As0(-300,-1600)[A^s_0\!=\!B^s_0]
\node At0(300,-1600)[A^t_0\!=\!B^t_0]
%
\arrow[AxB4`AxBs3;]
\arrow[AxB4`AxBt3;]
%
\arrow[AxBs3`As2;]
\arrow[AxBs3`Bt2;]
\arrow[AxBt3`As2;]
\arrow[AxBt3`Bt2;]
%
\arrow[As2`As1;]
\arrow[As2`At1;]
\arrow[Bt2`As1;]
\arrow[Bt2`At1;]
%
\arrow[As1`As0;]
\arrow[As1`At0;]
\arrow[At1`As0;]
\arrow[At1`At0;]
\efig
$$
\caption{\label{fig:composite-spans} A 2-composition of 4-spans $B \tensor_2 A$}
\end{figure}

We will often apply this in the case where $\C = \D^\op$, for some familiar $\D$; so then we will work with \emph{co-spans} in $\D$, whose composites are computed by pushouts, and so on.  

Although of course technically identical, there is an important difference in intuition between the two orientations.  When working in $\Spans[\C]$, the objects of a span are typically objects \emph{containing} cells, such as the sets of $n$-cells of a higher category.  When working $\Spans[\D^\op]$, we think of the objects of a span as objects \emph{representing} cells, such as the topological globes $D^n$ or the universal globes $\yon(n)$ in $\GSets$, with ``co-source'' and ``co-target'' maps $s,t \colon A^{s,t}_i \to A_{i+1}$ embedding the lower-dimensional globes as the sources or targets of higher ones. \\

The crucial function of monoidal globular categories is as the setting within which one can define globular operads and their algebras.  We will not need the former, only the latter.

\begin{definition}
A \emph{globular object} $\A$ of a monoidal globular category $\E$ is a globular functor $1 \to \E_\bullet$, where $1$ is the terminal globular gategory.  Concretely, a globular object in $\E$ is a sequence of objects $A_i \in \E_i$, with $S(A_{i+1}) = T(A_{i+1}) = A_i$.
\end{definition}

Globular objects in $\Spans[\C]$ correspond precisely to globular objects in $\C$.  (We often abuse notation here by using $A_n$ both for a span and for the object at its apex.)

Given a globular object $\A$ of a monoidal globular category $\E$, we may extend it to a monoidal globular functor $\pd \to \E$ (where the sets of $\pd$ are regarded here as discrete categories).  This follows immediately from the universal property of $T1$ together with the description of monoidal globular categories as certain pseudo-algebras for $\T$:
$$\xymatrix{\pd = \Tcal 1 \ar[r]^{\Tcal \A} & \Tcal \E_\bullet \ar[d] \\ 1 \ar[r]^{\A} &  \E_\bullet}$$

We denote objects in the image of this extended functor by $A^\pi$, for $\pi \in \pd$.  Intuitively, if $A_n$ is the object of $n$-cells of $\A$, then $A^\pi$ is the object of diagrams of shape $\pi$ in $\A$; or in the dual case, if $C_n$ is a representing object for $n$-cells, then $C^\pi$ represents diagrams of shape $\pi$.

In particular, $\yon \colon \G \to \GSets$ gives a globular object in $\Spans[\GSets^\op]$, whose diagram objects are exactly the realisations $\widehat{\pi}$ of pasting diagrams constructed above\\

Now if $\C$ is co-complete, and $\Cbu : \G \to \C$ is a co-globular object, we may form the left Kan extension of $\Cbu$ along the Yoneda embedding $\yon \colon \G \to/{ >->}/ \GSets$, analogous to the ``geometric realisation'' of globular or simplicial objects in $\Top$.  $\Lan_\yon \Cbu$ realises any globular set as a colimit in $\C$, using the objects $C_n$ as templates for the cells.

Since $\Lan_\yon \Cbu$ preserves colimits, it lifts to give a monoidal globular functor $\Spans[\Lan_\yon \Cbu] \colon \Spans[\GSets^\op] \to \Spans[\C^\op]$, and since it sends the basic globes $\yon$ to $\Cbu$, we have moreover a commutative (up to natural isomorphism) diagram:
$$\bfig
\node pd(0,0)[\pd]
\node GSets(600,300)[{\Spans[\GSets^\op]}]
\node C(1200,0)[{\Spans[\C^\op]}]
\arrow|a|[pd`GSets;\yon]
\arrow|b|[pd`C;\Cbu]
\arrow|a|[GSets`C;]
\efig$$

Thus we may compute the diagram objects of $\Cbu$ as the image under $\Lan_\yon \Cbu$ of those of $\yon$:
$$C^\pi = \Lan_\yon \Cbu (\widehat{\pi}) = \colim_{c \in \int\, \widehat{\pi}}  C_{\dim c}$$

This formula is easiest explained by example: if $\pi = (\xymatrix{ \bullet \rtwocell & \bullet \rtwocell & \bullet})$, then
\begin{eqnarray*} C_\pi & \iso & \colim \left( 
\bfig
\node C0l(0,0)[C_0]
\node C0m(700,0)[C_0]
\node C0r(1400,0)[C_0]
\node C1tl(350,300)[C_1]
\node C1tr(1050,300)[C_1]
\node C1bl(350,-300)[C_1]
\node C1br(1050,-300)[C_1]
\node C2l(350,0)[C_2]
\node C2r(1050,0)[C_2]
\arrow[C0l`C1tl;s]
\arrow[C0m`C1tl;t]
\arrow[C0m`C1tr;s]
\arrow[C0r`C1tr;t]
\arrow[C1tl`C2l;s]
\arrow[C1tr`C2r;s]
\arrow[C1bl`C2l;t]
\arrow[C1br`C2r;t]
\arrow[C0l`C1bl;s]
\arrow[C0m`C1bl;t]
\arrow[C0m`C1br;s]
\arrow[C0r`C1br;t]
\efig
\right) \\
& \iso & C_2 +_{C_0} C_2,
\end{eqnarray*}
or in the dual case,
\begin{eqnarray*} A_\pi & \iso & \lim \left( 
\bfig
\node A0l(0,0)[A_0]
\node A0m(700,0)[A_0]
\node A0r(1400,0)[A_0]
\node A1tl(350,300)[A_1]
\node A1tr(1050,300)[A_1]
\node A1bl(350,-300)[A_1]
\node A1br(1050,-300)[A_1]
\node A2l(350,0)[A_2]
\node A2r(1050,0)[A_2]
\arrow[A1tl`A0l;s]
\arrow[A1tl`A0m;t]
\arrow[A1tr`A0m;s]
\arrow[A1tr`A0r;t]
\arrow[A2l`A1tl;s]
\arrow[A2r`A1tr;s]
\arrow[A2l`A1bl;t]
\arrow[A2r`A1br;t]
\arrow[A1bl`A0l;s]
\arrow[A1bl`A0m;t]
\arrow[A1br`A0m;s]
\arrow[A1br`A0r;t]
\efig
\right) \\
& \iso & A_2 \times_{A_0} A_2,
\end{eqnarray*}
giving the object of 0-composable pairs of 2-cells in $A$.

A slightly more careful calculation shows that diagram objects are computed by this formula even when $\C$ is not co-complete. \\

We can now use this to define internal algebras for globular operads, in monoidal globular categories:
\begin{definition}[\protect{\cite[7.2]{batanin:natural-environment}}]
If $\A$ is a globular object in a monoidal globular category $\E$, there is a globular operad $\End_\E(\A)$, given by $\End_\E(\A)(\pi) = \E_n(A^\pi,A_n)$, for each $\pi \in \pd_n$.

A monoidal globular functor $F \colon \E \to \F$ induces an operad map $\End_\E(\A) \to \End_\F(F \A)$.
\end{definition}

In the case $\E = \Spans[\Sets]$, this agrees with our earlier description of the endomorphism operad of a globular set.

\begin{definition}
For a globular operad $P$, a $P$-algebra structure on a globular object $\A$ of $\E$ is an operad map $P \to \End_\E(\A)$.
\end{definition}

We then call $\A$ an \emph{(internal) $P$-algebra} in $\E$; or when $\E = \Spans[\C]$, a $P$-algebra in $\C$, or when $\E = \Spans[\D^\op]$, a $P$-coalgebra in $\D$.

As usual with algebraic structures, homming into internal operad algebras yields external ones.  If $X$ is any object of a category $\C$, then there is an obvious functor
$$ \Hom(X,-) \colon \Spans[\C] \to \Spans[\Sets],$$
which is moreover monoidal globular; so if $\A$ is an internal $P$-algebra in $\C$, we have operad maps
$$P \to \End_{\Spans[\C]}(\A) \to \End_{\Spans(\Sets)}(\Hom(X,\A))$$
and hence a $P$-algebra structure on the globular set $\Hom(X,\A)$.   Similarly, homming out of a coalgebra gives an algebra.

\subsection{Contracting pasting diagrams}

(Unlike the rest of this chapter, the material of this subsection is somewhat original; the basic concepts are not especially new, but the precise techniques used here do not seem to predate \cite{lumsdaine:tlca} and \cite{garner-van-den-berg}.)

\begin{para}Some early presentations of pasting diagrams introduced them as isomorphism classes of \emph{contractible globular sets}.  Here, we describe a methodical procedure for contracting them.  Specifically, we give a partial operation $\pi \mapsto \pi^-$, which under repeated application eventually reduces any pasting diagram $\pi$ to the trivial pasting diagram $\bullet$.

Using the free monoid representation, $()$ is the trivial pasting diagram $\bullet$; it is already contracted, so $()^-$ is not defined.  When $\pi$ is the path of length $l > 1$
$$(\underbrace{(),\ldots,()}_l )$$
we take $\pi^-$ to be the path of length $l-1$.  For any other pasting diagram $\pi = (\pi_1,\ldots,\pi_l)$, with $l \geq 1$ and $\pi_i$ not all equal to $()$, we take $\pi^- := (\pi_1,\ldots,\pi_i^-,\ldots,\pi_k)$, where $i$ is minimal such that $\pi_i$ is not already contracted.

Roughly, this operation removes the leftmost cell (in algebraic order, or rightmost in diagrammatic order) of dimension $> 1$ if there are any such, or of dimension $1$ if there are none higher.  It is clear by induction on e.g.\ dimension that under repeated application of $(-)^-$, any pasting diagram eventually reaches $()$.

$$\xy
(0,0)*+{\bullet}="a";
(400,0)*+{\bullet}="b";
{\ar@/^0.8pc/ "a";"b"};
{\ar@/_0.8pc/ "a";"b"};
{\ar@{=>} (200,70)*{};(200,-70)*{}} ;
(800,0)*+{\bullet}="c";
{\ar "b";"c"};
% (0,250)*{\ };
% (0,-220)*{\ };
(1200,0)*+{\bullet}="d";
{\ar@/^1.65pc/ "c";"d"};
{\ar@/^0.55pc/ "c";"d"};
{\ar@/_0.55pc/ "c";"d"};
{\ar@/_1.65pc/ "c";"d"};
{\ar@{=>} (1000,185)*{};(1000,85)*{}} ;
{\ar@{=>} (1000,50)*{};(1000,-50)*{}};
{\ar@{=>} (1000,-85)*{};(1000,-185)*{}};
(600,-250)*+{\pi};

\endxy
  \qquad \to/{|->}/ \qquad 
\xy
(0,0)*+{\bullet}="a";
(400,0)*+{\bullet}="b";
{\ar@/^0.8pc/ "a";"b"};
{\ar@/_0.8pc/ "a";"b"};
{\ar@{=>} (200,70)*{};(200,-70)*{}} ;
(800,0)*+{\bullet}="c";
{\ar "b";"c"};
% (0,250)*{\ };
% (0,-220)*{\ };
(1200,0)*+{\bullet}="d";
{\ar@/^1.65pc/ "c";"d"};
{\ar@/^0.55pc/ "c";"d"};
{\ar@/_0.55pc/ "c";"d"};
% {\ar@/_1.65pc/ "c";"d"};
{\ar@{=>} (1000,185)*{};(1000,85)*{}} ;
{\ar@{=>} (1000,50)*{};(1000,-50)*{}};
% {\ar@{=>} (1000,-85)*{};(1000,-185)*{}};
(600,-250)*+{\pi^-};
\endxy$$
\end{para}

\begin{para} \label{para:pruning-realisation} We wil need to know a little about the realisations of these; in fact, some apparently very crude statements will be enough.

For any non-trivial pasting diagram $\pi$, there is an inclusion
$$\widehat{\pi^-} \to/{ >->}/ \widehat{\pi}$$
whose image consists of all of $\pi$ except for two cells, one the target of the other.  When $\pi$ is just a path, this condition determines what the inclusion must be; otherwise it is constructed by recursion on $\pi^-$, using the definition of $\pi^-$ together with the explicit description of $\widehat{\pi}$ in \ref{def:pasting-diagrams}. 

It follows from this description of the image that we have a pushout square:
$$\bfig
\node yk1(0,400)[\yon(k-1)]
\node yk(0,0)[\yon(k)]
\node pim(500,400)[\widehat{\pi^-}]
\node pi(500,0)[\widehat{\pi}]
\arrow[yk1`yk;s]
\arrow[yk`pi;]
\arrow[yk1`pim;]
\arrow[pim`pi;]
\place(400,100)[\po]
\efig$$

Hence if $\A$ (resp.\ $C_\bullet$) is a globular (co-globular) object in any (co-)complete category, we obtain by applying the appropriate Kan extension a pullback (pushout) square:
$$\bfig
\node Ak1(500,0)[A_{k-1}]
\node Ak(500,400)[A_{k}]
\node Apim(0,0)[A^{\pi^-}]
\node Api(0,400)[A^{\pi}]
\arrow[Ak`Ak1;s]
\arrow[Api`Ak;]
\arrow[Apim`Ak1;]
\arrow[Api`Apim;]
\place(100,300)[\pb]
\efig 
  \qquad 
\bfig
\node Ck1(0,400)[C_{k-1}]
\node Ck(0,0)[C_{k}]
\node Cpim(500,400)[C^{\pi^-}]
\node Cpi(500,0)[C^{\pi}]
\arrow[Ck1`Ck;s]
\arrow[Ck`Cpi;]
\arrow[Ck1`Cpim;]
\arrow[Cpim`Cpi;]
\place(400,100)[\po]
\efig$$
(again, a slightly more careful calculation avoids the need for (co-)completeness, but we will only need the (co-)complete case).

Finally, for any non-trivial $\pi$, we either have $s(\pi^-) = (s\pi)^-$, or $s(\pi^-) = s\pi$; the resulting square
$$\bfig
\node spim(0,400)[\widehat{s(\pi^-)}]
\node spi(0,0)[\widehat{s(\pi)}]
\node pim(500,400)[\widehat{\pi^-}]
\node pi(500,0)[\widehat{\pi}]
\arrow[spim`spi;]
\arrow[spi`pi;s]
\arrow[spim`pim;s]
\arrow[pim`pi;]
\efig$$
commutes, as does its realisation over any (co-)globular object; and similarly with $t$ in place of $s$. 
\end{para}

\begin{para}As presented here, this construction is rather ad hoc.  A natural and flexible setting for this sort of step-by-step contraction is the Batanin tree representation of pasting diagrams: ways of contracting a pasting diagram in this sort of manner correspond precisely to ways of pruning leaves off its Batanin tree. 

However, since we do not require the Batanin tree representation for any other purpose, for brevity we give just this formulaic procedure, which suffices for the applications in the present work.
\end{para}
