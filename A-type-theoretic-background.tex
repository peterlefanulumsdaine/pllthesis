% this file is called up by thesis.tex
% content in this file will be fed into the main document

%: ----------------------- name of chapter  -------------------------
\chapter{Background: Martin-L\"o{}f Type Theory}

Possibly just make this a presentation of the type theory?

\section{Presentation overview}

\para The type theories we consider are all essentially variants on the basic type theory originally presented in \cite{ML:predicative-part}.  See also (TODO: pick out good references!)

The core of the type theory, its basic judgements and structural rules, is given in Section \ref{para:structural-core}; this will remain constant in all the theories we consider.

The type-constructors and various other rules that will be used are given in Section \ref{para:further-rules}; we will consider theories including various different subsets of these rules.

Section \ref{para:tt-lemmas} contains some fundamental lemmas concerning the resulting type theories: basic proof-theoretic properties, admissible rules, and so on.  TODO: this has changed!  Memo to self: don't write overviews of chapters before starting the chapters themselves...

For further background, see \cite{pitts:categorical-logic} (a notably thorough and precise presentation), \cite[\S6]{jacobs:categorical-logic} (for a thoughtful discussion of this system within the wider type-theoretic context), \cite{hofmann:syntax-and-semantics}, \cite[Ch.3]{n-p-s:programming} (discussion of the use of simply-typed calculus as the ``raw language'), and \cite{martin-loef:predicative-part} (the originial presentation of the theory).  The presentation uses essentially the formal presentation of \cite{pitts:categorical-logic}, in the notation of \cite{jacobs:categorical-logic}, slightly modified to include dependent contexts and their morphisms. 

(TODO: perhaps fully expound the presentation rather than just recapping?)

Notes (re-organise and/or remove later):

--- point of using $\lambda$-calculus as raw language: to separate the subtleties of binding and capture-avoiding substitution from those of dependency and well-formedness.

--- point of having the raw language already simply-typed, not completely un(i)typed: to maintain decidability of $\equiv$. 

--- abuse of notation: making variables explicit

--- dependent contexts and maps: to formulate Frobenius rule

--- dependent maps: really only need $\Gamma \types \vec f : \Delta \To \Delta'$ in case either $\Gamma = \diamond$ (for substitution rules) or $Delta = \diamond$ (for Frobenius rules).  However, it's simpler to give this than those two separately, plus natural for syntactic cwa.

--- syntactic sugar: $\x : \Gamma \types a(\x) : A(\x)$ really means $\Gamma \types a : A$, where all of these are \emph{closed} metaexpressions, but $a$, $A$ have arities $\term^n \to \term$, $\term^n \to \type$, where $n$ is the length of $\Gamma$; and so on.


\para{Raw syntax} \label{para:raw-syntax}

Define: ``raw language''.  (NB: sometimes called \emph{metalanguage}, but not meaning the language in which we reason about the type theory; rather, plays a similar r\^o{}le to that of strings (or trees) of symbols in simpler systems.)

Define: signature, pre-terms, pre-types, pre-terms, judgements.

\para{Judgement forms} \label{para:judgement-forms}

\begin{center}\begin{tabular}{|@{\ }c@{\qquad \qquad}c@{\ }|}
\hline
\multicolumn{2}{|c|}{Basic judgement forms} \\
\hline
$\Gamma \types A \ \type $ & $ \Gamma \types a:A $ \\
$\Gamma \types A = A' \ \type $& $ \Gamma \types a = a' : A $ \\
\hline
\end{tabular}

\begin{tabular}{|@{\ }c@{\qquad \qquad}c@{\ }|}
\hline
\multicolumn{2}{|c|}{Derived judgement forms} \\
\hline
$ \Gamma \types \Delta \ \cxt$ & $\Gamma \types {\vec f} : \Delta \To \Delta' $  \\
$ \Gamma \types \Delta = \Delta' \ \cxt$ & $\Gamma \types {\vec f} = {\vec f}' : \Delta \To \Delta'$ \\
\hline
\end{tabular}

Explain in what sense these are derived judgements!

(Discuss in parens why we use dependent cxts and morphisms.)

Perhaps (if feeling pedantic, or can find way to say it non-pedantically) discuss arities of things in judgements, etc.?

\para{Rules: Structural core} \label{para:structural-core}


\begin{tabular}{c}
Rules for contexts:
\\ \ \\
$\inferrule*[right=$\cxt$-$\diamond\diamond$]{\ }{\diamond \types \diamond \ \cxt}$ \quad $\inferrule*[right=$\cxt$-$\diamond$]{\diamond \types \Gamma\ \cxt}{\Gamma \types \diamond \ \cxt}$ \quad $\inferrule*[right={$\cxt$=-$\diamond$}]{\diamond \types \Gamma\ \cxt}{\diamond \types \diamond = \diamond \ \cxt}$
\\ \ \\
$\inferrule*[right=$\cxt$-$\cons$]{\Gamma \types \Delta\ \cxt \\\\ \Gamma , \Delta \types A \ \type}{\quad \Gamma \types \Delta, A\ \cxt\quad}$ \qquad
$\inferrule*[right={$\cxt$=-$\cons$}]{\Gamma \types \Delta = \Delta'\ \cxt \\\\ \Gamma, \Delta \types A = A'\ \type}{\quad \Gamma \types \Delta, A = \Delta', A'\ \cxt \quad}$ 
\\ \ \\
% $\inferrule*[right=$\cxt$-$\subst$]{\x : \Gamma \types \Delta(\x)\ \cxt \\\\ \diamond \types \f : \Gamma' \To \Gamma}{\quad \y : \Gamma' \types \Delta(\f(\y))\ \cxt \quad}$ \qquad 
% $\inferrule*[right={$\cxt$=-$\subst$}]{\x : \Gamma \types \Delta(\x) = \Delta'(\x)\ \cxt \\\\ \diamond \types \f = \f' : \Gamma' \To \Gamma}{\quad \y : \Gamma' \types \Delta(\f(\y)) = \Delta'(\f'(\y))\ \cxt\quad }$ 
% \\ \ \\ 
\end{tabular}

\begin{tabular}{c}
Rules for types: \\ \ \\
$\inferrule*[right={$\type$=-$\refl$}]{\Gamma \types A\ \type}{\quad \Gamma \types A = A\ \type \quad} \qquad
\inferrule*[right={$\type$=-$\sym$}]{\Gamma \types A = B\ \type}{\quad \Gamma \types B = A\ \type \quad}$ \\ \ \\
$\inferrule*[right={$\type$=-$\trans$}]{\Gamma \types A = B\ \type \\\\ \Gamma \types B = C\ \type}{\quad \Gamma \types A = C\ \type \quad} \qquad 
\inferrule*[right=$\type$-$\subst$]{\x : \Gamma \types A(\x)\ \type \\\\ \diamond \types \f : \Gamma' \To \Gamma}{\quad \y : \Gamma' \types A(\f(\y))\ \type \quad}$ \\ \ \\
$\inferrule*[right={$\type$=-$\subst$}]{\x : \Gamma \types A = A'\ \type \\\\ \diamond \types \f = \f' : \Gamma' \To \Gamma}{\quad \y:\Gamma' \types A(\f(\y)) = A'(\f'(\y))\ \type\quad }$ \\ \ \\
\end{tabular}

\begin{tabular}{c}
Rules for terms: \\ \ \\
$\inferrule*[right={\sf var}]{\Gamma \types A\ \type \\ \Gamma, A \types \Delta\ \cxt}{\quad \Gamma, x:A, \Delta \types x : A \quad}$ \\ \ \\
$\inferrule*[right={\sf term-coerce}]{\Gamma \types a : A \\\\ \Gamma \types A = A'\ \type}{\quad \Gamma \types a : A' \quad}$ \qquad \quad
$\inferrule*[right={\sf term=-coerce}]{\Gamma \types a = a' : A \\\\ \Gamma \types A = A'\ \type}{\quad \Gamma \types a = a' : A' \quad}$
\\ \ \\
$\inferrule*[right={$\term$=-$\refl$}]{\Gamma \types a : A}{\quad \Gamma \types a = a : A\quad} \qquad \qquad
\inferrule*[right={$\term$=-$\sym$}]{\Gamma \types a = b : A}{\quad \Gamma \types b = a : A \quad}$
\\ \ \\
$\inferrule*[right={$\term$=-$\trans$}]{\Gamma \types a = b : A \\\\ \Gamma \types b = c : A}{\quad \Gamma \types a = c : A \quad} \qquad 
\inferrule*[right=$\term$-$\subst$]{\x : \Gamma \types a(\x) : A(\x) \\\\ \diamond \types \f : \Gamma' \To \Gamma}{\quad \y : \Gamma' \types a(\f(\y)) : A(\f(\y)) \quad}$
\\ \ \\
$\inferrule*[right={$\term$=-$\subst$}]{\x : \Gamma \types a(\x) = a'(\x) : A(\x) \\\\ \diamond \types \f = \f' : \Gamma' \To \Gamma}{\quad \y:\Gamma' \types a(\f(\y)) = a'(\f'(\y)) : A(\f(\y)) \quad}$
\end{tabular}


\begin{tabular}{c}
Rules for context maps: 
\\ \ \\
$\inferrule*[right=$\cxt$-$\diamond$]{\diamond \types \Gamma\ \cxt}{\diamond \types \diamond : \Gamma \To \diamond}$ \qquad $\inferrule*[right={$\cxt$=-$\diamond$}]{\diamond \types \Gamma\ \cxt}{\diamond \types \diamond = \diamond : \Gamma \To \diamond}$
\\ \ \\
$\inferrule*[right=$\cxt$-$\cons$]{\Gamma \types \f : \Delta' \To \Delta \\\\ \x : \Gamma , \y : \Delta(\x) \types A(\x,\y) \ \type \\\\ \x : \Gamma, \z : \Delta'(\x) \types a(\x,\z) : A(\x,\f(\x,\z))}{\quad \Gamma \types (\f,a) : \Delta' \To (\Delta, A) \quad}$ \qquad
$\inferrule*[right=$\cxt$-$\cons$]{\Gamma \types \f : \Delta' \To \Delta \\\\ \x : \Gamma , \y : \Delta(\x) \types A(\x,\y) \ \type \\\\ \x : \Gamma, \z : \Delta'(\x) \types a(\x,\z) : A(\x,\f(\x,\z))}{\quad \Gamma \types (\f,a) : \Delta' \To (\Delta, A) \quad}$ \qquad
\\ \ \\
% $\inferrule*[right=$\cxt$-$\subst$]{\x : \Gamma \types \Delta(\x)\ \cxt \\\\ \diamond \types \f : \Gamma' \To \Gamma}{\quad \y : \Gamma' \types \Delta(\f(\y))\ \cxt \quad}$ \qquad 
% $\inferrule*[right={$\cxt$=-$\subst$}]{\x : \Gamma \types \Delta(\x) = \Delta'(\x)\ \cxt \\\\ \diamond \types \f = \f' : \Gamma' \To \Gamma}{\quad \y : \Gamma' \types \Delta(\f(\y)) = \Delta'(\f'(\y))\ \cxt\quad }$ 
% \\ \ \\ 
\end{tabular}
\end{center}

\para{Type constructors and miscellaneous rules} \label{para:further-rules}

Rules for type-constructors go here: $\Id$-, $\Pi$-, $\Sigma$-, with just a tip of the hat to $\NN$-, $\Bool$, \ldots

\begin{center}
\begin{tabular}{c}
Rules for $\Id$-types: 
\\ \ \\
$\inferrule*[right=$\Id$-$\form$]{\Gamma \types A\ \type}{\Gamma, x,y : A \types \Id_A(x,y)\ \type}$ \qquad $\inferrule*[right={$\Id$-$\intro$}]{\Gamma \types A\ \type}{\Gamma, x : A \types r(x) : \Id_A(x,x)}$
\\ \ \\
$\inferrule*[right=$\Id$-$\elim$]{\Gamma, x,y : A, p : \Id_A(x,y), \w : \Delta(x,y,p) \types C(x,y,p,\w)\ \type \\
\Gamma, z:A, \v : \Delta(z,z,r(z)) \types d(z,\v) : C(z,z,r(z),\v)}
{\Gamma, x,y:A, p:\Id_A(x,y), \w:\Delta(x,y,p) \types J_{A;\  x,y,p.\Delta;\ x,y,p,\w.C}(z,\v.\ d(z,\v);\ x,y,p,\w) : C(x,y,p,\w)}$
\\ \ \\
$\inferrule*[right=$\Id$-$\comp$]{\Gamma, x,y : A, p : \Id_A(x,y), \w : \Delta(x,y,p) \types C(x,y,p,\w)\ \type \\
\Gamma, z:A, \v : \Delta(z,z,r(z)) \types d(z,\v) : C(z,z,r(z),\v)}
{\Gamma, x:A, \w:\Delta(x,y,p) \types J_{A;\ x,y,p.\Delta}(z,\v.\ d(z,\v);\ x,x,r(x),\w) = d(x,\w): C(x,y,p,\w)}$
\\ \ \\
$\inferrule*[right={reflection}]{\Gamma \types e : \Id_A(a,b)}{\Gamma \types a = b : A}$
\\ \ \\
$\inferrule*[right=K]{\Gamma, x: A, p : \Id_A(x,x), \w : \Delta(x,p) \types C(x,p,\w)\ \type \\
\Gamma, z:A, \v : \Delta(z,r(z)) \types d(z,\v) : C(z,r(z),\v)}
{\Gamma, x:A, p:\Id_A(x,x), \w:\Delta(x,p) \types K_{A;\ x,p.\Delta;\ x,p,\w.C}(z,\v.\ d(z,\v);\ x,p,\w) : C(x,p,\w)}$
\\ \ \\
\end{tabular}
\end{center}

Include optional things: $\eta$- and extensionality rules for function-types; K and the truncation rules for $\Id$-types. 

Discuss implications between these?

\para{Translations} \label{para:translations}

Define translation, category of type theories.

\section{Categories with attributes}

TODO: chase up references on these!  In particular: names --- type-categories, categories with attributes, categories with families, split comprehension categories, \ldots ?

TODO: ask around about reachable vs.\ ``stratified'' cwa's.

\para{Equivalence with type theories}

Define: full split comprehension categories \cite[4.10]{jacobs:comprehension-categories} / categories with attributes.  \cite[\S 6.4]{pitts:categorical-logic}

Recall (from Pitts): \emph{equivalence} (honest 1-equivalence!) between type theories with no constructors and cwa's.  TO DO: read Jacobs in full (\& chase further\ldots) re theories with constructors!

Define: $\Id$-structure, $\Pi$-structure, etc. on cwa's.  (Reference: \cite{awodey-warren}.)

Extend equivalence above; define $\ThId$, $\ThIdPi$, etc.  

Fact: since all essentially algebraic (or otherwise), these categories are all co-complete!

\para{The ``strong $\Sigma$'s'' monad}  Explicitly give the monad on $\ThId$ throwing in all dependent contexts as new types, and interpreting equality as described in loc.~cit.  TODO: look up that interpretation of equality!  Refer to Thorsten's OTT also.

Point out how this corresponds closely (2-equivalence?) but not precisely to adding strong $\Sigma$-types in the type theory.

Ask: is this induced by the adjunction with $\Id$-type categories?  or categories with appropriate nwfs's, or something?  or Hyland-Pitts style ``categories with \ldots?''

Discuss the Kleisli of this monad: ``modelling types as contexts''.


% \section{Basic properties} \label{sec:tt-lemmas}
% 
% --- lemma: $\Gamma \types \Delta \cxt$ iff $\diamond \types \Gamma , \Delta \cxt$
% --- lemma: if $\Gamma$ appears on the left of any judgement, then $\diamond \types \Gamma \cxt$.  Generally: can derive boundaries of derived judgements.
% --- judgements are stable under equality of the boundary.
% --- lemma: can derive wkg, subst.
% --- lemma: can concatenate context maps like contexts.
% --- lemma: having given only substitution along closed context maps, can get it along all.  
% --- lemma: $\equiv$ decidable.
% --- lemma: if no equality axioms, then judgemental equality decidable.
% --- lemma: if no equality axioms, then typing unique.
% \para 



% ---------------------------------------------------------------------------
%: ----------------------- end of thesis sub-document ------------------------
% ---------------------------------------------------------------------------
