\documentclass{amsart}

\usepackage{color}
\usepackage{ifpdf}
\usepackage{mathpartir}
\usepackage{amssymb}
\usepackage{amsthm}
\usepackage{amsmath}
\usepackage[all]{xypic}
\usepackage{wrapfig}

\xyoption{2cell}
\xyoption{rotate}
%\xyoption{curve}
\UseTwocells
 \def\dated#1{\def\thedate{#1}}%
 \dated{2004-12-08}%
 % This is a package of commutative diagram macros built on top of Xy-pic%
 % by Michael Barr (email:  barr@barrs.org).  Its use is unrestricted.  It%
 % may be freely distributed, unchanged, for non-commercial or commercial%
 % use.  If changed, it must be renamed.  Inclusion in a commercial%
 % software package is also permitted, but I would appreciate receiving a%
 % free copy for my personal examination and use.  There are no guarantees%
 % that this package is good for anything.  I have tested it with LaTeX 2e,%
 % LaTeX 2.09 and Plain TeX.  Although I know of no reason it will not work%
 % with AMSTeX, I have not tested it.%
 % Added 2003-05-10: I now know that the only clash is with \square in%
 % amssymb, which should therefore be loaded first.  If the amssymb%
 % \square is required, you can \let\box\square and use \box instead.%

\newcount\atcode \atcode=\catcode`\@%
\catcode`\@=12%
\input xy
\xyoption{arrow}
\xyoption{curve}

\newdir{ >}{{ }*!/-.9em/@{>}}%
\newdir{ (}{{ }*!/-.5em/@{(}}%
\newdir^{ (}{{ }*!/-.5em/@^{(}}%
\newdir{< }{!/.9em/@{<}*{ }}%

\newdimen\high%
\newdimen\ul%
\newcount\deltax%
\newcount\deltay%
\newcount\deltaX%
\newcount\deltaXprime%
\newcount\deltaY%
\newcount\deltaYprime%

\newdimen\wdth%
\newcount\xend%
\newcount\yend%
\newcount\Xend%
\newcount\Yend%
\newcount\xpos%
\newcount\ypos%
\newcount\default \default=500%
\newcount\defaultmargin \defaultmargin=150%
\newcount\topw%
\newcount\botw%
\newcount\Xpos%
\newcount\Ypos%
\def\ratchet#1#2{\ifnum#1<#2\global #1=#2\fi}%

\catcode`\@=11%
\expandafter\ifx\csname @ifnextchar\endcsname\relax%
\def\ifnextchar#1#2#3{\let\@tempe%
#1\def\@tempa{#2}\def\@tempb{#3}\futurelet%
    \@tempc\@ifnch}%
\def\@ifnch{\ifx \@tempc \@sptoken \let\@tempd\@xifnch%
      \else \ifx \@tempc \@tempe\let\@tempd\@tempa\else\let\@tempd\@tempb\fi%
      \fi \@tempd}%
\def\:{\let\@sptoken= } \:  % this makes \@sptoken a space token%
\def\:{\@xifnch} \expandafter\def\: {\futurelet\@tempc\@ifnch}%
\else%
\let\ifnextchar\@ifnextchar%
\fi%
\ifx\check@mathfonts\undefined%
\else \check@mathfonts%
\fi%
\newdimen\axis \axis=\fontdimen22\textfont2%
\ul=.01em%
\X@xbase =.01em%
\Y@ybase =.01em%
\def\scalefactor#1{\ul=#1\ul \X@xbase=#1\X@xbase \Y@ybase=#1\Y@ybase}%
\catcode`\@=12%

\def\fontscale#1{%
\if#1h\relax%
\font\xydashfont=xydash10 scaled \magstephalf%
\font\xyatipfont=xyatip10 scaled \magstephalf%
\font\xybtipfont=xybtip10 scaled \magstephalf%
\font\xybsqlfont=xybsql10 scaled \magstephalf%
\font\xycircfont=xycirc10 scaled \magstephalf%
\else%
\font\xydashfont=xydash10 scaled \magstep#1%
\font\xyatipfont=xyatip10 scaled \magstep#1%
\font\xybtipfont=xybtip10 scaled \magstep#1%
\font\xybsqlfont=xybsql10 scaled \magstep#1%
\font\xycircfont=xycirc10 scaled \magstep#1%
\fi}%

\def\bfig{\vcenter\bgroup\xy}%
\def\efig{\endxy\egroup}%

\def\car#1#2\nil{#1}%

\def\morphism{\ifnextchar({\morphismp}{\morphismp(0,0)}}%
\def\morphismp(#1){\ifnextchar|{\morphismpp(#1)}{\morphismpp(#1)|a|}}%
\def\morphismpp(#1)|#2|{\ifnextchar/{\morphismppp(#1)|#2|}%
    {\morphismppp(#1)|#2|/>/}}%
\def\morphismppp(#1)|#2|/#3/{%
    \ifnextchar<{\morphismpppp(#1)|#2|/#3/}%
    {\morphismpppp(#1)|#2|/#3/<\default,0>}}%

\def\morphismpppp(#1,#2)|#3|/#4/<#5,#6>[#7`#8;#9]{%
\xend#1\advance \xend by #5%
\yend#2\advance \yend by #6%
\domorphism(#1,#2)|#3|/#4/<#5,#6>[{#7}`{#8};{#9}]}%


\def\domorphism(#1,#2)|#3|/#4/<#5,#6>[#7`#8;#9]{%
 % Check if arrow arg has an @; then don't add it.%
\def\next{\car#4.\nil}%
\if@\next\relax%
 \if#3l%
  \ifnum #6>0%
   \POS(#1,#2)*+!!<0ex,\axis>{#7}\ar#4^-{#9} (\xend,\yend)*+!!<0ex,\axis>{#8}%
  \else%
   \POS(#1,#2)*+!!<0ex,\axis>{#7}\ar#4_-{#9} (\xend,\yend)*+!!<0ex,\axis>{#8}%
  \fi%
 \else \if#3m%
    \setbox0\hbox{$#9$}%
   \ifdim \wd0=0pt%
     \POS(#1,#2)*+!!<0ex,\axis>{#7}\ar#4 (\xend,\yend)*+!!<0ex,\axis>{#8}%
   \else%
     \POS(#1,#2)*+!!<0ex,\axis>{#7}\ar#4|-*+<1pt,4pt>{\labelstyle#9}%
       (\xend,\yend)*+!!<0ex,\axis>{#8}%
   \fi%
 \else \if#3r%
  \ifnum #6<0%
   \POS(#1,#2)*+!!<0ex,\axis>{#7}\ar#4^-{#9} (\xend,\yend)*+!!<0ex,\axis>{#8}%
  \else%
   \POS(#1,#2)*+!!<0ex,\axis>{#7}\ar#4_-{#9} (\xend,\yend)*+!!<0ex,\axis>{#8}%
  \fi%
 \else \if#3a%
  \ifnum #5>0%
   \POS(#1,#2)*+!!<0ex,\axis>{#7}\ar#4^-{#9} (\xend,\yend)*+!!<0ex,\axis>{#8}%
  \else%
   \POS(#1,#2)*+!!<0ex,\axis>{#7}\ar#4_-{#9} (\xend,\yend)*+!!<0ex,\axis>{#8}%
  \fi%
 \else \if#3b%
  \ifnum #5<0%
   \POS(#1,#2)*+!!<0ex,\axis>{#7}\ar#4^-{#9} (\xend,\yend)*+!!<0ex,\axis>{#8}%
  \else%
   \POS(#1,#2)*+!!<0ex,\axis>{#7}\ar#4_-{#9} (\xend,\yend)*+!!<0ex,\axis>{#8}%
  \fi%
 \else%
   \POS(#1,#2)*+!!<0ex,\axis>{#7}\ar#4 (\xend,\yend)*+!!<0ex,\axis>{#8}%
 \fi\fi\fi\fi\fi%
 %% Otherwise, have to add @{..}%
\else%
 \if#3l%
  \ifnum #6>0%
   \POS(#1,#2)*+!!<0ex,\axis>{#7}\ar@{#4}^-{#9} (\xend,\yend)*+!!<0ex,\axis>{#8}%
  \else%
   \POS(#1,#2)*+!!<0ex,\axis>{#7}\ar@{#4}_-{#9} (\xend,\yend)*+!!<0ex,\axis>{#8}%
  \fi%
 \else \if#3m%
    \setbox0\hbox{$#9$}%
   \ifdim \wd0=0pt%
     \POS(#1,#2)*+!!<0ex,\axis>{#7}\ar@{#4} (\xend,\yend)*+!!<0ex,\axis>{#8}%
   \else%
     \POS(#1,#2)*+!!<0ex,\axis>{#7}\ar@{#4}|-*+<1pt,4pt>{\labelstyle#9}%
         (\xend,\yend)*+!!<0ex,\axis>{#8}%
   \fi%
 \else \if#3r%
  \ifnum #6<0%
   \POS(#1,#2)*+!!<0ex,\axis>{#7}\ar@{#4}^-{#9} (\xend,\yend)*+!!<0ex,\axis>{#8}%
  \else%
   \POS(#1,#2)*+!!<0ex,\axis>{#7}\ar@{#4}_-{#9} (\xend,\yend)*+!!<0ex,\axis>{#8}%
  \fi%
 \else \if#3a%
  \ifnum #5>0%
   \POS(#1,#2)*+!!<0ex,\axis>{#7}\ar@{#4}^-{#9} (\xend,\yend)*+!!<0ex,\axis>{#8}%
  \else%
   \POS(#1,#2)*+!!<0ex,\axis>{#7}\ar@{#4}_-{#9} (\xend,\yend)*+!!<0ex,\axis>{#8}%
  \fi%
 \else \if#3b%
  \ifnum #5<0%
   \POS(#1,#2)*+!!<0ex,\axis>{#7}\ar@{#4}^-{#9} (\xend,\yend)*+!!<0ex,\axis>{#8}%
  \else%
   \POS(#1,#2)*+!!<0ex,\axis>{#7}\ar@{#4}_-{#9} (\xend,\yend)*+!!<0ex,\axis>{#8}%
  \fi%
 \else%
   \POS(#1,#2)*+!!<0ex,\axis>{#7}\ar@{#4} (\xend,\yend)*+!!<0ex,\axis>{#8}%
 \fi\fi\fi\fi\fi%
\fi\ignorespaces}%

\def\vect(#1,#2)/#3/<#4,#5>{%
 \xend#1 \yend#2 \advance\xend by #4 \advance\yend by #5%
     \POS(#1,#2)\ar#3 (\xend,\yend)}%

\def\squarepppp(#1,#2)|#3|/#4`#5`#6`#7/<#8>[#9]{%
\xpos#1\ypos#2%
\def\next|##1##2##3##4|{%
 \def\xa{##1}\def\xb{##2}\def\xc{##3}\def\xd{##4}\ignorespaces}%
\next|#3|%
\def\next<##1,##2>{\deltax=##1\deltay=##2\ignorespaces}%
\next<#8>%
\def\next[##1`##2`##3`##4;##5`##6`##7`##8]{%
    \def\nodea{##1}\def\nodeb{##2}\def\nodec{##3}\def\noded{##4}%
    \def\labela{##5}\def\labelb{##6}\def\labelc{##7}\def\labeld{##8}\ignorespaces}%
\next[#9]%
\morphism(\xpos,\ypos)|\xd|/{#7}/<\deltax,0>[\nodec`\noded;\labeld]%
\advance \ypos by \deltay%
\morphism(\xpos,\ypos)|\xb|/{#5}/<0,-\deltay>[\nodea`\nodec;\labelb]%
\morphism(\xpos,\ypos)|\xa|/{#4}/<\deltax,0>[\nodea`\nodeb;\labela]%
 \advance \xpos by \deltax%
\morphism(\xpos,\ypos)|\xc|/{#6}/<0,-\deltay>[\nodeb`\noded;\labelc]%
\ignorespaces}%

\def\square{\ifnextchar({\squarep}{\squarep(0,0)}}%
\def\squarep(#1){\ifnextchar|{\squarepp(#1)}{\squarepp(#1)|alrb|}}%
\def\squarepp(#1)|#2|{\ifnextchar/{\squareppp(#1)|#2|}%
    {\squareppp(#1)|#2|/>`>`>`>/}}%
\def\squareppp(#1)|#2|/#3`#4`#5`#6/{%
    \ifnextchar<{\squarepppp(#1)|#2|/#3`#4`#5`#6/}%
    {\squarepppp(#1)|#2|/#3`#4`#5`#6/<\default,\default>}}%

\def\ptrianglepppp(#1,#2)|#3|/#4`#5`#6/<#7>[#8]{%
\xpos#1\ypos#2%
\def\next|##1##2##3|{\def\xa{##1}\def\xb{##2}\def\xc{##3}}%
\next|#3|%
\def\next<##1,##2>{\deltax=##1\deltay=##2\ignorespaces}%
\next<#7>%
\def\next[##1`##2`##3;##4`##5`##6]{%
    \def\nodea{##1}\def\nodeb{##2}\def\nodec{##3}%
    \def\labela{##4}\def\labelb{##5}\def\labelc{##6}}%
\next[#8]%
\advance\ypos by \deltay%
\morphism(\xpos,\ypos)|\xa|/{#4}/<\deltax,0>[\nodea`\nodeb;\labela]%
\morphism(\xpos,\ypos)|\xb|/{#5}/<0,-\deltay>[\nodea`\nodec;\labelb]%
\advance\xpos by \deltax%
\morphism(\xpos,\ypos)|\xc|/{#6}/<-\deltax,-\deltay>[\nodeb`\nodec;\labelc]%
\ignorespaces}%

\def\qtrianglepppp(#1,#2)|#3|/#4`#5`#6/<#7>[#8]{%
\xpos#1\ypos#2%
\def\next|##1##2##3|{\def\xa{##1}\def\xb{##2}\def\xc{##3}}%
\next|#3|%
\def\next<##1,##2>{\deltax=##1\deltay=##2\ignorespaces}%
\next<#7>%
\def\next[##1`##2`##3;##4`##5`##6]{%
    \def\nodea{##1}\def\nodeb{##2}\def\nodec{##3}%
    \def\labela{##4}\def\labelb{##5}\def\labelc{##6}}%
\next[#8]%
\advance\ypos by \deltay%
\morphism(\xpos,\ypos)|\xa|/{#4}/<\deltax,0>[\nodea`\nodeb;\labela]%
\morphism(\xpos,\ypos)|\xb|/{#5}/<\deltax,-\deltay>[\nodea`\nodec;\labelb]%
\advance\xpos by \deltax%
\morphism(\xpos,\ypos)|\xc|/{#6}/<0,-\deltay>[\nodeb`\nodec;\labelc]%
\ignorespaces}%

\def\dtrianglepppp(#1,#2)|#3|/#4`#5`#6/<#7>[#8]{%
\xpos#1\ypos#2%
\def\next|##1##2##3|{\def\xa{##1}\def\xb{##2}\def\xc{##3}}%
\next|#3|%
\def\next<##1,##2>{\deltax=##1\deltay=##2\ignorespaces}%
\next<#7>%
\def\next[##1`##2`##3;##4`##5`##6]{%
    \def\nodea{##1}\def\nodeb{##2}\def\nodec{##3}%
    \def\labela{##4}\def\labelb{##5}\def\labelc{##6}}%
\next[#8]%
\morphism(\xpos,\ypos)|\xc|/{#6}/<\deltax,0>[\nodeb`\nodec;\labelc]%
\advance\ypos by \deltay\advance \xpos by \deltax%
\morphism(\xpos,\ypos)|\xa|/{#4}/<-\deltax,-\deltay>[\nodea`\nodeb;\labela]%
\morphism(\xpos,\ypos)|\xb|/{#5}/<0,-\deltay>[\nodea`\nodec;\labelb]%
\ignorespaces}%

\def\btrianglepppp(#1,#2)|#3|/#4`#5`#6/<#7>[#8]{%
\xpos#1\ypos#2%
\def\next|##1##2##3|{\def\xa{##1}\def\xb{##2}\def\xc{##3}}%
\next|#3|%
\def\next<##1,##2>{\deltax=##1\deltay=##2\ignorespaces}%
\next<#7>%
\def\next[##1`##2`##3;##4`##5`##6]{%
    \def\nodea{##1}\def\nodeb{##2}\def\nodec{##3}%
    \def\labela{##4}\def\labelb{##5}\def\labelc{##6}}%
\next[#8]%
\morphism(\xpos,\ypos)|\xc|/{#6}/<\deltax,0>[\nodeb`\nodec;\labelc]%
\advance\ypos by \deltay%
\morphism(\xpos,\ypos)|\xa|/{#4}/<0,-\deltay>[\nodea`\nodeb;\labela]%
\morphism(\xpos,\ypos)|\xb|/{#5}/<\deltax,-\deltay>[\nodea`\nodec;\labelb]%
\ignorespaces}%

\def\Atrianglepppp(#1,#2)|#3|/#4`#5`#6/<#7>[#8]{%
\xpos#1\ypos#2%
\def\next|##1##2##3|{\def\xa{##1}\def\xb{##2}\def\xc{##3}}%
\next|#3|%
\def\next<##1,##2>{\deltax=##1\deltay=##2\ignorespaces}%
\next<#7>%
\def\next[##1`##2`##3;##4`##5`##6]{%
    \def\nodea{##1}\def\nodeb{##2}\def\nodec{##3}%
    \def\labela{##4}\def\labelb{##5}\def\labelc{##6}}%
\next[#8]%
\multiply\deltax by 2%
\morphism(\xpos,\ypos)|\xc|/{#6}/<\deltax,0>[\nodeb`\nodec;\labelc]%
\divide\deltax by 2%
\advance\ypos by \deltay\advance\xpos by \deltax%
\morphism(\xpos,\ypos)|\xa|/{#4}/<-\deltax,-\deltay>[\nodea`\nodeb;\labela]%
\morphism(\xpos,\ypos)|\xb|/{#5}/<\deltax,-\deltay>[\nodea`\nodec;\labelb]%
\ignorespaces}%

\def\Vtrianglepppp(#1,#2)|#3|/#4`#5`#6/<#7>[#8]{%
\xpos#1\ypos#2%
\def\next|##1##2##3|{\def\xa{##1}\def\xb{##2}\def\xc{##3}}%
\next|#3|%
\def\next<##1,##2>{\deltax=##1\deltay=##2\ignorespaces}%
\next<#7>%
\def\next[##1`##2`##3;##4`##5`##6]{%
    \def\nodea{##1}\def\nodeb{##2}\def\nodec{##3}%
    \def\labela{##4}\def\labelb{##5}\def\labelc{##6}}%
\next[#8]%
\advance\ypos by \deltay%
\morphism(\xpos,\ypos)|\xb|/{#5}/<\deltax,-\deltay>[\nodea`\nodec;\labelb]%
\multiply\deltax by 2%
\morphism(\xpos,\ypos)|\xa|/{#4}/<\deltax,0>[\nodea`\nodeb;\labela]%
\advance\xpos by \deltax \divide \deltax by 2%
\morphism(\xpos,\ypos)|\xc|/{#6}/<-\deltax,-\deltay>[\nodeb`\nodec;\labelc]%
\ignorespaces}%

\def\Ctrianglepppp(#1,#2)|#3|/#4`#5`#6/<#7>[#8]{%
\xpos#1\ypos#2%
\def\next|##1##2##3|{\def\xa{##1}\def\xb{##2}\def\xc{##3}}%
\next|#3|%
\def\next<##1,##2>{\deltax=##1\deltay=##2\ignorespaces}%
\next<#7>%
\def\next[##1`##2`##3;##4`##5`##6]{%
    \def\nodea{##1}\def\nodeb{##2}\def\nodec{##3}%
    \def\labela{##4}\def\labelb{##5}\def\labelc{##6}}%
\next[#8]%
\advance \ypos by \deltay%
\morphism(\xpos,\ypos)|\xc|/{#6}/<\deltax,-\deltay>[\nodeb`\nodec;\labelc]%
\advance\ypos by \deltay \advance \xpos by \deltax%
\morphism(\xpos,\ypos)|\xa|/{#4}/<-\deltax,-\deltay>[\nodea`\nodeb;\labela]%
\multiply\deltay by 2%
\morphism(\xpos,\ypos)|\xb|/{#5}/<0,-\deltay>[\nodea`\nodec;\labelb]%
\ignorespaces}%

\def\Dtrianglepppp(#1,#2)|#3|/#4`#5`#6/<#7>[#8]{%
\xpos#1\ypos#2%
\def\next|##1##2##3|{\def\xa{##1}\def\xb{##2}\def\xc{##3}}%
\next|#3|%
\def\next<##1,##2>{\deltax=##1\deltay=##2\ignorespaces}%
\next<#7>%
\def\next[##1`##2`##3;##4`##5`##6]{%
    \def\nodea{##1}\def\nodeb{##2}\def\nodec{##3}%
    \def\labela{##4}\def\labelb{##5}\def\labelc{##6}}%
\next[#8]%
\advance\xpos by \deltax \advance\ypos by \deltay%
\morphism(\xpos,\ypos)|\xc|/{#6}/<-\deltax,-\deltay>[\nodeb`\nodec;\labelc]%
\advance\xpos by -\deltax \advance\ypos by \deltay%
\morphism(\xpos,\ypos)|\xb|/{#5}/<\deltax,-\deltay>[\nodea`\nodeb;\labelb]%
\multiply \deltay by 2%
\morphism(\xpos,\ypos)|\xa|/{#4}/<0,-\deltay>[\nodea`\nodec;\labela]%
\ignorespaces}%

\def\ptriangle{\ifnextchar({\ptrianglep}{\ptrianglep(0,0)}}%
\def\ptrianglep(#1){\ifnextchar|{\ptrianglepp(#1)}{\ptrianglepp(#1)|alr|}}%
\def\ptrianglepp(#1)|#2|{\ifnextchar/{\ptriangleppp(#1)|#2|}%
    {\ptriangleppp(#1)|#2|/>`>`>/}}%
\def\ptriangleppp(#1)|#2|/#3`#4`#5/{%
    \ifnextchar<{\ptrianglepppp(#1)|#2|/#3`#4`#5/}%
    {\ptrianglepppp(#1)|#2|/#3`#4`#5/<\default,\default>}}%

\def\qtriangle{\ifnextchar({\qtrianglep}{\qtrianglep(0,0)}}%
\def\qtrianglep(#1){\ifnextchar|{\qtrianglepp(#1)}{\qtrianglepp(#1)|alr|}}%
\def\qtrianglepp(#1)|#2|{\ifnextchar/{\qtriangleppp(#1)|#2|}%
    {\qtriangleppp(#1)|#2|/>`>`>/}}%
\def\qtriangleppp(#1)|#2|/#3`#4`#5/{%
    \ifnextchar<{\qtrianglepppp(#1)|#2|/#3`#4`#5/}%
    {\qtrianglepppp(#1)|#2|/#3`#4`#5/<\default,\default>}}%

\def\dtriangle{\ifnextchar({\dtrianglep}{\dtrianglep(0,0)}}%
\def\dtrianglep(#1){\ifnextchar|{\dtrianglepp(#1)}{\dtrianglepp(#1)|lrb|}}%
\def\dtrianglepp(#1)|#2|{\ifnextchar/{\dtriangleppp(#1)|#2|}%
    {\dtriangleppp(#1)|#2|/>`>`>/}}%
\def\dtriangleppp(#1)|#2|/#3`#4`#5/{%
    \ifnextchar<{\dtrianglepppp(#1)|#2|/#3`#4`#5/}%
    {\dtrianglepppp(#1)|#2|/#3`#4`#5/<\default,\default>}}%

\def\btriangle{\ifnextchar({\btrianglep}{\btrianglep(0,0)}}%
\def\btrianglep(#1){\ifnextchar|{\btrianglepp(#1)}{\btrianglepp(#1)|lrb|}}%
\def\btrianglepp(#1)|#2|{\ifnextchar/{\btriangleppp(#1)|#2|}%
    {\btriangleppp(#1)|#2|/>`>`>/}}%
\def\btriangleppp(#1)|#2|/#3`#4`#5/{%
    \ifnextchar<{\btrianglepppp(#1)|#2|/#3`#4`#5/}%
    {\btrianglepppp(#1)|#2|/#3`#4`#5/<\default,\default>}}%

\def\Atriangle{\ifnextchar({\Atrianglep}{\Atrianglep(0,0)}}%
\def\Atrianglep(#1){\ifnextchar|{\Atrianglepp(#1)}{\Atrianglepp(#1)|lrb|}}%
\def\Atrianglepp(#1)|#2|{\ifnextchar/{\Atriangleppp(#1)|#2|}%
    {\Atriangleppp(#1)|#2|/>`>`>/}}%
\def\Atriangleppp(#1)|#2|/#3`#4`#5/{%
    \ifnextchar<{\Atrianglepppp(#1)|#2|/#3`#4`#5/}%
    {\Atrianglepppp(#1)|#2|/#3`#4`#5/<\default,\default>}}%

\def\Vtriangle{\ifnextchar({\Vtrianglep}{\Vtrianglep(0,0)}}%
\def\Vtrianglep(#1){\ifnextchar|{\Vtrianglepp(#1)}{\Vtrianglepp(#1)|alb|}}%
\def\Vtrianglepp(#1)|#2|{\ifnextchar/{\Vtriangleppp(#1)|#2|}%
    {\Vtriangleppp(#1)|#2|/>`>`>/}}%
\def\Vtriangleppp(#1)|#2|/#3`#4`#5/{%
    \ifnextchar<{\Vtrianglepppp(#1)|#2|/#3`#4`#5/}%
    {\Vtrianglepppp(#1)|#2|/#3`#4`#5/<\default,\default>}}%

\def\Ctriangle{\ifnextchar({\Ctrianglep}{\Ctrianglep(0,0)}}%
\def\Ctrianglep(#1){\ifnextchar|{\Ctrianglepp(#1)}{\Ctrianglepp(#1)|arb|}}%
\def\Ctrianglepp(#1)|#2|{\ifnextchar/{\Ctriangleppp(#1)|#2|}%
    {\Ctriangleppp(#1)|#2|/>`>`>/}}%
\def\Ctriangleppp(#1)|#2|/#3`#4`#5/{%
    \ifnextchar<{\Ctrianglepppp(#1)|#2|/#3`#4`#5/}%
    {\Ctrianglepppp(#1)|#2|/#3`#4`#5/<\default,\default>}}%

\def\Dtriangle{\ifnextchar({\Dtrianglep}{\Dtrianglep(0,0)}}%
\def\Dtrianglep(#1){\ifnextchar|{\Dtrianglepp(#1)}{\Dtrianglepp(#1)|alb|}}%
\def\Dtrianglepp(#1)|#2|{\ifnextchar/{\Dtriangleppp(#1)|#2|}%
    {\Dtriangleppp(#1)|#2|/>`>`>/}}%
\def\Dtriangleppp(#1)|#2|/#3`#4`#5/{%
    \ifnextchar<{\Dtrianglepppp(#1)|#2|/#3`#4`#5/}%
    {\Dtrianglepppp(#1)|#2|/#3`#4`#5/<\default,\default>}}%


\def\Atrianglepairpppp(#1)|#2|/#3`#4`#5`#6`#7/<#8>[#9]{%
\def\next(##1,##2){\xpos##1\ypos##2}%
\next(#1)%
\def\next|##1##2##3##4##5|{\def\xa{##1}\def\xb{##2}%
\def\xc{##3}\def\xd{##4}\def\xe{##5}}%
\next|#2|%
\def\next<##1,##2>{\deltax=##1\deltay=##2\ignorespaces}%
\next<#8>%
\def\next[##1`##2`##3`##4;##5`##6`##7`##8`##9]{%
 \def\nodea{##1}\def\nodeb{##2}\def\nodec{##3}\def\noded{##4}%
 \def\labela{##5}\def\labelb{##6}\def\labelc{##7}\def\labeld{##8}\def\labele{##9}}%
\next[#9]%
\morphism(\xpos,\ypos)|\xd|/{#6}/<\deltax,0>[\nodeb`\nodec;\labeld]%
\advance\xpos by \deltax%
\morphism(\xpos,\ypos)|\xe|/{#7}/<\deltax,0>[\nodec`\noded;\labele]%
\advance\ypos by \deltay%
\morphism(\xpos,\ypos)|\xa|/{#3}/<-\deltax,-\deltay>[\nodea`\nodeb;\labela]%
\morphism(\xpos,\ypos)|\xb|/{#4}/<0,-\deltay>[\nodea`\nodec;\labelb]%
\morphism(\xpos,\ypos)|\xc|/{#5}/<\deltax,-\deltay>[\nodea`\noded;\labelc]%
\ignorespaces}%

\def\Vtrianglepairpppp(#1)|#2|/#3`#4`#5`#6`#7/<#8>[#9]{%
\def\next(##1,##2){\xpos##1\ypos##2}%
\next(#1)%
\def\next|##1##2##3##4##5|{\def\xa{##1}\def\xb{##2}%
\def\xc{##3}\def\xd{##4}\def\xe{##5}}%
\next|#2|%
\def\next<##1,##2>{\deltax=##1\deltay=##2\ignorespaces}%
\next<#8>%
\def\next[##1`##2`##3`##4;##5`##6`##7`##8`##9]{%
 \def\nodea{##1}\def\nodeb{##2}\def\nodec{##3}\def\noded{##4}%
 \def\labela{##5}\def\labelb{##6}\def\labelc{##7}\def\labeld{##8}\def\labele{##9}}%
\next[#9]%
\advance\ypos by \deltay%
\morphism(\xpos,\ypos)|\xa|/{#3}/<\deltax,0>[\nodea`\nodeb;\labela]%
\morphism(\xpos,\ypos)|\xc|/{#5}/<\deltax,-\deltay>[\nodea`\noded;\labelc]%
\advance\xpos by \deltax%
\morphism(\xpos,\ypos)|\xb|/{#4}/<\deltax,0>[\nodeb`\nodec;\labelb]%
\morphism(\xpos,\ypos)|\xd|/{#6}/<0,-\deltay>[\nodeb`\noded;\labeld]%
\advance\xpos by \deltax%
\morphism(\xpos,\ypos)|\xe|/{#7}/<-\deltax,-\deltay>[\nodec`\noded;\labele]%
\ignorespaces}%

\def\Ctrianglepairpppp(#1)|#2|/#3`#4`#5`#6`#7/<#8>[#9]{%
\def\next(##1,##2){\xpos##1\ypos##2}%
\next(#1)%
\def\next|##1##2##3##4##5|{\def\xa{##1}\def\xb{##2}%
\def\xc{##3}\def\xd{##4}\def\xe{##5}}%
\next|#2|%
\def\next<##1,##2>{\deltax=##1\deltay=##2\ignorespaces}%
\next<#8>%
\def\next[##1`##2`##3`##4;##5`##6`##7`##8`##9]{%
 \def\nodea{##1}\def\nodeb{##2}\def\nodec{##3}\def\noded{##4}%
 \def\labela{##5}\def\labelb{##6}\def\labelc{##7}\def\labeld{##8}\def\labele{##9}}%
\next[#9]%
\advance\ypos by \deltay%
\morphism(\xpos,\ypos)|\xe|/{#7}/<0,-\deltay>[\nodec`\noded;\labele]%
\advance\xpos by -\deltax%
\morphism(\xpos,\ypos)|\xc|/{#5}/<\deltax,0>[\nodeb`\nodec;\labelc]%
\morphism(\xpos,\ypos)|\xd|/{#6}/<\deltax,-\deltay>[\nodeb`\noded;\labeld]%
\advance\ypos by \deltay%
\advance\xpos by \deltax%
\morphism(\xpos,\ypos)|\xa|/{#3}/<-\deltax,-\deltay>[\nodea`\nodeb;\labela]%
\morphism(\xpos,\ypos)|\xb|/{#4}/<0,-\deltay>[\nodea`\nodec;\labelb]%
\ignorespaces}%

\def\Dtrianglepairpppp(#1)|#2|/#3`#4`#5`#6`#7/<#8>[#9]{%
\def\next(##1,##2){\xpos##1\ypos##2}%
\next(#1)%
\def\next|##1##2##3##4##5|{\def\xa{##1}\def\xb{##2}%
\def\xc{##3}\def\xd{##4}\def\xe{##5}}%
\next|#2|%
\def\next<##1,##2>{\deltax=##1\deltay=##2\ignorespaces}%
\next<#8>%
\def\next[##1`##2`##3`##4;##5`##6`##7`##8`##9]{%
 \def\nodea{##1}\def\nodeb{##2}\def\nodec{##3}\def\noded{##4}%
 \def\labela{##5}\def\labelb{##6}\def\labelc{##7}\def\labeld{##8}\def\labele{##9}}%
\next[#9]%
\advance\ypos by \deltay%
\morphism(\xpos,\ypos)|\xc|/{#5}/<\deltax,0>[\nodeb`\nodec;\labelc]%
\morphism(\xpos,\ypos)|\xd|/{#6}/<0,-\deltay>[\nodeb`\noded;\labeld]%
\advance\ypos by \deltay%
\morphism(\xpos,\ypos)|\xa|/{#3}/<0,-\deltay>[\nodea`\nodeb;\labela]%
\morphism(\xpos,\ypos)|\xb|/{#4}/<\deltax,-\deltay>[\nodea`\nodec;\labelb]%
\advance\ypos by -\deltay%
\advance\xpos by \deltax%
\morphism(\xpos,\ypos)|\xe|/{#7}/<-\deltax,-\deltay>[\nodec`\noded;\labele]%
\ignorespaces}%
\def\Atrianglepair{\ifnextchar({\Atrianglepairp}{\Atrianglepairp(0,0)}}%
\def\Atrianglepairp(#1){\ifnextchar|{\Atrianglepairpp(#1)}%
{\Atrianglepairpp(#1)|lmrbb|}}%
\def\Atrianglepairpp(#1)|#2|{\ifnextchar/{\Atrianglepairppp(#1)|#2|}%
    {\Atrianglepairppp(#1)|#2|/>`>`>`>`>/}}%
\def\Atrianglepairppp(#1)|#2|/#3`#4`#5`#6`#7/{%
    \ifnextchar<{\Atrianglepairpppp(#1)|#2|/#3`#4`#5`#6`#7/}%
    {\Atrianglepairpppp(#1)|#2|/#3`#4`#5`#6`#7/<\default,\default>}}%

\def\Vtrianglepair{\ifnextchar({\Vtrianglepairp}{\Vtrianglepairp(0,0)}}%
\def\Vtrianglepairp(#1){\ifnextchar|{\Vtrianglepairpp(#1)}%
{\Vtrianglepairpp(#1)|aalmr|}}%
\def\Vtrianglepairpp(#1)|#2|{\ifnextchar/{\Vtrianglepairppp(#1)|#2|}%
    {\Vtrianglepairppp(#1)|#2|/>`>`>`>`>/}}%
\def\Vtrianglepairppp(#1)|#2|/#3`#4`#5`#6`#7/{%
    \ifnextchar<{\Vtrianglepairpppp(#1)|#2|/#3`#4`#5`#6`#7/}%
    {\Vtrianglepairpppp(#1)|#2|/#3`#4`#5`#6`#7/<\default,\default>}}%

\def\Ctrianglepair{\ifnextchar({\Ctrianglepairp}{\Ctrianglepairp(0,0)}}%
\def\Ctrianglepairp(#1){\ifnextchar|{\Ctrianglepairpp(#1)}%
{\Ctrianglepairpp(#1)|lrmlr|}}%
\def\Ctrianglepairpp(#1)|#2|{\ifnextchar/{\Ctrianglepairppp(#1)|#2|}%
    {\Ctrianglepairppp(#1)|#2|/>`>`>`>`>/}}%
\def\Ctrianglepairppp(#1)|#2|/#3`#4`#5`#6`#7/{%
    \ifnextchar<{\Ctrianglepairpppp(#1)|#2|/#3`#4`#5`#6`#7/}%
    {\Ctrianglepairpppp(#1)|#2|/#3`#4`#5`#6`#7/<\default,\default>}}%

\def\Dtrianglepair{\ifnextchar({\Dtrianglepairp}{\Dtrianglepairp(0,0)}}%
\def\Dtrianglepairp(#1){\ifnextchar|{\Dtrianglepairpp(#1)}%
{\Dtrianglepairpp(#1)|lrmlr|}}%
\def\Dtrianglepairpp(#1)|#2|{\ifnextchar/{\Dtrianglepairppp(#1)|#2|}%
    {\Dtrianglepairppp(#1)|#2|/>`>`>`>`>/}}%
\def\Dtrianglepairppp(#1)|#2|/#3`#4`#5`#6`#7/{%
    \ifnextchar<{\Dtrianglepairpppp(#1)|#2|/#3`#4`#5`#6`#7/}%
    {\Dtrianglepairpppp(#1)|#2|/#3`#4`#5`#6`#7/<\default,\default>}}%

\def\pplace[#1](#2,#3)[#4]{\POS(#2,#3)*+!!<0ex,\axis>!#1{#4}\ignorespaces}%
\def\cplace(#1,#2)[#3]{\POS(#1,#2)*+!!<0ex,\axis>{#3}\ignorespaces}%
\def\place{\ifnextchar[{\pplace}{\cplace}}%

\def\pullback#1]#2]{\square#1]\trident#2]\ignorespaces}%

\def\tridentppp|#1#2#3|/#4`#5`#6/<#7,#8>[#9]{%
\def\next[##1;##2`##3`##4]{\def\nodee{##1}\def\labele{##2}%
   \def\labelf{##3}\def\labelg{##4}}%
\next[#9]%
\advance \xpos by -\deltax%
\advance \xpos by -#7\advance \ypos by #8%
\advance\deltax by #7%
\morphism(\xpos,\ypos)|#1|/{#4}/<\deltax,-#8>[\nodee`\nodeb;\labele]%
\advance\deltax by -#7%
\morphism(\xpos,\ypos)|#2|/{#5}/<#7,-#8>[\nodee`\nodea;\labelf]%
\advance\deltay by #8%
\morphism(\xpos,\ypos)|#3|/{#6}/<#7,-\deltay>[\nodee`\nodec;\labelg]%
\ignorespaces}%

\def\trident{\ifnextchar|{\tridentp}{\tridentp|amb|}}%
\def\tridentp|#1|{\ifnextchar/{\tridentpp|#1|}{\tridentpp|#1|/{>}`{>}`{>}/}}%
\def\tridentpp|#1|/#2/{\ifnextchar<{\tridentppp|#1|/#2/}%
  {\tridentppp|#1|/#2/<500,500>}}%

\def\setmorphismwidth#1#2#3#4{%
 \setbox0=\hbox{$#1{\labelstyle#3#3}#2$}#4=\wd0%
 \divide #4 by 2 \divide #4 by \ul%
 \advance #4 by 350 \ratchet{#4}{500}}%

\def\setSquarewidth[#1`#2`#3`#4;#5`#6`#7`#8]{%
 \setmorphismwidth{#1}{#2}{#5}{\topw}%
 \setmorphismwidth{#3}{#4}{#8}{\botw}%
\ratchet{\topw}{\botw}}%

\def\Squarepppp(#1)|#2|/#3/<#4>[#5]{%
 \setSquarewidth[#5]%
 \squarepppp(#1)|#2|/#3/<\topw,#4>[#5]%
\ignorespaces}%

\def\Square{\ifnextchar({\Squarep}{\Squarep(0,0)}}%
\def\Squarep(#1){\ifnextchar|{\Squarepp(#1)}{\Squarepp(#1)|alrb|}}%
\def\Squarepp(#1)|#2|{\ifnextchar/{\Squareppp(#1)|#2|}%
    {\Squareppp(#1)|#2|/>`>`>`>/}}%
\def\Squareppp(#1)|#2|/#3`#4`#5`#6/{%
    \ifnextchar<{\Squarepppp(#1)|#2|/#3`#4`#5`#6/}%
    {\Squarepppp(#1)|#2|/#3`#4`#5`#6/<\default>}}%

\def\hsquarespppp(#1,#2)|#3|/#4/<#5>[#6;#7]{%
\Xpos=#1\Ypos=#2%
\def\next|##1##2##3##4##5##6##7|{%
 \def\Xa{##1}\def\Xb{##2}\def\Xc{##3}\def\Xd{##4}%
 \def\Xe{##5}\def\Xf{##6}\def\Xg{##7}}%
\next|#3|%
\def\next<##1,##2,##3>{\deltaX=##1\deltaXprime=##2\deltaY=##3}%
\next<#5>%
\def\next[##1`##2`##3`##4`##5`##6]{%
 \def\Nodea{##1}\def\Nodeb{##2}\def\Nodec{##3}%
 \def\Noded{##4}\def\Nodee{##5}\def\Nodef{##6}}%
\next[#6]%
\def\next[##1`##2`##3`##4`##5`##6`##7]{%
 \def\Labela{##1}\def\Labelb{##2}\def\Labelc{##3}\def\Labeld{##4}%
 \def\Labele{##5}\def\Labelf{##6}\def\Labelg{##7}}%
\next[#7]%
\dohsquares/#4/}%

\def\dohsquares/#1`#2`#3`#4`#5`#6`#7/{%
\squarepppp(\Xpos,\Ypos)|\Xa\Xc\Xd\Xf|/#1`#3`#4`#6/<\deltaX,\deltaY>%
 [\Nodea`\Nodeb`\Noded`\Nodee;\Labela`\Labelc`\Labeld`\Labelf]%
 \advance \Xpos by \deltaX%
\squarepppp(\Xpos,\Ypos)|\Xb\Xd\Xe\Xg|/#2``#5`#7/<\deltaXprime,\deltaY>%
[\Nodeb`\Nodec`\Nodee`\Nodef;\Labelb``\Labele`\Labelg]%
\ignorespaces}%

\def\hsquares{\ifnextchar({\hsquaresp}{\hsquaresp(0,0)}}%
\def\hsquaresp(#1){\ifnextchar|{\hsquarespp(#1)}{\hsquarespp%
(#1)|aalmrbb|}}%
\def\hsquarespp(#1)|#2|{\ifnextchar/{\hsquaresppp(#1)|#2|}%
    {\hsquaresppp(#1)|#2|/>`>`>`>`>`>`>/}}%
\def\hsquaresppp(#1)|#2|/#3/{%
    \ifnextchar<{\hsquarespppp(#1)|#2|/#3/}%
    {\hsquarespppp(#1)|#2|/#3/<\default,\default,\default>}}%

\def\hSquarespppp(#1,#2)|#3|/#4/<#5>[#6;#7]{%
\Xpos=#1\Ypos=#2%
\def\next|##1##2##3##4##5##6##7|{%
 \def\Xa{##1}\def\Xb{##2}\def\Xc{##3}\def\Xd{##4}%
 \def\Xe{##5}\def\Xf{##6}\def\Xg{##7}}%
\next|#3|%
\deltaY=#5%
\def\next[##1`##2`##3`##4`##5`##6]{%
 \def\Nodea{##1}\def\Nodeb{##2}\def\Nodec{##3}%
 \def\Noded{##4}\def\Nodee{##5}\def\Nodef{##6}}%
\next[#6]%
\def\next[##1`##2`##3`##4`##5`##6`##7]{%
 \def\Labela{##1}\def\Labelb{##2}\def\Labelc{##3}\def\Labeld{##4}%
 \def\Labele{##5}\def\Labelf{##6}\def\Labelg{##7}}%
\next[#7]%
\dohSquares/#4/}%

\def\dohSquares/#1`#2`#3`#4`#5`#6`#7/{%
\Squarepppp(\Xpos,\Ypos)|\Xa\Xc\Xd\Xf|/#1`#3`#4`#6/<\deltaY>%
 [\Nodea`\Nodeb`\Noded`\Nodee;\Labela`\Labelc`\Labeld`\Labelf]%
 \advance \Xpos by \topw%
\Squarepppp(\Xpos,\Ypos)|\Xb\Xd\Xe\Xg|/#2``#5`#7/<\deltaY>%
[\Nodeb`\Nodec`\Nodee`\Nodef;\Labelb``\Labele`\Labelg]%
\ignorespaces}%

\def\hSquares{\ifnextchar({\hSquaresp}{\hSquaresp(0,0)}}%
\def\hSquaresp(#1){\ifnextchar|{\hSquarespp(#1)}{\hSquarespp%
(#1)|aalmrbb|}}%
\def\hSquarespp(#1)|#2|{\ifnextchar/{\hSquaresppp(#1)|#2|}%
    {\hSquaresppp(#1)|#2|/>`>`>`>`>`>`>/}}%
\def\hSquaresppp(#1)|#2|/#3/{%
    \ifnextchar<{\hSquarespppp(#1)|#2|/#3/}%
    {\hSquarespppp(#1)|#2|/#3/<\default>}}%

\def\vsquarespppp(#1,#2)|#3|/#4/<#5>[#6;#7]{%
\Xpos=#1\Ypos=#2%
\def\next|##1##2##3##4##5##6##7|{%
 \def\Xa{##1}\def\Xb{##2}\def\Xc{##3}\def\Xd{##4}%
 \def\Xe{##5}\def\Xf{##6}\def\Xg{##7}}%
\next|#3|%
\def\next<##1,##2,##3>{\deltaX=##1\deltaY=##2\deltaYprime=##3}%
\next<#5>%
\def\next[##1`##2`##3`##4`##5`##6]{%
 \def\Nodea{##1}\def\Nodeb{##2}\def\Nodec{##3}%
 \def\Noded{##4}\def\Nodee{##5}\def\Nodef{##6}}%
\next[#6]%
\def\next[##1`##2`##3`##4`##5`##6`##7]{%
 \def\Labela{##1}\def\Labelb{##2}\def\Labelc{##3}\def\Labeld{##4}%
 \def\Labele{##5}\def\Labelf{##6}\def\Labelg{##7}}%
\next[#7]%
\dovsquares/#4/}%

\def\dovsquares/#1`#2`#3`#4`#5`#6`#7/{%
\squarepppp(\Xpos,\Ypos)|\Xd\Xe\Xf\Xg|/`#5`#6`#7/<\deltaX,\deltaYprime>%
[\Nodec`\Noded`\Nodee`\Nodef;`\Labele`\Labelf`\Labelg]%
 \advance\Ypos by \deltaYprime%
\squarepppp(\Xpos,\Ypos)|\Xa\Xb\Xc\Xd|/#1`#2`#3`#4/<\deltaX,\deltaY>%
 [\Nodea`\Nodeb`\Nodec`\Noded;\Labela`\Labelb`\Labelc`\Labeld]%
\ignorespaces}%

\def\vsquares{\ifnextchar({\vsquaresp}{\vsquaresp(0,0)}}%
\def\vsquaresp(#1){\ifnextchar|{\vsquarespp(#1)}{\vsquarespp%
(#1)|aalmrbb|}}%
\def\vsquarespp(#1)|#2|{\ifnextchar/{\vsquaresppp(#1)|#2|}%
    {\vsquaresppp(#1)|#2|/>`>`>`>`>`>`>/}}%
\def\vsquaresppp(#1)|#2|/#3/{%
    \ifnextchar<{\vsquarespppp(#1)|#2|/#3/}%
    {\vsquarespppp(#1)|#2|/#3/<\default,\default,\default>}}%


\def\vSquarespppp(#1,#2)|#3|/#4/<#5,#6>[#7;#8]{%
\Xpos=#1\Ypos=#2%
\def\next|##1##2##3##4##5##6##7|{%
 \def\Xa{##1}\def\Xb{##2}\def\Xc{##3}\def\Xd{##4}%
 \def\Xe{##5}\def\Xf{##6}\def\Xg{##7}}%
\next|#3|%
\deltaX=#5%
\deltaY=#6%
\def\next[##1`##2`##3`##4`##5`##6]{%
 \def\Nodea{##1}\def\Nodeb{##2}\def\Nodec{##3}%
 \def\Noded{##4}\def\Nodee{##5}\def\Nodef{##6}}%
\next[#7]%
\def\next[##1`##2`##3`##4`##5`##6`##7]{%
 \def\Labela{##1}\def\Labelb{##2}\def\Labelc{##3}\def\Labeld{##4}%
 \def\Labele{##5}\def\Labelf{##6}\def\Labelg{##7}}%
\next[#8]%
\dovSquares/#4/\ignorespaces}%

\def\dovSquares/#1`#2`#3`#4`#5`#6`#7/{%
\setmorphismwidth{\Nodea}{\Nodeb}{\Labela}{\topw}%
\setmorphismwidth{\Nodec}{\Noded}{\Labeld}{\botw}%
\ratchet{\topw}{\botw}%
\setmorphismwidth{\Nodee}{\Nodef}{\Labelg}{\botw}%
\ratchet{\topw}{\botw}%
\square(\Xpos,\Ypos)|\Xd\Xe\Xf\Xg|/`#5`#6`#7/<\topw,\deltaX>%
 [\Nodec`\Noded`\Nodee`\Nodef;`\Labele`\Labelf`\Labelg]%
\advance \Ypos by \deltaX%
\square(\Xpos,\Ypos)|\Xa\Xb\Xc\Xd|/#1`#2`#3`#4/<\topw,\deltaY>%
 [\Nodea`\Nodeb`\Nodec`\Noded;\Labela`\Labelb`\Labelc`\Labeld]%
}%

\def\vSquares{\ifnextchar({\vSquaresp}{\vSquaresp(0,0)}}%
\def\vSquaresp(#1){\ifnextchar|{\vSquarespp(#1)}{\vSquarespp%
(#1)|alrmlrb|}}%
\def\vSquarespp(#1)|#2|{\ifnextchar/{\vSquaresppp(#1)|#2|}%
    {\vSquaresppp(#1)|#2|/>`>`>`>`>`>`>/}}%
\def\vSquaresppp(#1)|#2|/#3/{%
    \ifnextchar<{\vSquarespppp(#1)|#2|/#3/}%
    {\vSquarespppp(#1)|#2|/#3/<\default,\default>}}%

\def\osquarepppp(#1)|#2|/#3`#4`#5`#6/<#7>[#8]{\squarepppp%
 (#1)|#2|/#3`#4`#5`#6/<#7>[#8]%
 \let\Nodea\nodea\let\Nodeb\nodeb%
\let\Nodec\nodec\let\Noded\noded\Xpos=\xpos\Ypos=\ypos%
\deltaX=\deltax \deltaY=\deltay \isquare}%

\def\cube{\ifnextchar({\osquarep}{\osquarep(0,0)}}%
\def\osquarep(#1){\ifnextchar|{\osquarepp(#1)}{\osquarepp(#1)|alrb|}}%
\def\osquarepp(#1)|#2|{\ifnextchar/{\osquareppp(#1)|#2|}%
    {\osquareppp(#1)|#2|/>`>`>`>/}}%
\def\osquareppp(#1)|#2|/#3`#4`#5`#6/{%
    \ifnextchar<{\osquarepppp(#1)|#2|/#3`#4`#5`#6/}%
    {\osquarepppp(#1)|#2|/#3`#4`#5`#6/<1500,1500>}}%

\def\isquarepppp(#1)|#2|/#3`#4`#5`#6/<#7>[#8]{%
 \squarepppp(#1)|#2|/#3`#4`#5`#6/<#7>[#8]%
\ifnextchar|{\cubep}{\cubep|mmmm|}}%
\def\cubep|#1|{\ifnextchar/{\cubepp|#1|}{\cubepp|#1|/>`>`>`>/}}%

\def\isquare{\ifnextchar({\isquarep}{\isquarep(\default,\default)}}%
\def\isquarep(#1){\ifnextchar|{\isquarepp(#1)}{\isquarepp(#1)|alrb|}}%
\def\isquarepp(#1)|#2|{\ifnextchar/{\isquareppp(#1)|#2|}%
    {\isquareppp(#1)|#2|/>`>`>`>/}}%
\def\isquareppp(#1)|#2|/#3`#4`#5`#6/{%
    \ifnextchar<{\isquarepppp(#1)|#2|/#3`#4`#5`#6/}%
    {\isquarepppp(#1)|#2|/#3`#4`#5`#6/<500,500>}}%

\def\cubepp|#1#2#3#4|/#5`#6`#7`#8/[#9]{%
\def\next[##1`##2`##3`##4]{\gdef\Labela{##1}%
\gdef\Labelb{##2}\gdef\Labelc{##3}\gdef\Labeld{##4}}\next[#9]%
\xend\xpos \yend\ypos%
\Xend\xend\advance\Xend by -\Xpos%
\Yend\yend\advance\Yend by -\Ypos%
\domorphism(\Xpos,\Ypos)|#2|/#6/<\Xend,\Yend>[\Nodeb`\nodeb;\Labelb]%
\advance\Xpos by-\deltaX%
\advance\xend by-\deltax%
\Xend\xend\advance\Xend by -\Xpos%
\domorphism(\Xpos,\Ypos)|#1|/#5/<\Xend,\Yend>[\Nodea`\nodea;\Labela]%
\advance\Ypos by-\deltaY%
\advance\yend by-\deltay%
\Yend\yend\advance\Yend by -\Ypos%
\domorphism(\Xpos,\Ypos)|#3|/#7/<\Xend,\Yend>[\Nodec`\nodec;\Labelc]%
\advance\Xpos by\deltaX%
\advance\xend by\deltax%
\Xend\xend\advance\Xend by -\Xpos%
\domorphism(\Xpos,\Ypos)|#4|/#8/<\Xend,\Yend>[\Noded`\noded;\Labeld]%
\ignorespaces}%

\def\setwdth#1#2{\setbox0\hbox{$\labelstyle#1$}\wdth=\wd0%
\setbox0\hbox{$\labelstyle#2$}\ifnum\wdth<\wd0 \wdth=\wd0 \fi}%

\def\topppp/#1/<#2>^#3_#4{\:%
\ifnum#2=0%
   \setwdth{#3}{#4}\deltax=\wdth \divide \deltax by \ul%
   \advance \deltax by \defaultmargin  \ratchet{\deltax}{200}%
\else \deltax #2%
\fi%
\xy\ar@{#1}^{#3}_{#4}(\deltax,0) \endxy%
\:}%

\def\toppp/#1/<#2>^#3{\ifnextchar_{\topppp/#1/<#2>^{#3}}{\topppp/#1/<#2>^{#3}_{}}}%
\def\topp/#1/<#2>{\ifnextchar^{\toppp/#1/<#2>}{\toppp/#1/<#2>^{}}}%
\def\toop/#1/{\ifnextchar<{\topp/#1/}{\topp/#1/<0>}}%
\def\to{\ifnextchar/{\toop}{\toop/>/}}%


\def\mon{\to/ >->/}%
\def\epi{\to/->>/}%
\def\toleft{\to/<-/}%
\def\monleft{\to/<-< /}%
\def\epileft{\to/<<-/}%


\def\twopppp/#1`#2/<#3>^#4_#5{\:%
\ifnum0=#3%
  \setwdth{#4}{#5}\deltax=\wdth \divide \deltax by \ul \advance \deltax%
  by \defaultmargin \ratchet{\deltax}{200}%
\else \deltax#3 \fi%
\xy\ar@{#1}@<2.5pt>^{#4}(\deltax,0)%
\ar@{#2}@<-2.5pt>_{#5}(\deltax,0)\endxy\:}%

\def\twoppp/#1`#2/<#3>^#4{\ifnextchar_{\twopppp/#1`#2/<#3>^{#4}}%
  {\twopppp/#1`#2/<#3>^{#4}_{}}}%
\def\twopp/#1`#2/<#3>{\ifnextchar^{\twoppp/#1`#2/<#3>}{\twoppp/#1`#2/<#3>^{}}}%
\def\twop/#1`#2/{\ifnextchar<{\twopp/#1`#2/}{\twopp/#1`#2/<0>}}%
\def\two{\ifnextchar/{\twop}{\twop/>`>/}}%

\def\twoleft{\two/<-`<-/}%

\def\threeppppp/#1`#2`#3/<#4>^#5|#6_#7{\:%
\ifnum0=#4%
\setbox0\hbox{$\labelstyle#5$}\wdth=\wd0%
\setbox0\hbox{$\labelstyle#6$}\ifnum\wdth<\wd0 \wdth=\wd0 \fi%
\setbox0\hbox{$\labelstyle#7$}\ifnum\wdth<\wd0 \wdth=\wd0 \fi%
\deltax=\wdth \divide \deltax by \ul \advance \deltax by%
\defaultmargin \ratchet{\deltax}{300}%
\else\deltax#4 \fi%
    \xy \ifnum\wd0=0 \ar@{#2}(\deltax,0)%
    \else \ar@{#2}|{#6}(\deltax,0)\fi%
\ar@{#1}@<4.5pt>^{#5}(\deltax,0)%
\ar@{#3}@<-4.5pt>_{#7}(\deltax,0)\endxy\:}%

\def\threepppp/#1`#2`#3/<#4>^#5|#6{\ifnextchar_{\threeppppp%
  /#1`#2`#3/<#4>^{#5}|{#6}}{\threeppppp/#1`#2`#3/<#4>^{#5}|{#6}_{}}}%
\def\threeppp/#1`#2`#3/<#4>^#5{\ifnextchar|{\threepppp%
  /#1`#2`#3/<#4>^{#5}}{\threepppp/#1`#2`#3/<#4>^{#5}|{}}}%
\def\threepp/#1`#2`#3/<#4>{\ifnextchar^{\threeppp/#1`#2`#3/<#4>}%
  {\threeppp/#1`#2`#3/<#4>^{}}}%
\def\threep/#1`#2`#3/{\ifnextchar<{\threepp/#1`#2`#3/}%
  {\threepp/#1`#2`#3/<0>}}%
\def\three{\ifnextchar/{\threep}{\threep/>`>`>/}}%


\def\twoar(#1,#2){{%
 \scalefactor{0.1}%
 \deltax#1\deltay#2%
 \deltaX=\ifnum\deltax<0-\fi\deltax%
 \deltaY=\ifnum\deltay<0-\fi\deltay%
 \Xend\deltax \multiply \Xend by \deltax%
 \Yend\deltay \multiply \Yend by \deltay%
 \advance\Xend by \Yend \multiply \Xend by 3%
 \ifnum \deltaX > \deltaY%
    \multiply \deltaX by 3 \advance \deltaX by \deltaY%
 \else%
    \multiply \deltaY by 3 \advance \deltaX by \deltaY%
 \fi%
 \multiply\deltax by 500%
 \multiply\deltay by 500%
 \xpos\deltax \multiply \xpos by 3 \divide\xpos by \deltaX%
 \Xpos\deltax \multiply \Xpos by \deltaX \divide \Xpos by \Xend%
 \advance \xpos by \Xpos%
 \ypos\deltay \multiply \ypos by 3 \divide\ypos by \deltaX%
 \Ypos\deltay \multiply \Ypos by \deltaX \divide \Ypos by \Xend%
 \advance \ypos by \Ypos%
 \xy \ar@{=>}(\xpos,\ypos) \endxy%
}\ignorespaces}%

\def\iiixiiipppppp(#1,#2)|#3|/#4/<#5>#6<#7>[#8;#9]{%
 \xpos#1\ypos#2\relax%
 \def\next|##1##2##3##4##5##6##7|{\def\xa{##1}\def\xb{##2}%
 \def\xc{##3}\def\xd{##4}\def\xe{##5}\def\xf{##6}\nextt|##7|}%
 \def\nextt|##1##2##3##4##5##6|{\def\xg{##1}\def\xh{##2}%
 \def\xi{##3}\def\xj{##4}\def\xk{##5}\def\xl{##6}}%
 \next|#3|%
 \def\next<##1,##2>{\deltax##1\deltay##2}%
 \next<#5>%
 \def\next<##1,##2>{\deltaX##1\deltaY##2}%
 \next<#7>%
 \def\next##1{\topw##1\relax%
 \ifodd\topw \def\zl{}\else\def\zl{\relax}\fi \divide\topw by 2
 \ifodd\topw \def\zk{}\else\def\zk{\relax}\fi \divide\topw by 2
 \ifodd\topw \def\zj{}\else\def\zj{\relax}\fi \divide\topw by 2
 \ifodd\topw \def\zi{}\else\def\zi{\relax}\fi \divide\topw by 2
 \ifodd\topw \def\zh{}\else\def\zh{\relax}\fi \divide\topw by 2
 \ifodd\topw \def\zg{}\else\def\zg{\relax}\fi \divide\topw by 2
 \ifodd\topw \def\zf{}\else\def\zf{\relax}\fi \divide\topw by 2
 \ifodd\topw \def\ze{}\else\def\ze{\relax}\fi \divide\topw by 2
 \ifodd\topw \def\zd{}\else\def\zd{\relax}\fi \divide\topw by 2
 \ifodd\topw \def\zc{}\else\def\zc{\relax}\fi \divide\topw by 2
 \ifodd\topw \def\zb{}\else\def\zb{\relax}\fi \divide\topw by 2
 \ifodd\topw \def\za{}\else\def\za{\relax}\fi}%
 \next{#6}%
 \def\next[##1`##2`##3`##4`##5`##6`##7`##8`##9]{%
 \def\nodea{##1}\def\nodeb{##2}\def\nodec{##3}%
 \def\noded{##4}\def\nodee{##5}\def\nodef{##6}%
 \def\nodeg{##7}\def\nodeh{##8}\def\nodei{##9}}%
 \next[#8]%
 \def\next[##1`##2`##3`##4`##5`##6`##7]{%
 \def\labela{##1}\def\labelb{##2}\def\labelc{##3}%
 \def\labeld{##4}\def\labele{##5}\def\labelf{##6}\nextt[##7]}%
 \def\nextt[##1`##2`##3`##4`##5`##6]{%
 \def\labelg{##1}\def\labelh{##2}\def\labeli{##3}%
 \def\labelj{##4}\def\labelk{##5}\def\labell{##6}}%
 \next[#9]%
 \def\next/##1`##2`##3`##4`##5`##6`##7/{%
\morphism(\xpos,\ypos)|\xe|/{##5}/<\deltax,0>[\nodeg`\nodeh;\labele]%
 \ifx\zi\empty\relax \morphism(\xpos,\ypos)||/<-/<-\deltaX,0>[\nodeg`0;]\fi%
 \ifx\zd\empty\relax \morphism(\xpos,\ypos)||<0,-\deltaY>[\nodeg`0;]\fi%
 \advance\xpos by \deltax%
 \morphism(\xpos,\ypos)|\xf|/{##6}/<\deltax,0>[\nodeh`\nodei;\labelf]%
 \ifx\ze\empty\relax \morphism(\xpos,\ypos)||<0,-\deltaY>[\nodeh`0;]\fi%
 \advance\xpos by \deltax%
 \ifx\zf\empty\relax \morphism(\xpos,\ypos)||<0,-\deltaY>[\nodei`0;]\fi%
 \ifx\zl\empty\relax \morphism(\xpos,\ypos)||<\deltaX,0>[\nodei`0;]\fi%
 \advance\ypos by \deltay%
 \ifx\zk\empty\relax \morphism(\xpos,\ypos)||<\deltaX,0>[\nodef`0;]\fi%
 \advance\xpos by -\deltax%
 \morphism(\xpos,\ypos)|\xd|/{##4}/<\deltax,0>[\nodee`\nodef;\labeld]%
 \advance\xpos by -\deltax%
 \morphism(\xpos,\ypos)|\xc|/{##3}/<\deltax,0>[\noded`\nodee;\labelc]%
 \ifx\zh\empty\relax \morphism(\xpos,\ypos)||/<-/<-\deltaX,0>[\noded`0;]\fi%
 \advance\ypos by \deltay%
 \morphism(\xpos,\ypos)|\xa|/{##1}/<\deltax,0>[\nodea`\nodeb;\labela]%
 \ifx\zg\empty\relax \morphism(\xpos,\ypos)||/<-/<-\deltaX,0>[\nodea`0;]\fi%
 \ifx\za\empty\relax \morphism(\xpos,\ypos)||/<-/<0,\deltaY>[\nodea`0;]\fi%
 \advance\xpos by \deltax%
 \morphism(\xpos,\ypos)|\xb|/{##2}/<\deltax,0>[\nodeb`\nodec;\labelb]%
 \ifx\zb\empty\relax \morphism(\xpos,\ypos)||/<-/<0,\deltaY>[\nodeb`0;]\fi%
 \advance\xpos by \deltax%
 \ifx\zc\empty\relax \morphism(\xpos,\ypos)||/<-/<0,\deltaY>[\nodec`0;]\fi%
 \ifx\zj\empty\relax \morphism(\xpos,\ypos)||<\deltaX,0>[\nodec`0;]\fi%
 \nextt/##7/}%
 \def\nextt/##1`##2`##3`##4`##5`##6/{%
 \morphism(\xpos,\ypos)|\xi|/{##3}/<0,-\deltay>[\nodec`\nodef;\labeli]%
 \advance\xpos by -\deltax%
 \morphism(\xpos,\ypos)|\xh|/{##2}/<0,-\deltay>[\nodeb`\nodee;\labelh]%
 \advance\xpos by -\deltax%
 \morphism(\xpos,\ypos)|\xg|/{##1}/<0,-\deltay>[\nodea`\noded;\labelg]%
 \advance\ypos by -\deltay%
 \morphism(\xpos,\ypos)|\xj|/{##4}/<0,-\deltay>[\noded`\nodeg;\labelj]%
 \advance\xpos by \deltax%
 \morphism(\xpos,\ypos)|\xk|/{##5}/<0,-\deltay>[\nodee`\nodeh;\labelk]%
 \advance\xpos by \deltax%
 \morphism(\xpos,\ypos)|\xl|/{##6}/<0,-\deltay>[\nodef`\nodei;\labell]}%
 \next/#4/\ignorespaces}%

\def\iiixiii{\ifnextchar({\iiixiiip}{\iiixiiip(0,0)}}%
\def\iiixiiip(#1){\ifnextchar|{\iiixiiipp(#1)}%
  {\iiixiiipp(#1)|aammbblmrlmr|}}%
\def\iiixiiipp(#1)|#2|{\ifnextchar/{\iiixiiippp(#1)|#2|}%
    {\iiixiiippp(#1)|#2|/>`>`>`>`>`>`>`>`>`>`>`>/}}%
\def\iiixiiippp(#1)|#2|/#3/{%
    \ifnextchar<{\iiixiiipppp(#1)|#2|/#3/}%
    {\iiixiiipppp(#1)|#2|/#3/<\default,\default>}}%
\def\iiixiiipppp(#1)|#2|/#3/<#4>{\ifnextchar[{\iiixiiippppp(#1)|#2|/#3/%
   <#4>0<0,0>}{\iiixiiippppp(#1)|#2|/#3/<#4>}}%
\def\iiixiiippppp(#1)|#2|/#3/<#4>#5{\ifnextchar<%
   {\iiixiiipppppp(#1)|#2|/#3/<#4>{#5}}%
   {\iiixiiipppppp(#1)|#2|/#3/<#4>{#5}<400,400>}}%

\def\iiixiipppppp(#1,#2)|#3|/#4/<#5>#6<#7>[#8;#9]{%
 \xpos#1\ypos#2\relax%
 \def\next|##1##2##3##4##5##6##7|{\def\xa{##1}\def\xb{##2}%
 \def\xc{##3}\def\xd{##4}\def\xe{##5}\def\xf{##6}\def\xg{##7}}%
 \next|#3|%
 \def\next<##1,##2>{\deltax##1\deltay##2}%
 \next<#5>%
 \deltaX#7
 \topw#6
 \def\next{%
 \ifodd\topw \def\za{}\else\def\za{\relax}\fi \divide\topw by 2
 \ifodd\topw \def\zb{}\else\def\zb{\relax}\fi \divide\topw by 2
 \ifodd\topw \def\zc{}\else\def\zc{\relax}\fi \divide\topw by 2
 \ifodd\topw \def\zd{}\else\def\zd{\relax}\fi}%
 \next%
 \def\next[##1`##2`##3`##4`##5`##6]{%
 \def\nodea{##1}\def\nodeb{##2}\def\nodec{##3}%
 \def\noded{##4}\def\nodee{##5}\def\nodef{##6}}%
 \next[#8]%
 \def\next[##1`##2`##3`##4`##5`##6`##7]{%
 \def\labela{##1}\def\labelb{##2}\def\labelc{##3}%
 \def\labeld{##4}\def\labele{##5}\def\labelf{##6}\def\labelg{##7}}%
 \next[#9]%
 \def\next/##1`##2`##3`##4`##5`##6`##7/{%
 \ifx\zc\empty\relax\morphism(\xpos,\ypos)<\deltaX,0>[0`\noded;]\fi%
 \advance\xpos by\deltaX%
 \morphism(\xpos,\ypos)|\xc|/##3/<\deltax,0>[\noded`\nodee;\labelc]%
 \advance\xpos by \deltax%
 \morphism(\xpos,\ypos)|\xd|/##4/<\deltax,0>[\nodee`\nodef;\labeld]%
 \advance\xpos by \deltax%
 \ifx\zd\empty\relax  \morphism(\xpos,\ypos)<\deltaX,0>[\nodef`0;]\fi%
 \advance\xpos by -\deltaX  \advance\xpos by -\deltax
 \advance\xpos by -\deltax  \advance\ypos by \deltay
 \ifx\za\empty\relax\morphism(\xpos,\ypos)<\deltaX,0>[0`\nodea;]\fi%
 \advance\xpos by\deltaX%
 \morphism(\xpos,\ypos)|\xa|/##1/<\deltax,0>[\nodea`\nodeb;\labela]%
 \morphism(\xpos,\ypos)|\xe|/##5/<0,-\deltay>[\nodea`\noded;\labele]%
 \advance\xpos by \deltax%
 \morphism(\xpos,\ypos)|\xb|/##2/<\deltax,0>[\nodeb`\nodec;\labelb]%
 \morphism(\xpos,\ypos)|\xf|/##6/<0,-\deltay>[\nodeb`\nodee;\labelf]%
 \advance\xpos by \deltax%
 \morphism(\xpos,\ypos)|\xg|/##7/<0,-\deltay>[\nodec`\nodef;\labelg]%
 \ifx\zb\empty\relax \morphism(\xpos,\ypos)<\deltaX,0>[\nodec`0;]\fi}%
 \next/#4/\ignorespaces}%


\def\iiixii{\ifnextchar({\iiixiip}{\iiixiip(0,0)}}%
\def\iiixiip(#1){\ifnextchar|{\iiixiipp(#1)}%
  {\iiixiipp(#1)|aabblmr|}}%
\def\iiixiipp(#1)|#2|{\ifnextchar/{\iiixiippp(#1)|#2|}%
    {\iiixiippp(#1)|#2|/>`>`>`>`>`>`>/}}%
\def\iiixiippp(#1)|#2|/#3/{%
    \ifnextchar<{\iiixiipppp(#1)|#2|/#3/}%
    {\iiixiipppp(#1)|#2|/#3/<\default,\default>}}%
\def\iiixiipppp(#1)|#2|/#3/<#4>{\ifnextchar[{\iiixiippppp(#1)|#2|/#3/%
   <#4>{0}<0>}{\iiixiippppp(#1)|#2|/#3/<#4>}}%
\def\iiixiippppp(#1)|#2|/#3/<#4>#5{\ifnextchar<%
   {\iiixiipppppp(#1)|#2|/#3/<#4>{#5}}%
   {\iiixiipppppp(#1)|#2|/#3/<#4>{#5}<400>}}%

\def\node#1(#2,#3)[#4]{%
\expandafter\gdef\csname x@#1\endcsname{#2}%
\expandafter\gdef\csname y@#1\endcsname{#3}%
\expandafter\gdef\csname ob@#1\endcsname{#4}%
\ignorespaces}%

\newcount\xfinish%
\newcount\yfinish%
\def\arrow{\ifnextchar|{\arrowp}{\arrowp|a|}}%
\def\arrowp|#1|{\ifnextchar/{\arrowpp|#1|}{\arrowpp|#1|/>/}}%
\def\arrowpp|#1|/#2/[#3`#4;#5]{%
\xfinish=\csname x@#4\endcsname%
\yfinish=\csname y@#4\endcsname%
\advance\xfinish by -\csname x@#3\endcsname%
\advance\yfinish by -\csname y@#3\endcsname%
\morphism(\csname x@#3\endcsname,\csname y@#3\endcsname)|#1|/{#2}/%
<\xfinish,\yfinish>[\csname ob@#3\endcsname`\csname ob@#4\endcsname;#5]%
}%


\def\Loop(#1,#2)#3(#4,#5){\POS(#1,#2)*+!!<0ex,\axis>{#3}\ar@(#4,#5)}%
\def\iloop#1(#2,#3){\xy\Loop(0,0)#1(#2,#3)\endxy}%


\catcode`\@=\atcode%
\endinput%
\entrymodifiers={+!!<0pt,\fontdimen22\textfont2>}%



   the \xybox  does not allow control over *where*,%
inside the <object> that it builds,%
the reference point is to be located.%

Accordingly, I've just devised a variant that builds%
the same kind of compound <object>, but also sets%
its reference-point to be at the <coord> of the%
last <POS> within the box; i.e., the <coord> for%
the <object> that has been built is at the current%
<POS> when the Xy-pic parsing has been completed.%
The LRUD extents are the size of the complete box;%
i.e., *not* the extents of the final <POS>.%

Here is coding that should go in your document's%
preamble -- eventually it should be added to  xy.tex%



---------  start of new Xy-pic definitions  -------%


> \makeatletter   % adjust the \catcode of @%

 this is a better definition for the new  \xyobjbox%

   \xydef@\xyobjbox#1{\xy%
     \let \PATHafterPOS\PATHafterPOS@default%
     \let \arsavedPATHafterPOS@@\relax%
     \let\afterar@@\relax%
     \POS#1\endxyobj\Edge@c={\rectangleEdge}\computeLeftUpness@}%

> \xydef@\endxyobj{\if\inxy@\else\xyerror@{Unexpected \string\endxy}{}\fi%
>  \relax%
>   \dimen@=\Y@max \advance\dimen@-\Y@min%
>   \ifdim\dimen@<\z@ \dimen@=\z@ \Y@min=\z@ \Y@max=\z@ \fi%
>   \dimen@=\X@max \advance\dimen@-\X@min%
>   \ifdim\dimen@<\z@ \dimen@=\z@ \X@min=\z@ \X@max=\z@ \fi%
>   \edef\tmp@{\egroup%
>     \setboxz@h{\kern-\the\X@min \boxz@}%
>     \ht\z@=\the\Y@max \dp\z@=-\the\Y@min \wdz@=\the\dimen@%
>     \noexpand\maybeunraise@ \raise\dimen@\boxz@%
>     \noexpand\recoverXyStyle@ \egroup \noexpand\xy@end%
>     \U@c=\the\Y@max \advance\U@c-\the\Y@c%
>     \D@c=-\the\Y@min \advance\D@c\the\Y@c%
>     \L@c=-\the\X@min  \advance\L@c\the\X@c%
>     \R@c=\the\X@max  \advance\R@c-\the\X@c%
>    }\tmp@}%
>%
> \makeatother   % revert \catcode of @%
>%
> ---------  end of new Xy-pic definitions  -------%
\makeatletter%
\gdef\xymerge@MinMax{}%
\xydef@\twocell{\hbox\bgroup\xysave@MinMax\@twocell}%
\xydef@\uppertwocell{\hbox\bgroup\xysave@MinMax\@uppertwocell}%
\xydef@\lowertwocell{\hbox\bgroup\xysave@MinMax\@lowertwocell}%
\xydef@\compositemap{\hbox\bgroup\xysave@MinMax\@compositemap}%
\xydef@\twocelll#1#{\hbox\bgroup\xysave@MinMax\xy@\save\save@\@twocelll{%
#1}}%

\xydef@\xysave@MinMax{\xdef\xymerge@MinMax{%
   \noexpand\ifdim\X@max<\the\X@max \X@max=\the\X@max\noexpand\fi%
   \noexpand\ifdim\X@min>\the\X@min \X@min=\the\X@min\noexpand\fi%
   \noexpand\ifdim\Y@max<\the\Y@max \Y@max=\the\Y@max\noexpand\fi%
   \noexpand\ifdim\Y@min>\the\Y@min \Y@min=\the\Y@min\noexpand\fi%
  }}%
\xydef@\drop@Twocell{\boxz@ \xymerge@MinMax}%


\xydef@\twocell@DONE{%
  \edef\tmp@{\egroup%
   \X@min=\the\X@min \X@max=\the\X@max%
   \Y@min=\the\Y@min \Y@max=\the\Y@max}\tmp@%
  \L@c=\X@c \advance\L@c-\X@min \R@c=\X@max \advance\R@c-\X@c%
  \D@c=\Y@c \advance\D@c-\Y@min \U@c=\Y@max \advance\U@c-\Y@c%
  \ht\z@=\U@c \dp\z@=\D@c \dimen@=\L@c \advance\dimen@\R@c \wdz@=\dimen@%
  \computeLeftUpness@%
  \setboxz@h{\kern-\X@p \raise-\Y@c\boxz@ }%
  \dimen@=\L@c \advance\dimen@\R@c \wdz@=\dimen@ \ht\z@=\U@c \dp\z@=\D@c%
  \Edge@c={\rectangleEdge}\Invisible@false \Hidden@false%
  \edef\Drop@@{\noexpand\drop@Twocell%
   \noexpand\def\noexpand\Leftness@{\Leftness@}%
   \noexpand\def\noexpand\Upness@{\Upness@}}%
  \edef\Connect@@{\noexpand\connect@Twocell%
   \noexpand\ifdim\X@max<\the\X@max \X@max=\the\X@max\noexpand\fi%
   \noexpand\ifdim\X@min>\the\X@min \X@min=\the\X@min\noexpand\fi%
   \noexpand\ifdim\Y@max<\the\Y@max \Y@max=\the\Y@max\noexpand\fi%
   \noexpand\ifdim\Y@min>\the\Y@min \Y@min=\the\Y@min\noexpand\fi }%
  \xymerge@MinMax%
}%
\makeatother%




% \usepackage{makeindex}

%%%%
% Theorem-type environments
%%%%

%% following Cisinski's style, which I found excellent, the theorem-like environments are set up to number _all_ paragraphs [in the conceptual rather than typographic sense] consecutively.  the major advantage of this is making any paragraph referenceable, and hence making the (always rather arbitrary) decision of what to pick out as theorems, definitions, etc. much less consequential and more flexible.


\makeatletter

\newtheoremstyle{mytheorem}{}{}{\itshape}{}{\bfseries}{.}{5\p@ plus\p@ minus\p@}{}

\newtheoremstyle{mydefinition}{}{}{}{}{\bfseries}{.}{5\p@ plus\p@ minus\p@}{}

%% proof environment taken almost verbatim from amsthm.sty, to remove the small caps and indentation that are used in amsbook.cls
\renewenvironment{proof}[1][Proof]{\par
  \pushQED{\qed}%
  \normalfont \topsep6\p@\@plus6\p@\relax
  \trivlist
  \item[\hskip\labelsep
        \itshape
    #1\@addpunct{.}]\ignorespaces
}{%
  \popQED\endtrivlist\@endpefalse
}

\makeatother



\theoremstyle{mytheorem} 
\newtheorem{thm}{Theorem}[section]
\newtheorem{theorem}[thm]{Theorem}
\newtheorem{proposition}[thm]{Proposition}
\newtheorem{lemma}[thm]{Lemma}
\newtheorem{corollary}[thm]{Corollary}
\newtheorem{scholium}[thm]{Scholium}
\newtheorem{conjecture}[thm]{Conjecture}

\theoremstyle{mydefinition}
\newtheorem{definition}[thm]{Definition}
\newtheorem{para}[thm]{}
\newtheorem{exercise}[thm]{Exercise}

%\theoremstyle{remark}
\newtheorem{remark}[thm]{Remark}
\newtheorem{notation}[thm]{Notations}
\newtheorem{example}[thm]{Example}
\newtheorem{examples}[thm]{Examples}

\newtheorem{mydefinition}[thm]{Definition}


\setcounter{tocdepth}{3}
\setcounter{secnumdepth}{2}

\renewcommand{\baselinestretch}{1.5}


% Peter LeFanu Lumsdaine, June 2010
% macros for my thesis

% Contents:
%
% - Binary relations
% - Category names
% - Single letters


%%%%
% Binary relations, operators
%%%%

% \newcommand{\coslice}{\!\!\; \backslash \!\!\!\; \backslash}
\newcommand{\coslice}{\!\!\; \backslash}
\newcommand{\cotensor}{\pitchfork}
\renewcommand{\equiv}{\simeq}
\newcommand{\Iff}{\Leftrightarrow}
\newcommand{\Imp}{\Rightarrow}
\newcommand{\into}{\hookrightarrow}
\newcommand{\iso}{\cong}
\newcommand{\propeq}{\approx}
\newcommand{\mono}{\to/ >->/}
\newcommand{\slice}{\!\!\; / \!\!\!\; /}
\newcommand{\tensor}{\otimes}
\newcommand{\tightcdot}{\! \cdot \!}
\newcommand{\tightcolon}{\!\!\, : \!\!\,}
\newcommand{\To}{\Rightarrow}
\newcommand{\types}{\vdash}

%%%% 
% Single styled characters (or almost single) and character-like symbols
%%%%

\newcommand{\Two}{\mathbf{2}}
\newcommand{\A}{A_\bullet}
\newcommand{\abar}{\overline{a}}
\newcommand{\uA}{\underline{A}}
\newcommand{\uAbu}{\underline{A}_\bullet}
\newcommand{\B}{B_\bullet}
% \newcommand{\ML}{\mathit{ML_I}}
% \newcommand{\MLfrag}{\mathit{ML}^\Id}
\renewcommand{\c}{\vec c}
\newcommand{\C}{\mathcal{C}}
\newcommand{\CC}{\mathbb{C}}
\newcommand{\Cbf}{\mathcal{C}}
\newcommand{\Cbu}{C_\bullet}
\newcommand{\D}{\mathcal{D}}
% \newcommand{\bC}{\mathbf{C}}
% \newcommand{\Chat}{\widehat{\mathbb{C}}}
% \newcommand{\D}{\mathbb{D}}
% \newcommand{\bigD}{\mathcal{D}}
% \newcommand{\bD}{\mathbf{D}}
\newcommand{\diag}{\delta}
% \renewcommand{\d}{\partial}
\newcommand{\E}{\mathcal{E}}
\newcommand{\Ebu}{\E_\bullet}
\newcommand{\f}{\vec f}
\newcommand{\fbf}{\mathbf{f}}
\newcommand{\F}{\mathcal{F}}
\newcommand{\FF}{\mathbb{F}}
\newcommand{\g}{\vec g}
\newcommand{\gbf}{\mathbf{g}}
\newcommand{\G}{\mathbb{G}}
\newcommand{\h}{\vec h}
\newcommand{\I}{\mathcal{I}}   % generating cofibrations.  mathscr is prettier,
\newcommand{\J}{\mathcal{J}}   % but I find its I, J confusing.
\newcommand{\K}{\mathcal{K}}    % A class of left maps
\renewcommand{\L}{\mathcal{L}}    % A class of left maps
\newcommand{\N}{\mathbb{N}}   % Natural numbers
\newcommand{\NN}{\mathbb{N}}   % ditto
\newcommand{\Ncal}{\mathcal{N}}   % Nerve
% \renewcommand{\P}{P_{\MLfrag}}
\newcommand{\PML}{P_{\MLId}}
% \newcommand{\Pfull}{P_{\ML}}
\newcommand{\PARA}{\textparagraph}
\newcommand{\pow}{\mathcal{P}}
\newcommand{\p}{\vec p}
\newcommand{\SEC}{\textsection}
\newcommand{\R}{\mathcal{R}}    % A class of right maps
\renewcommand{\r}{\vec r}
\renewcommand{\S}{\textsf{\textbf{S}}}    % Another generic type theory
\newcommand{\T}{\textbf{\textsf{T}}}      % A generic type theory
\newcommand{\Tcal}{\mathcal{T}}      % A 2-monad
\newcommand{\TT}{\mathbb{T}}    % A generic type theory, seen as a categorical structure
\newcommand{\Tsf}{\mathsf{T}}
\newcommand{\tsf}{\mathsf{t}}
\renewcommand{\u}{\vec u}
\newcommand{\V}{\mathcal{V}}
% \renewcommand{\v}{\vec v}
\newcommand{\W}{\mathcal{W}}
\newcommand{\WW}{\mathbb{W}}
\newcommand{\w}{\vec w}
\newcommand{\Xcal}{\mathcal{X}}
\newcommand{\X}{X_\bullet}
\newcommand{\Xbullet}{X_\bullet}
\newcommand{\x}{\vec x}
\newcommand{\uX}{\underline{X}}
\newcommand{\uXbu}{\underline{X}_\bullet}
\newcommand{\Ycal}{\mathcal{Y}}
\newcommand{\Y}{Y_\bullet}
\newcommand{\y}{\vec y}
\newcommand{\yon}{\mathbf{y}}
\newcommand{\z}{\vec z}

%%%%
% Styled words: general
%%%%

\newcommand{\Alg}[1]{#1\mbox{-}\mathbf{Alg}}
\newcommand{\IntAlg}[2]{\mathbf{Alg}_{#2}(#1)}
\newcommand{\AMS}{AMS}
\newcommand{\AWFS}{AWFS}
\newcommand{\Arr}{\mathrm{Arr}}
\newcommand{\Cat}{\mathbf{Cat}}
\newcommand{\intCat}[1][-]{\mathbf{Cat}(#1)}
\newcommand{\enrCat}[1][\V]{#1\mbox{-}\mathbf{Cat}}
\newcommand{\nCat}[1][n]{#1\mbox{-}\mathbf{Cat}}
\newcommand{\cl}{\mathcal{C}\! \ell}
\newcommand{\clpi}{{}^\Pi \cl}
\newcommand{\ClovFib}{\mathbf{ClovFib}}
\newcommand{\Coll}{\mathbf{Coll}}
\newcommand{\CwA}{\mathbf{CwA}}
\newcommand{\CwAId}{\mathbf{CwA}^{\Id}}
\newcommand{\CwF}{\mathbf{CwF}}
\newcommand{\CwFId}{\mathbf{CwF}^{\Id}}
\newcommand{\Cxt}{\mathrm{Cxt}}
\newcommand{\cxl}{\mathit{cxl}}
\newcommand{\CofCosps}{\mathbf{CofCosps}}
\newcommand{\cod}{\mathrm{cod}}
\newcommand{\cof}{\mathit{cof}}
\newcommand{\colim}{\varinjlim}
\newcommand{\del}{\partial}
\newcommand{\dom}{\mathrm{dom}}
\newcommand{\DTT}{\mathbf{DTT}}
\newcommand{\End}{\mathrm{End}}
% \newcommand{\ev}{\mathbf{ev}}
\newcommand{\Fib}{\mathbf{Fib}}
\newcommand{\FibSpans}{\mathbf{FibSpans}}
\newcommand{\FSCC}{\mathbf{FSCC}}
\newcommand{\fscc}{\textsc{fscc}}
\newcommand{\fsccs}{\textsc{fscc}'s}
\newcommand{\FSCS}{\mathbf{FSCS}}
\newcommand{\fscs}{\textsc{fscs}}
\newcommand{\fscss}{\textsc{fscs}'s}
\newcommand{\globe}[1][n]{\textsf{\textbf{G}}_{#1}}
\newcommand{\piglobe}[1][n]{{}^\Pi \textsf{\textbf{G}}_{#1}}
  % oh dear, I keep reading this as "pig lobe".  It's clearly too late at night…
\newcommand{\globefig}[1]{\globe[#1]}
  % needed because commands with optional[arguments] seem to interact badly with parsing of square brackets inside \bfig...\efig.
\newcommand{\globes}{\textsf{\textbf{G}}_\bullet}
\newcommand{\piglobes}{{}^\Pi \textsf{\textbf{G}}_\bullet}
\newcommand{\maybepiglobes}{{}^{(\Pi)} \textsf{\textbf{G}}_\bullet}
% \newcommand{\longGSets}{[\mathbb{G}^\op,\mathbf{Sets}]}
\newcommand{\GSets}{\widehat{\mathbb{G}}}
\newcommand{\GnSets}[1][n]{\widehat{\mathbb{G}}_{#1}}
\newcommand{\Hom}{\mathrm{Hom}}
\renewcommand{\lim}{\varprojlim}
\newcommand{\Lan}{\mathrm{Lan}}
\newcommand{\lax}{\mathrm{lax}}
\newcommand{\Mon}{\mathbf{Mon}}
\newcommand{\MonGlobCat}{\mathbf{MonGlobCat}}
\newcommand{\ML}{\textsf{\textbf{ML}}}
\newcommand{\MLId}{\textsf{\textbf{ML}}^{\Id}}
\newcommand{\ob}{\operatorname{ob}}
\newcommand{\op}{\mathrm{op}}
\newcommand{\nOpd}[1][n]{#1\mbox{-}\mathbf{Opd}}
% \newcommand{\Operads}{\mathbf{Operads}}
\newcommand{\pd}{\mathbf{pd}}
\newcommand{\pathpd}{\mathit{path}}
\newcommand{\PsAlg}[2][]{\mathbf{Ps}_{#1}\mbox{-}{#2}\mbox{-}\mathbf{Alg}}
\newcommand{\QCat}{\mathbf{QCat}}
\newcommand{\qcat}{\mathit{qcat}}
\newcommand{\Ran}{\mathrm{Ran}}
% \newcommand{\RefGlob}[1][]{\mathbf{RefGlob}_{#1}}
% \newcommand{\RefOneGlob}[1][]{\mathbf{Ref}_1\mathbf{Glob}_{#1}}
\newcommand{\RefnGlob}[2]{\mathbf{Ref}_{#1}\mathbf{Glob}(#2)}
\newcommand{\Sets}{\mathbf{Sets}}
\newcommand{\Spans}[1][]{\mathbf{Spans}(#1)}
\newcommand{\Spansplain}{\mathbf{Spans}}
\newcommand{\str}{\mathrm{str}}
\newcommand{\strat}{\textrm{strat}}
\newcommand{\pdfDTT}{\texorpdfstring{$\DTT$}{DTT}}
\newcommand{\pdfeta}{\texorpdfstring{$\eta$}{η}}
\newcommand{\pdfId}{\texorpdfstring{$\Id$}{Id}}
\newcommand{\pdfJbar}{\texorpdfstring{$\Jbar$}{J-bar}}
\newcommand{\pdfPi}{\texorpdfstring{$\Pi$}{Π}}
\newcommand{\pdfomega}{\texorpdfstring{$\omega$}{ω}}
\renewcommand{\th}{\mathbf{th}}
\newcommand{\Th}{\mathbf{Th}}
\newcommand{\ThId}{\mathbf{Th}^{\Id}}
\newcommand{\ThIdPi}{\mathbf{Th}^{\Id,\Pi}}
\newcommand{\Tm}{\mathrm{Tm}}
\newcommand{\tm}{\textrm{tm}}
\newcommand{\Top}{\mathbf{Top}}
\newcommand{\Ty}{\mathrm{Ty}}
\newcommand{\ty}{\textrm{ty}}
\newcommand{\strMonGlobCat}{\mathbf{MonGlobCat}}
\newcommand{\strwCat}{\mathbf{str}\mbox{-}\omega\mbox{-}\mathbf{Cat}}
\newcommand{\strnCat}[1][n]{\mathbf{str}\mbox{-}#1\mbox{-}\mathbf{Cat}}
\newcommand{\SynPres}{\mathbf{SynPres}}
\newcommand{\SynThy}{\mathbf{SynThy}}
\newcommand{\wkwCat}{\mathbf{wk}\mbox{-}\omega\mbox{-}\mathbf{Cat}}
\newcommand{\wkwGpd}{\mathbf{wk}\mbox{-}\omega\mbox{-}\mathbf{Gpd}}
\newcommand{\wknCat}[1][n]{\mathbf{wk}\mbox{-}#1\mbox{-}\mathbf{Cat}}


%%%%
% Styled words: type theory syntax
%%%%

\newcommand{\app}{\textsc{app}}
\newcommand{\apprule}{\textsc{app}}
\newcommand{\Bool}{\mathsf{Bool}}
\newcommand{\cellrule}{\mathsf{cell}}
\newcommand{\comp}{\textsc{comp}}
\newcommand{\CONG}{\textsc{cong}}
\newcommand{\contr}{\textsc{contr}}
\newcommand{\cons}{\textsc{cons}}
\newcommand{\Cterm}{\mathsf{C}}
\newcommand{\cterm}{\mathsf{c}}
\newcommand{\cxt}{\mathsf{cxt}}
\newcommand{\cxtmap}{\mathsf{cxtmap}}
\newcommand{\cxtrule}{\cxt}
\newcommand{\defrule}{\textsc{def}}
\newcommand{\exch}{\textsc{exch}}
\newcommand{\elim}{\textsc{elim}}
\newcommand{\empt}{\textsc{empty}}
\newcommand{\extrule}{\textsc{ext}}
\newcommand{\extterm}{\mathsf{ext}}
\newcommand{\form}{\textsc{form}}
\newcommand{\Id}{\mathsf{Id}}
% \newcommand{\varidelim}[5]{#4\mathsf{ for }#3\mathsf{ in }#1.#2\mathsf{ via }#5}
% \newcommand{\idelim}[5]{J_{#1.#2}(#3,#4,#5)}
\newcommand{\Jterm}{\mathsf{J}}
\newcommand{\Jbar}{\overline{J}}
\newcommand{\Kbar}{\overline{K}}
\newcommand{\Jbarrule}{\overline{J}\mbox{-}2\mbox{-}\textsc{o}}
\newcommand{\Jbarterm}{\overline{\mathsf{J}}}
\newcommand{\Kterm}{\mathsf{K}}
\newcommand{\LF}{\mathrm{LF}}
\newcommand{\Lterm}{\mathsf{L}}
\newcommand{\Lang}{\textbf{\textsf{Lang}}}
\newcommand{\intro}{\textsc{intro}}
\newcommand{\PiIdelim}{{\Pi\mbox{-}\Id\mbox{-}\elim}}
\newcommand{\Piext}{{\Pi\mbox{-}\extrule}}
\newcommand{\Piextapp}{{\Pi\mbox{-}\extrule\mbox{-}\apprule}}
\newcommand{\PiIdcomp}{{\Pi\mbox{-}\Id\mbox{-}\comp}}
\newcommand{\Nat}{\mathsf{Nat}}
\newcommand{\One}{\mathsf{1}}
\newcommand{\prop}{\textsc{prop}}
\newcommand{\refl}{\textsc{refl}}
\newcommand{\sourcerule}{\mathsf{src}}
\newcommand{\subst}{\textsc{subst}}
\newcommand{\substterm}{\mathsf{subst}}
\newcommand{\src}{\mathsf{src}}
\newcommand{\Sterm}{\mathsf{S}}
\newcommand{\sterm}{\mathsf{s}}
% \newcommand{\scterm}{\textsc{term}}
\newcommand{\sym}{\textsc{sym}}
\newcommand{\targetrule}{\mathsf{tgt}}
\newcommand{\termrule}{\textsc{term}}
\newcommand{\term}{\mathsf{term}}
\newcommand{\tgt}{\mathsf{tgt}}
\newcommand{\trans}{\textsc{trans}}
\newcommand{\Tterm}{\mathsf{T}}
\newcommand{\tterm}{\mathsf{t}}
\newcommand{\type}{\mathsf{type}}
\newcommand{\typerule}{\textsc{type}}
% \newcommand{\sctype}{\textsc{type}}
\newcommand{\wkg}{\textsc{wkg}}
\newcommand{\var}{\textsc{var}}
\newcommand{\Zero}{\mathsf{0}}

%%%%
% Other operators
%%%%

\newcommand{\Clw}{\mathbf{Cl}_\omega}
\newcommand{\ClwQCat}{\mathbf{Cl}^\qcat_\omega}


%%%%
% Other symbols
%%%%

\newcommand{\lscott}{[\![}
\newcommand{\rscott}{]\!]}


%%%
%%% Diagram annotations, work with diagxy
%%%


%\newdir{|>}{!/4.7pt/@{|}
%        *:(1,-.2)\dir^{>}
%        *:(1,+.2)@_{>}}

\newdir{|>}{!/4.5pt/@{|}*:(1,-.2)@^{>}*:(1,+.2)@_{>}}
       
\newdir{ |>}{{}*!/-1pt/@{|}*!/-6pt/:(1,-.2)@^{>}*!/-6pt/:(1,+.2)@_{>}}

\newbox\pbbox
\setbox\pbbox=\hbox{\xy \POS(75,0)\ar@{-} (0,0) \ar@{-} (75,75)\endxy}
\def\pb{\copy\pbbox}
\newbox\urpbbox
\setbox\urpbbox=\hbox{\xy \POS(0,0)\ar@{-} (75,0) \ar@{-} (0,75)\endxy}
\def\urpb{\copy\urpbbox}
\newbox\upbbox
\setbox\upbbox=\hbox{\xy \POS(0,0)\ar@{-} (-40,40) \ar@{-} (40,40)\endxy}
\def\upb{\copy\upbbox}
\newbox\pobox
\setbox\pobox=\hbox{\xy \POS(0,75)\ar@{-} (0,0) \ar@{-} (75,75) \endxy}
\def\po{\copy\pobox}

% \newbox\tiltvdashbox
% \setbox\tiltvdashbox{\xy \POS( 

%% typical usage:
%
% $$\bfig \square[A`B`C`D;```]
% \place(100,400)[\pb]
% \place(400,100)[\po]
% \efig$$

% \newcommand{\todo}[1]{\marginpar{\textcolor{red}{#1}}}
\newcommand{\todo}[1]{}
\newcommand{\oldtodo}[1]{\todo{#1}}
% \newcommand{\padding}[1]{\textcolor{purple}{#1}}
% \newcommand{\comment}[1]{\textcolor{blue}{#1}}
\newcommand{\comment}[1]{}


\newcommand{\CompCat}{\mathbf{CompCat}}

\newcommand{\arr}{\mathrm{arr}}
\newcommand{\ext}{\mathrm{ext}}
% \newcommand{\Jbar}{\overline{J}}
\newcommand{\tree}{\mathrm{tree}}
\newcommand{\tr}{\mathrm{tr}}
\newcommand{\stuff}{{\Phi}}
% \makeindex

%%
%% PDFJUNK
%% Can add /CreationDate, /Creator, /Subject, /Keywords
%%
\ifpdf
\pdfinfo{
  /Author (Peter LeFanu Lumsdaine) 
  /Title (TODO: put thesis title here when decided!)
}
\fi
%%
%% BEGIN DOCUMENT:
%\onehalfspacing
\begin{document}



%% TITLE INFORMATION

\title{The classifying weak $\omega$-category of a type theory}

\author[P. LeF. Lumsdaine]{Peter LeFanu Lumsdaine}

\maketitle
\tableofcontents

\section{Type-theoretic background}

\comment{To include: give basic type theory syntactically---first the structural core, then constructors and various extensionality axioms.  Then give categorical equivalent; crucially, set $\DTT$ up as ess.\ alg.\ and show that extra rules/constructors are ess.\ alg.\ extensions.  Discuss the various constructions: dependent contexts, slicing, co-slicing.  Finally, discuss normalisation/canonicity!}

\subsection{Syntactic presentation}

\begin{para}Several different, mostly equivalent, syntactic presentations of Martin-Löf-style dependent type theory exist in the literature; the present one is essentially based (up to notation) on those of \cite{pitts:categorial-logic} and \cite{hofmann:syntax-and-semantics}.

A word about the range of theories we will consider is, however, in order here.  All theories we consider will share the common basic syntax and structural core presented in \ref{para:basic-syntax} and \ref{para:structural-core}.  We will, however, vary the constructors we add on top of that, in two stages.  Firstly, by a \emph{type system} we will mean the extension of this structural core by some selection $\stuff$ of the constructors and rules of \ref{para:constructors} and \ref{para:rules}, and of other standard constructors and rules, within certain limitations---a precise statement will appear in \ref{para:types-systems-as-ess-alg}.  Secondly, over a given such system $\stuff$, we will consider arbitrary extensions by \emph{algebraic} axioms, organised into a category $\DTT_\stuff$ of \emph{theories over $\stuff$}.
\end{para}

\begin{para}[Basic syntax]
Since we will eventually work in a presentation-agnostic category of type theories, the precise formalism we use for the raw syntax will not be of importance; but for the sake of definiteness, let us suppose a simply-typed metalanguage in which the syntax of our theory is formalised, as described in e.g.\ \cite[6.1]{pitts:categorial-logic}, and in which we have defined notions of free variables, capture-avoiding substitution, etc.

The basic judgements of our theory are now minimal---\emph{types}, \emph{terms}, and \emph{definitional equalities} between each of these:
\end{para}
\begin{center}\begin{tabular}{@{\ }c@{\hskip 0.75in}c@{\ }}
$\Gamma \types A \ \type $ & $ \Gamma \types a:A $ \\ \rule{0pt}{3ex} 
$\Gamma \types A = A' \ \type $& $ \Gamma \types a = a' : A $ \\
\end{tabular}
\end{center}

On top of these, we work with derived judgements of \emph{contexts} and \emph{context morphisms}\footnote{also sometimes known as \emph{subtitutions} or \emph{telescopes}} between them:
\begin{center}\begin{tabular}{@{\ }c@{\hskip 0.75in}c@{\ }}
$\types \Gamma \ \cxt$ & $ \types  f : \Gamma \To \Gamma' $  \\ \rule{0pt}{3ex} 
$\types \Gamma = \Gamma' \ \cxt$ &  $\types { f} = { f}' : \Gamma \To \Gamma'$ \\
\end{tabular}
\end{center}

(In all these judgements, we put the obvious restrictions on the free variables of the objects involved.)

In the sequel, we will sometimes also use dependent versions of these derived judgements, though they will not be needed in the presentation of the theory:
\begin{center}\begin{tabular}{@{\ }c@{\hskip 0.75in}c@{\ }}
$ \Gamma \types \Delta \ \cxt$ & $\Gamma \types { f} : \Delta \To \Delta' $  \\ \rule{0pt}{3ex} 
$ \Gamma \types \Delta = \Delta' \ \cxt$ & $\Gamma \types { f} = { f}' : \Delta \To \Delta'$ \\
\end{tabular}
\end{center}

In slight abuse of notation, we will often explicitly display free variables for emphasis or readability, writing e.g.\ $\x : \Gamma \types A(\x)\ \type$ interchangeably with $\Gamma \types A\ \type$.

Also, we will often abbreviate multiple judgements of the same form by writing e.g.\ $\Gamma \types a,a' : A$.

\begin{para}[Structural core] Rules for contexts:
$$\begin{array}{c}
\inferrule*[right=$\cxt$-$\empt$]{ \ }{\ \types \diamond \ \cxt\ } \qquad \inferrule*[right={$\cxt$=-$\empt$}]{\ }{\ \types \diamond = \diamond \ \cxt\ }
\\ \ \\
\inferrule*[right=$\cxt$-$\cons$]{\types \Gamma\ \cxt \\\\ \Gamma \types A \ \type}{\ \types \Gamma, y:A\ \cxt\ } \qquad
\inferrule*[right={$\cxt$=-$\cons$}]{ \types \Gamma = \Gamma'\ \cxt \\\\ \Gamma \types A = A'[ \x / \x' ]\ \type}{\ \types \Gamma, y:A = \Gamma', y':A'\ \cxt\ } 
\end{array}
$$

Rules for types: 

$$\begin{array}{c}
\inferrule*[right=$\type$-$\subst$]{\x : \Gamma \types A(\x)\ \type \\\\ \types f : \Gamma' \To \Gamma}{\  \Gamma' \types A[f/\x] \type \ } \qquad
\inferrule*[right={$\type$=-$\subst$}]{ \Gamma \types A = A'\ \type \\\\  \types f = f' : \Gamma' \To \Gamma}{\  \Gamma' \types A[f/\x] = A'[f'/\x]\ \type\ }
\\ \ \\
\inferrule*[right={$\type$=-$\refl$}]{\Gamma \types A\ \type}{\ \Gamma \types A = A\ \type\ } \qquad
\inferrule*[right={$\type$=-$\sym$}]{\Gamma \types A = B\ \type}{\ \Gamma \types B = A\ \type\ } 
\\ \ \\
\inferrule*[right={$\type$=-$\trans$}]{\Gamma \types A = B\ \type \\\\ \Gamma \types B = C\ \type}{\ \Gamma \types A = C\ \type\ }
\end{array}$$


Rules for terms: 
$$\begin{array}{c}
\inferrule*[right={\sf var}]{\Gamma \types A\ \type \\ \Gamma, A \types \Delta\ \cxt}{\ \Gamma, x:A, \Delta \types x : A\ }
\qquad \qquad
\\ \ \\
\inferrule*[right={\sf term-coerce}]{\Gamma \types a : A \\\\ \Gamma \types A = A'\ \type}{\ \Gamma \types a : A'\ } \qquad \quad
\inferrule*[right={\sf term=-coerce}]{\Gamma \types a = a' : A \\\\ \Gamma \types A = A'\ \type}{\ \Gamma \types a = a' : A'\ }
\\ \ \\
\inferrule*[right=$\term$-$\subst$]{ \Gamma \types a : A \\\\  \types f : \Gamma' \To \Gamma}{\ \Gamma' \types a[f/\x] : A[f/\x]\ } \qquad
\inferrule*[right={$\term$=-$\subst$}]{ \Gamma \types a = a' : A \\\\ \types f = f' : \Gamma' \To \Gamma}{\ \Gamma' \types a[f/\x] = a'[f'/\x] : A[f/\x]\ }
\\ \ \\
\inferrule*[right={$\term$=-$\refl$}]{\Gamma \types a : A}{\ \Gamma \types a = a : A\ }
\qquad \qquad 
\inferrule*[right={$\term$=-$\sym$}]{\Gamma \types a = b : A}{\ \Gamma \types b = a : A\ }
\\ \ \\
\inferrule*[right={$\term$=-$\trans$}]{\Gamma \types a = b : A \\\\ \Gamma \types b = c : A}{\ \Gamma \types a = c : A\ } 
\end{array}$$

Rules for context maps: 
$$\begin{array}{c}
\inferrule*[right={$\cxtmap$-$\empt$}]{ \types \Gamma\ \cxt}{ \types \diamond : \Gamma \To \diamond} 
\qquad \qquad
\inferrule*[right={$\cxtmap$=-$\empt$}]{ \types \Gamma\ \cxt}{ \types \diamond = \diamond : \Gamma \To \diamond}
\\ \ \\
\inferrule*[right={$\cxtmap$-$\cons$}]{ \types f : \Gamma' \To \Gamma \\\\ \Gamma \types A\ \type \\\\ \Gamma' \types a : A[f/\x] }{\ \types f,a : \Gamma' \To \Gamma, y:A \ } \qquad
\inferrule*[right={$\cxtmap$=-$\cons$}]{ \types f = f' : \Gamma' \To \Gamma \\\\ \Gamma \types A\ \type \\\\ \Gamma' \types a = a' : A[f/\x] }{\ \Gamma \types (f,a) = (f',a') : \Delta' \To (\Delta, A) \ }
\end{array}$$

\end{para}

\begin{para}[Type constructors]

Core rules for $\Id$-types: 
$$\begin{array}{c}
\inferrule*[right=$\Id$-$\form$]{\Gamma \types A\ \type \\\\ \Gamma \types a, b : A}{\ \Gamma \types \Id_A(a,b)\ \type\ }$ \qquad $\inferrule*[right={$\Id$-$\intro$}]{\Gamma \types A\ \type \\\\ \Gamma \types a : A}{\ \Gamma \types r(a) : \Id_A(a,a)\ }
\\ \ \\
\inferrule*[right=$\Id$-$\elim$]{
\Gamma,\, x,y : A,\, u : \Id_A(x,y),\, \w : \Delta(x,y,u)\ \types\ C(x,y,u,\w)\ \type \\
\Gamma,\, z:A,\, \v : \Delta(z,z,r(z))\ \types\ d(z,\v) : C(z,z,r(z),\v) \\
\Gamma \types a, b : A \qquad \Gamma \types p : \Id_A(a,b) \qquad \Gamma \types \c : \Delta(a,b,p)}
{\Gamma\ \types\ \Jterm_{(A;\ x,y,u.\Delta(x,y,u);\ x,y,u,\w.C(x,y,u,\w))\,}(z,\v.\ d(z,\v);\ a,b,p,\c)\,: C(a,b,p,\c)}
\end{array}$$
Here all free variables are displayed, to emphasise the full formal binding that occurs in terms involving $\Jterm$; we will usually abbreviate the concluding term above to e.g.\ $\Jterm_C(d; a,b,p,\c)$, as for instance:
$$\begin{array}{c}
\inferrule*[right=$\Id$-$\comp$]{
\Gamma,\, x,y : A,\, u : \Id_A(x,y),\, \w : \Delta(x,y,u)\ \types\ C(x,y,u,\w)\ \type \\
\Gamma,\, z:A,\, \v : \Delta(z,z,r(z))\ \types\ d(z,\v) : C(z,z,r(z),\v) \qquad
\Gamma \types a : A \qquad \Gamma \types \c : \Delta(a,a,r(a))}
{\Gamma\ \types\ \Jterm_C(d;\ a,a,r(a),\c) = d(a) : C(a,a,r(a),\c)}
\end{array}$$

Core rules for $\Pi$-types:

$$\begin{array}{c}
\inferrule*[right=$\Pi$-$\form$]{\Gamma \types A\ \type \quad \Gamma\, x:A \types B \ \type }{\ \Gamma\ \types\ \Pi_{x : A} B\ \type\ }
\qquad
\inferrule*[right=$\Pi$-$\intro$]{\Gamma\, x:A \types b: B }{\ \Gamma\ \types\ \lambda x \tightcolon A.\ b : \Pi_{x : A} B\ }
\\ \ \\
\inferrule*[right=$\Pi$-$\app$]{\Gamma \types t : \Pi_{x : A} B \quad \Gamma \types a : A}{\ \Gamma\ \types\ t \tightcdot a : B[a/x]\ \ }
\inferrule*[right=$\Pi$-$\beta$]{\Gamma\, x:A \types b: B \quad \Gamma \types a : A}{\ \Gamma\ \types\ (\lambda x \tightcolon A.\ b) \tightcdot a = b[a/x] :  B[a/x]\ }
\end{array}$$

\end{para}

\begin{para}[Variant rules for $\Id$-types] \label{para:id-variants}
Two rules which will not be part of our core concern but which are worth mentioning here are the stronger eliminators for identity types: the \emph{reflection principle} of extensional type theory (which collapses propositional and definitional equality), and Thomas Streicher's eliminator $\Kterm$ (introduced in \cite{streicher:habilitationsschrift}):
$$\begin{array}{c}
\inferrule*[right={reflection}]{\Gamma \types e : \Id_A(a,b)}{\Gamma \types a = b : A}
\\ \ \\
\inferrule*[right=$\Kterm$]{\Gamma, x: A, u : \Id_A(x,x), \w : \Delta(x,u) \types C(x,u,\w)\ \type \\
\Gamma, z:A, \v : \Delta(z,r(z)) \types d(z,\v) : C(z,r(z),\v) \\
\Gamma \types a : A \qquad \Gamma \types p : \Id_A(a,a) \qquad \Gamma \types \c : \Delta(a,p)}
{\Gamma \types \Kterm_{A;\ x,u.\Delta;\ x,u,\w.C}(z,\v.\ d(z,\v);\ a,p,\c) : C(a,p,\c)}
\\ \ \\
\inferrule*[right=$\Kterm$-$\comp$]{\Gamma, x: A, u : \Id_A(x,x), \w : \Delta(x,u) \types C(x,u,\w)\ \type \\
\Gamma, z:A, \v : \Delta(z,r(z)) \types d(z,\v) : C(z,r(z),\v) \\
\Gamma \types a : A \qquad \Gamma \types \c : \Delta(a,r(a))}
{\Gamma \types \Kterm_{A;\ x,u.\Delta;\ x,u,\w.C}(z,\v.\ d(z,\v);\ a,r(a),\c) = d(a) : C(a,r(a),\c)}
\end{array}$$

These eliminators both essentially trivialise the higher-categorical structure in which we are interested, so we will mainly consider theories without them; however, they will provide interesting comparisons at times.  In the case of the reflection rule, this trivialisation is immediate.  In the case of $\Kterm$, it is slightly less obvious; but it turns out that from $\Kterm$ we can derive the ``(propositional) uniqueness of identity proofs'' principle, asserting that all elements of any identity type are equal (\cite{streicher:habilitationsschrift}, \cite{warren:thesis}).  See \ref{para:j-and-k-homotopically} for an alternative point of view on $\Kterm$.

Two less desctructive variations on the $\Id$-elimination rule are worth mentioning; we will not use these as rules per se, but in Section \ref{sec:Jbar} and elsewhere we will discuss some analogous principles.

Firstly, the dependent context $\Delta(x,y,u)$ which we have included in the premises is often omitted.  (Its inclusion is, in categorical terms, a Frobenius condition.) In the presence of $\Pi$-types, this is interderivable with our version ($\Delta$ can simply be curried over to the right-hand side).  In the absence of $\Pi$-types, the non-Frobenius version of the rule is simply not strong enough to be much use: one cannot even derive, for instance, the transitivity of propositional equality.

Secondly, \todo{[ask Thorsten about reference for this!]} there is a ``one-ended'' form of $\Id$-$\elim$ (in contrast to which the version above may be seen as ``two-ended''):
$$\inferrule*[right=$\Id$-$\elim^1$]{
\Gamma \types a : A \\
\Gamma,\, y : A,\, u : \Id_A(a,y),\, \w : \Delta(y,u)\ \types\ C(y,u,\w)\ \type \\
\Gamma,\, \v : \Delta(a,r(a))\ \types\ d(\v) : C(a,r(a),\v) \\
\Gamma \types b : A \qquad \Gamma \types p : \Id_A(a,b) \qquad \Gamma \types \c : \Delta(b,p)}
{\Gamma\ \types\ \Jterm^1_C( d;\, b,p,\c)\,: C(b,p,\c)}
$$
and with computation rule concluding $\Jterm^1_C( d;\, a,r(a),\c) = d(\c)$.

This easily implies the original form; and in the presence of $\Sigma$-types, they are inter-derivable.
\end{para}

\begin{para}[Functional extensionalty and $\eta$-rules] \label{para:ext-rules}
\todo{To include: $\Pi$-$\eta$, $\Pi$-$\eta$-$\prop$, $\Pi$-$\extrule$, $\Pi$-$\extrule$-$\app$(-$\defrule$), and $\PiIdelim$.}

It turns out that with just the core rules of \ref{para:Pi-core}, propositional equality on $\Pi$-types is a rather more exclusive relation than one might desire.  Various different rules have been considered relaxing it, and in particular relating to the principle of ``functional extensionality'': that two functions are equal if they are equal on values.  A systematic comparison is given in \cite{garner:on-the-strength}, the terminology of which we mostly follow here.  Several rules asserting equalities may be given in both propositional and definitional flavours.

$$
\inferrule*[right={$\Pi$-$\eta$}]{\Gamma \types f : \Pi_{x : A} B}{\ \Gamma \types f = (\lambda x. f \cdot x) : \Pi_{x : A} B \ } 
\qquad \inferrule*[right={$\Pi$-$\prop$-$\eta$}]{\Gamma \types f : \Pi_{x : A} B}{\ \Gamma \types \eta(f) : \Id_{\Pi_{x : A} B} (f, (\lambda x. f \cdot x))\ } 
$$

 $$ \inferrule*[right={$\Pi$-$\prop$-$\eta$-$\comp$}]{\Gamma,\,x:A \types b(x) : B(x) }{\ \Gamma \types \eta(\lambda x. b(x)) = r(\lambda x. b(x)) : \Id_{\Pi_{x : A} B} (\lambda x. b(x), \lambda x. b(x))\ }
$$

 $$ \inferrule*[right={$\Pi$-$\extrule$}]{\Gamma \types f, g : \Pi_{x : A} B \\ \ \Gamma \types k : \Pi_{x : A} \Id_B(f \cdot x, g \cdot x)}{\ \Gamma \types \extterm(f,g,k) : \Id_{\Pi_A B}(f,g)\ }
$$ 

 $$ \inferrule*[right={$\Pi$-$\extrule$-$\comp$}]{\Gamma,\,x:A \types b : B }{\ \Gamma \types \extterm(\lambda x. b, \lambda x. b, \lambda x. r(b)) = r(\lambda x. b) : \Id (\lambda x. b, \lambda x. b)\ } 
$$ 

 $$ \inferrule*[right={$\Pi$-$\extrule$-$\apprule$}]{\Gamma \types f, g : \Pi_{x : A} B \\ \Gamma \types k : \Pi_{x : A} \Id_B(f \tightcdot x, g \tightcdot x) \\ \Gamma \types a : A}{\ \Gamma \types \mu(f,g,k,a) : \Id(\extterm(f,g,k) \star a , k \cdot a)\ }
$$ 

 $$ \inferrule*[right={$\Pi$-$\extrule$-$\apprule$-$\comp$}]{\Gamma,\,x:A \types b(x) : B(x) \\ \Gamma \types a : A}{\ \Gamma \types \mu(\lambda x. b(x), \lambda x. b(x), \lambda x. rb(x), a) = rrb(a) : \Id_{B(a)} (b(a), b(a))\ } 
$$
\end{para}


\begin{para}[Algebraic rules] \label{para:alg-rules}
Beyond these specific constructors, we also consider extensions by arbitrary rules of a simpler form: \emph{algebraic} axioms of each of the basic judgements.

An \emph{algebraic type-forming axiom} is specified by a basic type-former $\Tsf$, unique to the axiom, together with a pre-context $\Gamma_\Tsf$; it then has introductory rule
$$\inferrule{\types \Gamma_\Tsf\ \cxt}{\ \x : \Gamma_\Tsf \types \Tsf(\x)\ \type\ }.$$

An \emph{algebraic term-forming axiom} is similarly given by a basic term-former $\tsf$, and pre-context $\Gamma_\tsf$ and pre-type $A_\tsf$; its rule is then
$$\inferrule{ \x : \Gamma_\Tsf \types A_\tsf(\x) \ \type}{\ \x : \Gamma_\Tsf \types \tsf(\x) : A_\tsf(\x)\ }.$$

And \emph{algebraic type-} and \emph{term-equality axioms}, similarly, are of the form:
$$ \inferrule{ \Gamma \types A,\, A'\ \type}{\ \x : \Gamma \types A = A'\ \type\ } 
  \qquad \qquad
  \inferrule{ \Gamma \types a, a' : A }{\ \Gamma \types a = a' : A\ }$$
for some given $\Gamma,\,A,\,A'$ or $\Gamma,\,A,\,a,\,a'$.
\end{para}



\begin{para}[Logical frameworks]
An alternate presentation of the syntax is via a \emph{logical framework}: rather than starting with raw syntax and then defining the judgements of well-formedness on top of that, we define well-formed types, terms etc.\ directly from the start; the logistical cost of this is using a dependently-typed metalanguage, the \emph{logical framework}, assumed to have $\Pi$-types and their $\eta$-rule, plus one ``universe'' in which we encode our object language.  \cite{hofmann:syntax-and-semantics} gives a useful overview of the two presentations, together with a crucial comparison of their formal strengths, which we recall in Example \ref{ex:hofmann-contractibility}.

We will briefly have cause to work in the logical framework presentation, and in particular to discuss the rule $\PiIdelim$ (and its associated computation rule $\PiIdcomp$), from \ref{garner:on-the-strength}, which involves second-order quantification and so cannot be directly expressed in the earlier presentation.  This rule is a strong extensionality principle, asserting that the type $u, v : \Pi_{x:A}B(x), \types \Pi_{x:A} \Id_{B(x)}( u \tightcdot x,v \tightcdot x)$ is generated by canonical elements of the form $\lambda x. r(b(x))$: 
$$ \inferrule*[right={$\PiIdelim$}]{
\Gamma,\ u, v : \Pi_{x:A}B(x), w : \Pi_{x:A} \Id_{B(x)}(u \cdot x,v \cdot x)\ \types\ C(u,v,w)\ \type \\ 
\Gamma,\ f : (x \tightcolon A) B(x)\ \types\ d(f) : C (\lambda f, \lambda f, \lambda (r \circ f)) \\
\Gamma\ \types\ k, k' : \Pi_{x:A} B(x) \qquad \Gamma\ \types\ l : (x \tightcolon A) \Id_{B(x)}(k x, k' x) }
{ \Gamma\ \types\ \Lterm(C,d,k,k',l) : C(k,k',l) } $$

Its computation rule concludes from appropriate premises that
$$ \Lterm(C,d, \lambda h, \lambda h, \lambda ( r \circ h)) = d(h) : C( \lambda h, \lambda h, \lambda (r \circ h)).$$

(Here $(x \tightcolon A)B(x)$ and $[x \tightcolon A] b(x)$ denote abstraction in the metalanguage.)

\cite[5.11]{garner:on-the-strength} shows that (over the core $\Pi$-type rules) the rules $\PiIdelim$(-$\comp$) are inter-derivable with the conjunction of the first-order rules $\Pi$-$\extrule$(-$\comp$) and $\Pi$-$\extrule$-$\app$(-$\comp$).
\end{para}























% Yarright!  To do in categorical representations:
% 
% --- Say: many!  Name names, praise comprehension cats.
% DONE
% --- Define: CwA, accessible, stratified.
% DONE
% --- ``Honest 1-equivalence'' of \CwA_strat with \SynDTT; adjoint, so univ props of theories: maps out in terms of axioms.
% DONE.
% 
% --- constructors!  /DON'T GO INTO DETAILS!/ --- leave that as \pad{...}
% DONE (possibly in too much detail!)
% --- extend equivalence.
% --- ``presentation-agnostic'': DTT_\Phi
% 
% --- note: ess. alg!  Hence: adjunctions.
% --- define extent of generality of \Phi: ess. alg. extensions of CwA's (_not_ just of CwAstrat???  yerk; think a bit, to make sure it really holds up under the LF encoding and things)
% 
% --- constructions between stratified and non (mostly leave as at present)
% DONE WELL.
% 
% --- normalisation results: overview
% DONE WELL.


\subsection{Categorical representation}

The syntax is an excellent tool for intuitive and computationally tractable presentations of theories; however, for semantic/categorical purposes its formal complexity can be a hindrance.  To this end, various categorical structures have been introduced which correspond, more or less closely, to dependent type theories as presented above\footnote{Most relevantly for our purposes, \emph{comprehension categories} (\cite{jacobs:comprehension-categories}), \emph{categories with attributes} (\cite{cartmell-thesis}, \cite{moggi:program-modules}, \cite{pitts:categorial-logic}), \emph{categories with attributes} (\cite{dybjer:internal-type-theory}), \cite{hofmann:syntax-and-semantics}, \emph{contextual categories} (\cite{cartmell:gats}, \cite{streicher:semantics}); see \cite{jacobs:comprehension-categories}, \cite{hofmann:syntax-and-semantics} for useful overviews.}.

Of these, the most categorically flexible are probably Jacobs' \emph{comprehension categories}, which admit many useful variations; since in the present work we will stick closely to the type theory, however, we will use Cartmell's \emph{categories with attributes} as our main model, as these are very elementarily presented.  The relationships between various models are summed up below in Fig.~\ref{fig:big-picture}.

\begin{definition}A \emph{category with attributes} consists of:
\begin{itemize}
\item a category $\C$, with a distinguished terminal object $\diamond$;
\item a functor $\Ty \colon \C^\op \to \Sets$; that is, for each object $\Gamma$, a set $\Ty(\Gamma)$, and for $f \colon \Delta \to \Gamma$, actions $f^* \colon \Ty(\Gamma) \to \Ty(\Delta)$, functorial in $\Gamma$;
\item for each $A \in \Ty(\Gamma)$, an object $\Gamma . A$ and projection $\pi_{\Gamma;A} \colon \Gamma . A \to \Gamma$;
\item for each $A \in \Ty(\Gamma)$ and $f \colon \Delta \to \Gamma$, a pullback square
$$\bfig
\node Delta(0,0)[\Delta]
\node Gamma(600,0)[\Gamma]
\node DeltaA(0,400)[\Delta.f^*A]
\node GammaA(600,400)[\Gamma.A]
\arrow[Delta`Gamma;f]
\arrow[DeltaA`Delta;\pi_{\Delta;f^*A}]
\arrow[GammaA`Gamma;\pi_{\Gamma;A}]
\arrow[DeltaA`GammaA;q(f,A)]
\place(100,400)[\pb]
\efig$$
again functorial in $f$, in that $q(1_\Gamma,A) = 1_{\Gamma . A}$, $q(f \circ g, A) = q(f,A) \circ q(g, f^*A)$.
\end{itemize}
\end{definition}

\begin{para} We'll use a few pieces of obvious terminology for working in CwA's.  Objects and maps of $\C$ we call \emph{contexts}, and \emph{context maps}.  Elements of $\Ty(\Gamma)$ we call \emph{types over $\Gamma$}.

For a type $A \in \Ty(\Gamma)$, a \emph{term of type $A$ in context $\Gamma$} is a section $a : \Gamma \to \Gamma.A$ of the dependent projection $\Gamma.A \to \Gamma$; we write $\Tm_\Gamma(A)$ for the set of these.

For an object $\Gamma \in \C$, a \emph{dependent context} over $\Gamma$ is a sequence $A_1 \in \Ty(\Gamma)$, $A_2 \in \Ty(\Gamma.A_1)$, \ldots $A_l \in \Ty(\Gamma.A_1.\ldots.A_{l-1})$, for some $l \geq 0$; we write $\Cxt(\Gamma)$ for the set of these.
\end{para}

\begin{definition}A category with attributes $\C$ is \emph{contextual} if the natural map $\Cxt(\diamond) \to \ob \C$ is a bijection, and \emph{accessible} \cite{pitts:categorial-logic} when this map is a surjection.

Equivalently, $\C$ is contextual if there is a \emph{length} function $l \colon \ob \C \to \N$, such that
\begin{itemize}
\item $\diamond$ is the unique context of length $0$;
\item for any context $\Gamma$ and type $A$, $l(\Gamma.A) = l(\Gamma) + 1$; and
\item for any context $\Delta$ of length $n > 0$, there are unique $\Gamma$, $A$ such that $\Delta = \Gamma.A$.
\end{itemize}
\end{definition}

Contextual CwA's correspond precisely to the \emph{contextual categories} of Cartmell \cite{cartmell:gats} and Streicher \cite{streicher:semantics}, and hence, crucially, to dependent type theories without constructors:

\begin{proposition}[Cartmell \cite{cartmell:gats}] \comment{[I had a note here attributing it also to Fourman, but no details!  Is it also partly due to Fourman??]} \label{prop:CwA-equivalence}
The category of small CwA's is \emph{equivalent} to the category of type theories presented syntactically by a small set of purely algebraic axioms (i.e.\ dependent terms and types, and equality axioms between them) and interpretations between such theories.

Specifically, there is an adjoint equivalence
$$\bfig 
\node CwA(800,0)[\CwA_\cxl]
\node SynThy(0,0)[\SynThy]
\arrow|b|/@/^0.5em//[CwA`SynThy;\Lang]
\arrow|a|/@/^0.5em//[SynThy`CwA;\cl]
\place(400,0)[\equiv]
\efig$$
taking a syntactically presented type theory $\T$ to its \emph{classifying category} $\cl(\T)$, and a CwA $\C$ to its \emph{internal language} $\Lang(\C)$.
\end{proposition}

(Note that this is an honest 1-equivalence; the various categories involved carry natural 2-category structures, but for the present we study them purely 1-categorically.  This is our reason for working specifically with contextual CwA's, rather than just e.g.~admissible)

\begin{para}This equivalence justifies working with a \emph{presentation-agnostic} category of type theories, which we denote $\DTT$: we will construct and work with objects of $\DTT$ (\emph{theories}) sometimes as syntactic presentations, sometimes as categories with attributes.  Given any construction either on syntactically presented theories or on contextual CwA's, we'll transfer it without comment to $\DTT$, and so forth.  In subsequent chapters, we will work more often with syntactic presentations, but for the constructions of this section it will be convenient to work primarily via $\CwA_\cxl$, for the sake of its connections to other categories of CwA's.

The major power of the $\CwA_\cxl$ presentation, however, is that it displays $\DTT$ as the category of models of a (small) \emph{essentially algebraic theory}\todo{[Ref!  Elephant?]}; this implies many useful categorical properties, including in particular that it is complete, cocomplete, and moreover locally presentable---in fact, finitely presentable, since contextual CwA's are a finitary theory.

The construction of the equivalence also gives us a useful universal property for theories presented syntactically: a map out of such a theory is just an interpretation, and hence is determined precisely by the interpretations of the axioms.
\end{para}

\begin{para}Of course, we want categories not just of \emph{algebraic} dependent type theories, but of type theories with \emph{constructors}; in particular, $\Id$- and $\Pi$-types.  These too can be succinctly and profitably defined in terms of CwA's.  In particular:
\end{para}

\begin{definition} An \emph{elim-structure} $J$ on a map $i : \Gamma \to \Theta$ is a function assigning, to every type $C \in \Ty(\Theta)$ and every term $d : \Gamma \to \Gamma.i^*C$ of type $i^*C$ (equivalently, every map $\hat{d} : \Gamma \to \Theta.C$ over $\Theta$) a term $J_{C,d} : \Theta \to \Theta.C$ of type $C$

An \emph{absolute (Frobenius?) elim-structure} on $i : \Gamma \to \Theta$ is an elim-structure $J_\Delta$ on $i.\Delta$ for each $\Delta \in \Cxt(\Theta)$.
\end{definition}

This axiomatises the structure provided by the elimination/computation rules for an inductive type with just a single introduction form $i$.  (It can be nicely generalised to deal with multiple introduction forms, but we will not need that.) In particular:

\begin{definition}
A \emph{CwA with $\Id$-types} is a CwA $\C$, together with:
\begin{itemize}
\item for each context $\Gamma \in \C$ and type $A \in \Ty(\Gamma)$, a type $\Id_A \in \Ty(\Gamma.A.A)$, and a morphism $r_A : \Gamma.A \to \Gamma.A.A.\Id_A$ over $\Delta_A : \Gamma.A \to \Gamma.A.A$ with an absolute elim-structure $J_A$,
\item all stable in $\Gamma$, in that for $f:\Gamma' \to \Gamma$ and $A \in \Ty(\Gamma)$, we have $(f.A.A)^*\Id_A = \Id_{f^*A} \in \Ty(\Gamma'.f^*A.f^*A)$, and so on.
\end{itemize}
\end{definition}

Write $\CwA^\Id$, $\CwA_\cxl^\Id$, etc.\ for the various categories of CwA's with $\Id$-types.  

\begin{proposition}[\cite{hofmann:syntax-and-semantics}, \cite{pitts:categorical-logic} \todo{[Original source?]}] This structure really does correspond precisely to the $\Id$-type rules: the equivalence of Proposition \ref{prop:CwA-equivalence} lifts to an equivalence between $\CwA^\Id_\cxl$ and a category $\SynThy^\Id$, of theories presented syntactically by the core $\Id$-type rules plus a set of algebraic axioms. 
\end{proposition}

\begin{para}Similarly, we may define structure on the categorical side correponding to the core $\Pi$-type rules, or any of the other rules of \ref{para:id-variants}, \ref{para:ext-rules} above (with obvious dependencies: the structure for $\extrule$ rules refers to the $\Pi$- and $\Id$-types structure, and so on).

In particular, if $\stuff$ is any appropriate collection of the above rules, then we can obtain a category $\CwA_\cxl^\stuff$ and an equivalence $\CwA_\cxl^\stuff \equiv \SynThy^\stuff$ as before, and justified by this we introduce a presentation-agnostic category $\DTT_\stuff$ of theories ove $\stuff$.

In the case of all the rules above, theories over $\stuff$ are again models of an essentially algebraic theory extending the base theory of contextual CwA's.  This motivates the level of generality we work at: when we speak informally of a \emph{collection of rules/constructors} $\stuff$, what we mean formally is an algebraic theory extending the theory of contextual CwA's, with the same sorts.

(This is not a terribly natural definition.  It allows type-theoretically unnatural constructions, such as rules which hold only over contexts of length 17, and the like; and, worse, important constructions such as the logical framework embedding cannot defined in this generality.  However, as a crude demarcation, it will suffice for the scope of the present work.) \todo{[make sure this is really true! I know nothing substantial depends on the issue, but some phrasings may have to be restricted]}

This immediately gives us comparison functors between different categories of theories.  If $\stuff'$ is any collection of rules/constructors extending $\stuff$, then we have a map of algebraic theories, and hence an adjunction:
$$\bfig 
\node DTT1(0,0)[\DTT_\Phi]
\node DTT2(800,0)[\DTT_{\Phi'}]
\arrow|a|/@/^0.6em//[DTT1`DTT2;F]
\arrow|b|/@/^0.6em//[DTT2`DTT1;U]
\place(400,0)[\bot]
\efig$$
where, in syntactic terms, $F$ adds the extra rules/constructors of $\stuff$ to a theory, while $U$ forgets structure.  In particular, if $\T$ is a theory over $\Phi$ presented by some collection of axioms, then $F(\T)$ is the theory presented bythe same axioms over $\Phi'$. 

This is the main reason why we consider second-order rules such as $\PiIdelim$ separately.  They do \emph{not} (at least a priori) correspond to extra essentially algebraic structure over contextual CwA's; in the present categorical setup, they can be discussed only within their logical framework embeddings.
\end{para}

\begin{para}{Left and right maps in a CwA} \label{subsec:left-right-in-CwA}
The above presentation of $\Id$-types via elim-structures is based on ideas of Gambino and Garner (\cite{gambino-garner}) which will be very useful to us: that any CwA has an important class of left and right maps, which in the presence of $\Id$-types is moreover a wfs.

The right maps are just (maps isomorphic to) (compositions of basic) dependent projections.  The left maps are maps admitting an (or algebraically: with a chosen) elim-structure; following observations above, this gives exactly $\L = {}^\pitchfork \R$.  

(\cite{gambino-garner} moreover gives an alternative type-theoretic characterisation of each class of maps, and uses $\Id$-types to construct $\L,\R$ factorisations, but we will not need these.)

\comment{Find a place for this remark, of scrap it: compare the elim-structure with the last refinement of the ``left-maps-from-right'' described in Example \ref{ex:left-maps-from-right} below, with the basic dependent projections $\Theta.A \to \Theta$ as the generating $\R$-maps.}


\end{para}


\subsection{Constructions on dependent type theories}

There are several interesting and important ways to construct new type theories from old:

\begin{para} First, the \emph{dependent contexts monad} on $\CwA$ over $\Cat$, sending $\C = (\C,\Ty)$ to $\C^\Cxt := (\C,\Cxt)$.  So types of the new theory are just dependent contexts in the old theory; the base category is unchanged, and context extension is just by concatenation.  This can be seen as the monad for ``very strong, strictly associative $\Sigma$-types''.)  

Moreover, given $\Id$-types structure on $\C$, we can extend this to an $\Id$-types structure on $\C^\Cxt$: this is the ``identity contexts'' of \cite{streicher:habilitationsthesis} or \cite{gambino-garner}.  \todo{[Give more specific citations, and/or present details?]}  So we have an endofunctor $(-)^\Cxt$ on $\CwA^\Id$. However, the monad structure does \emph{not} lift to $\CwA^\Id$: the multiplication turns out not to preserve $\Id$-contexts strictly.  (This could presumably be accommodated by developing a theory of pseudo-maps of $\CwA^\Id$'s.)

Similar considerations let us lift it to act as an endofunctor on CwA's with any of $\Sigma$-types, $\Pi$-types, the functional extensionality rules, and most other standard constructors.

Varying our CwA's along the other axis, ``contextuality'', for each set of constructors $(-)^\Cxt$ lifts to an endofunctor on $\CwA_\diamond$, but not on $\CwA_\cxl$: its result is almost never contextual, since adjoining two types $A$ and $B$ to a context in succesion has the same result as adjoining $(A,B)$ in one step, so (as long as the original theory had any types at all) there can be no well-defined notion of length.
\end{para}

\begin{para} The \emph{slice} construction is one of the fundamental tools of the category--type~theory correspondence; however, in terms of CwA's, it is not exactly the ordinary categorical slice.

For $\C$ any CwA, and $\Gamma$ any object of $\C$, the \emph{(type-theoretic) slice} $\C \slice \Gamma$ has as objects dependent contexts over $\Gamma$, and morphisms and attributes structure induced by pullback along the map $\ob (\C \slice \Gamma) \to \ob \C$ sending $\Delta \in \Cxt_\C(\Gamma)$ to $\Gamma,\Delta$.

Slices are always contextual; in particular, by slicing each $\C \in \CwA_\diamond$ over $\diamond$, we obtain a right adjoint $- \slice \diamond \CwA_\diamond \to \CwA_\cxl$ to the inclusion $\CwA_\cxl \mono \CwA$.

In syntactic terms, slicing corresponds to taking variables into the context: judgements $\Delta \types \J$ in $\T \slice \Gamma$ correspond exactly to judgements $\Gamma, \Delta \types \J$ in $\T$.

(There is also a natural CwA structure on the \emph{categorical slice} $\C/\Gamma$, but this will not be of importance to us.) 
\end{para}

\begin{para} \label{par:types-to-cxts}Combining this with the dependent contexts construction gives us an endofunctor of $\CwA_\cxl$:
$$ \CwA_\cxl \mono<300> \CwA_\diamond \to^{(-)^\Cxt} \CwA_\diamond \to^{- \slice \diamond} \CwA_\cxl$$

An object of the resulting $\C^\Cxt \slice \diamond$ can be seen as an object $A_1,\ldots,A_k$ of $\C$ together with a partition $k = \sum_{i \leq l} k_i$, thought of as dividing the context into chunks: 
$$(A_1,\ldots,A_{k_1});\ (A_{k_1 + 1}, \ldots , A_{k_1 + k_2});\ \ldots ;\  (A_{\sum_{i < l}k_i + 1}, \ldots, A_k).$$
A ``type'' over such an object is then a dependent context over the full (un-chunked) context in $\C$; context extension adjoins the dependend context as a single additional chunk.  There is a natural inclusion $\C \mono \C^\Cxt \slice \diamond$, chunking each context into individual types.

This construction will be briefly but crucially useful to us, in Corollary \ref{cor:types-to-cxts}.
\end{para}

\begin{para} With some of our constructors, we can also construct \emph{co-slice theories}.  In a co-slice $\Theta \coslice \C$, an object is a map $g \colon \Theta \to \Gamma$ of $\C$ (a \emph{$\Theta$-pointed object of $\C$}); a map $(g,\Gamma) \to (d,\Delta)$ is a map $f \colon \Gamma \to \Delta$ of $\C$ with $fg = d$ (i.e.\ preserving the ``point''); and a type over $(g,\Gamma)$ is a type $A \in \Ty(\Gamma)$ together with a term $a$ of $g^*A$, or equivalently a point $(g,a) \colon \Theta \to \Gamma, A$ for which $\pi_{\Gamma;A} (g,a) = g$.  

There is an obvious projection $\Theta \coslice \C \to \C$, forgetting points.

This construction preserves contextuality; it also extends to act on $\Id$-types, $\Sigma$-types, $\One$, and more generally on inductive types with a single unary constructor; but it does \emph{not} act on $\Pi$-types, nor on $\Bool$, $\Zero$, or most other type-formers.  This is familiar categorically: co-slices retain e.g.\ binary products and terminal objects, but not coproducts or exponentials.
\end{para}

\begin{para}[Normalisation results]  It is a fundamental fact, going back to \cite{martin-lof:predicative-part}, that the basic structural theory together with any subset of the standard constructors ($\Id$-, $\Sigma$-, $\Pi$-types, and also $\Nat$ and $\Bool$) is strongly normalising.

Moreover, it is easy to see \todo{[but is it proven anywhere citable??]} that this result extends to theories including algebraic type- and term-forming axioms.  (It can fail, however, under the addition of algebraic definitional equality axioms.)

Normalisation is especially informative for theories also posessing the \emph{canonicity} property, that all normal forms are ``canonical'', i.e.\ are terms formed from the $\intro$-rule constructors.  Unfortunately, this will fail in many theories of interest to us---in particular, to theories with terms of identity types adjoined.  We will discuss canonicity of particular theories as and when we require it.

It is currently somewhat unclear to what extent one can retain strong normalisation in conjunction with the functional extensionality rules.  The Observational Type Theory of Altenkirch and collaborators (\cite{altenkirch:ott}, \cite{altenkirch-mcbride-swierstra}) achieves this, but has \emph{defined} rather than axiomatic identity types (and therefore does not easily permit extension by further type axioms), and moreover forces these to be trivial: any two terms of an identity type are equal (the UIP axiom of \cite{hofmann:extensional??}).  However, the OTT system is an encouraging step towards the development of a fully intensional system with functional extensionality.  (The difficulty lies essentially in defining the computational behaviour of the extensionality combinator; this seems to be related to the admissibility of the principle $\Jbar$ of Section \ref{sec:jbar}.)
\end{para}














\clearpage

%% Bibliography Info

\bibliographystyle{amsalpha}
\bibliography{pll-thesis-bib}



\end{document}